\documentclass{article}
\usepackage[a4paper, margin=1in]{geometry}
\usepackage{graphicx} % Required for inserting images
\usepackage{longtable}
\usepackage[hidelinks]{hyperref} %custom
\hypersetup{
colorlinks,
linktoc=all,
citecolor=black,
filecolor=black,
linkcolor=black,
urlcolor=blue
}

\title{Foclóir Tráchtais v1.6}
\author{Jeffrey Seathrún Sardina}
\date{Márta 2025}

% setup bibliography
\usepackage[
backend=biber,
style=numeric,
sorting=ynt
]{biblatex}
\addbibresource{refs.bib}

\begin{document}

\maketitle

\newpage
\tableofcontents
% Phantom sections explanation: https://tex.stackexchange.com/questions/364010/hyperref-and-titlesec-conflict-and-warning"

\newpage \section*{Achoimre na dTéarmaí}
\begin{longtable}{|l|l|}
	\hline
		\textbf{Béarla} & \textbf{Gaeilge}\\ \hline 
		0-shot&0-sonra\\ \hline 
		ablation experiment&turgnamh ionadaithe\\ \hline 
		to abstract&teibigh\\ \hline 
		abstraction&teibiú\\ \hline 
		activation function&feidhm ghníomhachtaithe\\ \hline 
		adjacency matrix&maitrís chóngarachta\\ \hline 
		aggregate&tionól\\ \hline 
		algorithm&algartam\\ \hline 
		alignment&ailíniú\\ \hline 
		antecedent&réamhthéarma\\ \hline 
		application (in practice)&úsáid (phraiticiúil)\\ \hline 
		application domain&réimse úsáide\\ \hline 
		to approximate&meastachán a dhéanamh (ar)\\ \hline 
		approximation&meastachán\\ \hline 
		arbitrary&treallach\\ \hline 
		architecture&dearadh\\ \hline 
		artificial intelligence (AI) (concept)&intleacht shaorga (IS) (coincheap)\\ \hline 
		artificial intelligence (AI) (system)&córas intleachta saorga (córas IS)\\ \hline 
		assumption&foshuíomh\\ \hline 
		atomic&adamhach\\ \hline 
		to attend to&aird a thabhairt ar\\ \hline 
		attention&aird\\ \hline 
		attention head&ceann airde\\ \hline 
		attention layer&ciseal airde\\ \hline 
		attention mechanism&oibriú airde\\ \hline 
		baseline model&samhail bhunlíne\\ \hline 
		batch&baisc\\ \hline 
		batch size&méid na mbaisceanna\\ \hline 
		Bayesian&Bayes\\ \hline 
		benchmark dataset&tacar sonraí comparáide\\ \hline 
		bin&eatramh\\ \hline 
		binary&dénártha\\ \hline 
		binary cross entropy loss (BCEL)&pionós tras-eantrópachta dénártha (PTED)\\ \hline 
		block&modúl\\ \hline 
		to bound&cuimsigh\\ \hline 
		class&aicme\\ \hline 
		classification&aicmiú\\ \hline 
		classifier&aicmitheoir\\ \hline 
		to classify&aicmigh\\ \hline 
		Closed World Assumption&Foshuíomh an Domhain Dhúnta\\ \hline 
		cluster&braisle (pointí)\\ \hline 
		to cluster&aicmiú ionduchtach a dhéanamh\\ \hline 
		clustering&aicmiú ionduchtach\\ \hline 
		co-frequency&cóimhinicíocht\\ \hline 
		coefficient&comhéifeacht\\ \hline 
		computer network&líonra ríomhairí\\ \hline 
		computer science&ríomheolaíocht\\ \hline 
		connectivity&(frása le 'ceangailte')\\ \hline 
		consequent&iarmhairt\\ \hline 
		constant&buan-\\ \hline 
		correct fitting&foghlaim cheart\\ \hline 
		to correlate&comhghaolaigh\\ \hline 
		correlation&comhghaol\\ \hline 
		correlation coefficient&comhéifeacht comhghaolúcháin\\ \hline 
		to corrupt&malartaigh\\ \hline 
		counterexample&frith-shampla\\ \hline 
		cross entropy loss (CEL)&pionós tras-eantrópachta (PTE)\\ \hline 
		cross-attention&tras-aird\\ \hline 
		cross-entropy&tras-eantrópacht\\ \hline 
		curve&cuar\\ \hline 
		data&sonraí\\ \hline 
		data leak&sceitheadh sonraí\\ \hline 
		database&bunachar sonraí\\ \hline 
		dataset&tacar sonraí\\ \hline 
		decision boundary&teorainn chinnidh\\ \hline 
		decision tree&crann cinnte\\ \hline 
		deep learning&foghlaim dhomhain\\ \hline 
		degree&céim\\ \hline 
		dense&dlúth\\ \hline 
		dense layer&ciseal lán-cheangailte\\ \hline 
		density&dlús\\ \hline 
		dependency graph&graf spleáchais\\ \hline 
		DIKW Pyramid&Pirimid SFEE\\ \hline 
		dimension&toise\\ \hline 
		dimensionality&(frása le 'toise')\\ \hline 
		directed&dírithe\\ \hline 
		distribution&dáileadh\\ \hline 
		domain&fearann\\ \hline 
		dropout layer&ciseal nialas\\ \hline 
		edge&ceangal\\ \hline 
		efficiency&éifeachtacht (ama, fhuinnimh)\\ \hline 
		element&ball\\ \hline 
		to embed&leabaigh\\ \hline 
		embedding&leabú\\ \hline 
		end condition&coinníoll críochnaithe\\ \hline 
		entity&aonad\\ \hline 
		entity alignment&ailíniú aonad\\ \hline 
		entropy&eantrópacht\\ \hline 
		environment&comhthéacs cóid\\ \hline 
		epoch&seal\\ \hline 
		equation&cothromóid\\ \hline 
		error&earráid\\ \hline 
		estimate&meastachán\\ \hline 
		to estimate (about)&meastachán a dhéanamh (ar)\\ \hline 
		to evaluate&measúnaigh\\ \hline 
		evaluation&measúnú\\ \hline 
		expression&slonn\\ \hline 
		family&(frása le 'gaolmhar')\\ \hline 
		feature&airí\\ \hline 
		few-shot&cúpla-sonra\\ \hline 
		filter&scagaire\\ \hline 
		to filter&scag\\ \hline 
		filtering&scagaireacht\\ \hline 
		to finetune&mion-fheabhsú\\ \hline 
		foundation model&samhail fhorais\\ \hline 
		framework&creatlach\\ \hline 
		to freeze&buanaigh\\ \hline 
		frequency&minicíocht\\ \hline 
		function&feidhm\\ \hline 
		generality&ginearáltacht\\ \hline 
		generative&cumadóireachta\\ \hline 
		genetic algorithm&algartam géiniteach\\ \hline 
		global&uilíoch\\ \hline 
		global maximum&uasluach uilíoch\\ \hline 
		global minimum&íosluach uilíoch\\ \hline 
		graph&graf\\ \hline 
		graph foundation model&samhail fhorais ghraf\\ \hline 
		graph neural network (GNN)&líonra néarach graif (LNG)\\ \hline 
		grid search&cuardach ar eangach\\ \hline 
		ground truth&bun-fhírinneach\\ \hline 
		ground truth (data)&bun-fhírinne\\ \hline 
		histogram&histeagram\\ \hline 
		hits@k (H@k)&fíricí@k (F@k)\\ \hline 
		hop&coiscéim\\ \hline 
		hyper-graph&hipear-ghraf\\ \hline 
		hyperparameter&hipear-pharaiméadar\\ \hline 
		hyperparameter combination&teaglaim hipear-pharaiméadar\\ \hline 
		hyperparameter grid&eangach hipear-pharaiméadar\\ \hline 
		hyperparameter search&cuardach hipear-pharaiméadar\\ \hline 
		implementation&leagan infheidhmithe\\ \hline 
		information content&lucht faisnéise\\ \hline 
		input&ionchur\\ \hline 
		to input (into)&cuir isteach (i)\\ \hline 
		to instantiate&cruthaigh\\ \hline 
		instantiation (concept)&leagan\\ \hline 
		instantiation (object)&réad\\ \hline 
		instantiation (process)&cruthú\\ \hline 
		interface&comhéadan\\ \hline 
		intersection&idirmhír\\ \hline 
		interval&eatramh\\ \hline 
		inverse&inbhéartach\\ \hline 
		inverse (relation)&inbhéarta\\ \hline 
		knowledge graph (KG)&graf eolais (GE)\\ \hline 
		knowledge graph embedding (KGE)&leabú graif eolais (LGE)\\ \hline 
		knowledge graph embedding model (KGEM)&samhail leabaithe graif eolais (SLGE)\\ \hline 
		Kullback-Leibler (KL) divergence&dibhéirseacht Kullback-Leibler (KL)\\ \hline 
		label&lipéad\\ \hline 
		labelled&le lipéad\\ \hline 
		language model (LM)&samhail theanga (ST)\\ \hline 
		large language model (LLM)&samhail theanga mhór (STM)\\ \hline 
		latent&folaigh\\ \hline 
		layer&ciseal\\ \hline 
		learning rate&ráta foghlama\\ \hline 
		library&leabharlann (chóid)\\ \hline 
		linear&líneach\\ \hline 
		link prediction (LP)&réamhinsint nasc (RN)\\ \hline 
		link prediction query&ceist réamhinsinte nasc\\ \hline 
		link prediction task&tasc réamhinsinte nasc\\ \hline 
		link predictor&réamhinsteoir nasc\\ \hline 
		linked data (LD)&sonraí nasctha (SN)\\ \hline 
		linked open data (LOD)&sonraí nasctha saor-rochtana (SNSR)\\ \hline 
		to load&lódáil\\ \hline 
		local&logánta\\ \hline 
		local maximum&uasluach logánta\\ \hline 
		local minimum&íosluach logánta\\ \hline 
		logic&loighic\\ \hline 
		loss&pionós (foghlama)\\ \hline 
		loss function&feidhm phionóis\\ \hline 
		machine&ríomh-\\ \hline 
		machine learning&ríomhfhoghlaim\\ \hline 
		to map&mapáil\\ \hline 
		mapping&mapa\\ \hline 
		margin&bearna\\ \hline 
		margin ranking loss (MRL)&pionós rangaithe le bearna (PRB)\\ \hline 
		matrix&maitrís\\ \hline 
		maximum&uasluach\\ \hline 
		mean&meán- nó meánach\\ \hline 
		mean (statistic)&meán\\ \hline 
		mean rank (MR)&meán-rang (MR)\\ \hline 
		mean reciprocal rank (MRR)&meán na ranganna deilíneacha (MRD)\\ \hline 
		mean squared error (MSE)&meán na n-earráidí cearnaithe (MEC)\\ \hline 
		measure&tomhas\\ \hline 
		to measure&tomhais\\ \hline 
		median&airmheánach\\ \hline 
		median (statistic)&airmheán\\ \hline 
		message passing&seachadadh teachtaireachtaí\\ \hline 
		metric&tomhas\\ \hline 
		minimum&íosluach\\ \hline 
		to model&samhlaigh\\ \hline 
		model&samhail\\ \hline 
		modular&modúlach\\ \hline 
		module&modúl\\ \hline 
		motif&móitíf\\ \hline 
		multi-headed attention&aird il-cheannach\\ \hline 
		n-shot&n-sonra\\ \hline 
		negative (of number)&diúltach\\ \hline 
		negative (sample)&frith-shampla\\ \hline 
		negative sampler&frith-shamplóir\\ \hline 
		negative sampling&frith-shampláil\\ \hline 
		neighbour&comharsa\\ \hline 
		network&líonra\\ \hline 
		neural&néarach\\ \hline 
		neural network (NN)&líonra néarach (LN)\\ \hline 
		neuro-symbolic&néar-shiombalach\\ \hline 
		node&nód\\ \hline 
		noise&torann\\ \hline 
		noisy&torannach\\ \hline 
		non-linear&neamh-líneach\\ \hline 
		object&cuspóir\\ \hline 
		object corruption&malartú an chuspóra\\ \hline 
		object prediction&réamhinsint an chuspóra\\ \hline 
		one-hot encoding&códú aon-innéacs\\ \hline 
		ontology&ointeolaíocht\\ \hline 
		open source&saor-rochtana\\ \hline 
		Open World Assumption&Foshuíomh an Domhain Oscailte\\ \hline 
		operation&oibriú\\ \hline 
		operator&oibreoir\\ \hline 
		optimisation&feabhsúchán\\ \hline 
		to optimise&feabhsaigh\\ \hline 
		optimiser&córas feabhsúcháin\\ \hline 
		outlier&asluiteach\\ \hline 
		output&aschur\\ \hline 
		to output&cuir amach\\ \hline 
		overfitting&ró-fhoghlaim\\ \hline 
		pairwise function&feidhm de phéire (pointí)\\ \hline 
		parameter&paraiméadar\\ \hline 
		percentile&peircintíl\\ \hline 
		performance&éifeachtacht (ama, taisc)\\ \hline 
		plausibility&inchreidteacht\\ \hline 
		plausibility score&scór inchreidteachta\\ \hline 
		pointwise function&feidhm de phointe\\ \hline 
		positive (of number)&deimhneach\\ \hline 
		positive (triple)&fíor-abairt (thriarach)\\ \hline 
		predicate&faisnéis\\ \hline 
		to predict&réamhinis\\ \hline 
		prediction&réamhinsint\\ \hline 
		predictor&réamhinsteoir\\ \hline 
		to pretrain&réamh-thraenáil\\ \hline 
		pretraining&réamh-thraenáil\\ \hline 
		probability&dóchúlacht\\ \hline 
		proportion&comhréir\\ \hline 
		pseudotype&aicme chumtha\\ \hline 
		pseudotyped&le haicme chumtha\\ \hline 
		qualitative&cineálach\\ \hline 
		to quantify&cainníochtaigh\\ \hline 
		quantitative&cainníochtúil\\ \hline 
		query&ceist\\ \hline 
		random&randamach\\ \hline 
		random number generator&gineadóir uimhreacha randamacha\\ \hline 
		random sample&sampla fánach\\ \hline 
		random search&cuardach randamach\\ \hline 
		random seed&fréamh randamach\\ \hline 
		random walk&siúlóid fhánach\\ \hline 
		range&raon\\ \hline 
		rank&rang\\ \hline 
		to rank&rangaigh\\ \hline 
		ranked list&liosta ranganna\\ \hline 
		ranking&rangú\\ \hline 
		ratio&cóimheas\\ \hline 
		to reason (on)&réasúnaíocht a dhéanamh (ar)\\ \hline 
		reasoner&córas réasúnaíochta\\ \hline 
		reasoning&réasúnaíocht\\ \hline 
		reference implementation&leagan infheidhmithe caighdeánach\\ \hline 
		region&réigiún\\ \hline 
		to regress&cúlaigh\\ \hline 
		regression&cúlú\\ \hline 
		regressor&cúlaitheoir\\ \hline 
		regularisation&tabhairt chun rialtachta\\ \hline 
		to regularise&tabhair chun rialtachta\\ \hline 
		regulariser&córas rialtachta\\ \hline 
		reification&tearcú\\ \hline 
		to reify&tearcaigh\\ \hline 
		relation(ship)&ceangal\\ \hline 
		relative entropy&eantrópacht choibhneasta\\ \hline 
		ReLU&ReLU\\ \hline 
		repository&stór (cóid)\\ \hline 
		representation&leagan\\ \hline 
		representative&ionadaíochta\\ \hline 
		reproducible&in-athdhéanta\\ \hline 
		rule (in logic)&riail (loighce)\\ \hline 
		rule-based&bunaithe ar rialacha\\ \hline 
		to run&cuir ar siúl\\ \hline 
		running&ar siúl\\ \hline 
		to sample&sampláil\\ \hline 
		sample&sampla\\ \hline 
		sampler&samplóir\\ \hline 
		scalar&scálach\\ \hline 
		scatter plot (or diagram)&scaipléaráid\\ \hline 
		to score&scóráil\\ \hline 
		score&scór\\ \hline 
		scoring function&feidhm scórála\\ \hline 
		self-attention&féin-aird\\ \hline 
		self-supervised&faoi fhéin-mhaoirseacht\\ \hline 
		semantics&séimeantaic\\ \hline 
		semi-supervised&faoi leath-mhaoirseacht\\ \hline 
		set&tacar\\ \hline 
		setwise function&feidhm de thacar (pointí)\\ \hline 
		sigmoid function&feidhm shiogmóideach\\ \hline 
		signal&comhartha\\ \hline 
		to simulate&insamhail\\ \hline 
		simulation&insamhladh\\ \hline 
		social network&líonra cairdis\\ \hline 
		softmax&softmax\\ \hline 
		source code&(bun-)chód\\ \hline 
		sparse&éadlúth\\ \hline 
		sparsity&éadlús\\ \hline 
		standard deviation&diall caighdeánach\\ \hline 
		state of the art (best)&scoth an réimse\\ \hline 
		state of the art (current)&staid an réimse\\ \hline 
		state of the art (the literature)&litríocht an réimse\\ \hline 
		structural alignment&ailíniú struchtúir\\ \hline 
		structural alignment framework&creatlach ailínithe struchtúir\\ \hline 
		structural alignment hypothesis&hipitéis ar ailíniú struchtúir\\ \hline 
		structure&struchtúr\\ \hline 
		subgraph&fo-ghraf\\ \hline 
		subject&ainmní\\ \hline 
		subject corruption&malartú an ainmní\\ \hline 
		subject prediction&réamhinsint an ainmní\\ \hline 
		subset&fo-thacar\\ \hline 
		supervised&faoi mhaoirseacht\\ \hline 
		supervision&maoirseacht\\ \hline 
		support vector&veicteoir tacaíochta\\ \hline 
		support vector classifier&aicmitheoir bunaithe ar veicteoirí tacaíochta\\ \hline 
		support vector machine (SVM)&samhail (bunaithe ar) veicteoirí tacaíochta (SVT)\\ \hline 
		support vector regressor&cúlaitheoir bunaithe ar veicteoirí tacaíochta\\ \hline 
		symbol&siombail\\ \hline 
		symbolic&siombalach\\ \hline 
		symbolic logic&loighic shiombalach\\ \hline 
		symmetric&siméadrach\\ \hline 
		symmetry&siméadracht\\ \hline 
		to test&teisteáil\\ \hline 
		testing&teisteáil\\ \hline 
		testing set&tacar teisteála\\ \hline 
		topology&toipeolaíocht\\ \hline 
		to train&traenáil\\ \hline 
		training&traenáil\\ \hline 
		training loop&timthriall traenála\\ \hline 
		training set&tacar traenála\\ \hline 
		transfer learning&tras-fhoghlaim\\ \hline 
		transform&trasfhoirm\\ \hline 
		to transform&trasfhoirmigh\\ \hline 
		transformation&trasfhoirmiú\\ \hline 
		transformer&trasfhoirmeoir\\ \hline 
		transitive&aistreach\\ \hline 
		tree&crann\\ \hline 
		triple&abairt thriarach\\ \hline 
		underfitting&foghlaim easnamhach\\ \hline 
		union&aontas\\ \hline 
		universe&domhan\\ \hline 
		unseen&gan feiceáil\\ \hline 
		unsupervised&gan mhaoirseacht\\ \hline 
		to validate&deimhnigh\\ \hline 
		validation&deimhniú\\ \hline 
		validation set&tacar deimhnithe\\ \hline 
		variance&athraitheas\\ \hline 
		vector&veicteoir\\ \hline 
		walk&siúlóid\\ \hline 
		to weight&ualaigh\\ \hline 
		weight&ualach\\ \hline 
		weighted&ualaithe\\ \hline 
		weighted random search&cuardach randamach ualaithe\\ \hline 
		world&domhan\\ \hline 
		z-score&scór-z\\ \hline 
\caption{Liosta na dtéarmaí sa bhFoclóir Tráchtas.}
\label{tab-terms}
\end{longtable}

\newpage \section*{Téarmaí}
\phantomsection \subsection*{\#}
\addcontentsline{toc}{subsection}{\#}
\markboth{\#}{\#}

\subsubsection*{0-shot (aidiacht): 0-sonra}
 \noindent \textit{Sainmhíniú (ga):} Cur chuige mion-fheabsaithe ina dteisteáiltear samhail réamh-thraenáilte ar thacar sonraí nua gan mion-fheabsú ar bith.
\\
 \noindent \textit{Sainmhíniú (en):} A finetuning protocol in which a pretrained model is directly tested on a new dataset without any finetuning.
\\
 \noindent \textit{Tagairtí:}
\begin{itemize}
	\item sonra: féach ar an téarma `database / bunachar sonraí'
\end{itemize}

 \noindent \textit{Nótaí Aistriúcháin:}
\begin{itemize}
	\item Féach ar an téarma `n-shot / n-sonra'.
\end{itemize}


\phantomsection \subsection*{A}
\addcontentsline{toc}{subsection}{A}
\markboth{A}{A}

\subsubsection*{ablation experiment (ainmfhocal): turgnamh ionadaithe}
 \noindent \textit{Sainmhíniú (ga):} Turgnamh a dhéantar ar shamhail ríomhfhoghlama ina mbíonn gach aon mhodúl sa tsamhail bainte nó ionadaithe le modúl eile. Déantar é seo chun fáil amach cén tionchar atá ag gach uile mhodúl ar chumas foghlama na samhla.
\\
 \noindent \textit{Sainmhíniú (en):} An experiment performed on machine learning models in which each module in the model is removed or replaced with another module. This is done to determine the exact effects of each module on the model's performance.
\\
 \noindent \textit{Tagairtí:}
\begin{itemize}
	\item turgnamh: De Bhaldraithe (1978) \cite{de-bhaldraithe}, Ó Dónaill et al. (1991) \cite{focloir-beag}, Ó Dónaill (1977) \cite{odonaill}
	\item ionadaigh: De Bhaldraithe (1978) \cite{de-bhaldraithe}, Ó Dónaill et al. (1991) \cite{focloir-beag}, Ó Dónaill (1977) \cite{odonaill}
\end{itemize}

 \noindent \textit{Nótaí Aistriúcháin:}
\begin{itemize}
	\item Tá `eisídiú' ar Téarma.ie. Ach i bhFoclóir Uí Dhónaill, tá `eisídiú' luaite mar théarma geolaíochta amháin. Thairis sin, is brí iomlán ar leith atá ag `ablation' i gcomhthéacs na geolaíochta i gcomparáid le comhthéacs na ríomheolaíochta. Ní ghlactar le `eisídiú' mar sin.
	\item Toisc nach bhfuil téarma ar fail dó seo ó na foclóirí dúchasacha, cruthaíodh téarma as nua chun an bhrí seo a chur in iúl. Is é `ionadú' seachas `ionadaíocht' an fréamh a úsáideadh toisc an turgnamh a bheith ag baint le próiseas an ionadaithe -- seachas le hionadaíocht mar choincheap.
	\item Tá `turgnamh' luaite i bhFoclóir Uí Dhónaill mar théarma eolaíochta, agus sna foclóirí eile i gcomhthéacs níos ginearálta.
	\item Is é `ionadú' seachas `ionadaigh' atá i bhFoclóir Uí Dhónaill agus Uí Mhaoileoin.
	\item Tá am téarma `ionadaigh' i bhFoclóir Uí Dhuinín, ach ní luann sé `replace / substitute' mar bhrí leis.
\end{itemize}


\subsubsection*{to abstract (ainmfhocal): teibigh}
 \noindent \textit{Sainmhíniú (ga):} Ginearálú a dhéanamh ar bhun-choincheap chun teacht ar choincheap nua ar leibhéal níos airde. Nó, i gcomhthéacs ríomheolaíochta, comhéadan simplí a chur timpeall ar phróiseas nó ar réad casta chun rochtain air a éascú.
\\
 \noindent \textit{Sainmhíniú (en):} To generalise a base concept to arrive at a higher-order concept. Or, in the context of computer science, to put a simple interface around a complex process or object to facilitate access to it.
\\
 \noindent \textit{Tagairtí:}
\begin{itemize}
	\item teibigh: De Bhaldraithe (1978) \cite{de-bhaldraithe}, Ó Dónaill (1977) \cite{odonaill}
\end{itemize}

 \noindent \textit{Nótaí Aistriúcháin:}
\begin{itemize}
	\item Téarma díreach ar fáil leis an mbrí chéanna (ach i gcomhthéacs fealsúnachta seachas ríomheolaíochta).
	\item Féach chomh maith ar an téarma `abstraction / teibiú'.
\end{itemize}


\subsubsection*{abstraction (ainmfhocal): teibiú}
 \noindent \textit{Sainmhíniú (ga):} Ginearálú ar bhun-choincheap chun teacht ar choincheap nua ar leibhéal níos airde. Nó, i gcomhthéacs ríomheolaíochta, comhéadan simplí a chuirtear timpeall ar phróiseas nó ar réad casta chun rochtain air a éascú.
\\
 \noindent \textit{Sainmhíniú (en):} Generalisation of a base concept to arrive at a higher-order concept. Or, in the context of computer science, a simple interface put around a complex process or object to facilitate access to it.
\\
 \noindent \textit{Tagairtí:}
\begin{itemize}
	\item teibiú: De Bhaldraithe (1978) \cite{de-bhaldraithe}, Ó Dónaill (1977) \cite{odonaill}
\end{itemize}

 \noindent \textit{Nótaí Aistriúcháin:}
\begin{itemize}
	\item Téarma díreach ar fáil leis an mbrí chéanna (ach i gcomhthéacs fealsúnachta seachas ríomheolaíochta).
	\item Féach chomh maith ar an téarma `to abstract / teibigh'.
\end{itemize}


\subsubsection*{activation function (ainmfhocal): feidhm ghníomhachtaithe}
 \noindent \textit{Sainmhíniú (ga):} Feidhm neamh-líneach, m.sh.  \noindent \textit{softmax} nó  \noindent \textit{ReLU}, a úsáidtear idir chisil i líonra néarach.
\\
 \noindent \textit{Sainmhíniú (en):} A non-linear function, such as softmax or ReLU, that is used between layers in a neural network.
\\
 \noindent \textit{Tagairtí:}
\begin{itemize}
	\item feidhm: féach ar an téarma `function / feidhm'
	\item ghníomhachtaigh: De Bhaldraithe (1978) \cite{de-bhaldraithe}, Ó Dónaill (1977) \cite{odonaill}
\end{itemize}

 \noindent \textit{Nótaí Aistriúcháin:}
\begin{itemize}
	\item Úsáidtear `gníomhachtaithe' le cúis mhaith -- is éard a dhéanann feidhm ghníomhachtaithe ná ligean do líonra néarach níos mó ná ciseal amháin a bheith aige. Gan í, ní bheifí in ann ach ciseal amháin a úsáid i gcaoi fhiúntach ar leibhéal matamaiticiúil. Sin é rá, bíonn cisil eile gníomhachtaithe ag an bhfeidhm ghníomhachtaithe.
	\item Is é `gníomhachtaigh' atá ar Téarma.ie chomh maith.
	\item Féach ar an téarma `function / feidhm'.
\end{itemize}


\subsubsection*{adjacency matrix (ainmfhocal): maitrís chóngarachta}
 \noindent \textit{Sainmhíniú (ga):} I gcomhthéacs graif, Maitrís ina bhfuil gach uile nód mar lipéad ar cholúin agus ar shraitheanna, agus ina léiríonn luachanna na maitríse cén ceangal atá idir gach péire nód (má tá cheangal ar bith ann).
\\
 \noindent \textit{Sainmhíniú (en):} In the context of graphs, a matrix in which every node is a column and row label, and in which the matrix values describe what sort of connection (if any) exists between each pair of nodes. 
\\
 \noindent \textit{Tagairtí:}
\begin{itemize}
	\item maitrís: féach ar an téarma `matrix / maitrís'
	\item cóngaracht: Williams et al. (2023) \cite{storchiste}
\end{itemize}

 \noindent \textit{Nótaí Aistriúcháin:}
\begin{itemize}
	\item Luann Foclóir Uí Dhónaill agus Foclóir De Bhaldraithe `cóngarach' mar théarma geoiméadrachta. Cé nach ionann sin agus comhthéacs na ngraf eolais, meastar go mba léire coinneáil leis seachas téarma eile (amhail `teagmháil', nach bhfuil luaite i gcomhthéacs matamaiticiúil) a úsáid. Is é `maitrís chóngarachta' atá ar Téarma.ie chomh maith leis sin.
	\item Féach chomh maith ar an téarma `matrix / maitrís'.
\end{itemize}


\subsubsection*{aggregate (ainmfhocal): tionól}
 \noindent \textit{Sainmhíniú (ga):} I gcomhthéacs sonraí, staitistic amháin atá mar fheidhm de roinnt pointí sonraí ar leith.
\\
 \noindent \textit{Sainmhíniú (en):} In the context of data, a single statistic that is a function of multiple data points.
\\
 \noindent \textit{Tagairtí:}
\begin{itemize}
	\item tionól: De Bhaldraithe (1978) \cite{de-bhaldraithe}, Dineen (1934) \cite{dineen}, Ó Dónaill et al. (1991) \cite{focloir-beag}, Ó Dónaill (1977) \cite{odonaill}
\end{itemize}

 \noindent \textit{Nótaí Aistriúcháin:}
\begin{itemize}
	\item Téarma díreach ar fáil i gcomhthéacs matamaitice i bhFoclóir Uí Dhuinín agus i bhFoclóir De Bhaldraithe. I bhfoclóir Uí Dhuinín agus i bhFoclóir Uí Dhónaill agus Uí Mhaoileoin, ní luaitear an téarma seo mar théarma matamaitice, ach luaitear é le brí chomhchosúil.
	\item Tá an téarma `comhiomlán' ar Téarma.ie, gan trácht ar an bhfocal `tionól'. Ní ghlactar leis sin toisc nach n-aontaíonn sé lena bhfuil ar fáil sna foclóirí dúchasacha. 
\end{itemize}


\subsubsection*{algorithm (ainmfhocal): algartam}
 \noindent \textit{Sainmhíniú (ga):} I gcomhthéacs ríomheolaíochta, próiseas sainmhínithe a bhfuil déanamh taisc mar aidhm aige.
\\
 \noindent \textit{Sainmhíniú (en):} In the context of computer science, a fully-defined process that aims to solve some task.
\\
 \noindent \textit{Tagairtí:}
\begin{itemize}
	\item algartam: Williams et al. (2023) \cite{storchiste}
\end{itemize}

 \noindent \textit{Nótaí Aistriúcháin:}
\begin{itemize}
	\item Téarma díreach ar fáil i gcomhthéacs matamaiticiúil i Stórchiste.
\end{itemize}


\subsubsection*{alignment (ainmfhocal): ailíniú}
 \noindent \textit{Sainmhíniú (ga):} I gcomhthéacs matamaitice, cé chomh maith is a luíonn dhá thomhas / dhá cháilíocht lena chéile ar leibhéal ginearálta.
\\
 \noindent \textit{Sainmhíniú (en):} In a mathematical context, how well two measures / quantities relate to each other in general terms.
\\
 \noindent \textit{Tagairtí:}
\begin{itemize}
	\item ailíniú: De Bhaldraithe (1978) \cite{de-bhaldraithe}, Ó Dónaill (1977) \cite{odonaill}
\end{itemize}

 \noindent \textit{Nótaí Aistriúcháin:}
\begin{itemize}
	\item Téarma ar fáil i gcomhthéacs comhchosúil sna foclóirí thuas.
\end{itemize}


\subsubsection*{antecedent (ainmfhocal): réamhthéarma}
 \noindent \textit{Sainmhíniú (ga):} I gcomhthéacs rialach loighce A $\rightarrow$ B, an téarma A atá mar thús ag an riail.
\\
 \noindent \textit{Sainmhíniú (en):} In the context of a logical rule A $\rightarrow$ B, the term A that is the beginning of the rule.
\\
 \noindent \textit{Tagairtí:}
\begin{itemize}
	\item réamhthéarma: De Bhaldraithe (1978) \cite{de-bhaldraithe}, Ó Dónaill (1977) \cite{odonaill}
\end{itemize}

 \noindent \textit{Nótaí Aistriúcháin:}
\begin{itemize}
	\item Téarma díreach ar fáil mar théarma loighce agus matamaitice ó Fhoclóir De Bhaldraithe agus ó Fhoclóir Uí Dhónaill.
\end{itemize}


\subsubsection*{application (in practice) (ainmfhocal): úsáid (phraiticiúil)}
 \noindent \textit{Sainmhíniú (ga):} Úsáid ruda sa bhfíorshaol, i gcontrárthacht le húsáid theoiriciúil gan chur-i-bhfeidhm.
\\
 \noindent \textit{Sainmhíniú (en):} The real-world use of something, as contrasted with theoretical use that is not put into practice.
\\
 \noindent \textit{Tagairtí:}
\begin{itemize}
	\item úsáid: De Bhaldraithe (1978) \cite{de-bhaldraithe}, Dineen (1934) \cite{dineen}, Ó Dónaill et al. (1991) \cite{focloir-beag}, Ó Dónaill (1977) \cite{odonaill}
	\item praiticiúil: De Bhaldraithe (1978) \cite{de-bhaldraithe}, Ó Dónaill et al. (1991) \cite{focloir-beag}, Ó Dónaill (1977) \cite{odonaill}
\end{itemize}

 \noindent \textit{Nótaí Aistriúcháin:}
\begin{itemize}
	\item Téarmaí díreach ar fáil le brí chomhchosúil.
	\item Ní raibh aon ghá le téarma nua (.i. focal amháin, seachas téarma bunaithe ar `úsáid') a chumadh toisc gur léire agus gur dírí `úsáid phraiticiúil' ná téarma cumtha as nua.
\end{itemize}


\subsubsection*{application domain (ainmfhocal): réimse úsáide}
 \noindent \textit{Sainmhíniú (ga):} An réimse taighde, tionscail, nó eile ina bhfuil córas úsáidte.
\\
 \noindent \textit{Sainmhíniú (en):} The research, industry, or other domain in which a system is used.
\\
 \noindent \textit{Tagairtí:}
\begin{itemize}
	\item réimse: De Bhaldraithe (1978) \cite{de-bhaldraithe}, Ó Dónaill et al. (1991) \cite{focloir-beag}, Ó Dónaill (1977) \cite{odonaill}
	\item úsáid: féach ar an téarma `application (in practice) / úsáid (phraiticiúil)'
\end{itemize}

 \noindent \textit{Nótaí Aistriúcháin:}
\begin{itemize}
	\item Féach ar an téarma `application (in practice) / úsáid (phraiticiúil)'.
\end{itemize}


\subsubsection*{to approximate (ainmfhocal): meastachán a dhéanamh (ar)}
 \noindent \textit{Sainmhíniú (ga):} Luach a mheas.
\\
 \noindent \textit{Sainmhíniú (en):} To estimate a value.
\\
 \noindent \textit{Tagairtí:}
\begin{itemize}
	\item meastachán: féach ar an téarma `to estimate (about) / meastachán a dhéanamh (ar)'
\end{itemize}

 \noindent \textit{Nótaí Aistriúcháin:}
\begin{itemize}
	\item Féach ar an téarma `to estimate (about) / meastachán a dhéanamh (ar)'.
\end{itemize}


\subsubsection*{approximation (ainmfhocal): meastachán}
 \noindent \textit{Sainmhíniú (ga):} I gcomhthéacs matamaitice, luach a dhéanann cur síos cainníochtúil ar fheiniméan éigin, gan a bheith iomlán cruinn.
\\
 \noindent \textit{Sainmhíniú (en):} In a mathematical context, a value that gives a quantitative description of some phenomenon, but which may not be entirely exact.
\\
 \noindent \textit{Tagairtí:}
\begin{itemize}
	\item meastachán: féach ar an téarma `estimate / meastachán'
\end{itemize}

 \noindent \textit{Nótaí Aistriúcháin:}
\begin{itemize}
	\item Féach ar an téarma `estimate / meastachán'.
\end{itemize}


\subsubsection*{arbitrary (aidiacht): treallach}
 \noindent \textit{Sainmhíniú (ga):} I gcomhthéacs luacha, roghnaithe gan chúis léirithe, go háirithe más féidir an luach a ionadú le rogha threallach ar bith eile gan an ráiteas / cothromóid / feidhm / srl ina bhfuil sé a athrú go mór. Is féidir le rogha threallach a bheith randamach nó a bheith déanta gan a bheith randamach.
\\
 \noindent \textit{Sainmhíniú (en):} In the context of a value, chosen without a given justification, especially if the value can be replaced with any other arbitrary value without significantly changing the statement / equation / function / etc it is in. An arbitrary choice can be random or non-random.
\\
 \noindent \textit{Tagairtí:}
\begin{itemize}
	\item treallach: De Bhaldraithe (1978) \cite{de-bhaldraithe}, Ó Dónaill et al. (1991) \cite{focloir-beag}, Ó Dónaill (1977) \cite{odonaill}
\end{itemize}

 \noindent \textit{Nótaí Aistriúcháin:}
\begin{itemize}
	\item Téarma díreach ar fáil le brí chomhchosúil ó na foclóirí thuas.
\end{itemize}


\subsubsection*{architecture (ainmfhocal): dearadh}
 \noindent \textit{Sainmhíniú (ga):} I gcomhthéacs ríomhfhoghlama, an struchtúr matamaiticiúil atá ar líonra néarach.
\\
 \noindent \textit{Sainmhíniú (en):} In the context of machine learning, the mathematical structure of a neural network.
\\
 \noindent \textit{Tagairtí:}
\begin{itemize}
	\item dearadh: De Bhaldraithe (1978) \cite{de-bhaldraithe}, Dineen (1934) \cite{dineen}, Ó Dónaill et al. (1991) \cite{focloir-beag}, Ó Dónaill (1977) \cite{odonaill}
\end{itemize}

 \noindent \textit{Nótaí Aistriúcháin:}
\begin{itemize}
	\item Úsáidtear `dearadh' seachas `ailtireacht' toisc gurb é an rud is tábhachtaí ná cé chaoi a rinneadh an líonra néarach a dhearadh; sin le ré, rogha an dearthóra.
	\item Níl cúis ar bith ann, ar thaobh na ríomheolaíochta de, idirdhealú a dhéanamh idir `design' agus `architecture' (agus úsáidtear mar an gcéanna an dá fhocal). Tá `dearadh' níos léire i nGaeilge toisc go mbíonn ailtireacht ag trácht, den chuid is mó, ar ailtireacht foirgneamh (mar a fheictear i bhFoclóir Uí Dhónaill). Roghnaítear `dearadh' seachas `ailtireacht' mar sin.
\end{itemize}


\subsubsection*{artificial intelligence (AI) (concept) (ainmfhocal): intleacht shaorga (IS) (coincheap)}
 \noindent \textit{Sainmhíniú (ga):} An coincheap taobh thiar de chórais intleachta saorga; nó, an intleacht atá acu mar choincheap teibí.
\\
 \noindent \textit{Sainmhíniú (en):} The concept behind artificial intelligence systems; or, the intelligence they have as an abstract concept.
\\
 \noindent \textit{Tagairtí:}
\begin{itemize}
	\item intleacht: féach ar an téarma `artificial intelligence (AI) (system) / córas intleachta saorga'
	\item saorga: féach ar an téarma `artificial intelligence (AI) (system) / córas intleachta saorga'
\end{itemize}

 \noindent \textit{Nótaí Aistriúcháin:}
\begin{itemize}
	\item Ní féidir an téarma seo a úsáid chun trácht a dhéanamh ar chóras intleachta shaorga. Ní bhaineann an focal `intleacht' ach le coincheap teibí na hintleachta (féach ar Foclóir Uí Dhónaill mar fhianaise air sin).
	\item Féach chomh maith ar an téarma `artificial intelligence (AI) (system) / córas intleachta saorga'.
\end{itemize}


\subsubsection*{artificial intelligence (AI) (system) (ainmfhocal): córas intleachta saorga (córas IS)}
 \noindent \textit{Sainmhíniú (ga):} Córas ríomhaireachta a bhfuil mar aidhm aige gníomhú ar nós go bhfuil intleacht nó cumas smaointe aige. Ní bhíonn intleacht shaorga bunaithe ar chórais ríomhfhoghlama i gcónaí, ach is sórt intleachta saorga iad chuile chóras ríomhfhoghlama.
\\
 \noindent \textit{Sainmhíniú (en):} A computational system that aims to act as though it has intelligence or the ability to think. Artificial intelligence is not always based on machine learning, but all machine learning systems are forms of artificial intelligence.
\\
 \noindent \textit{Tagairtí:}
\begin{itemize}
	\item córas: De Bhaldraithe (1978) \cite{de-bhaldraithe}, Ó Dónaill et al. (1991) \cite{focloir-beag}, Ó Dónaill (1977) \cite{odonaill}
	\item intleacht: De Bhaldraithe (1978) \cite{de-bhaldraithe}, Dineen (1934) \cite{dineen}, Ó Dónaill et al. (1991) \cite{focloir-beag}, Ó Dónaill (1977) \cite{odonaill}
	\item saorga: De Bhaldraithe (1978) \cite{de-bhaldraithe}, Ó Dónaill et al. (1991) \cite{focloir-beag}, Ó Dónaill (1977) \cite{odonaill}
\end{itemize}

 \noindent \textit{Nótaí Aistriúcháin:}
\begin{itemize}
	\item Tagann an téarma seo as an aidhm atá le hintleacht shaorga .i. gníomhú ar nós go bhfuil intleacht ann ('intleacht') gan í a bheith ann dáiríre ('saorga').
\end{itemize}


\subsubsection*{assumption (ainmfhocal): foshuíomh}
 \noindent \textit{Sainmhíniú (ga):} Ráiteas a chuirtear i gcás gur fíor é.
\\
 \noindent \textit{Sainmhíniú (en):} A statement that is taken to be true.
\\
 \noindent \textit{Tagairtí:}
\begin{itemize}
	\item foshuíomh: De Bhaldraithe (1978) \cite{de-bhaldraithe}, Ó Dónaill (1977) \cite{odonaill}
\end{itemize}

 \noindent \textit{Nótaí Aistriúcháin:}
\begin{itemize}
	\item De réir Fhoclóirí Uí Dhónaill agus De Bhaldraithe, is téarma fealsúnachta é seo. Cloíonn sé sin leis an gcomhthéacs eolaíochta atá i gceist leis an tráchtas seo.
	\item Is féidir `cur i gcás' nó `beirtear leis' a úsáid chomh maith i bhfrása chun brí chomhchosúil leis seo a chur in iúl.
\end{itemize}


\subsubsection*{atomic (aidiacht): adamhach}
 \noindent \textit{Sainmhíniú (ga):} Gan a bheith in ann a bheith briste síos a thuilleadh chun teacht ar aonad níos bunúsaí.
\\
 \noindent \textit{Sainmhíniú (en):} Unable to be broken down further to arrive at a more basic unit.
\\
 \noindent \textit{Tagairtí:}
\begin{itemize}
	\item adamhach: De Bhaldraithe (1978) \cite{de-bhaldraithe}, Ó Dónaill et al. (1991) \cite{focloir-beag}, Ó Dónaill (1977) \cite{odonaill}
\end{itemize}

 \noindent \textit{Nótaí Aistriúcháin:}
\begin{itemize}
	\item Tagann an téarma seo as an bhfocal `atom / adamh' sa gceimic. Sin ráite, meastar go bhfuil brí an fhocail `adamhach' sa gceimic cosúil go leor leis an mbrí atá i gceist anseo.
\end{itemize}


\subsubsection*{to attend to (ainmfhocal): aird a thabhairt ar}
 \noindent \textit{Sainmhíniú (ga):} I gcomhthéacs oibrithe airde i líonra néarach, díriú ar chodanna ar leith den ionchur de réir an oibrithe airde.
\\
 \noindent \textit{Sainmhíniú (en):} In the context of an attention mechanism in a neural network, to focus on some parts of the input based on the attention mechanism.
\\
 \noindent \textit{Tagairtí:}
\begin{itemize}
	\item aird: féach ar an téarma `attention / aird'
	\item aird a thabhairt ar: De Bhaldraithe (1978) \cite{de-bhaldraithe}, Ó Dónaill et al. (1991) \cite{focloir-beag}, Ó Dónaill (1977) \cite{odonaill}
\end{itemize}

 \noindent \textit{Nótaí Aistriúcháin:}
\begin{itemize}
	\item Féach ar an téarma `attention / aird'.
\end{itemize}


\subsubsection*{attention (ainmfhocal): aird}
 \noindent \textit{Sainmhíniú (ga):} I gcomhthéacs ríomhfhoghlama, córas a dhéanann meastachán ar thábhacht gach uile airí i veicteoir chun ligean do shamhail ríomhfhoghlama díriú ar an hairíonna is tábhachtaí ann.
\\
 \noindent \textit{Sainmhíniú (en):} In the context of machine learning, a mechanism that estimates the importance of every feature in a vector, so as to allow a machine learning model to focus on the most important features.
\\
 \noindent \textit{Tagairtí:}
\begin{itemize}
	\item aird: De Bhaldraithe (1978) \cite{de-bhaldraithe}, Dineen (1934) \cite{dineen}, Ó Dónaill et al. (1991) \cite{focloir-beag}, Ó Dónaill (1977) \cite{odonaill}
\end{itemize}

 \noindent \textit{Nótaí Aistriúcháin:}
\begin{itemize}
	\item Is é `áird' seachas `aird' atá i bhFoclóir Uí Dhuinín, ach glactar leis gurb in an focal céanna.
	\item Ní luaitear an focal seo i gcomhthéacs ríomhfhoghlama -- rud a bhfuil ciall leis, toisc gur coincheap nua go leor é. Sin ráite, cruthaíodh `attention' i gcomhthéacs ríomhfhoghlama chun a bheith cosúil leis an gcaoi ar féidir le daoine aird a thabhairt ar rudaí mórthimpeall orthu. Toisc gurb in an bhrí dhíreach a ghabhann le `aird' i nGaeilge, meastar gurb é `aird' an focal ceart chun `attention' a chur in iúl.
\end{itemize}


\subsubsection*{attention head (ainmfhocal): ceann airde}
 \noindent \textit{Sainmhíniú (ga):} I gcomhthéacs trasfhoirmeora (nó modúl airde eile), cuid den mhodúl inar féidir sonraí ionchuir a chur isteach.
\\
 \noindent \textit{Sainmhíniú (en):} In the context of a transformer (or other attention block), a part of the block in which data can be input.
\\
 \noindent \textit{Tagairtí:}
\begin{itemize}
	\item ceann: De Bhaldraithe (1978) \cite{de-bhaldraithe}, Dineen (1934) \cite{dineen}, Ó Dónaill et al. (1991) \cite{focloir-beag}, Ó Dónaill (1977) \cite{odonaill}
	\item aird: féach ar an téarma `attention / aird'
\end{itemize}

 \noindent \textit{Nótaí Aistriúcháin:}
\begin{itemize}
	\item Roghnaíodh `ceann' ní toisc gur leagan Gaeilge den fhocal `head' é, ach toisc go bhfuil an bhrí `end, extremity' luaite leis chomh maith (féach ar Fhoclóir Uí Dhónaill). Sin go díreach a bhfuil i gceist le `attention head' -- is é an `extremity' den mhodúl airde ina gcuirtear sonraí ann mar ionchur. Meastar, mar sin, go bhfuil úsáid an fhocail `ceann' cuí chun `(attention) head' a chur in iúl.
	\item Féach chomh maith ar an téarma `attention / aird'.
\end{itemize}


\subsubsection*{attention layer (ainmfhocal): ciseal airde}
 \noindent \textit{Sainmhíniú (ga):} I gcomhthéacs ríomhfhoghlama, ciseal i líonra néarach a chuireann oibriú airde i gcrích.
\\
 \noindent \textit{Sainmhíniú (en):} In the context of machine learning, a layer in a neural network that implements attention.
\\
 \noindent \textit{Tagairtí:}
\begin{itemize}
	\item ciseal: féach ar an téarma `layer / ciseal'
	\item aird: féach ar an téarma `attention / aird'
\end{itemize}

 \noindent \textit{Nótaí Aistriúcháin:}
\begin{itemize}
	\item Féach ar an téarma `layer / ciseal'.
	\item Féach chomh maith ar an téarma `attention / aird'.
\end{itemize}


\subsubsection*{attention mechanism (ainmfhocal): oibriú airde}
 \noindent \textit{Sainmhíniú (ga):} I gcomhthéacs ríomhfhoghlama, na hoibrithe a úsáidtear chun ciseal airde a chur i bhfeidhm.
\\
 \noindent \textit{Sainmhíniú (en):} In the context of machine learning, the operations used to implement an attention layer.
\\
 \noindent \textit{Tagairtí:}
\begin{itemize}
	\item oibriú: féach ar an téarma `operation / oibriú'
	\item aird: féach ar an téarma `attention / aird'
\end{itemize}

 \noindent \textit{Nótaí Aistriúcháin:}
\begin{itemize}
	\item Ní úsáidtear an focal `meicníocht' (nó mar sin) anseo toisc é a bheith ró-litriúil. Ní bhaineann an téarma seo le meicníocht fhisiceach in aon chor. Ach ní mheastar gur ceart `córas' a úsáid ach an oiread -- ní bhíonn `attention mechanism' ina `attention system'. Toisc gur sórt oibrithe é `attention mechanism', áfach, meastar go bhfuil ciall leis an bhfocal `oibriú' a úsáid anseo.
	\item Féach chomh maith ar an téarma `operation / oibriú'.
	\item Féach chomh maith ar an téarma `attention / aird'.
	\item Féach chomh maith ar an téarma `attention layer / ciseal airde'.
\end{itemize}


\phantomsection \subsection*{B}
\addcontentsline{toc}{subsection}{B}
\markboth{B}{B}

\subsubsection*{baseline model (ainmfhocal): samhail bhunlíne}
 \noindent \textit{Sainmhíniú (ga):} I gcomhthéacs ríomhfhoghlama, samhail (chaighdeánach) a úsáidtear chun comparáid a dhéanamh le samhlacha (nua) eile agus iad á meas.
\\
 \noindent \textit{Sainmhíniú (en):} In the context of machine learning, a (standard) model that is used to compare to other (new) models to evaluate them.
\\
 \noindent \textit{Tagairtí:}
\begin{itemize}
	\item samhail: féach ar an téarma `model / samhail'
	\item bunlíne: De Bhaldraithe (1978) \cite{de-bhaldraithe}, Ó Dónaill (1977) \cite{odonaill}
\end{itemize}

 \noindent \textit{Nótaí Aistriúcháin:}
\begin{itemize}
	\item Cé go bhfuil an téarma `baseline' $\rightarrow$ `bunlíne' ar fáil i bhFoclóir De Bhaldraithe agus i bhFoclóir Uí Dhónaill, ní luann siad comhthéacs áirithe ar bith leis. Sin ráite, ní bheadh téarma eile (m.sh. bun-samhail) cuí don úsáid seo toisc nach ionann bun-samhail agus samhail bhunlíne i gcónaí. Mar shampla, d'fhéadfadh trácht a dhéanamh ar shean-shamhail (m.sh. TransE) ar bhunlíne ar shamhail nua (m.sh. TransH nó TransR) a bhí bunaithe air. Ach ní hionann sin is go mbeadh TransE úsáidte mar shamhail bhunlíne sa taighde sin -- cé gur dócha sin, níl sé sin cinnte. Is dá bharr sin gur léire `samhail bhunlíne' mar théarma. Glactar le úsáid `bunlíne' seachas `bun-' mar sin.
	\item Sin ráite, níl locht ar bith ar úsáid `bun-samhail' chun `baseline model' a chur in iúl má tá comhthéacs léir leis.
\end{itemize}


\subsubsection*{batch (ainmfhocal): baisc}
 \noindent \textit{Sainmhíniú (ga):} Tacar sonraí tógtha as an tacar traenála a thugtar do shamhail ríomhfhoghlama agus í ag foghlaim (.i. le linn an phróisis traenála). Uaireanta, úsáidtear fo-thacar de thacar teisteála / deimhnithe le linn an phróisis teisteála / deimhnithe chomh maith.
\\
 \noindent \textit{Sainmhíniú (en):} A set of data points taken from the training set that is given to a machine learning model as it is learning (i.e. as a part of the training process). Sometimes, batches of the testing and validation sets are used during testing and validation as well.
\\
 \noindent \textit{Tagairtí:}
\begin{itemize}
	\item baisc: Ó Dónaill (1977) \cite{odonaill}
\end{itemize}

 \noindent \textit{Nótaí Aistriúcháin:}
\begin{itemize}
	\item Is é `baisc' atá ar Téarma.ie ina chomhair seo. Tá an téarma sin le feiceáil i bhFoclóir Uí Dhónaill le brí chomhchiallach. Cé go mbeadh ciall le go leor téarmaí eile, glactar leis sin toisc go bhfuil fianise ann dó i bhFoclóir Uí Dhónaill agus toisc go bhfuil sé in úsáid cheana ar Téarma.ie.
	\item Is ionann baisc agus fo-thacar den tacar traenála, teisteála, nó deimhnithe. Mar sin, bheadh sé ceart go leor `fo-thacar den tacar traenála' (srl) a rá go litriúil -- ach moltar `baisc' toisc go bhfuil sé níos léire go ginearálta.
\end{itemize}


\subsubsection*{batch size (ainmfhocal): méid na mbaisceanna}
 \noindent \textit{Sainmhíniú (ga):} Cé mhéid pointí sonraí atá i chuile fho-thacar den tacar traenála / teisteála / deimhnithe.
\\
 \noindent \textit{Sainmhíniú (en):} How many data points are in each batch of the training / testing / validation set.
\\
 \noindent \textit{Tagairtí:}
\begin{itemize}
	\item méid: De Bhaldraithe (1978) \cite{de-bhaldraithe}, Dineen (1934) \cite{dineen}, Ó Dónaill et al. (1991) \cite{focloir-beag}, Ó Dónaill (1977) \cite{odonaill}
	\item fo-thacar: féach ar an téarma `batch / fo-thacar'
\end{itemize}

 \noindent \textit{Nótaí Aistriúcháin:}
\begin{itemize}
	\item Féach ar an téarma `batch / baisc'.
\end{itemize}


\subsubsection*{Bayesian (aidiacht): Bayes}
 \noindent \textit{Sainmhíniú (ga):} I gcomhthéacs staitistice, bunaithe ar Theoirim Bayes.
\\
 \noindent \textit{Sainmhíniú (en):} In the context of statistics, based on Bayes' Theorem.
\\
 \noindent \textit{Tagairtí:}
\begin{itemize}
	\item Bayes: Williams et al. (2023) \cite{storchiste}
\end{itemize}

 \noindent \textit{Nótaí Aistriúcháin:}
\begin{itemize}
	\item Ní luann Stórchiste `Bayes' mar leagan de `Bayesian', ach bíonn sé ann sa téarma `Bayes Theorem / Teoirim Bayes'. Thairis sin, i gcás gach uile théarma eile a bhfuil ainm duine leis (m.sh. `sraith Fourier', `foirmle de Moivre', srl) bíonn sloinne an duine úsáidte gan athrú ar bith air. Is mar sin a moltar `Bayes' a úsáid ar nós aidiachta chun `Bayesian' a chur in iúl.
	\item Is é `Bayesach' atá ar Téarma.ie, ach ní ghlactar leis sin toisc nach bhfaightear fianaise dó sna foclóirí dúchasacha.
\end{itemize}


\subsubsection*{benchmark dataset (ainmfhocal): tacar sonraí comparáide}
 \noindent \textit{Sainmhíniú (ga):} Tacar sonraí a úsáidtear chun samhlacha ríomhfhoghlama a mheasúnú agus a chur i gcomparáid lena chéile.
\\
 \noindent \textit{Sainmhíniú (en):} A dataset that is used to evaluate and compare machine learning models.
\\
 \noindent \textit{Tagairtí:}
\begin{itemize}
	\item tacar sonraí: féach ar an téarma `dataset / tacar sonraí'
	\item comparáid: De Bhaldraithe (1978) \cite{de-bhaldraithe}, Dineen (1934) \cite{dineen}, Ó Dónaill et al. (1991) \cite{focloir-beag}, Ó Dónaill (1977) \cite{odonaill}
\end{itemize}

 \noindent \textit{Nótaí Aistriúcháin:}
\begin{itemize}
	\item Níl an téarma Béarla `benchmark' ar fáil i gceann ar bith de na foclóirí atá in úsáid chun an Foclóir Tráchtais a chur le chéile. Bhí gá le téarma cumtha as nua mar sin. Roghnaíodh `tacar sonraí comparáide' toisc gurb é aidhm `benchmark dataset' ná a bheith úsáidte chun samhlacha ríomhfhoghlama a chur i gcomparáid lena chéile.
	\item Tá `tagarmharc' ar Focloir.ie -- ach ní féidir bunús ar bith a fháil dó sin sna foinse dúchasacha atá in úsáid anseo. Ní ghlactar leis mar sin.
\end{itemize}


\subsubsection*{bin (ainmfhocal): eatramh}
 \noindent \textit{Sainmhíniú (ga):} I gcomhthéacs histeagraim, ceann ar bith de na raonta luachanna atá úsáidte chun luachanna a bhailiú le chéile agus chun a minicíocht a chomhaireamh.
\\
 \noindent \textit{Sainmhíniú (en):} In the context of a histogram, any one of the intervals that is used to group values and calculate their frequency.
\\
 \noindent \textit{Tagairtí:}
\begin{itemize}
	\item eatramh: féach ar an téarma `interval / eatramh'
\end{itemize}

 \noindent \textit{Nótaí Aistriúcháin:}
\begin{itemize}
	\item Bíonn gach uile `bin' ina `interval' -- níl aon ghá idirdhealú a dhéanamh nuair is léir gur i gcomhthéacs histeagraim atáthar ag caint.
	\item Féach chomh maith ar an téarma `interval / eatramh'.
\end{itemize}


\subsubsection*{binary (aidiacht): dénártha}
 \noindent \textit{Sainmhíniú (ga):} I gcomhthéacs ríomheolaíochta, bunaithe ar dhá luach / rogha / aicme / srl amháin (0 agus 1, go hiondúil).
\\
 \noindent \textit{Sainmhíniú (en):} In the context of computer science, based on only two values / choices / classes / etc (typically 0 and 1).
\\
 \noindent \textit{Tagairtí:}
\begin{itemize}
	\item dénártha: De Bhaldraithe (1978) \cite{de-bhaldraithe}, Ó Dónaill (1977) \cite{odonaill}
\end{itemize}

 \noindent \textit{Nótaí Aistriúcháin:}
\begin{itemize}
	\item Téarma díreach le fáil ó na foclóirí thuas.
\end{itemize}


\subsubsection*{binary cross entropy loss (BCEL) (ainmfhocal): pionós tras-eantrópachta dénártha (PTED)}
 \noindent \textit{Sainmhíniú (ga):} Feidhm phionóis bunaithe ar thras-eantrópacht dénártha. Úsáidtear go minic í i gcomhair an taisc aicmithe agus an taisc réamhinsinte nasc.
\\
 \noindent \textit{Sainmhíniú (en):} A loss function based on binary cross entropy. It is typically used in classification and link prediction.
\\
 \noindent \textit{Tagairtí:}
\begin{itemize}
	\item pionós: féach ar an téarma `loss function / feidhm phionóis'
	\item tras-eantrópachta: féach ar an téarma `cross-entropy / tras-eantrópacht'
	\item dénártha: féach ar an téarma `binary / dénártha'
\end{itemize}

 \noindent \textit{Nótaí Aistriúcháin:}
\begin{itemize}
	\item Úsáidtear `pionós' seachas `feidhm phionóis' thuas mar giorrúchán. Tá sé ceart go leor `feidhm phionóis' a úsáid freisin.
	\item Féach chomh maith ar an téarma `loss function / feidhm phionóis'.
	\item Féach chomh maith ar an téarma `cross-entropy / tras-eantrópacht'.
	\item Féach chomh maith ar an téarma `binary / dénártha'.
\end{itemize}


\subsubsection*{block (ainmfhocal): modúl}
 \noindent \textit{Sainmhíniú (ga):} Cuid de shamhail nó de phróiseas a bhfuil feidhm ar leith aici agus ar féidir í a úsáid i samhail nó i bpróiseas eile, gan í athrú, chun an tasc céanna a dhéanamh.
\\
 \noindent \textit{Sainmhíniú (en):} Part of a model or process that has a specific function and that can be used in other models or processes, without changing it, to do the same task.
\\
 \noindent \textit{Tagairtí:}
\begin{itemize}
	\item modúl: féach ar an téarma `module / modúl'.
\end{itemize}

 \noindent \textit{Nótaí Aistriúcháin:}
\begin{itemize}
	\item Tá an téarma seo comhchiallach le `modúl'.
	\item Féach ar an téarma `module / modúl'.
\end{itemize}


\subsubsection*{to bound (briathar): cuimsigh}
 \noindent \textit{Sainmhíniú (ga):} I gcomhthéacs matamaitice, aschur feidhme a choinneáil ar eatramh.
\\
 \noindent \textit{Sainmhíniú (en):} In the context of mathematics, to keep the output of a function on an interval.
\\
 \noindent \textit{Tagairtí:}
\begin{itemize}
	\item cuimsigh: Dineen (1934) \cite{dineen}, Ó Dónaill (1977) \cite{odonaill}, Williams et al. (2023) \cite{storchiste}
\end{itemize}

 \noindent \textit{Nótaí Aistriúcháin:}
\begin{itemize}
	\item Téarma díreach le fáil leis an mbrí cheannann chéanna i gcomhthéacs matamaiticiúil i bhFoclóir Uí Dhónaill agus i Stórchiste. Is i gcomhthéacs comhchosúil, ach níos leithne, a luaitear an téarma seo i bhFoclóir Uí Dhuinín.
	\item Féach chomh maith ar an téarma `interval / eatramh'.
\end{itemize}


\phantomsection \subsection*{C}
\addcontentsline{toc}{subsection}{C}
\markboth{C}{C}

\subsubsection*{class (ainmfhocal): aicme}
 \noindent \textit{Sainmhíniú (ga):} I gcomhthéacs an taisc aicmithe, ceann ar bith de na lipéad ar féidir é a chur le sonraí ionchuir le linn aicmithe. I gcomhthéacs cliarlathais, ointeolaíochta, nó tacsanomaíochta, sórt nó cineál réada sa gcliarlathas / san ointeolaíocht / sa tacsanomaíocht.
\\
 \noindent \textit{Sainmhíniú (en):} In the context of classification, any one of the labels that can be given to input data during classification. In the context of a hierarchy, ontology, or taxonomy, a sort or type of object present in the hierarchy / ontology / taxonomy.
\\
 \noindent \textit{Tagairtí:}
\begin{itemize}
	\item aicme: De Bhaldraithe (1978) \cite{de-bhaldraithe}, Dineen (1934) \cite{dineen}, Ó Dónaill et al. (1991) \cite{focloir-beag}, Ó Dónaill (1977) \cite{odonaill}, Williams et al. (2023) \cite{storchiste}
\end{itemize}

 \noindent \textit{Nótaí Aistriúcháin:}
\begin{itemize}
	\item Téarma díreach ar fáil le brí chomhchosúil ó na foclóirí thuas.
	\item Tá an téarma seo le fáil i gcomhthéacs staitistice i Stórchiste sa téarma `class interval / eatramh aicme'.
\end{itemize}


\subsubsection*{classification (ainmfhocal): aicmiú}
 \noindent \textit{Sainmhíniú (ga):} Tasc ríomhfhoghlama a bhfuil mar aidhm aige lipéad ('aicme') a chur le chuile phointe sonraí ionchuir. Mar shampla, íomhánna a bhfuil madra nó cat iontu a aicmiú de réir an t-ainmhí atá san íomhá.
\\
 \noindent \textit{Sainmhíniú (en):} The machine learning task that aims to assign a label ('class') to every input data point. For example, classifying images of dogs or cats based on the animal in the image.
\\
 \noindent \textit{Tagairtí:}
\begin{itemize}
	\item aicmigh: De Bhaldraithe (1978) \cite{de-bhaldraithe}, Ó Dónaill et al. (1991) \cite{focloir-beag}, Ó Dónaill (1977) \cite{odonaill}
\end{itemize}

 \noindent \textit{Nótaí Aistriúcháin:}
\begin{itemize}
	\item Téarma díreach ar fáil le brí chomhchosúil ó na foclóirí thuas.
\end{itemize}


\subsubsection*{classifier (ainmfhocal): aicmitheoir}
 \noindent \textit{Sainmhíniú (ga):} I gcomhthéacs ríomhfhoghlama, samhail a chuireann aicmiú i gcrích.
\\
 \noindent \textit{Sainmhíniú (en):} In the context of machine learning, a model that performs classification.
\\
 \noindent \textit{Tagairtí:}
\begin{itemize}
	\item aicmigh: féach ar an téarma `to classify / aicmigh'
	\item -eoir: De Bhaldraithe (1978) \cite{de-bhaldraithe}, Dineen (1934) \cite{dineen}, Ó Dónaill et al. (1991) \cite{focloir-beag}, Ó Dónaill (1977) \cite{odonaill}, Williams et al. (2023) \cite{storchiste}
\end{itemize}

 \noindent \textit{Nótaí Aistriúcháin:}
\begin{itemize}
	\item Níl iontráil ar leith ag an iarmhír `-eoir' sna foclóirí thuas, ach luann siad uilig go leor focal a úsáideann í díreach mar a úsáidtear anseo.
	\item Féach chomh maith ar an téarma `to classify / aicmigh'.
\end{itemize}


\subsubsection*{to classify (ainmfhocal): aicmigh}
 \noindent \textit{Sainmhíniú (ga):} I gcomhthéacs ríomhfhoghlama, an tasc aicmithe a chur i gcrích.
\\
 \noindent \textit{Sainmhíniú (en):} In the context of machine learning, to perform the classification task.
\\
 \noindent \textit{Tagairtí:}
\begin{itemize}
	\item aicmigh: féach ar an téarma `classification / aicmiú'
\end{itemize}

 \noindent \textit{Nótaí Aistriúcháin:}
\begin{itemize}
	\item * Is é `aicmiú' seachas `aicmigh' atá i bhFoclóir Uí Dhónaill agus Uí Mhaoileoin.
	\item Féach chomh maith ar an téarma `classification / aicmiú'.
\end{itemize}


\subsubsection*{Closed World Assumption (ainmfhocal): Foshuíomh an Domhain Dhúnta}
 \noindent \textit{Sainmhíniú (ga):} I gcomhthéacs graf eolais, an foshuíomh a deir go bhfuil gach uile fhíric i réimse graif eolais sa ngraf cheana.
\\
 \noindent \textit{Sainmhíniú (en):} In the context of knowledge graphs, the assumption that says that all facts in the domain of the knowledge graph are already in it.
\\
 \noindent \textit{Tagairtí:}
\begin{itemize}
	\item foshuíomh: féach ar an téarma `assumption / foshuíomh'
	\item domhan: féach ar an téarma `world / domhan'
	\item dún: De Bhaldraithe (1978) \cite{de-bhaldraithe}, Dineen (1934) \cite{dineen}, Ó Dónaill et al. (1991) \cite{focloir-beag}, Ó Dónaill (1977) \cite{odonaill}
\end{itemize}

 \noindent \textit{Nótaí Aistriúcháin:}
\begin{itemize}
	\item Is é `dúnadh' seachas `dún' atá i bhFoclóir Uí Dhuinín agus Uí Mhaoileoin.
	\item Féach chomh maith ar an téarma `assumption / foshuíomh'.
	\item Féach chomh maith ar an téarma `Open World Assumption / Foshuíomh an Domhain Oscailte'.
	\item Féach chomh maith ar an téarma `world / domhan'.
\end{itemize}


\subsubsection*{cluster (ainmfhocal): braisle (pointí)}
 \noindent \textit{Sainmhíniú (ga):} I gcomhthéacs matamaitice, grúpa pointí atá gar dá chéile, nó cosúil lena chéile, de réir tomhais éigin.
\\
 \noindent \textit{Sainmhíniú (en):} In the context of mathematics, a group of points that are close to each other, or similar to each other, according to some metric.
\\
 \noindent \textit{Tagairtí:}
\begin{itemize}
	\item braisle: De Bhaldraithe (1978) \cite{de-bhaldraithe}, Ó Dónaill (1977) \cite{odonaill}
	\item pointe: De Bhaldraithe (1978) \cite{de-bhaldraithe}, Dineen (1934) \cite{dineen}, Ó Dónaill et al. (1991) \cite{focloir-beag}, Ó Dónaill (1977) \cite{odonaill}, Williams et al. (2023) \cite{storchiste}
\end{itemize}

 \noindent \textit{Nótaí Aistriúcháin:}
\begin{itemize}
	\item De réir Fhoclóir Uí Dhuinín, is éard is braisle ann ná `mass, lump, nó cluster'. Luaitear é i measc roinnt focal eile (.i. triopall, mogall, crobhaing, clibín, agus cloigín) a bhfuil an bhrí chéanna (nó chomhchosúil) luaite leo. Astu siúd, bíonn `triopall' luaite i comhthéacs plandaí (m.sh. triopall caor), agus gan a bheith luaite i gcomhthéacs níos leithne. Bíonn `mogall' luaite mar théarma teicniúil na luibheolaíocht, agus mar fhocal níos ginearálta le brí eile. Bíonn na focail eile luaite i gcomhthéacsanna níos ginearálta. Sin ráite, bíonn `mass' luaite le `braisle' mar bhrí eile -- brí a luíonn le húsáid `braisle' mar `mass of points' (a deirtear uaireanta chun trácht ar `clusters'). Níl a leithéid de bhrí luaite leis na téarmaí eile. Meastar gurb é `braisle' is fearr mar sin chun `cluster' a chur in iúl.
	\item Ní gá `braisle pointí' a rá, ach is dócha gur léire é mar théarma, toisc nach bhfuil `braisle' luaite mar théarma matamaitice cheana sna foclóirí thuas.
	\item Tá `pointe' le fáil mar théarma matamaitice i bhFoclóir Uí Dhónaill agus i Stórchiste, agus le brí níos leithne sna foclóirí eile.
\end{itemize}


\subsubsection*{to cluster (briathar): aicmiú ionduchtach a dhéanamh}
 \noindent \textit{Sainmhíniú (ga):} I gcomhthéacs ríomhfhoghlama, an tasc aicmithe ionduchtaigh a chur i gcrích.
\\
 \noindent \textit{Sainmhíniú (en):} In the context of machine learning, to perform the clustering task.
\\
 \noindent \textit{Tagairtí:}
\begin{itemize}
	\item aicmigh: féach ar an téarma `classification / aicmiú'
	\item ionduchtach: féach ar an téarma `clustering / aicmiú ionduchtach'
\end{itemize}

 \noindent \textit{Nótaí Aistriúcháin:}
\begin{itemize}
	\item Féach ar an téarma `clustering / aicmiú ionduchtach'.
\end{itemize}


\subsubsection*{clustering (ainmfhocal): aicmiú ionduchtach}
 \noindent \textit{Sainmhíniú (ga):} I gcomhthéacs ríomhfhoghlama, an próiseas a bhaineann le pointí sonraí a aicmiú nuair nach bhfuil eolas ar bith céard iad na haicmí cearta (ná ar an méid aicmí fiú) ar fáil roimh ré, sa gcaoi gur gá iad sin a fhoghlaim go hionduchtach.
\\
 \noindent \textit{Sainmhíniú (en):} In the context of machine learning, the process related to assigning a class to data points when there is no knowledge in advance of what the correct labels are (or even how many there are), such that the learning must be done inductively.
\\
 \noindent \textit{Tagairtí:}
\begin{itemize}
	\item aicmigh: féach ar an téarma `classification / aicmiú'
	\item ionduchtach: De Bhaldraithe (1978) \cite{de-bhaldraithe}, Ó Dónaill (1977) \cite{odonaill}, Williams et al. (2023) \cite{storchiste}
\end{itemize}

 \noindent \textit{Nótaí Aistriúcháin:}
\begin{itemize}
	\item Luann Stórchiste an téarma `ionduchtach' i gcomhthéacs matamaitice. Luann na foclóirí eile é i gcomhthéacs níos ginearálta, ach leis an mbrí chéanna.
	\item Is éard is `clustering' ann mar thasc ríomhfhoghlama ná lipéad aicme a chur le pointí sonraí nach bhfuil lipéad ar bith leo. Tá sé an-cheangailte le aicmiú mar thasc, ach amháin nach bhfuil fios na n-aicmí cearta ar fáil don chóras ríomhfhoghlama roimh ré. Sin le rá, is tasc ionduchtach é. Is as an mbun-choincheap seo a roghnaítear `aicmiú ionduchtach' mar théarma.
	\item Tá `braisliú' le feiceáil ar Focloir.ie -- focal a thagann as an bhfréamh `braisle'. Cé is moite de sin, ní bhíonn an leagan briathair `braisliú' le feiceáil i bhfoclóir dúchasach ar bith. Ní ghlactar leis mar sin sa gcomhthéacs matamaitice seo.
	\item Féach chomh maith ar an téarma `classification / aicmiú'.
\end{itemize}


\subsubsection*{co-frequency (ainmfhocal): cóimhinicíocht}
 \noindent \textit{Sainmhíniú (ga):} I gcomhthéacs graif eolais, cé chomh minic is a bhíonn dhá nód / cheangal (nó níos mó) le chéile sna habairtí triaracha céanna sa ngraf.
\\
 \noindent \textit{Sainmhíniú (en):} In the context of a knowledge graph, how often two (or more) nodes / edges are part of the same triples in the graph.
\\
 \noindent \textit{Tagairtí:}
\begin{itemize}
	\item minicíocht: féach ar an téarma `frequency / minicíocht'
	\item comh-: De Bhaldraithe (1978) \cite{de-bhaldraithe}*, Dineen (1934) \cite{dineen}, Ó Dónaill (1977) \cite{odonaill}
\end{itemize}

 \noindent \textit{Nótaí Aistriúcháin:}
\begin{itemize}
	\item * Níl an réimír Béarla `co-' i bhFoclóir De Bhaldraithe mar théarma ar leith, ach úsáidtear é (agus a leagan Gaeilge `comh-' i roinnt maith focail ann.
	\item Is é `cómh' atá i bhFoclóir Uí Dhuinín, ach glactar leis gurb in an focal céanna.
	\item De réir Fhoclóir Uí Dhónaill, scríobhtar có(i)- seachas comh- nuair atá an réimír seo curtha roimh focal a thosaíonn le `m' nó le `n', mar a tharlaíonn anseo.
	\item Féach chomh maith ar an téarma `frequency / minicíocht'.
\end{itemize}


\subsubsection*{coefficient (ainmfhocal): comhéifeacht}
 \noindent \textit{Sainmhíniú (ga):} Uimhir scálach a úsáidtear chun uimhir nó athróg eile a mhéadú fúithi.
\\
 \noindent \textit{Sainmhíniú (en):} A scalar value multiplied with another number or variable.
\\
 \noindent \textit{Tagairtí:}
\begin{itemize}
	\item comhéifeacht: De Bhaldraithe (1978) \cite{de-bhaldraithe}, Ó Dónaill (1977) \cite{odonaill}, Williams et al. (2023) \cite{storchiste}
\end{itemize}

 \noindent \textit{Nótaí Aistriúcháin:}
\begin{itemize}
	\item Téarma luaite mar théarma matamaitice sna foclóirí thuas.
\end{itemize}


\subsubsection*{computer network (ainmfhocal): líonra ríomhairí}
 \noindent \textit{Sainmhíniú (ga):} Graf nó graf eolais ina gcuireann nóid ríomhairí in iúl, agus ina mbíonn ceangail ann a léiríonn gur féidir le dá ríomhaire nascadh lena chéile.
\\
 \noindent \textit{Sainmhíniú (en):} A graph or knowledge graph in which nodes represent computers, and edges represent the that two computers can connect to each other.
\\
 \noindent \textit{Tagairtí:}
\begin{itemize}
	\item líonra: féach ar an téarma `network / líonra'
	\item ríomhaire: De Bhaldraithe (1978) \cite{de-bhaldraithe}, Dineen (1934) \cite{dineen}*, Ó Dónaill et al. (1991) \cite{focloir-beag}, Ó Dónaill (1977) \cite{odonaill}
\end{itemize}

 \noindent \textit{Nótaí Aistriúcháin:}
\begin{itemize}
	\item * Tá `ríomhaire' i bhFoclóir Uí Dhuinín le brí níos sine (ag trácht ar áireamhán seachas ar ríomhaire an lae inniu).
	\item Tá an dá fhocal (líonra agus ríomhaire) díreach ar fáil ó na foclóirí thuas le brí chomhchosúil.
	\item Féach chomh maith ar an téarma `network / líonra.'
\end{itemize}


\subsubsection*{computer science (ainmfhocal): ríomheolaíocht}
 \noindent \textit{Sainmhíniú (ga):} an réimse staidéar atá dírithe ar algartaim, ar bhogearraí, agus ar ríomhchlárúchán.
\\
 \noindent \textit{Sainmhíniú (en):} the field of study that focuses on algorithms, software, and computer programing.
\\
 \noindent \textit{Tagairtí:}
\begin{itemize}
	\item ríomheolaíocht: Ó Dónaill (1977) \cite{odonaill}
\end{itemize}

 \noindent \textit{Nótaí Aistriúcháin:}
\begin{itemize}
	\item Téarma iomlán ar fáil i bhfoclóir iontaofa. Glactar leis.
\end{itemize}


\subsubsection*{connectivity (ainmfhocal): (frása le `ceangailte')}
 \noindent \textit{Sainmhíniú (ga):} Cé chomh ceangailte is atá cuid de ghraf eolais (.i. nód nó ceangal) le codanna eile den ghraf céanna.
\\
 \noindent \textit{Sainmhíniú (en):} How connected one part of a knowledge graph (i.e. a node or edge) is with other parts of the same graph.
\\
 \noindent \textit{Tagairtí:}
\begin{itemize}
	\item ceangailte: De Bhaldraithe (1978) \cite{de-bhaldraithe}, Dineen (1934) \cite{dineen}, Ó Dónaill et al. (1991) \cite{focloir-beag}, Ó Dónaill (1977) \cite{odonaill}
\end{itemize}

 \noindent \textit{Nótaí Aistriúcháin:}
\begin{itemize}
	\item Níl téarma dó seo ar fáil go díreach ó Fhoclóir Uí Dhónaill, De Bhaldraithe, Uí Dhónaill agus Uí Maoileoin, ná Uí Dhuinín. Cé go bhfuil dlús ann (mar `density' i gcomhthéacs eolaíochta), is iomaí saghsanna dlúis atá ann i ngraf eolais, agus níl `connectivity' ach ar cheann amháin acu sin. Fágtar gan téarma ar leith ina chomhair seo mar sin. Moltar frása leis an téarma `ceangailte' a úsáid.
	\item Samplaí: Cé chomh ceangailte is atá rud, nód atá an-cheangailte le nóid eile, srl.
\end{itemize}


\subsubsection*{consequent (ainmfhocal): iarmhairt}
 \noindent \textit{Sainmhíniú (ga):} I gcomhthéacs rialach loighce A $\rightarrow$ B, an téarma B atá mar chuid dheireanach den riail.
\\
 \noindent \textit{Sainmhíniú (en):} In the context of a logical rule A $\rightarrow$ B, the term B that is the ending part of the rule.
\\
 \noindent \textit{Tagairtí:}
\begin{itemize}
	\item iarmhairt: De Bhaldraithe (1978) \cite{de-bhaldraithe}, Ó Dónaill (1977) \cite{odonaill}
\end{itemize}

 \noindent \textit{Nótaí Aistriúcháin:}
\begin{itemize}
	\item Téarma díreach ar fáil mar théarma loighce ó Fhoclóir De Bhaldraithe agus ó Fhoclóir Uí Dhónaill. Is é `iarmairt' seachas `iarmhairt' atá luaite i bhFoclóir De Bhaldraithe, ach luann Foclóir Uí Dhónaill gur leagan eile den fhocal céanna atá ann. Glactar leis an téarma atá i bhFoclóir Uí Dhónaill toisc gur nua é mar fhoclóir.
	\item Luann Foclóir De Bhaldraithe (agus Foclóir Uí Dhónaill) trí théarma a chuireann `consequent' in iúl: iarthéarma, iarmhairt, agus iarbheart. Cé go nglactar le `réamhthéarma' mar `antecedent', is léir ón dá fhoclóir anseo nár cheart glacadh le `iarthéarma' -- is focal matamaitice amháin é sin dar leo araon, a dhéanann cur síos ar an dara uimhir i gcóimheas (.i. an $b$ in $a:b$). Bíonn idir `iarmhairt'' agus iarbheart' luaite mar théarmaí loighce, ach glactar le `iarmhairt' toisc gurb in atá i bhfoinsí eile (.i. Téarma.ie). Toisc nach bhfuil difríocht ar bith idir `iarmhairt' agus `iarbheart' sna foclóirí dúchasacha, meastar nach bhfuil cúis ar bith gan glacadh leis an téarma a bhfuil an úsáid is mó bainte as.
\end{itemize}


\subsubsection*{constant (aidiacht): buan-}
 \noindent \textit{Sainmhíniú (ga):} Ag trácht ar luach éigin, gan athrú.
\\
 \noindent \textit{Sainmhíniú (en):} Regarding a value, unchanging.
\\
 \noindent \textit{Tagairtí:}
\begin{itemize}
	\item buan-: De Bhaldraithe (1978) \cite{de-bhaldraithe}, Dineen (1934) \cite{dineen}, Ó Dónaill et al. (1991) \cite{focloir-beag}, Ó Dónaill (1977) \cite{odonaill}, Williams et al. (2023) \cite{storchiste}
\end{itemize}

 \noindent \textit{Nótaí Aistriúcháin:}
\begin{itemize}
	\item Tá an réimír `buan-' úsáidte i gcomhthéacs matamaitice i bhFoclóir Uí Dhónaill, i bhFoclóir De Bhaldraithe, agus i Stórchiste sa téarma `buanuimhir'. Sna foclóirí eile, is i gcomhthéacs níos leithe atá sé luaite.
\end{itemize}


\subsubsection*{correct fitting (ainmfhocal): foghlaim cheart}
 \noindent \textit{Sainmhíniú (ga):} I gcomhthéacs ríomhfhoghlama, foghlaim réasúnta iomlán atá in ann patrúin ghinearálta an tacair thraenála a fhoghlaim gan a bheith ag ró-fhoghlaim ná an foghlaim go heasnamhach.
\\
 \noindent \textit{Sainmhíniú (en):} In the context of machine learning, reasonably complete learning that can extract general patterns from the training set without overfitting or underfitting.
\\
 \noindent \textit{Tagairtí:}
\begin{itemize}
	\item foghlaim: féach ar an téarma `machine learning / ríomhfhoghlaim'
	\item ceart: De Bhaldraithe (1978) \cite{de-bhaldraithe}, Dineen (1934) \cite{dineen}, Ó Dónaill et al. (1991) \cite{focloir-beag}, Ó Dónaill (1977) \cite{odonaill}
\end{itemize}

 \noindent \textit{Nótaí Aistriúcháin:}
\begin{itemize}
	\item Féach ar an téarma `machine learning / ríomhfhoghlaim'.
\end{itemize}


\subsubsection*{to correlate (briathar): comhghaolaigh}
 \noindent \textit{Sainmhíniú (ga):} I gcomhthéacs matamaitice, comhghaol a fháil idir dhá shraith sonraí.
\\
 \noindent \textit{Sainmhíniú (en):} In a mathematical context, to find the correlation between two datasets.
\\
 \noindent \textit{Tagairtí:}
\begin{itemize}
	\item comhghaolaigh: Ó Dónaill (1977) \cite{odonaill}
\end{itemize}

 \noindent \textit{Nótaí Aistriúcháin:}
\begin{itemize}
	\item Téarma luaite mar théarma matamaitice i bhFoclóir Uí Dhónaill.
\end{itemize}


\subsubsection*{correlation (ainmfhocal): comhghaol}
 \noindent \textit{Sainmhíniú (ga):} Cáilíocht mhatamaiticiúil a dhéanann cur síos ar cé chomh maith is a luíonn dhá shraith uimhreacha lena chéile.
\\
 \noindent \textit{Sainmhíniú (en):} A mathematical quantity that describes how well two lists of numbers relate.
\\
 \noindent \textit{Tagairtí:}
\begin{itemize}
	\item comhghaol: De Bhaldraithe (1978) \cite{de-bhaldraithe}, Ó Dónaill (1977) \cite{odonaill}, Williams et al. (2023) \cite{storchiste}
\end{itemize}

 \noindent \textit{Nótaí Aistriúcháin:}
\begin{itemize}
	\item Luann Foclóir Uí Dhónaill `comhghaol' mar théarma matamaitice. Luann Stórchiste `comhéifeacht comhghaolúcháin' mar `correlation coefficient', téarma bunaithe as an bhfréamh céanna.
	\item Más é `correlation coefficient' atá i gceist, ba cheart `comhéifeacht comhghaolúcháin' a úsáid i bhfianaise a bhfuil le feiceáil i Stórchiste.
\end{itemize}


\subsubsection*{correlation coefficient (ainmfhocal): comhéifeacht comhghaolúcháin}
 \noindent \textit{Sainmhíniú (ga):} I gcomhthéacs matamaitice, luach (a scríobhtar go minic mar `r') idir -1 agus 1 léiríonn cé chomh mór, agus i cén treo, is atá an comhghaol idir dhá shraith sonraí.
\\
 \noindent \textit{Sainmhíniú (en):} In a mathematical context, a value (often written as `r') between -1 and 1 that describes how strong, and in what direction, correlation is between two datasets.
\\
 \noindent \textit{Tagairtí:}
\begin{itemize}
	\item comhéifeacht: féach ar an téarma `coefficient / comhéifeacht'
	\item comhghaolúcháin: Williams et al. (2023) \cite{storchiste}
\end{itemize}

 \noindent \textit{Nótaí Aistriúcháin:}
\begin{itemize}
	\item Féach ar an téarma `coefficient / comhéifeacht'
	\item Féach chomh maith ar an téarma `correlation / comhghaol'
	\item Féach chomh maith ar an téarma `to correlate / comhghaolaigh'
\end{itemize}


\subsubsection*{to corrupt (briathar): malartaigh}
 \noindent \textit{Sainmhíniú (ga):} I gcomhthéacs frith-shamplála, frith-shampla a chruthú trí nód (an t-ainmní nó an cuspóir) in abairt thriarach bhun-fhírinneach a athrú go nód eile.
\\
 \noindent \textit{Sainmhíniú (en):} In the context negative sampling, to create a negative sample by changing a node (the subject or object) in a triple to a different node.
\\
 \noindent \textit{Tagairtí:}
\begin{itemize}
	\item malartaigh: De Bhaldraithe (1978) \cite{de-bhaldraithe}, Dineen (1934) \cite{dineen}, Ó Dónaill et al. (1991) \cite{focloir-beag}, Ó Dónaill (1977) \cite{odonaill}
\end{itemize}

 \noindent \textit{Nótaí Aistriúcháin:}
\begin{itemize}
	\item Téarma díreach ar fáil le brí chomhchosúil (ach i gcomhthéacs níos leithne).
\end{itemize}


\subsubsection*{counterexample (ainmfhocal): frith-shampla}
 \noindent \textit{Sainmhíniú (ga):} Sonra a úsáidtear chun cur in iúl do shamhail ríomhaireachta rud atá mícheart nó nár cheart dó a fhoghlaim.
\\
 \noindent \textit{Sainmhíniú (en):} A data point that is used to instruct a machine learning model about something that is incorrect or that should not be learned.
\\
 \noindent \textit{Tagairtí:}
\begin{itemize}
	\item frith-: De Bhaldraithe (1978) \cite{de-bhaldraithe}, Dineen (1934) \cite{dineen}, Ó Dónaill et al. (1991) \cite{focloir-beag}, Ó Dónaill (1977) \cite{odonaill}
	\item sampla: féach ar an téarma `sample / sampla'
\end{itemize}

 \noindent \textit{Nótaí Aistriúcháin:}
\begin{itemize}
	\item Ní fheictear `frith-shampla' i bhfoclóir ar bith, ach amháin an réimír `firth-' agus focal `sampla'.
	\item Féach chomh maith ar an téarma `negative (sample) / frith-shampla'.
\end{itemize}


\subsubsection*{cross entropy loss (CEL) (ainmfhocal): pionós tras-eantrópachta (PTE)}
 \noindent \textit{Sainmhíniú (ga):} Feidhm phionóis bunaithe ar thras-eantrópacht. Úsáidtear go minic í i gcomhair an taisc réamhinsinte nasc (i measc tascanna ríomhfhoghlama eile).
\\
 \noindent \textit{Sainmhíniú (en):} A loss function based on cross entropy. It is often used in link prediction, among other machine learning tasks.
\\
 \noindent \textit{Tagairtí:}
\begin{itemize}
	\item pionós: féach ar an téarma `loss function / feidhm phionóis'
	\item tras-eantrópachta: féach ar an téarma `cross-entropy / tras-eantrópacht'
\end{itemize}

 \noindent \textit{Nótaí Aistriúcháin:}
\begin{itemize}
	\item Úsáidtear `pionós' seachas `feidhm phionóis' thuas mar giorrúchán. Tá sé ceart go leor `feidhm phionóis' a úsáid freisin.
	\item Féach chomh maith ar an téarma `loss function / feidhm phionóis'.
	\item Féach chomh maith ar an téarma `cross-entropy / tras-eantrópacht'.
\end{itemize}


\subsubsection*{cross-attention (ainmfhocal): tras-aird}
 \noindent \textit{Sainmhíniú (ga):} Oibriú airde ina dtagann an t-ionchur ar fad ó fhoinsí éagsúla.
\\
 \noindent \textit{Sainmhíniú (en):} An attention operation in which inputs come from different sources.
\\
 \noindent \textit{Tagairtí:}
\begin{itemize}
	\item tras-: De Bhaldraithe (1978) \cite{de-bhaldraithe}, Ó Dónaill et al. (1991) \cite{focloir-beag}, Ó Dónaill (1977) \cite{odonaill}, Williams et al. (2023) \cite{storchiste}
	\item aird: féach ar an téarma `attention / aird'
\end{itemize}

 \noindent \textit{Nótaí Aistriúcháin:}
\begin{itemize}
	\item Luann Stórchiste `tras-' mar `cross-' i gcomhthéacs matamaitice. Sna foclóirí eile, is i gcomhthéacs níos leithne atá sé luaite.
	\item Féach chomh maith ar an téarma `attention / aird'.
\end{itemize}


\subsubsection*{cross-entropy (ainmfhocal): tras-eantrópacht}
 \noindent \textit{Sainmhíniú (ga):} Tomhas ar cé chomh neamhordaithe is atá dáileadh staitistiúil (de réir dáileadh tagartha eile).
\\
 \noindent \textit{Sainmhíniú (en):} A measure of disorder in statistical distributions (in comparison to a reference distribution).
\\
 \noindent \textit{Tagairtí:}
\begin{itemize}
	\item tras-: De Bhaldraithe (1978) \cite{de-bhaldraithe}, Ó Dónaill et al. (1991) \cite{focloir-beag}, Ó Dónaill (1977) \cite{odonaill}, Williams et al. (2023) \cite{storchiste}
	\item eantrópacht: féach ar an téarma `entropy / eantrópacht'
\end{itemize}

 \noindent \textit{Nótaí Aistriúcháin:}
\begin{itemize}
	\item Luann Stórchiste `tras-' mar `cross-' i gcomhthéacs matamaitice. Sna foclóirí eile, is i gcomhthéacs níos leithne atá sé luaite.
	\item Féach ar an téarma `entropy / eantrópacht'.
\end{itemize}


\subsubsection*{curve (ainmfhocal): cuar}
 \noindent \textit{Sainmhíniú (ga):} I gcomhthéacs matamaitice, breacadh cothromóide nó tacair phointí a leanann patrún éigin.
\\
 \noindent \textit{Sainmhíniú (en):} In the context of mathematics, a plot of an equation or of various points that follow some pattern.
\\
 \noindent \textit{Tagairtí:}
\begin{itemize}
	\item cuar: De Bhaldraithe (1978) \cite{de-bhaldraithe}, Dineen (1934) \cite{dineen}, Ó Dónaill et al. (1991) \cite{focloir-beag}, Ó Dónaill (1977) \cite{odonaill}
	\item cuaire: Williams et al. (2023) \cite{storchiste}
\end{itemize}

 \noindent \textit{Nótaí Aistriúcháin:}
\begin{itemize}
	\item Cé go bhfuil an téarma `cuar' luaite sna foclóirí thuas mar leagan den fhocal Béarla `curve', ní luann foclóir ar bith acu comhthéacs matamaitice leis. Sin ráite, is é `cuaire', bunaithe ar an bhfréamh céanna, atá ag Stórchiste i gcomhair `curvature' i gcomhthéacs matamaitice. Toisc brí chomhchiallach a bheith ag `cuar', agus focal gaolmhar leis a bheith luaite i gcomhthéacs matamaitice, glactar leis.
\end{itemize}


\phantomsection \subsection*{D}
\addcontentsline{toc}{subsection}{D}
\markboth{D}{D}

\subsubsection*{data (ainmfhocal): sonraí}
 \noindent \textit{Sainmhíniú (ga):} léiriú cainníochtúil nó cineálach ar rud.
\\
 \noindent \textit{Sainmhíniú (en):} a quantitative or qualitative description of something.
\\
 \noindent \textit{Tagairtí:}
\begin{itemize}
	\item sonra: De Bhaldraithe (1978) \cite{de-bhaldraithe}, Dineen (1934) \cite{dineen}, Ó Dónaill et al. (1991) \cite{focloir-beag}, Ó Dónaill (1977) \cite{odonaill}
\end{itemize}

 \noindent \textit{Nótaí Aistriúcháin:}
\begin{itemize}
	\item Úsáidtear an leagan iolra de ghnáth, toisc gur annamh a bhíonn trácht ar shonra amháin, ach ar thacar sonraí.
\end{itemize}


\subsubsection*{data leak (ainmfhocal): sceitheadh sonraí}
 \noindent \textit{Sainmhíniú (ga):} I gcomhthéacs ríomhfhoghlama, an rud a tharlaíonn nuair atá sonraí ón tacar deimhnithe / teisteála le feiceáil le linn an phróisis thraenála, rud a bhfuil ró-fhoghlaim mar thoradh air.
\\
 \noindent \textit{Sainmhíniú (en):} In the context machine leaning, what happens when data from the validation / testing set can be seen during the training process, leading to overfitting.
\\
 \noindent \textit{Tagairtí:}
\begin{itemize}
	\item sonra: féach ar an téarma `data / sonraí'
	\item sceith: De Bhaldraithe (1978) \cite{de-bhaldraithe}, Dineen (1934) \cite{dineen}, Ó Dónaill (1977) \cite{odonaill}
\end{itemize}

 \noindent \textit{Nótaí Aistriúcháin:}
\begin{itemize}
	\item Téarmaí díreach ar fáil leis an mbrí chéanna (ach i gcomhthéacs neamh-theicniúil).
	\item Is é `scéith' seachas `sceith' atá i bhFoclóir Uí Dhuinín, ach glactar leis gurb in an focal céanna.
	\item Tá an frása `rún a sceitheadh, to divulge a secret' i bhFoclóir Uí Dhónaill. Tá sceitheadh sonraí díreach cosúil le sceitheadh rúin -- ba cheart go mbeadh sonraí teisteála rúnda le linn an phróisis thraenála. Muna bhfuil, is sceitheadh rúin atá ann -- agus an rún ná na sonraí teisteála iad féin.
	\item Féach chomh maith ar an téarma `data / sonraí'.
\end{itemize}


\subsubsection*{database (ainmfhocal): bunachar sonraí}
 \noindent \textit{Sainmhíniú (ga):} Bailiúchán sonraí ar ríomhaire a bhfuil struchtúr loighciúil righin leis.
\\
 \noindent \textit{Sainmhíniú (en):} A collection of data on a computer with a strict logical structure.
\\
 \noindent \textit{Tagairtí:}
\begin{itemize}
	\item bunachar*: Dineen (1934) \cite{dineen}, Ó Dónaill (1977) \cite{odonaill}
	\item sonra: féach ar an téarma `data / sonraí'
\end{itemize}

 \noindent \textit{Nótaí Aistriúcháin:}
\begin{itemize}
	\item * Úsáidtear `bunachair' mar leagan iolra den téarma seo, cé nach bhfuil leagan iolra den fhocal `bunachar' luaite i bhFoclóir Uí Dhónaill.
	\item Tá `bunachar sonraí' ar fáil mar aistriúchán ar `database' ar Focloir.ie agus ar Téarma.ie, ach níl i bhfoclóir ar bith eile a úsáidtear sa tráchtas seo (.i. Ó Dónall, Ua Duinnín, srl). Sin ráite, tá idir `bunachar' agus `sonra' ar fáil i go leor foclóirí eile, agus níl fianaise ar bith ann go mbeadh an téarma `bunachar sonraí' mí-nádúrtha dá bharr sin. Móide sin, níl cúis ar bith téarma eile le brí gaolmhar (m.sh. `foras sonraí') a chumadh nuair atá téarma cuí ann cheana féin. Glactar le `bunachar sonraí' mar sin.
	\item Féach chomh maith ar an téarma `data / sonraí'.
\end{itemize}


\subsubsection*{dataset (ainmfhocal): tacar sonraí}
 \noindent \textit{Sainmhíniú (ga):} Grúpa sonraí curtha le chéile, go háirithe chun samhlacha ríomhfhoghlama a thraenáil nó a mheasúnú.
\\
 \noindent \textit{Sainmhíniú (en):} A set of data that is gathered together, especially to be used for training or evaluating machine learning models.
\\
 \noindent \textit{Tagairtí:}
\begin{itemize}
	\item tacar: féach ar an téarma `set / tacar'.
	\item sonra: féach ar an téarma `data / sonraí'.
\end{itemize}

 \noindent \textit{Nótaí Aistriúcháin:}
\begin{itemize}
	\item Téarma cruthaithe go díreach as an dá théarma thuas.
	\item Féach chomh maith ar an téarma `data / sonraí'.
	\item Féach chomh maith ar an téarma `set / tacar'.
\end{itemize}


\subsubsection*{decision boundary (ainmfhocal): teorainn chinnidh}
 \noindent \textit{Sainmhíniú (ga):} I gcomhthéacs aicmithe, cuar a scarann dhá aicme nó níos mó óna chéile. I gcomhthéacs ríomhfhoghlama níos leithne, cuar a scarann dhá shórt réamhinsint (nó níos mó) óna chéile.
\\
 \noindent \textit{Sainmhíniú (en):} In the context of classification, a curve that separates two or more classes from each other. In the context of machine learning more generally, a curve that separates two (or more) sorts of predictions from each other.
\\
 \noindent \textit{Tagairtí:}
\begin{itemize}
	\item teorainn: De Bhaldraithe (1978) \cite{de-bhaldraithe}, Dineen (1934) \cite{dineen}, Ó Dónaill et al. (1991) \cite{focloir-beag}, Ó Dónaill (1977) \cite{odonaill}
	\item cinneadh: De Bhaldraithe (1978) \cite{de-bhaldraithe}, Dineen (1934) \cite{dineen}, Ó Dónaill et al. (1991) \cite{focloir-beag}, Ó Dónaill (1977) \cite{odonaill}
\end{itemize}

 \noindent \textit{Nótaí Aistriúcháin:}
\begin{itemize}
	\item Is é `teora' seachas `teorainn' atá i bhFoclóir Uí Dhuinín.
	\item Ní luaitear ceachtar den dá théarma thuas i gcomhthéacs ríomhfhoghlama. Sin ráite, is fíor gur sórt teorann í an teorainn chinnidh -- ach is teorainn idir dhá chinneadh, seachas idir dhá thír, atá i gceist. De réir a bhfuil le feiceáil i bhFoclóir Uí Dhónaill, is féidir `teorainn' a úsáid i gcomhthéacseanna leithne -- agus glactar leis an úsáid seo de `teorainn' mar sin.
\end{itemize}


\subsubsection*{decision tree (ainmfhocal): crann cinnte}
 \noindent \textit{Sainmhíniú (ga):} I gcomhthéacs ríomheolaíochta, uirlis ríomhfhoghlama a dhéanann réamhinsintí (nó `cinntí') de réir rialacha atá samhlaithe ar chraobhacha an chrainn.
\\
 \noindent \textit{Sainmhíniú (en):} In the context of computer science, a machine learning tool that makes predictions (or `decisions') based on rules modelled on the branches of the tree.
\\
 \noindent \textit{Tagairtí:}
\begin{itemize}
	\item crann: féach ar an téarma `tree / crann'
	\item cinn: De Bhaldraithe (1978) \cite{de-bhaldraithe}, Dineen (1934) \cite{dineen}, Ó Dónaill et al. (1991) \cite{focloir-beag}, Ó Dónaill (1977) \cite{odonaill}
\end{itemize}

 \noindent \textit{Nótaí Aistriúcháin:}
\begin{itemize}
	\item Is é `cinneadh' seachas `cinn' atá i bhFoclóir Uí Dhónaill agus Uí Mhaoileoin.
	\item Is é `crann cinnteoireachta' atá ar Téarma.ie. Sin ráite, ní fheictear cúis ar bith nach leor crann cinnte (sin crann + cinneadh (ainmfhocal), seachas crann + cinnte (aidiacht)). Is éard is crann cinnte ann ná crann ríomheolaíochta a dhéanann cinntí -- mar sin, crann cinnte.
	\item Féach chomh maith ar an téarma `tree / crann'.
\end{itemize}


\subsubsection*{deep learning (ainmfhocal): foghlaim dhomhain}
 \noindent \textit{Sainmhíniú (ga):} Cur chuige ríomhfhoghlama a úsáideann líonraí néaracha móra (a bhfuil an-chuid ciseal iontu).
\\
 \noindent \textit{Sainmhíniú (en):} A machine learning approach that uses large neural networks (with many layers).
\\
 \noindent \textit{Tagairtí:}
\begin{itemize}
	\item foghlaim: féach ar an téarma `machine learning / ríomhfhoghlaim'
	\item domhain: De Bhaldraithe (1978) \cite{de-bhaldraithe}, Dineen (1934) \cite{dineen}, Ó Dónaill et al. (1991) \cite{focloir-beag}, Ó Dónaill (1977) \cite{odonaill}, Williams et al. (2023) \cite{storchiste}
\end{itemize}

 \noindent \textit{Nótaí Aistriúcháin:}
\begin{itemize}
	\item Tá `domhainfhoghlaim' ar Téarma.ie. Bíonn úsáid `domhain' mar réimír le feiceáil i bhFoclóir Uí Dhónaill in dá théarma ar leith -- `domhainmhachnamh' agus `domhainiascaireacht' -- rud a chuireann le fios go bhfuil an úsáid sin inghlactha. Sin ráite, cé go bhfuil an-chuid samplaí d'úsáid an téarma `domhain' mar aidiacht, níl sé le feiceáil mar réimír ach go hannamh (sa dá fhocal luaite cheana). Meastar go mbeadh `foghlaim dhomhain' níos léire ó thús deireadh toisc nach mbíonn an cleachtadh ann i nGaeilge réimíreanna a úsáid in ainneoin aidiachtaí ach go hannamh.
	\item Féach chomh maith ar an téarma `machine learning / ríomhfhoghlaim'.
\end{itemize}


\subsubsection*{degree (ainmfhocal): céim}
 \noindent \textit{Sainmhíniú (ga):} I gcomhthéacs nóid i ngraf eolais, ca mhéad ceangal atá aige le nóid eile sa ngraf.
\\
 \noindent \textit{Sainmhíniú (en):} In the context of a node in a knowledge graph, how many connections it has with other nodes in the graph.
\\
 \noindent \textit{Tagairtí:}
\begin{itemize}
	\item céim: De Bhaldraithe (1978) \cite{de-bhaldraithe}, Dineen (1934) \cite{dineen}, Ó Dónaill et al. (1991) \cite{focloir-beag}, Ó Dónaill (1977) \cite{odonaill}
\end{itemize}

 \noindent \textit{Nótaí Aistriúcháin:}
\begin{itemize}
	\item Luann na foclóirí thuas (seachas Foclóir Uí Dhuinín) `céim' mar téarma geoiméadrachta / eolaíochta. Ní hionann `céim' geoiméadrachta agus `céim' nóid i ngraf eolais. Cé is moite de sin, is féidir `céim' a úsáid i gcomhthéacs eolaíochta chun trácht a dhéanamh ar cé chomh fásta / láidir / srl is atá rud (.i. céim teochta). Luíonn sé seo lé `céim' mhinicíochta i ngraf -- cé chomh coitianta is atá nód amháin.
\end{itemize}


\subsubsection*{dense (aidiacht): dlúth}
 \noindent \textit{Sainmhíniú (ga):} I gcomhthéacs graif eolais (nó fo-ghraif), an-cheangailte le codanna eile den ghraf / den fho-ghraf.
\\
 \noindent \textit{Sainmhíniú (en):} In the context of a knowledge graph (or subgraph), highly connected with other parts of the same graph / subgraph.
\\
 \noindent \textit{Tagairtí:}
\begin{itemize}
	\item dlúth: De Bhaldraithe (1978) \cite{de-bhaldraithe}, Dineen (1934) \cite{dineen}, Ó Dónaill et al. (1991) \cite{focloir-beag}, Ó Dónaill (1977) \cite{odonaill}, Williams et al. (2023) \cite{storchiste}
\end{itemize}

 \noindent \textit{Nótaí Aistriúcháin:}
\begin{itemize}
	\item Téarma díreach ar fáil le brí chomhchosúil.
\end{itemize}


\subsubsection*{dense layer (ainmfhocal): ciseal lán-cheangailte}
 \noindent \textit{Sainmhíniú (ga):} Ciseal i líonra néarach ina bhfuil chuile néaróg ionchuir ceangailte le chuile néaróg aschuir.
\\
 \noindent \textit{Sainmhíniú (en):} A layer in a neural network in which every input neuron is connected to every output neuron.
\\
 \noindent \textit{Tagairtí:}
\begin{itemize}
	\item ciseal: féach ar an téarma `layer / ciseal'
	\item lán-: De Bhaldraithe (1978) \cite{de-bhaldraithe}, Dineen (1934) \cite{dineen}, Ó Dónaill et al. (1991) \cite{focloir-beag}, Ó Dónaill (1977) \cite{odonaill}
	\item ceangailte: De Bhaldraithe (1978) \cite{de-bhaldraithe}, Dineen (1934) \cite{dineen}, Ó Dónaill (1977) \cite{odonaill}
\end{itemize}

 \noindent \textit{Nótaí Aistriúcháin:}
\begin{itemize}
	\item Tá `ciseal lán-cheangailte' níos dírí mar théarma. Sin ráite, níl fadhb ar bith le `ciseal dlúth' ach an oiread -- is ionann an dá théarma.
\end{itemize}


\subsubsection*{density (ainmfhocal): dlús}
 \noindent \textit{Sainmhíniú (ga):} I gcomhthéacs graif eolais (nó fo-ghraif), cé chomh dlúth is atá sé.
\\
 \noindent \textit{Sainmhíniú (en):} In the context of a knowledge graph (or subgraph), how dense it is.
\\
 \noindent \textit{Tagairtí:}
\begin{itemize}
	\item dlúth: De Bhaldraithe (1978) \cite{de-bhaldraithe}, Ó Dónaill et al. (1991) \cite{focloir-beag}, Ó Dónaill (1977) \cite{odonaill}
\end{itemize}

 \noindent \textit{Nótaí Aistriúcháin:}
\begin{itemize}
	\item Téarma díreach ar fáil le brí chomhchosúil.
\end{itemize}


\subsubsection*{dependency graph (ainmfhocal): graf spleáchais}
 \noindent \textit{Sainmhíniú (ga):} I gcomhthéacs ríomheolaíochta, graf a léiríonn ceangail spleáchais idir a chuid nód.
\\
 \noindent \textit{Sainmhíniú (en):} In the context of computer science, a graph that shows dependency relationships among its nodes.
\\
 \noindent \textit{Tagairtí:}
\begin{itemize}
	\item graf: féach ar an téarma `graph / graf'
	\item spleáchas: De Bhaldraithe (1978) \cite{de-bhaldraithe}, Dineen (1934) \cite{dineen}, Ó Dónaill et al. (1991) \cite{focloir-beag}, Ó Dónaill (1977) \cite{odonaill}
\end{itemize}

 \noindent \textit{Nótaí Aistriúcháin:}
\begin{itemize}
	\item Téarma cruthaithe go díreach as na focail thuas.
	\item Is é `spleádhachas' atá i bhFoclóir Uí Dhuinín, ach glactar leis gurb in an focal céanna.
	\item Féach chomh maith ar an téarma `graph / graf'.
\end{itemize}


\subsubsection*{DIKW Pyramid (ainmfhocal): Pirimid SFEE}
 \noindent \textit{Sainmhíniú (ga):} Pirimid ceithre chisil ina bhfuil ann, ó bhun go barr, sonraí, faisnéis, eolas, agus eagna.
\\
 \noindent \textit{Sainmhíniú (en):} A four-layer pyramid which contains, from the base to the top, data, information, knowledge, and wisdom.
\\
 \noindent \textit{Tagairtí:}
\begin{itemize}
	\item pirimid: De Bhaldraithe (1978) \cite{de-bhaldraithe}, Ó Dónaill et al. (1991) \cite{focloir-beag}, Ó Dónaill (1977) \cite{odonaill}
	\item sonra: féach ar an téarma `data / sonraí'
	\item faisnéis: féach ar an téarma `information content / lucht faisnéise'
	\item eolas: féach ar an téarma `knowledge graph (KG) / graf eolais (GE)'
	\item eagna: De Bhaldraithe (1978) \cite{de-bhaldraithe}, Dineen (1934) \cite{dineen}, Ó Dónaill et al. (1991) \cite{focloir-beag}, Ó Dónaill (1977) \cite{odonaill}
\end{itemize}

 \noindent \textit{Nótaí Aistriúcháin:}
\begin{itemize}
	\item Téarma cruthaithe go díreach as a bhfuil thuas. Cé go dtugtar leagan Gaeilge den ghiorrúchán `DIKW', toisc gur ainm dílis é `DIKW Pyramid' ag an bpointe seo, is dócha go mba cheart an litriú Béarla a chur leis chomh maith ar an gcéad úsáid. Mar shampla `an Phirimid SFEE (nó DIKW, de réir litriú an Bhéarla)'. I gcomhthéacsanna áirithe, is dócha go mbeadh fiú `Pirimid DIKW' (le litreacha an Bhéarla amháin) níos éasca le tuiscint ó thús deireadh.
	\item Úsáideadh `eagna' mar `wisdom' seachas focal ar bith eile (m.sh. gaois, eagnaíocht, srl) toisc é a bheith luaite ní hamháin leis an mbrí `wisdom' i bhFoclóir Uí Dhónaill, ach le `intelligence, understanding' chomh maith. Sin an bhrí is giorra do bhrí an fhocail `wisdom' sa gcomhthéacs seo, agus glactar leis mar sin.
	\item Féach chomh maith ar an téarma `data / sonraí'.
	\item Féach chomh maith ar an téarma `information content / lucht faisnéise'.
	\item Féach chomh maith ar an téarma `knowledge graph (KG) / graf eolais (GE)'.
\end{itemize}


\subsubsection*{dimension (ainmfhocal): toise}
 \noindent \textit{Sainmhíniú (ga):} Ag trácht ar spás veicteora, líon na n-uimhreacha atá i ngach uile veicteoir sa spás céanna; nó, ais amháin den spás sin.
\\
 \noindent \textit{Sainmhíniú (en):} Regarding a vector space, the number of elements contained in each vector in that space; or, one axis of that space.
\\
 \noindent \textit{Tagairtí:}
\begin{itemize}
	\item toise: De Bhaldraithe (1978) \cite{de-bhaldraithe}, Dineen (1934) \cite{dineen}*, Ó Dónaill et al. (1991) \cite{focloir-beag}*, Ó Dónaill (1977) \cite{odonaill}, Williams et al. (2023) \cite{storchiste}
\end{itemize}

 \noindent \textit{Nótaí Aistriúcháin:}
\begin{itemize}
	\item * Is mar sórt tomhais a fheictear an téarma sna foclóirí seo (seachas mar théarma a dhéanann cur síos ar ais i spás veicteora).
	\item I bhFoclóir Uí Dhónaill agus i Stórchiste, luaitear `toise' mar théarma matamaitice.
\end{itemize}


\subsubsection*{dimensionality (ainmfhocal): (frása le `toise')}
 \noindent \textit{Sainmhíniú (ga):} Ag trácht ar spás veicteora, líon na n-uimhreacha atá i ngach uile veicteoir sa spás céanna; nó, an t-airí a bhaineann le rud a bhfuil toisí aige.
\\
 \noindent \textit{Sainmhíniú (en):} Regarding a vector space, the number of elements contained in each vector in that space; or, the property of having dimensions.
\\
 \noindent \textit{Tagairtí:}
\begin{itemize}
	\item toise: féach ar an téarma `dimension / toise'
\end{itemize}

 \noindent \textit{Nótaí Aistriúcháin:}
\begin{itemize}
	\item féach ar an téarma `dimension / toise'
\end{itemize}


\subsubsection*{directed (aidiacht): dírithe}
 \noindent \textit{Sainmhíniú (ga):} Ag tagairt ar ceangal in abairt thriarach, ag ceangal an nód tosaigh (an ainmfhocal) leis an nód deiridh (an cuspóir) in ord.
\\
 \noindent \textit{Sainmhíniú (en):} Regarding an edge in a triple, providing an order-aware mapping of a source node (the subject) to a target node (the object).
\\
 \noindent \textit{Tagairtí:}
\begin{itemize}
	\item dírithe: Dineen (1934) \cite{dineen}, Ó Dónaill (1977) \cite{odonaill}
\end{itemize}

 \noindent \textit{Nótaí Aistriúcháin:}
\begin{itemize}
	\item Úsáidtear an téarma seo i gcomhthéacs treo radhairc, aidhm ghunna, agus ar eile. Sin ráite, tá an bhrí sin an-ghar don bhrí atá i gceist anseo. Glactar leis an téarma mar sin.
\end{itemize}


\subsubsection*{distribution (ainmfhocal): dáileadh}
 \noindent \textit{Sainmhíniú (ga):} I gcomhthéacs tacar / liosta uimhreacha, léiriú staitistiúil ar mhinicíocht na luachanna atá ann.
\\
 \noindent \textit{Sainmhíniú (en):} In the context of a set / list of numbers, a statistical description of how often each value occurs.
\\
 \noindent \textit{Tagairtí:}
\begin{itemize}
	\item dáileadh: Williams et al. (2023) \cite{storchiste}
\end{itemize}

 \noindent \textit{Nótaí Aistriúcháin:}
\begin{itemize}
	\item Téarma ar fáil sa gcomhthéacs matamaiticiúil céanna i Stórchiste.
	\item Níl an téarma seo luaite i comhthéacs comhchosúil i bhFoclóir ar bith eile atá á úsáid anseo.
\end{itemize}


\subsubsection*{domain (ainmfhocal): fearann}
 \noindent \textit{Sainmhíniú (ga):} I gcomhthéacs matamaitice, tacar luacha ar féidir iad a úsáid mar ionchur ar fheidhm éigin. I gcomhthéacs faisnéise i ngraf eolais, tacar nód ar féidir leo bheith mar ainmnithe in abairtí triaracha leis an bhfaisnéis sin.
\\
 \noindent \textit{Sainmhíniú (en):} In the context of mathematics, the set of values that can be input into some function. In the context of a predicate in a knowledge graph, the set of nodes that can be used as subjects in triples with that predicate.
\\
 \noindent \textit{Tagairtí:}
\begin{itemize}
	\item fearann: De Bhaldraithe (1978) \cite{de-bhaldraithe}*, Williams et al. (2023) \cite{storchiste}
\end{itemize}

 \noindent \textit{Nótaí Aistriúcháin:}
\begin{itemize}
	\item Téarma díreach ar fáil ó Stórchiste leis an mbrí chéanna i gcomhthéacs matamaitice.
	\item * Cé go bhfuil an téarma seo i bhFoclóir De Bhaldraithe, ní luaitear comhthéacs ar bith leis, agus níl sé cinnte mar sin an raibh brí mhatamaiticiúil i gceist ann nó nach raibh.
	\item Cé go bhfuil an focal seo i bhFoclóir Uí Dhónaill, i bhFoclóir Uí Dhónaill agus Uí Mhaoileoin, agus i bhFoclóir Uí Dhuinín, is mar limistéar talún seachas mar thacar luacha matamaitice a bhíonn sé luaite iontu.
\end{itemize}


\subsubsection*{dropout layer (ainmfhocal): ciseal nialas}
 \noindent \textit{Sainmhíniú (ga):} Ciseal i líonra néarach ina bhfuil luach roinnt néaróg ionadaithe le 0 go randamach.
\\
 \noindent \textit{Sainmhíniú (en):} A layer in a neural network in which the value of some neurons is randomly replaced by 0.
\\
 \noindent \textit{Tagairtí:}
\begin{itemize}
	\item ciseal: féach ar an téarma `layer / ciseal'
	\item nialas: De Bhaldraithe (1978) \cite{de-bhaldraithe}, Dineen (1934) \cite{dineen}, Ó Dónaill et al. (1991) \cite{focloir-beag}, Ó Dónaill (1977) \cite{odonaill}
\end{itemize}

 \noindent \textit{Nótaí Aistriúcháin:}
\begin{itemize}
	\item Ní iarrtar `dropout' a aistriú go litriúil toisc é sin a bheith i bhfad níos foclaí, gan buntáiste léir ag baint leis.
\end{itemize}


\phantomsection \subsection*{E}
\addcontentsline{toc}{subsection}{E}
\markboth{E}{E}

\subsubsection*{edge (ainmfhocal): ceangal}
 \noindent \textit{Sainmhíniú (ga):} I gcomhthéacs graf nó graf eolais, cuid de ghraf a nascann (nó a cheanglaíonn) dhá nód le chéile.
\\
 \noindent \textit{Sainmhíniú (en):} In the context of graphs or knowledge graphs, an element of a graph that serves to connect two nodes.
\\
 \noindent \textit{Tagairtí:}
\begin{itemize}
	\item ceangal: De Bhaldraithe (1978) \cite{de-bhaldraithe}, Dineen (1934) \cite{dineen}, Ó Dónaill et al. (1991) \cite{focloir-beag}, Ó Dónaill (1977) \cite{odonaill}
\end{itemize}

 \noindent \textit{Nótaí Aistriúcháin:}
\begin{itemize}
	\item Is mar thagairt d'fheistiú (le rópa) a úsáidtear an téarma seo den chuid is mó sna foclóirí thus. Sin ráite, is féidir a rá chomh maith go bhfuil dhá nód a bhfuil ceangal eatarthu `feistithe' lena chéile. Ní mheasann an t-údar gur bac ar bith é sin ar úsáid an fhocail `ceangal' leis an mbrí nua seo.
	\item Seo an téarma céanna is a úsáidtear chun `relation(ship)' a chur in iúl, toisc go bhfuil an bhrí chéanna leis i gcomhthéacs graf.
\end{itemize}


\subsubsection*{efficiency (ainmfhocal): éifeachtacht (ama, fhuinnimh)}
 \noindent \textit{Sainmhíniú (ga):} I gcomhthéacs ríomheolaíochta, cé chomh maith is atá samhail nó próiseas ar thaobh an ama / úsáid fhuinnimh de.
\\
 \noindent \textit{Sainmhíniú (en):} In the context of computer science, how effective a model is according to time taken or energy used.
\\
 \noindent \textit{Tagairtí:}
\begin{itemize}
	\item éifeachtacht: De Bhaldraithe (1978) \cite{de-bhaldraithe}, Ó Dónaill et al. (1991) \cite{focloir-beag}, Ó Dónaill (1977) \cite{odonaill}
\end{itemize}

 \noindent \textit{Nótaí Aistriúcháin:}
\begin{itemize}
	\item Téarma díreach ar fáil leis an mbrí chéanna ó na foclóirí thuas.
	\item Féach chomh maith ar an téarma `performance / éifeachtacht (ama, taisc)'.
\end{itemize}


\subsubsection*{element (ainmfhocal): ball}
 \noindent \textit{Sainmhíniú (ga):} I gcomhthéacs tacar, uimhir nó réad atá i dtacar.
\\
 \noindent \textit{Sainmhíniú (en):} In the context of sets, a number or object in a set.
\\
 \noindent \textit{Tagairtí:}
\begin{itemize}
	\item ball: De Bhaldraithe (1978) \cite{de-bhaldraithe}, Dineen (1934) \cite{dineen}*, Ó Dónaill et al. (1991) \cite{focloir-beag}*, Ó Dónaill (1977) \cite{odonaill}, Williams et al. (2023) \cite{storchiste}
\end{itemize}

 \noindent \textit{Nótaí Aistriúcháin:}
\begin{itemize}
	\item * Níl `ball' luaite mar théarma matamaitice sna foclóirí seo.
	\item Téarma díreach le fáil i gcomhthéacs tacar / matamaitice sna foclóirí eile thuas.
\end{itemize}


\subsubsection*{to embed (briathar): leabaigh}
 \noindent \textit{Sainmhíniú (ga):} I gcomhthéacs ríomhfhoghlama, leabuithe a chruthú i gcomhair nód / ceangal / focal, srl.
\\
 \noindent \textit{Sainmhíniú (en):} In the context of machine learning, to ceate embeddigns for nodes / edges / words / etc.
\\
 \noindent \textit{Tagairtí:}
\begin{itemize}
	\item leabaigh: De Bhaldraithe (1978) \cite{de-bhaldraithe}, Ó Dónaill et al. (1991) \cite{focloir-beag}, Ó Dónaill (1977) \cite{odonaill}
\end{itemize}

 \noindent \textit{Nótaí Aistriúcháin:}
\begin{itemize}
	\item Is é `leabú' seachas `leabaigh' atá luaite i bhFoclóir Uí Dhónaill agus Uí Mhaoileoin.
	\item Féach chomh maith ar an téarma `embedding / leabú'.
\end{itemize}


\subsubsection*{embedding (ainmfhocal): leabú}
 \noindent \textit{Sainmhíniú (ga):} Próiseas ríomhfhoghlama a dhéanann nód nó ceangal a chur i spás veicteora; nó, an veicteoir é féin sa spás veicteora a chuireann nód nó ceangal in iúl.
\\
 \noindent \textit{Sainmhíniú (en):} A machine learning process that places nodes or edges into a vector space; or, the vector itself in embedding space that represents a node or edge.
\\
 \noindent \textit{Tagairtí:}
\begin{itemize}
	\item leabú: De Bhaldraithe (1978) \cite{de-bhaldraithe}, Dineen (1934) \cite{dineen}, Ó Dónaill et al. (1991) \cite{focloir-beag}, Ó Dónaill (1977) \cite{odonaill}
\end{itemize}

 \noindent \textit{Nótaí Aistriúcháin:}
\begin{itemize}
	\item Úsáidtear `leabú' sna foclóirí chun tagairt a dhéanamh ar leabú fisiciúil: mar shampla, cloch a leabú i moirtéal. Sin ráite, níl sa mbrí nua (ríomheolaíochta) seo ach leabú i rud neamh-fhisiciúil -- spás veicteora. Mar sin, glactar le húsáid an téarma seo.
\end{itemize}


\subsubsection*{end condition (ainmfhocal): coinníoll críochnaithe}
 \noindent \textit{Sainmhíniú (ga):} I gcomhthéacs ríomheolaíochta, coinníoll a chinneann cathain ar cheart próiseas a stopadh.
\\
 \noindent \textit{Sainmhíniú (en):} In the context of computer science, a condition that determines when a process should be stopped.
\\
 \noindent \textit{Tagairtí:}
\begin{itemize}
	\item coinníoll: De Bhaldraithe (1978) \cite{de-bhaldraithe}, Dineen (1934) \cite{dineen}*, Ó Dónaill et al. (1991) \cite{focloir-beag}, Ó Dónaill (1977) \cite{odonaill}
	\item críochnaigh: De Bhaldraithe (1978) \cite{de-bhaldraithe}, Dineen (1934) \cite{dineen}*, Ó Dónaill et al. (1991) \cite{focloir-beag}**, Ó Dónaill (1977) \cite{odonaill}
\end{itemize}

 \noindent \textit{Nótaí Aistriúcháin:}
\begin{itemize}
	\item * Is é `coingheall' agus `críochnuighim' atá i bhFoclóir Uí Dhuinín, ach glactar leis gurb iad na focail chéanna le sean-litriú.
	\item ** Is é `críochnú' atá i bhFoclóir Uí Dhónaill agus Uí Mhaoileoin.
	\item Ní roghnaíodh `coinníoll deiridh' (nó `coinníoll deireanach', srl) anseo toisc go dtabharfadh sé sin le fios go bhfuil an coinníoll seo ag deireadh liosta coinníollacha, seachas mar choinníoll a chinneann cathain ar ceart próiseas a chríochnú.
\end{itemize}


\subsubsection*{entity (ainmfhocal): aonad}
 \noindent \textit{Sainmhíniú (ga):} Coincheap nó réad amháin, go háirithe agus é samhlaithe mar nód i ngraf eolais, mar shonra i mbunachar sonraí, nó mar ainm i dtéacs scríofa go nádúrtha.
\\
 \noindent \textit{Sainmhíniú (en):} A single concept or object, especially when represented by a node in a knowledge graph, a data point in a database, or a name in a natural language text.
\\
 \noindent \textit{Tagairtí:}
\begin{itemize}
	\item aonad: De Bhaldraithe (1978) \cite{de-bhaldraithe}, Ó Dónaill et al. (1991) \cite{focloir-beag}, Ó Dónaill (1977) \cite{odonaill}
\end{itemize}

 \noindent \textit{Nótaí Aistriúcháin:}
\begin{itemize}
	\item Roghnaíodh `aonad' seachas `aonán' toisc brí níos leithne a bheith i gceist leis. Feictear i bhFoclóir Uí Dhónaill agus Uí Mhaoileoin gurb ionann aonad agus `rud amháin ann féin' -- a bhrí díreach atá de dhíth. Ní shin mar a bhíonn le `aonán', áfach -- bíonn an téarma sin luaite i bhFoclóir Uí Dhónaill agus i bhFoclóir De Bhaldraithe mar théarma bitheolaíochta; sin le rá, déanann sé trácht ar aonán beo (amháin). Cinntear `aonad' a úsáid mar sin.
	\item I gcomhthéacs graf eolais, más é nód i ngraf atá i gceist, is dócha gur léire `nód' a úsáid seachas `aonad'. Más é an coincheap taobh thiar den nód atá i gceist, áfach, bheadh an-chúis ann an focal `aonad' a úsáid.
\end{itemize}


\subsubsection*{entity alignment (ainmfhocal): ailíniú aonad}
 \noindent \textit{Sainmhíniú (ga):} I gcomhthéacs ríomhfhoghlama, tasc a bhfuil i gceist aige réamhinsint a dhéanamh ar cén aonaid ar leith i dtacar sonraí atá dáiríre ag seasamh don choincheap nó don réad céanna.
\\
 \noindent \textit{Sainmhíniú (en):} In the context of machine learning, the task of predicting which distinct entities in a dataset actually stand for the same concept or object.
\\
 \noindent \textit{Tagairtí:}
\begin{itemize}
	\item ailíniú: féach ar an téarma `alignment / ailíniú'
	\item aonad: féach ar an téarma `entity / aonad'
\end{itemize}

 \noindent \textit{Nótaí Aistriúcháin:}
\begin{itemize}
	\item Cuirtear san uimhir iolra an focal `aonad' tosc go mbaineann ailíniú aonad le ríonnt aonad, seachas le haonad amháin, i gcónaí
	\item Féach chomh maith ar an téarma `alignment / ailíniú'.
	\item Féach chomh maith ar an téarma `entity / aonad'.
\end{itemize}


\subsubsection*{entropy (ainmfhocal): eantrópacht}
 \noindent \textit{Sainmhíniú (ga):} Tomhas ar cé chomh neamhordaithe is atá dáileadh staitistiúil.
\\
 \noindent \textit{Sainmhíniú (en):} A measure of disorder in statistical distributions.
\\
 \noindent \textit{Tagairtí:}
\begin{itemize}
	\item eantrópacht: Ó Dónaill (1977) \cite{odonaill}
\end{itemize}

 \noindent \textit{Nótaí Aistriúcháin:}
\begin{itemize}
	\item Téarma díreach ar fáil i bhFoclóir Uí Dhónaill.
\end{itemize}


\subsubsection*{environment (ainmfhocal): comhthéacs cóid}
 \noindent \textit{Sainmhíniú (ga):} Comhthéacs a úsáidtear chun tionscadal cóid a scaradh ó thionscadal eile, agus ina bhfuil leaganacha áirithe de gach uile leabharlann chóid atá in úsáid stóráilte agus sainmhínithe.
\\
 \noindent \textit{Sainmhíniú (en):} A context that is used to keep various coding projects isolated, and in which specific versions of coding libraries are stored and defined.
\\
 \noindent \textit{Tagairtí:}
\begin{itemize}
	\item comhthéacs: De Bhaldraithe (1978) \cite{de-bhaldraithe}, Ó Dónaill et al. (1991) \cite{focloir-beag}, Ó Dónaill (1977) \cite{odonaill}
	\item cód: De Bhaldraithe (1978) \cite{de-bhaldraithe}, Dineen (1934) \cite{dineen}*, Ó Dónaill et al. (1991) \cite{focloir-beag}, Ó Dónaill (1977) \cite{odonaill}
\end{itemize}

 \noindent \textit{Nótaí Aistriúcháin:}
\begin{itemize}
	\item * Tá an focal `cód' i bhFoclóir Uí Dhuinín, ach is le brí `a code, a codex, a book' atá sé luaite. Ní dócha go rabhthas ag trácht ar cód ríomhaireachta sa bhFoclóir sin, toisc é a bheith níos sine, agus aidhm níos ginearálta (seachas teicniúil) a bheith aige.
	\item Tá `timpeallacht' ar Focloir.ie agus ar Téarma.ie lena aghaidh seo. Sin ráite, meastar go mbeadh sé sin rud beag ró-litriúil -- tá an cuma ar an scéal ó Fhoclóir Uí Dhónaill agus Uí Mhaoileoin (mar shampla) go mbíonn `timpeallacht' úsáidte chun trácht ar chúrsaí a bhaineann leis an saol. Luann Foclóir Uí Dhónaill le `surroundings' é -- sin le rá, baineann sé leis an rudaí atá timpeall ar rud. Ní shin atá i gceist le `environment' cóid. Is éard atá i gceist le `environment' cód ná cnuasach leabharlanna atá úsáidte mar chuid de thionscadal cóid amháin. Ní hé go bhfuil siad `timpeall' air. An choincheap is tábhachtaí ná, i gcomhthéacs cóid amháin, is ionann an t-ordú `python' agus Python 3.7. I gcomhthéacs eile, is ionann an t-ordú `python' agus Python 3.9. Agus ar eile. Mar sin, séard a dhéanann comhthéacs cóid ná comhthéacs a thabhairt don ríomhaire le go dtuigfeadh sé na horduithe atá tugtha dó.
\end{itemize}


\subsubsection*{epoch (ainmfhocal): seal}
 \noindent \textit{Sainmhíniú (ga):}  Geábh iomlán feabhsaithe ina fheiceann an tsamhail ríomhfhoghlama chuile shonra sa tacar traenála aon uair amháin.
\\
 \noindent \textit{Sainmhíniú (en):} A full pass of optimisation in which the machine learning model sees every data point in the training set exactly once.
\\
 \noindent \textit{Tagairtí:}
\begin{itemize}
	\item seal: De Bhaldraithe (1978) \cite{de-bhaldraithe}, Dineen (1934) \cite{dineen}, Ó Dónaill et al. (1991) \cite{focloir-beag}, Ó Dónaill (1977) \cite{odonaill}
\end{itemize}

 \noindent \textit{Nótaí Aistriúcháin:}
\begin{itemize}
	\item Is iomaí focal ar féidir a úsáid chun an bhrí seo a chur in iúl (geábh, babhta, timthriall, sealad, srl). Roghnaíodh `seal' toisc é a bheith ag tagairt ar geábh foghlama agus ar tréimhse ama, díreach mar a bhíonn lucht ríomhaireachta ag samhlú `epoch' ríomhfhoghlama.
	\item Bíonn cúpla focal luaite le bríonna comhchosúla ar Téarma.ie: `tardhul' agus `sealad'. Ní léir cad as a thagann `tardhul', agus níl rian air mar fhocal i bhfoclóir dúchasach ar bith. Ní ghlactar leis mar sin. Tá `sealad' le feiceáil ar Téarma.ie chomh maith, ach ní luíonn sé le comhthéacs ríomheolaíochta. Déanann “sealad” trácht ar thréimhse ama de réir Fhoclóir Uí Dhónaill. Tá an-chiall leis sin i gcomhthéacs `epoch' na Geolaíochta. Ach i gcomhthéacs ríomheolaíochta, is ionann `epoch' agus `iteration' amháin tríd an tacar traenála. Sin an bhrí atá luaite le `seal' ar Teanglann, agus is mar sin a ghlactar leis thar `sealad' mar théarma.
\end{itemize}


\subsubsection*{equation (ainmfhocal): cothromóid}
 \noindent \textit{Sainmhíniú (ga):} I gcomhthéacs na matamaitice, abairt matamaiticiúil scoilte ina dhá leath ag an gcomhartha `=', agus a deir go bhfuil an luach céanna ag dá thaobh.
\\
 \noindent \textit{Sainmhíniú (en):} In the context of mathematics, a mathematical statement split into two parts by the sign `=', and that says that the two halves of the statement on either side have the same value.
\\
 \noindent \textit{Tagairtí:}
\begin{itemize}
	\item cothromóid: De Bhaldraithe (1978) \cite{de-bhaldraithe}, Ó Dónaill (1977) \cite{odonaill}
\end{itemize}

 \noindent \textit{Nótaí Aistriúcháin:}
\begin{itemize}
	\item Téarma díreach ar fáil leis an mbrí cheannann chéanna.
\end{itemize}


\subsubsection*{error (ainmfhocal): earráid}
 \noindent \textit{Sainmhíniú (ga):} I gcomhthéacs ríomhfhoghlama, luach a dhéanann tomhas ar cé chomh minic is a dhéanann samhail ríomhfhoghlama réamhinsint mícheart, agus ar cé chomh mícheart is atá na réamhinsintí sin.
\\
 \noindent \textit{Sainmhíniú (en):} In the context of machine learning, a value that measures how often a machine learning model errs in its predictions, and how bad those predictions are.
\\
 \noindent \textit{Tagairtí:}
\begin{itemize}
	\item earráid: De Bhaldraithe (1978) \cite{de-bhaldraithe}, Dineen (1934) \cite{dineen}, Ó Dónaill et al. (1991) \cite{focloir-beag}, Ó Dónaill (1977) \cite{odonaill}, Williams et al. (2023) \cite{storchiste}
\end{itemize}

 \noindent \textit{Nótaí Aistriúcháin:}
\begin{itemize}
	\item Téarma díreach ar fáil i gcomhthéacs matamaitice leis an mbrí cheannann chéanna ó Stórchiste, agus le bríonna comhchosúl (ach i gcomhthéacs níos leithna) sna foclóirí eile.
	\item Bheadh ciall le `tomhas earráide' chomh maith, go háirithe má tá níos mó ná tomhas earráide amháin á phlé.
\end{itemize}


\subsubsection*{estimate (ainmfhocal): meastachán}
 \noindent \textit{Sainmhíniú (ga):} I gcomhthéacs matamaitice, luach a dhéanann cur síos cainníochtúil ar fheiniméan éigin, gan a bheith iomlán cruinn.
\\
 \noindent \textit{Sainmhíniú (en):} In a mathematical context, a value that gives a quantitative description of some phenomenon, but which may not be entirely exact.
\\
 \noindent \textit{Tagairtí:}
\begin{itemize}
	\item meastachán: De Bhaldraithe (1978) \cite{de-bhaldraithe}, Ó Dónaill et al. (1991) \cite{focloir-beag}, Ó Dónaill (1977) \cite{odonaill}
\end{itemize}

 \noindent \textit{Nótaí Aistriúcháin:}
\begin{itemize}
	\item Téarma ar fáil mar téarma airgeadais / matamaitice ó na foclóirí thuas.
	\item Is féidir `tomhas' a úsáid chomh maith (go háirithe sa gcaint) chun brí chomhchosúil leis seo a chur in iúl. Cé is moite de sin, roghnaíodh `meastachán' mar théarma dó seo chun idirdhealú léir a dhéanamh idir `metric / tomhas' agus `*estimate / tomhas'. Ar leibhéal neamh-fhoirmeálta, tá sé ceart go leor tomhas a úsáid leis an mbrí sin nuair is léir ón gcomhthéacs cén bhrí atá i gceist leis.
	\item Nuair is meastachán míchruinn (nach ionann agus mícheart) atá i gceist, moltar an téarma `garmheastachán' (de réir Fhoclóir Uí Dhónaill agus Fhoclóir De  Bhaldraithe).
	\item Tá an téarma seo comhchiallach leis an téarma `approximation / meastachán' sa gcomhthéacs matamaitice / ríomheolaíochta atá i gceist anseo.
	\item Féach chomh maith ar an téarma `approximation / meastachán'.
	\item Féach chomh maith ar an téarma `metric / tomhas'.
\end{itemize}


\subsubsection*{to estimate (about) (ainmfhocal): meastachán a dhéanamh (ar)}
 \noindent \textit{Sainmhíniú (ga):} Luach a mheas.
\\
 \noindent \textit{Sainmhíniú (en):} To estimate a value.
\\
 \noindent \textit{Tagairtí:}
\begin{itemize}
	\item meastachán: féach ar an téarma `estimate / meastachán'
\end{itemize}

 \noindent \textit{Nótaí Aistriúcháin:}
\begin{itemize}
	\item Tá an téarma seo comhchiallach leis an téarma `to approximate / meastachán a dhéanamh (ar)' sa gcomhthéacs matamaitice / ríomheolaíochta atá i gceist anseo.
	\item Féach ar an téarma `to approximate / meastachán a dhéanamh (ar)'.
	\item Féach ar an téarma `estimate / meastachán'.
\end{itemize}


\subsubsection*{to evaluate (briathar): measúnaigh}
 \noindent \textit{Sainmhíniú (ga):} Próiseas teisteála nó deimhnithe a dhéanamh ar shamhail ríomhfhoghlama.
\\
 \noindent \textit{Sainmhíniú (en):} To perform testing or validation on a machine learning model.
\\
 \noindent \textit{Tagairtí:}
\begin{itemize}
	\item measúnaigh: féach ar an téarma `evaluation / measúnú'
\end{itemize}

 \noindent \textit{Nótaí Aistriúcháin:}
\begin{itemize}
	\item Féach ar an téarma `evaluation / measúnú'
\end{itemize}


\subsubsection*{evaluation (ainmfhocal): measúnú}
 \noindent \textit{Sainmhíniú (ga):} Próiseas a úsáidtear chun fáil amach cé chomh maith (nó cé chomh dona) is a fheidhmíonn samhail ríomhfhoghlama le linn a traenála, nó tar a éis sin.
\\
 \noindent \textit{Sainmhíniú (en):} A process that is used to determine how well (or how poorly) a machine learning model works during its training, or after it has been trained.
\\
 \noindent \textit{Tagairtí:}
\begin{itemize}
	\item measúnú: De Bhaldraithe (1978) \cite{de-bhaldraithe}, Ó Dónaill (1977) \cite{odonaill}
\end{itemize}

 \noindent \textit{Nótaí Aistriúcháin:}
\begin{itemize}
	\item Úsáidtear `measúnú' seachas `meas' toisc é a bheith úsáidte i gcomhthéacs níos teicniúla, agus chun débhrí a sheachaint idir meas (mar smaoineamh) agus meas (mar mheasúnú).
	\item Cé is moite de sin, ní luaitear an téarma seo i gcomhthéacs ríomheolaíochta sna foclóirí thuas.
\end{itemize}


\subsubsection*{expression (ainmfhocal): slonn}
 \noindent \textit{Sainmhíniú (ga):} I gcomhthéacs matamaitice, abairt a bhfuil luach nó brí mhatamaiticiúil léi.
\\
 \noindent \textit{Sainmhíniú (en):} In the context of mathematics, a statement that has a value or mathematical meaning attached to it.
\\
 \noindent \textit{Tagairtí:}
\begin{itemize}
	\item pharaiméadar: Ó Dónaill (1977) \cite{odonaill}
\end{itemize}

 \noindent \textit{Nótaí Aistriúcháin:}
\begin{itemize}
	\item Téarma díreach ar fáil leis an mbrí cheannann chéanna.
\end{itemize}


\phantomsection \subsection*{F}
\addcontentsline{toc}{subsection}{F}
\markboth{F}{F}

\subsubsection*{family (ainmfhocal): (frása le `gaolmhar')}
 \noindent \textit{Sainmhíniú (ga):} I gcomhthéacs ríomhfhoghlama, grúpa samhlacha atá cosúil lena chéile, agus a thagann ón smaoineamh / ón mbun-samhail chéanna.
\\
 \noindent \textit{Sainmhíniú (en):} In the context of machine learning, a group of models that are similar to each other, and that come from the same idea / base model.
\\
 \noindent \textit{Tagairtí:}
\begin{itemize}
	\item gaolmhar: De Bhaldraithe (1978) \cite{de-bhaldraithe}, Dineen (1934) \cite{dineen}, Ó Dónaill et al. (1991) \cite{focloir-beag}, Ó Dónaill (1977) \cite{odonaill}, Williams et al. (2023) \cite{storchiste}
\end{itemize}

 \noindent \textit{Nótaí Aistriúcháin:}
\begin{itemize}
	\item Moltar `gaolmhar' a úsáid i bhfrása. Mar shampla `The TransE family of models' $\rightarrow$ `TransE agus na samhlacha gaolmhara leis'.
	\item Cé go mbíonn `gaolmhar' bainteach go minic le clanna, is léir ó Fhoclóir Uí Dhónaill agus ó Fhoclóir De Bhaldraithe nach mar sin amháin a bhíonn sé. Glactar leis mar théarma toisc go ndéanann sé cur síor ar an gcaoi atá samhlacha ríomhfhoghlama gaolmhara ceangailte lena chéile (ar nós clann, beagnach) gan bheith ag braith ar théarma míchuí / sa gcomhthéacs mícheart (.i. teaghlach / clann / muintir).
	\item Tá `fine' luaite ar Téarma.ie (mar shampla, tá `fine clófhoirne' acu mar `font family'). Ní luíonn an úsáid seo le sampla Fhoclóir Uí Dhónaill ná le sampla Fhoclóir Uí Dhónaill agus Uí Mhaoileoin, a luann `fine' mar `family' nó cine daoine amháin, gan trácht ar úsáid níos leithne ná sin. Ní ghlactar le `fine' anseo mar sin.
\end{itemize}


\subsubsection*{feature (ainmfhocal): airí}
 \noindent \textit{Sainmhíniú (ga):} Ionchur amháin ar shamhail ríomhfhoghlama, nó cuid uimhriúil den tsamhail chéanna, a sheasann do shonra nithiúil nó folaigh den tacar sonraí atá á fhoghlaim.
\\
 \noindent \textit{Sainmhíniú (en):} A single input to, or numerical element of, a machine learning model that represents a concrete or latent element of the dataset being learned.
\\
 \noindent \textit{Tagairtí:}
\begin{itemize}
	\item airí: De Bhaldraithe (1978) \cite{de-bhaldraithe}, Dineen (1934) \cite{dineen}, Ó Dónaill et al. (1991) \cite{focloir-beag}, Ó Dónaill (1977) \cite{odonaill}
\end{itemize}

 \noindent \textit{Nótaí Aistriúcháin:}
\begin{itemize}
	\item Ní mar théarma eolaíochta atá `airí' luaite i gceann ar bith de na foclóirí thuas. Sin ráite, meastar ó shamplaí atá le fáil ann go gcuireann `airí' an bhrí cheart in iúl agus é á úsáid i gcomhthéacs eolaíochta.
	\item Tá réimse leathan focal eile (.i. tréith, gné, srl) a bheadh inúsáidte sa gcomhthéacs seo (agus is é `gné' atá ar Téarma.ie). Cé is moite de sin, is minice a úsáidtear iad sin i gcomhthéacs duine, seachas i gcomhthéacs eolaíochta, de réir a bhfuil le feiceáil sna foclóirí thuas .
\end{itemize}


\subsubsection*{few-shot (aidiacht): cúpla-sonra}
 \noindent \textit{Sainmhíniú (ga):} Cur chuige mion-fheabsaithe ina bhfuil an tsamhail réamh-thraenáilte in ann cúpla sonra ó thacar sonraí nua a fheiceáil le linn á mion-fheabhsaithe.
\\
 \noindent \textit{Sainmhíniú (en):} A finetuning protocol in which the pretrained model is able to see a few data points from the new data set during finetuning.
\\
 \noindent \textit{Tagairtí:}
\begin{itemize}
	\item cúpla: De Bhaldraithe (1978) \cite{de-bhaldraithe}, Dineen (1934) \cite{dineen}, Ó Dónaill et al. (1991) \cite{focloir-beag}, Ó Dónaill (1977) \cite{odonaill}
	\item sonra: féach ar an téarma `database / bunachar sonraí'
\end{itemize}

 \noindent \textit{Nótaí Aistriúcháin:}
\begin{itemize}
	\item Féach ar an téarma `n-shot / n-sonra'.
\end{itemize}


\subsubsection*{filter (ainmfhocal): scagaire}
 \noindent \textit{Sainmhíniú (ga):} I gcomhthéacs próiseála sonraí, córas nó próiseas a scagann sonraí -- sin le rá, a bhaineann roinnt sonraí ó thacar mór sonraí.
\\
 \noindent \textit{Sainmhíniú (en):} In the context of data processing, a system or process that filters data -- that is, that removes some data from a larger dataset.
\\
 \noindent \textit{Tagairtí:}
\begin{itemize}
	\item scagaire: De Bhaldraithe (1978) \cite{de-bhaldraithe}, Dineen (1934) \cite{dineen}, Ó Dónaill et al. (1991) \cite{focloir-beag}, Ó Dónaill (1977) \cite{odonaill}
\end{itemize}

 \noindent \textit{Nótaí Aistriúcháin:}
\begin{itemize}
	\item Téarma díreach ar fáil le brí chomhchosúil ó na foclóirí thuas (i gcomhthéacs níos ginearálta). I bhFoclóir Uí Dhuinín agus i bhFoclóir Uí Dhónaill agus Uí Mhaoileoin, is i gcomhthéacs scagaireachta fisicí amháin atá sé.
	\item Féach chomh maith ar an téarma `filtering / scagaireacht'.
	\item Féach chomh maith ar an téarma `to filter / scag'.
\end{itemize}


\subsubsection*{to filter (briathar): scag}
 \noindent \textit{Sainmhíniú (ga):} I gcomhthéacs próiseála sonraí, roinnt sonraí ó thacar sonraí a bhaint de de réir tomhais nó algartaim éigin.
\\
 \noindent \textit{Sainmhíniú (en):} In the context of data processing, to remove some data from a dataset based on some metric or algorithm.
\\
 \noindent \textit{Tagairtí:}
\begin{itemize}
	\item scag: De Bhaldraithe (1978) \cite{de-bhaldraithe}, Dineen (1934) \cite{dineen}, Ó Dónaill et al. (1991) \cite{focloir-beag}, Ó Dónaill (1977) \cite{odonaill}, Williams et al. (2023) \cite{storchiste}
\end{itemize}

 \noindent \textit{Nótaí Aistriúcháin:}
\begin{itemize}
	\item Téarma díreach ar fáil le brí chomhchosúil ó na foclóirí thuas (i gcomhthéacs níos ginearálta). I bhFoclóir Uí Dhuinín agus i bhFoclóir Uí Dhónaill agus Uí Mhaoileoin, is i gcomhthéacs scagaireachta fisicí amháin atá sé.
	\item Is é `scagadh' seachas `scag' atá i bhFoclóir Uí Dhónaill agus Uí Mhaoileoin.
\end{itemize}


\subsubsection*{filtering (ainmfhocal): scagaireacht}
 \noindent \textit{Sainmhíniú (ga):} I gcomhthéacs próiseála sonraí, próiseas a úsáidtear chun cuid de na sonraí ó thacar sonraí a bhaint de réir tomhais nó algartaim éigin.
\\
 \noindent \textit{Sainmhíniú (en):} In the context of data processing, a process that is used to remove some of the data from a dataset based on some metric or algorithm.
\\
 \noindent \textit{Tagairtí:}
\begin{itemize}
	\item scagaireacht: Dineen (1934) \cite{dineen}, Ó Dónaill et al. (1991) \cite{focloir-beag}, Ó Dónaill (1977) \cite{odonaill}
\end{itemize}

 \noindent \textit{Nótaí Aistriúcháin:}
\begin{itemize}
	\item Téarma díreach ar fáil le brí chomhchosúil ó Fhoclóir Uí Dhónaill (i gcomhthéacs níos ginearálta). I bhFoclóir Uí Dhónaill agus Uí Mhaoileoin, is i gcomhthéacs scagaireachta fisicí amháin atá sé.
	\item Níl an téarma seo le feiceáil i bhFoclóir De Bhaldraithe, ach tá `scagaire' agus `scag' ann, agus bríonna comhchosúla luaite leo.
	\item Féach chomh maith ar an téarma `filter / scagaire'.
	\item Féach chomh maith ar an téarma `to filter / scag'.
\end{itemize}


\subsubsection*{to finetune (briathar): mion-fheabhsú}
 \noindent \textit{Sainmhíniú (ga):} Samhail ríomhfhoghlama atá traenáilte cheana a thraenáil ar shonraí nua.
\\
 \noindent \textit{Sainmhíniú (en):} To take a pre-trained machine learning model and train it further on new data.
\\
 \noindent \textit{Tagairtí:}
\begin{itemize}
	\item mion-: De Bhaldraithe (1978) \cite{de-bhaldraithe}, Dineen (1934) \cite{dineen}, Ó Dónaill et al. (1991) \cite{focloir-beag}*, Ó Dónaill (1977) \cite{odonaill}
	\item feabhsú: féach ar an téarma `to optimise / feabhsaigh'
\end{itemize}

 \noindent \textit{Nótaí Aistriúcháin:}
\begin{itemize}
	\item * Ní luaitear an téarma `mion-' mar réimír i bhFoclóir Uí Dhónaill agus Uí Mhaoileoin.
	\item Ní bhítear ag caint ar mion-fheabhsúchán (nach ionann agus mion-fheabhsú) ná ar `chórais mhion-fheabhsúchán' go minic.
	\item Is é `mionchoigeartú' atá ar Téarma.ie ina chomhair seo. Cé go bhfuil bunús leis sin de réir Fhoclóir Uí Dhónaill, is doiléire é mar théarma toisc nach mbíonn coigeartú in úsáid i gcomhthéacs ríomheolaíochta ar bith (fiú ar Téarma.ie, ní luaitear an comhthéacs sin leis). Thairis sin, ní mheastar go bhfuil gá le fréamh nua anseo -- níl i gceist le `finetuning' ar an leibhéal is bunúsaí ach feabhsú a dhéanamh arís.
	\item Féach chomh maith ar an téarma `to optimise / feabhsaigh'.
\end{itemize}


\subsubsection*{foundation model (ainmfhocal): samhail fhorais}
 \noindent \textit{Sainmhíniú (ga):} I gcomhthéacs ríomheolaíochta, samhail ríomhfhoghlama amháin ar féidir í a úsáid mar fhoras foghlama i réimse leathan, agus a mbíonn cur chuige tras-fhoghlama bunaithe uirthi chuige sin.
\\
 \noindent \textit{Sainmhíniú (en):} In the context of computer science, a single machine learning model that can be used as the foundation of learning in a broad domain, and which is applied in the transfer learning setting to do so.
\\
 \noindent \textit{Tagairtí:}
\begin{itemize}
	\item foras: Dineen (1934) \cite{dineen}, Ó Dónaill et al. (1991) \cite{focloir-beag}, Ó Dónaill (1977) \cite{odonaill}
	\item samhail: féach ar an téarma `model / samhail'
\end{itemize}

 \noindent \textit{Nótaí Aistriúcháin:}
\begin{itemize}
	\item Cé go bhfuil an focal `foras' i bhFoclóir De Bhaldraithe, is le brí ar leith atá sé luaite ann. Sna foclóirí eile, is leis an mbrí chéanna (i gcomhthéacs níos leithne) atá sé luaite.
	\item Féach chomh maith ar an téarma `model / samhail'.
\end{itemize}


\subsubsection*{framework (ainmfhocal): creatlach}
 \noindent \textit{Sainmhíniú (ga):} Struchtúr teibí a úsáidtear chun feiniméan a léiriú nó a thuiscint i bhfoirm ghinearálta.
\\
 \noindent \textit{Sainmhíniú (en):} An abstract structure used to describe or understand a phenomenon in a general form.
\\
 \noindent \textit{Tagairtí:}
\begin{itemize}
	\item creatlach: De Bhaldraithe (1978) \cite{de-bhaldraithe}, Ó Dónaill et al. (1991) \cite{focloir-beag}, Ó Dónaill (1977) \cite{odonaill}
\end{itemize}

 \noindent \textit{Nótaí Aistriúcháin:}
\begin{itemize}
	\item Téarma díreach ar fáil le brí chomhchosúil.
\end{itemize}


\subsubsection*{to freeze (briathar): buanaigh}
 \noindent \textit{Sainmhíniú (ga):} I gcomhthéacs líonraí néaracha, cuid de na paraiméadair i líonra néarach a dhéanamh buan, gan ligean dóibh athrú le linn traenála.
\\
 \noindent \textit{Sainmhíniú (en):} In the context of neural networks, to make some of the parameters in the neural network constant, not allowing them to change during training.
\\
 \noindent \textit{Tagairtí:}
\begin{itemize}
	\item buanaigh: De Bhaldraithe (1978) \cite{de-bhaldraithe}, Ó Dónaill et al. (1991) \cite{focloir-beag}, Ó Dónaill (1977) \cite{odonaill}
\end{itemize}

 \noindent \textit{Nótaí Aistriúcháin:}
\begin{itemize}
	\item Luann Foclóir De Bhaldraithe an téarma `buanaigh' i gcomhthéacs praghsanna (.i. `buanaím praghsanna'). Cé nach ionann sin agus paraiméadar i líonra néarach a bhuanú, tá an bhun-bhrí cheannann chéanna i gceist -- uimhir a dhéanamh buan le nach mbeadh sí ag athrú. Glactar leis an téarma seo mar sin.
	\item Is é `buanú' seachas `buanaigh' atá i bhFoclóir Uí Dhónaill agus Uí Mhaoileoin.
	\item Tá `reoigh' ar Téarma.ie, ach ní ghlactar leis sin toisc na fianaise thuas.
\end{itemize}


\subsubsection*{frequency (ainmfhocal): minicíocht}
 \noindent \textit{Sainmhíniú (ga):} I gcomhthéacs graif eolais, cé chomh minic is a bhíonn nód / ceangal mar chuid d'abairtí triaracha sa ngraf.
\\
 \noindent \textit{Sainmhíniú (en):} In the context of a knowledge graph, how often a node / edge is part of triples in the graph.
\\
 \noindent \textit{Tagairtí:}
\begin{itemize}
	\item minicíocht: De Bhaldraithe (1978) \cite{de-bhaldraithe}, Ó Dónaill et al. (1991) \cite{focloir-beag}, Ó Dónaill (1977) \cite{odonaill}, Williams et al. (2023) \cite{storchiste}
\end{itemize}

 \noindent \textit{Nótaí Aistriúcháin:}
\begin{itemize}
	\item Tá an focal `minic' (gan trácht ar `minicíocht') i bhFoclóir Uí Dhuinín.
	\item Luann Foclóir Uí Dhónaill agus Foclóir De Bhaldraithe `minicíocht' mar théarma leictreachais, agus le brí níos leithne (.i. minice).
	\item Luann Stórchiste `minicíocht' mar théarma matamaitice.
\end{itemize}


\subsubsection*{function (ainmfhocal): feidhm}
 \noindent \textit{Sainmhíniú (ga):} Próiseas nó cur chuige ríomhaireachta atá sainmhínithe (m.sh. mar chód).
\\
 \noindent \textit{Sainmhíniú (en):} A computational process or algorithm that can be precisely defined (i.e. in code).
\\
 \noindent \textit{Tagairtí:}
\begin{itemize}
	\item feidhm: De Bhaldraithe (1978) \cite{de-bhaldraithe}, Dineen (1934) \cite{dineen}, Ó Dónaill et al. (1991) \cite{focloir-beag}, Ó Dónaill (1977) \cite{odonaill}, Williams et al. (2023) \cite{storchiste}
\end{itemize}

 \noindent \textit{Nótaí Aistriúcháin:}
\begin{itemize}
	\item Ní i gcomhthéacs matamaiticiúil a luaitear an téarma seo ach amháin i Stórchiste. Cé is moite de sin, is léir go bhfuil úsáid teicniúil leis (.i. `Vital functions, feidhmiú an choirp.' i bhFoclóir De Bhaldraithe).
	\item Is é `feidhm de' a úsáidtear chun trácht ar ionchur feidhmeanna matamaitice, de réir Fhoclóir De Bhaldraithe. (.i. feidhm de X, feidhm d'athróg éigin).
\end{itemize}


\phantomsection \subsection*{G}
\addcontentsline{toc}{subsection}{G}
\markboth{G}{G}

\subsubsection*{generality (ainmfhocal): ginearáltacht}
 \noindent \textit{Sainmhíniú (ga):} I gcomhthéacs matamaitice, cé chomh ginearálta is atá cothromóid nó slonn.
\\
 \noindent \textit{Sainmhíniú (en):} In the context of mathematics, how general an equation or statement is.
\\
 \noindent \textit{Tagairtí:}
\begin{itemize}
	\item ginearáltacht: De Bhaldraithe (1978) \cite{de-bhaldraithe}, Ó Dónaill et al. (1991) \cite{focloir-beag}, Ó Dónaill (1977) \cite{odonaill}
\end{itemize}

 \noindent \textit{Nótaí Aistriúcháin:}
\begin{itemize}
	\item Téarma díreach ar fáil le brí chomhchosúil ó na foclóirí thuas.
	\item Is minic a úsáidtear an téarma seo i bhfrása. Seo a leanas cúpla sampla dá úsáid: `without loss of generality, assume...' $\rightarrow$ `gan ginearáltacht a chailleadh, abair go...', `loss of generality' (mar choincheap) $\rightarrow$ `caillteanas ginearáltachta'
\end{itemize}


\subsubsection*{generative (aidiacht): cumadóireachta}
 \noindent \textit{Sainmhíniú (ga):} I gcomhthéacs intleachta saorga, in ann aschur casta (m.sh. téacs fada, íomhánna, srl), a bhfuil cosúlacht éigin aige le healaín nó saothar duine, a chruthú.
\\
 \noindent \textit{Sainmhíniú (en):} In the context of artificial intelligence, able to create complex output (such as long text or images) that has some resemblance to human works or art.
\\
 \noindent \textit{Tagairtí:}
\begin{itemize}
	\item cumadóireacht: De Bhaldraithe (1978) \cite{de-bhaldraithe}, Dineen (1934) \cite{dineen}, Ó Dónaill et al. (1991) \cite{focloir-beag}, Ó Dónaill (1977) \cite{odonaill}
\end{itemize}

 \noindent \textit{Nótaí Aistriúcháin:}
\begin{itemize}
	\item Tá giniúnach ar Téarma.ie os a chomhair seo, ach ní léir ó Fhoclóir Uí Dhónaill, Uí Dhónaill agus Uí Mhaoileoin, Uí Dhuinín, ná De Bhaldraithe go bhfuil bunús leis sin. De réir na bhfoclóirí sin, bíonn `giniúint' níos cosúla le giniúint páistí ná le hiarracht ar chruthú ealaíne / téacs. Ach luann siad uilig `cumadóireacht' mar théarma a bhfuil an bhrí sin go díreach leis, agus glactar leis sin mar sin.
	\item Is focal sa tuiseal ginideach é seo; an fréamh atá leis ná `cumadóireacht'.
	\item Cé go bhfuil cathú ann `cruthaitheach' nó `cruthaíocht' a úsáid (as an bhfréamh `cruthú'), tá ciall ar leith acu siúd cheana nach n-oireann don téarma seo.
\end{itemize}


\subsubsection*{genetic algorithm (ainmfhocal): algartam géiniteach}
 \noindent \textit{Sainmhíniú (ga):} I gcomhthéacs ríomheolaíochta, algartam ríomhfhoghlama bunaithe ar oibrithe géineolaíochta.
\\
 \noindent \textit{Sainmhíniú (en):} In the context of computer science, a machine learning algorithm based on genetic operations.
\\
 \noindent \textit{Tagairtí:}
\begin{itemize}
	\item algartam: féach ar an téarma `algorithm / algartam'
	\item géiniteach: De Bhaldraithe (1978) \cite{de-bhaldraithe}, Ó Dónaill (1977) \cite{odonaill}
\end{itemize}

 \noindent \textit{Nótaí Aistriúcháin:}
\begin{itemize}
	\item Téarma cruthaithe go díreach as an dhá fhocal thuas.
	\item Féach chomh maith ar an téarma `algorithm / algartam'.
\end{itemize}


\subsubsection*{global (aidiacht): uilíoch}
 \noindent \textit{Sainmhíniú (ga):} I gcomhthéacs graif, sonraí, samhla, nó eile, ag trácht ar airíonna an graif / na sonraí / na samhla mar aonad amháin ar leibhéal leathan. Frithchiallach leis an téarma `logánta'.
\\
 \noindent \textit{Sainmhíniú (en):} In the context of a graph, data, a model, etc, relating to features of the graph / data / model as a whole at a very broad level. Antonym to the term `local'.
\\
 \noindent \textit{Tagairtí:}
\begin{itemize}
	\item uilíoch: De Bhaldraithe (1978) \cite{de-bhaldraithe}, Dineen (1934) \cite{dineen}, Ó Dónaill et al. (1991) \cite{focloir-beag}, Ó Dónaill (1977) \cite{odonaill}
\end{itemize}

 \noindent \textit{Nótaí Aistriúcháin:}
\begin{itemize}
	\item Téarma díreach ar fáil leis an mbrí chéanna ó na foclóirí thuas.
	\item Níor cheart `domhanda' a úsáid, toisc go mbíonn sé sin úsáidte chun trácht a dhéanamh ar an domhan.
\end{itemize}


\subsubsection*{global maximum (ainmfhocal): uasluach uilíoch}
 \noindent \textit{Sainmhíniú (ga):} I gcomhthéacs feabhsaithe nó ríomhfhoghlama, an luach is airde ar féidir é a fháil riamh.
\\
 \noindent \textit{Sainmhíniú (en):} In the context of optimisation or machine learning, the highest value that can ever be obtained.
\\
 \noindent \textit{Tagairtí:}
\begin{itemize}
	\item uaslauch: féach ar an téarma `maximum / uaslauch'
	\item uilíoch: féach ar an téarma `global / uilíoch'
\end{itemize}

 \noindent \textit{Nótaí Aistriúcháin:}
\begin{itemize}
	\item Féach ar an téarma `maximum / uaslauch'
	\item Féach chomh maith ar an téarma `global / uilíoch'
\end{itemize}


\subsubsection*{global minimum (ainmfhocal): íosluach uilíoch}
 \noindent \textit{Sainmhíniú (ga):} I gcomhthéacs feabhsaithe nó ríomhfhoghlama, an luach is  ísle ar féidir é a fháil riamh.
\\
 \noindent \textit{Sainmhíniú (en):} In the context of optimisation or machine learning, the lowest value that can ever be obtained.
\\
 \noindent \textit{Tagairtí:}
\begin{itemize}
	\item íosluach: féach ar an téarma `minimum / íosluach'
	\item uilíoch: féach ar an téarma `global / uilíoch'
\end{itemize}

 \noindent \textit{Nótaí Aistriúcháin:}
\begin{itemize}
	\item Féach ar an téarma `minimum / íosluach'
	\item Féach chomh maith ar an téarma `global / uilíoch'
\end{itemize}


\subsubsection*{graph (ainmfhocal): graf}
 \noindent \textit{Sainmhíniú (ga):} Struchtúr sonraí ina shamhlaítear sonraí mar nóid agus mar cheangail eatarthu.
\\
 \noindent \textit{Sainmhíniú (en):} A data structure in which data is modelled as nodes and the connections between them.
\\
 \noindent \textit{Tagairtí:}
\begin{itemize}
	\item graf: De Bhaldraithe (1978) \cite{de-bhaldraithe}, Ó Dónaill et al. (1991) \cite{focloir-beag}*, Ó Dónaill (1977) \cite{odonaill}
\end{itemize}

 \noindent \textit{Nótaí Aistriúcháin:}
\begin{itemize}
	\item * Cé go bhfuil `graf' istigh sa bhFoclóir Beag, is i gcomhthéacs graif líne amháin a luaitear é.
	\item Focal díreach ar fáil ó na foclóirí thuas.
\end{itemize}


\subsubsection*{graph foundation model (ainmfhocal): samhail fhorais ghraf}
 \noindent \textit{Sainmhíniú (ga):} Samhail fhorais a chruthaítear chun taisc éagsúla ar ghraif a chur i gcrích.
\\
 \noindent \textit{Sainmhíniú (en):} A foundation model built to solve diverse tasks on graphs.
\\
 \noindent \textit{Tagairtí:}
\begin{itemize}
	\item samhail: féach ar an téarma `model / samhail'
	\item foras: féach ar an téarma `foundation model / samhail fhorais'
	\item graf: féach ar an téarma `graph / graf'
\end{itemize}

 \noindent \textit{Nótaí Aistriúcháin:}
\begin{itemize}
	\item Cuirtear `graf' sa tuiseal ginideach iolra toisc gurb é an príomh-phointe atá le samhlacha forais graf ná go n-oibríonn siad go forleathan roinnt mhaith graf ar leith.
	\item Féach chomh maith ar an téarma `model / samhail'
	\item Féach chomh maith ar an téarma `foundation model / samhail fhorais'.
	\item féach chomh maith ar an téarma `graph / graf'
\end{itemize}


\subsubsection*{graph neural network (GNN) (ainmfhocal): líonra néarach graif (LNG)}
 \noindent \textit{Sainmhíniú (ga):} Líonra néarach atá in ann eolas ó ghraif a fhoghlaim, agus taisc ar graif a chur i gcrích (m.sh. aicmiú nód nó réamhinsint nasc), go háirithe nuair atá an líonra néarach sin bunaithe ar chomhéadan sheachadta teachtaireachtaí.
\\
 \noindent \textit{Sainmhíniú (en):} A neural network that is able to learn graph-based information and solve graph-based tasks (such as node classification or link prediction), especially when that neural network is based on a message-passing interface.
\\
 \noindent \textit{Tagairtí:}
\begin{itemize}
	\item líonra néarach: féach ar an téarma `neural network (NN) / líonra néarach (LN)'
	\item graf: féach ar an téarma `graph / graf'
\end{itemize}

 \noindent \textit{Nótaí Aistriúcháin:}
\begin{itemize}
	\item Féach chomh maith ar an téarma `neural network (NN) / líonra néarach (LN)'.
\end{itemize}


\subsubsection*{grid search (ainmfhocal): cuardach ar eangach}
 \noindent \textit{Sainmhíniú (ga):} I gcomhthéacs cuardaithe hipear-pharaiméadar, cuardach a dhéantar ar eangach hipear-pharaiméadar agus ina mbíonn gach uile theaglaim hipear-pharaiméadar measúnaithe.
\\
 \noindent \textit{Sainmhíniú (en):} In the context of a hyperparameter search, a search that is done on a grid of hyperparameters and in which every possible combination of hyperparameters is evaluated.
\\
 \noindent \textit{Tagairtí:}
\begin{itemize}
	\item cuardach: féach ar an téarma `hyperparameter search / cuardach hipear-pharaiméadar'
	\item eangach: De Bhaldraithe (1978) \cite{de-bhaldraithe}, Ó Dónaill (1977) \cite{odonaill}
\end{itemize}

 \noindent \textit{Nótaí Aistriúcháin:}
\begin{itemize}
	\item Cé go bhfuil an focal `eangach' i bhFoclóir Uí Dhónaill agus Uí Mhaoileoin agus i bhFoclóir Uí Dhuinín, is le brí neamhchosúil atá sé luaite iontu.
	\item Is é `cuardach ar eangach' a roghnaíodh, seachas `cuardach eangaí', toisc nach í an eangach atá á cuardach, ach an teaglaim hipear-pharaiméadar is fearr dá bhfuil inti.
	\item Féach chomh maith ar an téarma `hyperparameter search / cuardach hipear-pharaiméadar'.
\end{itemize}


\subsubsection*{ground truth (aidiacht): bun-fhírinneach}
 \noindent \textit{Sainmhíniú (ga):} I gcomhthéacs sonraí a úsáidtear i gcomhair ríomhfhoghlama, fíor agus ar fáil lena bheith úsáidte le linn traenála, teisteála, nó deimhnithe mar shampla den aschur ceart.
\\
 \noindent \textit{Sainmhíniú (en):} In the context of data for machine learning, true and available to be used during training, testing, or validation as an example of correct output.
\\
 \noindent \textit{Tagairtí:}
\begin{itemize}
	\item bun-: De Bhaldraithe (1978) \cite{de-bhaldraithe}, Dineen (1934) \cite{dineen}, Ó Dónaill et al. (1991) \cite{focloir-beag}, Ó Dónaill (1977) \cite{odonaill}
	\item fírinneach: De Bhaldraithe (1978) \cite{de-bhaldraithe}, Dineen (1934) \cite{dineen}, Ó Dónaill et al. (1991) \cite{focloir-beag}, Ó Dónaill (1977) \cite{odonaill}
\end{itemize}

 \noindent \textit{Nótaí Aistriúcháin:}
\begin{itemize}
	\item Is minic agus `bun-' úsáidte mar réimír ar ainmfhocail. Sin ráite, is léir ó Fhoclóir Uí Dhónaill gur féidir é a úsáid mar réimír ar aidiachtaí chomh maith -- féach ar `bun-ard' mar shampla.
	\item Is minic gur féidir `fírinneach' a úsáid in áit `bun-fírinneach' -- óir bíonn an bhun-fhírinne mar chuid den fhírinne i gcónaí, cé nach mbíonn an fhírinne ar fad mar chuid den bhun-fhírinne i gcónaí.
	\item Féach chomh maith ar an téarma `ground truth (data) / bun-fhírinne'.
\end{itemize}


\subsubsection*{ground truth (data) (ainmfhocal): bun-fhírinne}
 \noindent \textit{Sainmhíniú (ga):} I gcomhthéacs sonraí a úsáidtear i gcomhair ríomhfhoghlama, sonraí atá fíor agus ar fáil lena bheith úsáidte le linn traenála, teisteála, nó deimhnithe mar shamplaí den aschur ceart.
\\
 \noindent \textit{Sainmhíniú (en):} In the context of data for machine learning, data that is true and available to be used during training, testing, or validation as an example of correct output.
\\
 \noindent \textit{Tagairtí:}
\begin{itemize}
	\item bun-: De Bhaldraithe (1978) \cite{de-bhaldraithe}, Dineen (1934) \cite{dineen}, Ó Dónaill et al. (1991) \cite{focloir-beag}, Ó Dónaill (1977) \cite{odonaill}
	\item fírinne: De Bhaldraithe (1978) \cite{de-bhaldraithe}, Dineen (1934) \cite{dineen}, Ó Dónaill et al. (1991) \cite{focloir-beag}, Ó Dónaill (1977) \cite{odonaill}
\end{itemize}

 \noindent \textit{Nótaí Aistriúcháin:}
\begin{itemize}
	\item Déanann `bun-fhírinne' trácht ar an gcoincheap teibí agus ar na sonraí iad féin (.i. an bhun-fhírinne). 
	\item Ní hionann bun-fhírinne agus fírinne. Is féidir go bhfuil fírinne ann (mar shampla, trí dhéaduchtú loighce) nach bhfuil mar chuid den bhun-fhírinne. Is mar sin nach leor `fírinne' amháin chun trácht a dhéanamh air seo.
	\item Sin ráite, is minic gur féidir `fírinne' a úsáid in áit `bun-fhírinne' -- óir bíonn an bhun-fhírinne mar chuid den fhírinne i gcónaí, cé nach mbíonn an fhírinne ar fad mar chuid den bhun-fhírinne i gcónaí.
	\item Féach chomh maith ar an téarma `ground truth / bun-fhírinneach'.
\end{itemize}


\phantomsection \subsection*{H}
\addcontentsline{toc}{subsection}{H}
\markboth{H}{H}

\subsubsection*{histogram (ainmfhocal): histeagram}
 \noindent \textit{Sainmhíniú (ga):} Breacadh a dhéanann cur síos ar dháileadh staitistiúil le colúin mhinicíochta i gcomhair chuile eatramh luachanna sa dáileadh.
\\
 \noindent \textit{Sainmhíniú (en):} A plot that summarises a statistical distribution using frequency columns for every range of values in the distribution.
\\
 \noindent \textit{Tagairtí:}
\begin{itemize}
	\item histeagram: Ó Dónaill (1977) \cite{odonaill}, Williams et al. (2023) \cite{storchiste}
\end{itemize}

 \noindent \textit{Nótaí Aistriúcháin:}
\begin{itemize}
	\item Téarma díreach ar fáil ó Fhoclóir Uí Dhónaill agus ó Stórchiste.
\end{itemize}


\subsubsection*{hits@k (H@k) (ainmfhocal): fíricí@k (F@k)}
 \noindent \textit{Sainmhíniú (ga):} I gcomhthéacs liosta ranganna, comhréir na ranganna atá comhionann le k, nó níos ísle ná é.
\\
 \noindent \textit{Sainmhíniú (en):} In the context of a ranked list, the proportion of ranks that are equal to, or lesser than, k.
\\
 \noindent \textit{Tagairtí:}
\begin{itemize}
	\item fíric: Ó Dónaill et al. (1991) \cite{focloir-beag}, Ó Dónaill (1977) \cite{odonaill}
\end{itemize}

 \noindent \textit{Nótaí Aistriúcháin:}
\begin{itemize}
	\item Ní léir ó na foinsí dúchasacha go mbeadh ciall ar bith le `hits' a aistriú mar `buailtí' nó mar sin. Ina ainneoin sin, aistrítear brí an téarma -- gurb é `hits@k' ná cá mhéad ranganna atá ann comhionann le, nó níos ísle ná, k. Is ionann rang i liosta ranganna agus rang a tugadh don fhreagra ceart ar cheist réamhinsinte nasc. Toisc go bhfuil gach uile fhreagra ceart ina fhíric, is ionann rang an fhreagra agus innéacs fírice. Mar sin, is ionann `hits@k' agus líon na bhfíricí a bhfuil rang comhionann le, nó níos ísle ná, k acu. Is as sin a thagann an téarma seo go díreach.
	\item Féach chomh maith ar an téarma `ranked list / liosta ranganna'.
\end{itemize}


\subsubsection*{hop (ainmfhocal): coiscéim}
 \noindent \textit{Sainmhíniú (ga):} I gcomhthéacs graf, fad ceangail amháin idir dhá nód.
\\
 \noindent \textit{Sainmhíniú (en):} In the context of graphs, the distance of a single edge between two nodes.
\\
 \noindent \textit{Tagairtí:}
\begin{itemize}
	\item coiscéim: De Bhaldraithe (1978) \cite{de-bhaldraithe}, Dineen (1934) \cite{dineen}, Ó Dónaill et al. (1991) \cite{focloir-beag}, Ó Dónaill (1977) \cite{odonaill}, Williams et al. (2023) \cite{storchiste}
\end{itemize}

 \noindent \textit{Nótaí Aistriúcháin:}
\begin{itemize}
	\item Cé go mbeadh ciall le húsáid an fhocail `céim' anseo, úsáidtear an focal sin cheana chun cur síos a dhéanamh ar `degree' nóid. Ní mheastar go mba cheart `céim' a ath-úsáid anseo toisc go mbeadh sé an-deacair débhrí a sheachaint -- bíonn `hop' agus `céim' á n-úsáid go minic sna comhthéacsanna céanna. Mar shampla, uaireanta sa litríocht déantar fo-ghraf de ghraf eolais ina mbíonn ar a laghad céim éigin ag gach uile nód. Uaireanta eile, déantar fo-ghraf de ghraf eolais bunaithe ar gach uile nód atá laistigh de mhéid éigin coiscéimeanna ó nód amháin. Dá mbeadh an focal `céim' aistrithe mar `degree' agus mar `hop', d'úsáidfí an téarma `fo-ghraf n-chéime' don dá fho-graf sin -- cé gur fo-ghraif dhifriúla iad.
	\item Úsáidtear an téarma `siúlóid' chun cur síos a dhéanamh ar `walk' ar graf. Agus is minic a dhéantar fad na siúlóidí a chomhaireamh de réir coiscéimeanna; m.sh `a 3-hop walk / siúlóid 3-choiscéim'. Thairis sin, luann Foclóir De Bhaldraithe an frása `within a step of the house, faoi choiscéim den teach' mar úsáid féideartha leis an bhfocal `coiscéim'. Tá an úsáid sin díreach cosúil le húsáid an fhocail `coiscéim' chun trácht a dhéanamh ar fhad i ngraf -- an t-aon difríocht ná nach bhfuil fad i ngraf fisiceach. Cé is moite de sin, meastar go bhfuil `coiscéim' léir mar théarma (go háirithe ar an analach sin a ghlacadh), agus glactar leis mar sin.
	\item Féach chomh maith ar an téarma `degree / céim'.
	\item Féach chomh maith ar an téarma `walk / siúlóid'.
\end{itemize}


\subsubsection*{hyper-graph (ainmfhocal): hipear-ghraf}
 \noindent \textit{Sainmhíniú (ga):} I gcomhthéacs graf, graf inar féidir le ceangail 3+ nód a cheangal san am céanna, seachas dhá nód amháin.
\\
 \noindent \textit{Sainmhíniú (en):} In the context of graphs, a graph in which edges can connect 3+ nodes at once, instead of only two nodes.
\\
 \noindent \textit{Tagairtí:}
\begin{itemize}
	\item hipear-: Ó Dónaill (1977) \cite{odonaill}
	\item graf: féach ar an téarma `graph / graf'
\end{itemize}

 \noindent \textit{Nótaí Aistriúcháin:}
\begin{itemize}
	\item Téarma cruthaithe go díreach as na fréamhacha thuas.
	\item Féach chomh maith ar an téarma `graph / graf'.
\end{itemize}


\subsubsection*{hyperparameter (ainmfhocal): hipear-pharaiméadar}
 \noindent \textit{Sainmhíniú (ga):} paraiméadar (nach bhfuil fhoghlamtha) nó socrú atá úsáidte chun algartam samhla foghlama a rith.
\\
 \noindent \textit{Sainmhíniú (en):} a (non-learnable) parameter or setting that is used to run a machine learning algorithm.
\\
 \noindent \textit{Tagairtí:}
\begin{itemize}
	\item hipear-: Ó Dónaill (1977) \cite{odonaill}
	\item pharaiméadar: féach ar an téarma `parameter / pharaiméadar'
\end{itemize}

 \noindent \textit{Nótaí Aistriúcháin:}
\begin{itemize}
	\item Níl an téarma iomlán `hipear-pharaiméadar' ar fáil i bhfoclóir ar bith. Sin ráite, tá idir `hipear-' (mar réimír) agus `paraiméadar' (mar ainmfhocal) i bhFoclóir Uí Dhónaill, rud a spreagann an téarma seo go díreach.
	\item Is é hipearpharaiméadar a bhíonn ar Téarma.ie chomh maith.
\end{itemize}


\subsubsection*{hyperparameter combination (ainmfhocal): teaglaim hipear-pharaiméadar}
 \noindent \textit{Sainmhíniú (ga):} Tacar amháin (as iliomad tacar eile) luachanna hipear-pharaiméadar ar féidir iad a úsáid chun samhail ríomhfhoghlama shainmhíniú.
\\
 \noindent \textit{Sainmhíniú (en):} A set (out of many possible sets) of all hyperparameter values needed to fully define a machine learning model.
\\
 \noindent \textit{Tagairtí:}
\begin{itemize}
	\item teaglaim: De Bhaldraithe (1978) \cite{de-bhaldraithe}, Ó Dónaill et al. (1991) \cite{focloir-beag}, Ó Dónaill (1977) \cite{odonaill}
	\item hipear-pharaiméadar: féach ar an téarma `hyperparameter / hipear-pharaiméadar'
\end{itemize}

 \noindent \textit{Nótaí Aistriúcháin:}
\begin{itemize}
	\item Luann Foclóir Uí Dhónaill `teaglaim' mar théarma matamaitice -- is cosúil gurb in atá i gceist acu leis sin ná matamaitic theaglamach (.i. teaglamaí agus iomalartuithe). Sin ráite, luaitear `teaglaim' leis an mbrí `collection, gathering, compilation' chomh maith (féach ar Fhoclóir Uí Dhónaill, mar shampla). Luíonn an dá bhrí seo lena bhfuil i gceist. Is minic agus teaglaim hipear-pharaiméadar déanta trí sampla a thógáil ó eangach hipear-pharaiméadar -- sa gcaoi sin, is teaglaim litriúil (i gcomhthéacs matamaitice teaglamaí) í. Agus is cnuasach hipear-pharaiméadar í chomh maith -- brí a luíonn le húsáid an fhocail teaglaim chun cur síos a dhéanamh ar `collection, gathering, compilation'. Glactar leis an téarma seo mar sin.
	\item Féach chomh maith ar an téarma `hyperparameter / hipear-pharaiméadar'.
\end{itemize}


\subsubsection*{hyperparameter grid (ainmfhocal): eangach hipear-pharaiméadar}
 \noindent \textit{Sainmhíniú (ga):} Tacar teaglamaí hipear-pharaiméadar a úsáidtear chun cuardach hipear-pharaiméadar a chur i gcrích.
\\
 \noindent \textit{Sainmhíniú (en):} A set of hyperparameter combinations that is used for hyperparameter searches.
\\
 \noindent \textit{Tagairtí:}
\begin{itemize}
	\item eangach: féach ar an téarma `grid search / cuardach ar eangach'
	\item hipear-pharaiméadar: féach ar an téarma `hyperparameter / hipear-pharaiméadar'
\end{itemize}

 \noindent \textit{Nótaí Aistriúcháin:}
\begin{itemize}
	\item Bíonn níos mó ná hipear-pharaiméadar amháin i ngach uile eangach hipear-pharaiméadar. Cuirtear an focal `hipear-pharaiméadar' sa tuiseal ginideach iolra mar sin.
	\item Féach chomh maith ar an téarma `grid search / cuardach ar eangach'.
	\item Féach chomh maith ar an téarma `hyperparameter / hipear-pharaiméadar'.
\end{itemize}


\subsubsection*{hyperparameter search (ainmfhocal): cuardach hipear-pharaiméadar}
 \noindent \textit{Sainmhíniú (ga):} an cur chuige a úsáidtear chun na hipear-pharaiméadair is fearr a fháil do shamhail ríomhfhoghlama.
\\
 \noindent \textit{Sainmhíniú (en):} the approach used to find the optimal hyperparameters for a machine learning model.
\\
 \noindent \textit{Tagairtí:}
\begin{itemize}
	\item cuardach: De Bhaldraithe (1978) \cite{de-bhaldraithe}, Dineen (1934) \cite{dineen}, Ó Dónaill et al. (1991) \cite{focloir-beag}, Ó Dónaill (1977) \cite{odonaill}
	\item hipear-pharaiméadar: féach ar an téarma `hyperparameter / hipear-pharaiméadar'
\end{itemize}

 \noindent \textit{Nótaí Aistriúcháin:}
\begin{itemize}
	\item Cuirtear `hipear-pharaiméadar' sa nginideach iolra toisc go mbíonn cuardach déanta, den chuid is mó, chun níos mó ná hipear-pharaiméadar amháin a fháil.
	\item Féach chomh maith ar an téarma `hyperparameter / hipear-pharaiméadar'.
\end{itemize}


\phantomsection \subsection*{I}
\addcontentsline{toc}{subsection}{I}
\markboth{I}{I}

\subsubsection*{implementation (ainmfhocal): leagan infheidhmithe}
 \noindent \textit{Sainmhíniú (ga):} I gcomhthéacs ríomheolaíochta, cód infheidhmithe de shamhail ríomhfhoghlama nó d'algartam eile i dteanga ríomheolaíochta.
\\
 \noindent \textit{Sainmhíniú (en):} In the context of computer science, executable code of a machine learning model or other algorithm in a coding language.
\\
 \noindent \textit{Tagairtí:}
\begin{itemize}
	\item leagan: De Bhaldraithe (1978) \cite{de-bhaldraithe}, Dineen (1934) \cite{dineen}, Ó Dónaill et al. (1991) \cite{focloir-beag}, Ó Dónaill (1977) \cite{odonaill}
	\item feidhmigh: De Bhaldraithe (1978) \cite{de-bhaldraithe}, Ó Dónaill et al. (1991) \cite{focloir-beag}, Ó Dónaill (1977) \cite{odonaill}
\end{itemize}

 \noindent \textit{Nótaí Aistriúcháin:}
\begin{itemize}
	\item Is féidir an-chuid leaganacha ar leith a chruthú de gach uile shamhail ríomhfhoghlama (i dteangacha ríomheolaíochta ar leith, mar shampla). Ní bhíonn leagan ar bith `níos cirte' ná leagan eile -- déanann siad an rud céanna. Is mar seo a chinneadh an focal `leagan' a úsáid -- tugann sé le fios gur féidir leaganacha eile, atá ar chomhchéim lena chéile, a chruthú.
	\item Ar an leibhéal is bunúsaí, bíonn feidhm le gach uile `implementation'. Ní `implementation' gan bheith in ann é a chur ar siúl ar ríomhaire. Mar sin, baineann feidhmiú mar choincheap go díreach le `implementation' mar théarma.
\end{itemize}


\subsubsection*{information content (ainmfhocal): lucht faisnéise}
 \noindent \textit{Sainmhíniú (ga):} An t-eolas ar fad atá istigh i mbunachar nó tacar sonraí mar choincheap teibí nó mar réad neamhspleách ón gcaoi ina bhfuil na sonraí stóráilte / samhlaithe.
\\
 \noindent \textit{Sainmhíniú (en):} All of the information contained in a database or dataset as an abstract concept or as an object independent of the way in which the data is stored or modelled.
\\
 \noindent \textit{Tagairtí:}
\begin{itemize}
	\item lucht: De Bhaldraithe (1978) \cite{de-bhaldraithe}, Dineen (1934) \cite{dineen}, Ó Dónaill et al. (1991) \cite{focloir-beag}, Ó Dónaill (1977) \cite{odonaill}
	\item faisnéis: De Bhaldraithe (1978) \cite{de-bhaldraithe}, Dineen (1934) \cite{dineen}, Ó Dónaill et al. (1991) \cite{focloir-beag}, Ó Dónaill (1977) \cite{odonaill}
\end{itemize}

 \noindent \textit{Nótaí Aistriúcháin:}
\begin{itemize}
	\item De réir Fhoclóir Uí Dhónaill, is ionann `lucht' agus (i measc bríonna eile) ` Content, charge; fill, capacity; cargo, load.' Ina measc sin airítear lucht leictreach agus lucht loinge, rud a léiríonn go bhfuil úsáid leathan go leor aige mar fhocal. Ní fheictear téarma ar bith eile ann a bheadh ní ba oiriúnaí. Glactar le `lucht' mar sin mar leagan de `content' an Bhéarla sa gcomhthéacs seo.
	\item Ní hionann lucht faisnéise agus ábhar na faisnéise -- is coincheap staitistice é lucht faisnéise.
\end{itemize}


\subsubsection*{input (ainmfhocal): ionchur}
 \noindent \textit{Sainmhíniú (ga):} I gcomhthéacs córais, próisis, nó feidhme, sonraí a chuirtear isteach lena bheith úsáidte chun sprioc éigin a bhaint amach (m.sh. áireamh luach éigin).
\\
 \noindent \textit{Sainmhíniú (en):} In the context of a system, process, or function, data that is provided to be used to achieve an end (such as the calculation of a certain value).
\\
 \noindent \textit{Tagairtí:}
\begin{itemize}
	\item ionchur: De Bhaldraithe (1978) \cite{de-bhaldraithe}, Ó Dónaill (1977) \cite{odonaill}
\end{itemize}

 \noindent \textit{Nótaí Aistriúcháin:}
\begin{itemize}
	\item Luann Foclóir De Bhaldraithe `ionchur' mar théarma teileachumarsáide -- sin le rá, i gcomhthéacs an-chosúil leis an gcomhthéacs seo. Sin ráite, is cosúil go bhfuil an téarma `ionchur' (i gcomhthéacs teileachumarsáide) ag trácht ar ionchur mar choincheap, seachas mar shonraí nó mar réad ríomhaireachta. Mar sin, is dócha gur cirte a rá `tá dhá uimhir mar ionchur ag an bhfeidhm' nó `tá dhá uimhir ionchuir ag an bhfeidhm' seachas `* tá dhá ionchur ag an bhfeidhm'.
	\item Ní léir ó na Foclóirí thuas an féidir briathar a dhéanamh as seo (.i. *ionchuir). Mar sin, moltar frása le `ionchur' a úsáid nuair atá briathar de dhíth; m.sh. `tógann an fheidhm dhá uimhir isteach (mar ionchur)'.
	\item Féach chomh maith ar an téarma `to input (into) / cuir isteach (i)'.
\end{itemize}


\subsubsection*{to input (into) (ainmfhocal): cuir isteach (i)}
 \noindent \textit{Sainmhíniú (ga):} I gcomhthéacs córais, próisis, nó feidhme, sonraí ionchuir a tabhairt dó.
\\
 \noindent \textit{Sainmhíniú (en):} In the context of a system, process, or function, to give it input data.
\\
 \noindent \textit{Tagairtí:}
\begin{itemize}
	\item cuir isteach: De Bhaldraithe (1978) \cite{de-bhaldraithe}, Ó Dónaill (1977) \cite{odonaill}
\end{itemize}

 \noindent \textit{Nótaí Aistriúcháin:}
\begin{itemize}
	\item Téarma a fáil le brí chomhchosúil ó na foclóirí thuas.
	\item Mar shampla, is féidir `cuireadh X isteach sa bhfeidhm' a úsáid chun `X was input into the function' a chur in iúl.
	\item Tá an frása `cuir isteach' ar fad le fáil i bhFoclóir Uí Dhónaill agus i bhFoclóir De Bhaldraithe i gcomhthéacs comhchosúil (ach ní i gcomhthéacs ríomheolaíochta, cinnte).
	\item Féach chomh maith ar an téarma `input / ionchur'.
\end{itemize}


\subsubsection*{to instantiate (briathar): cruthaigh}
 \noindent \textit{Sainmhíniú (ga):} Cruth a chur ar réad (matamaiticiúil nó ríomhaireachta), go háirithe de réir creatlaí sainmhínithe éigin.
\\
 \noindent \textit{Sainmhíniú (en):} To create a (mathematical or computational) object, especially according to a strictly defined framework.
\\
 \noindent \textit{Tagairtí:}
\begin{itemize}
	\item cruthaigh: De Bhaldraithe (1978) \cite{de-bhaldraithe}, Dineen (1934) \cite{dineen}, Ó Dónaill (1977) \cite{odonaill}
\end{itemize}

 \noindent \textit{Nótaí Aistriúcháin:}
\begin{itemize}
	\item Den chuid is mó, is ionann `instantiation' (mar phróiseas) agus rud a chruthú. Úsáidtear `cruthaigh' mar sin.
	\item Toisc gur coincheap ginearálta agus leathan atá i gceist leis an téarma seo, tá sé iomlán ceart go leor focail agus frásaí comhchiallacha a úsáid ina ionad seo, m.sh. cum, cruth a chur ar, srl. ag braith ar an gcomhthéacs.
	\item Tá `áscaigh', as an  bhfocal `ásc' ar Téarma.ie -- focal nach bhfuil bunús ar bith leis i bhFoclóir Uí Dhónaill, De Bhaldraithe, ná Uí Dhuinín. Ní ghlactar le `áscaigh' mar sin.
\end{itemize}


\subsubsection*{instantiation (concept) (ainmfhocal): leagan}
 \noindent \textit{Sainmhíniú (ga):} Sampla de choincheap nó de chreatlach, nach ionann agus réad matamaiticiúil / ríomhaireachta.
\\
 \noindent \textit{Sainmhíniú (en):} An instance or example of a concept or framework, other than a mathematical / computational object.
\\
 \noindent \textit{Tagairtí:}
\begin{itemize}
	\item leagan: De Bhaldraithe (1978) \cite{de-bhaldraithe}, Dineen (1934) \cite{dineen}, Ó Dónaill et al. (1991) \cite{focloir-beag}, Ó Dónaill (1977) \cite{odonaill}
\end{itemize}

 \noindent \textit{Nótaí Aistriúcháin:}
\begin{itemize}
	\item Ní i gcomhthéacs eolaíochta a luaitear an téarma seo, ach is le brí chomhchosúil atá sé luaite.
	\item Más gá a léiriú gur rud coincréiteach atá i gceist le `leagan', seachas rud teibí, moltar `coincréiteach' a úsáid, i dtaca le Foclóir Uí Dhónaill agus Foclóir De Bhaldraithe.
	\item Tá an t-aistriúchán céanna luaite i dtéarma eile sa bhfoclóir seo. Sin ráite, tá an bhrí a bhaineann le `leagan' i nGaeilge leathan go leor le go bhfuil sé in ann seasamh isteach sa dá chás. Beidh sé léir go leor ón gcomhthéacs cén ceann acu atá i gceist.
	\item Is é `ascú' atá ar Téarma.ie Ní ghlactar leis sin toisc nach bhfuil fiansise dó sna foclóirí dúchasacha. Thairis sin, is cosúil go dtagann sé as an bhfréamh `ásc' atá luaite i bhFoclóir Uí Dhónaill mar fhocal nach mbíonn úsáid leis ach amháin i gcúpla frása áirithe ar leith. Ní mheastar go bhfuil sé  oiriúnach mar théarma mar sin.
	\item Féach chomh maith ar an téarma `representation / leagan'.
\end{itemize}


\subsubsection*{instantiation (object) (ainmfhocal): réad}
 \noindent \textit{Sainmhíniú (ga):} An rud (matamaiticiúil nó ríomhaireachta) atá cruthaithe de réir creatlaí matamaitice / ríomhaireachta.
\\
 \noindent \textit{Sainmhíniú (en):} The (mathematical or computational) object created by the process of instantiation from a mathematical / computational framework.
\\
 \noindent \textit{Tagairtí:}
\begin{itemize}
	\item réad: De Bhaldraithe (1978) \cite{de-bhaldraithe}, Dineen (1934) \cite{dineen}*, Ó Dónaill (1977) \cite{odonaill}, Williams et al. (2023) \cite{storchiste}
\end{itemize}

 \noindent \textit{Nótaí Aistriúcháin:}
\begin{itemize}
	\item * Ní luann Foclóir Uí Dhuinín an téarma `réad' ach i gcomhthéacs filíochta.
	\item Focal luaite i gcomhthéacs comhchosúil na foclóirí eile thuas.
	\item Féach chomh maith ar an téarma `to instantiate / cruthaigh'.
\end{itemize}


\subsubsection*{instantiation (process) (ainmfhocal): cruthú}
 \noindent \textit{Sainmhíniú (ga):} An próiseas a bhaineann le réad (matamaiticiúil nó ríomhaireachta) a chruthú.
\\
 \noindent \textit{Sainmhíniú (en):} The process relating to the instantiation of a (mathematical or computational) object.
\\
 \noindent \textit{Tagairtí:}
\begin{itemize}
	\item cruthaigh: féach ar an téarma `to instantiate / cruthaigh'
\end{itemize}

 \noindent \textit{Nótaí Aistriúcháin:}
\begin{itemize}
	\item Déanann an téarma seo trácht ar an bpróiseas a bhaineann le rud a chruthú; .i. an cruthú féin. Ní féidir an téarma seo a úsáid chun trácht ar an réad atá mar thoradh / aschur ar an bpróiseas céanna.
	\item Féach chomh maith ar an téarma `to instantiate / cruthaigh'.
\end{itemize}


\subsubsection*{interface (ainmfhocal): comhéadan}
 \noindent \textit{Sainmhíniú (ga):} Conradh a dhéantar idir dhá (nó níos mó) cuid de chód, de phróiseas, nó de chóras ríomhfhoghlama a shainmhíníonn cén chaoi a oibríonn siad lena chéile.
\\
 \noindent \textit{Sainmhíniú (en):} A protocol made between two (or more) parts of code, of a process, or of a machine learning system that defines how they interoperate.
\\
 \noindent \textit{Tagairtí:}
\begin{itemize}
	\item comhéadan: Ó Dónaill (1977) \cite{odonaill}
\end{itemize}

 \noindent \textit{Nótaí Aistriúcháin:}
\begin{itemize}
	\item Luann Foclóir Uí Dhónaill `comhéadan' mar théarma geolaíochta. I gcomhthéacs geolaíochta, is ionann comhéadan agus an áit (fhisiceach) a thagann dhá rud le chéile. I gcomhthéacs ríomheolaíochta, sin an coincheap céanna atá taobh thiar de `interface' -- an áit ina thagann dhá chuid de chód nó de phróiseas le chéile.
	\item Tá `comhéadan' ar Téarma.ie (i gcomhthéacs ríomheolaíochta) chomh maith.
\end{itemize}


\subsubsection*{intersection (ainmfhocal): idirmhír}
 \noindent \textit{Sainmhíniú (ga):} I gcomhthéacs dhá thacar (nó níos mó) fo-thacar na mball atá sna tacar sin ar fad.
\\
 \noindent \textit{Sainmhíniú (en):} In the context of two or more sets, the subset of elements that are in all of those sets.
\\
 \noindent \textit{Tagairtí:}
\begin{itemize}
	\item idirmhír: De Bhaldraithe (1978) \cite{de-bhaldraithe}, Ó Dónaill (1977) \cite{odonaill}
\end{itemize}

 \noindent \textit{Nótaí Aistriúcháin:}
\begin{itemize}
	\item Téarma le fáil i bhFoclóir Uí Dhónaill agus i bhFoclóir De Bhaldraithe leis an mbrí cheannann chéanna (.i. i gcomhthéacs tacar) is atá in gceist anseo.
\end{itemize}


\subsubsection*{interval (ainmfhocal): eatramh}
 \noindent \textit{Sainmhíniú (ga):} I gcomhthéacs matamaitice, réimse sainmhínithe luacha.
\\
 \noindent \textit{Sainmhíniú (en):} In the context of mathematics, a specific range of values.
\\
 \noindent \textit{Tagairtí:}
\begin{itemize}
	\item eatramh: De Bhaldraithe (1978) \cite{de-bhaldraithe}, Ó Dónaill et al. (1991) \cite{focloir-beag}, Ó Dónaill (1977) \cite{odonaill}, Williams et al. (2023) \cite{storchiste}
\end{itemize}

 \noindent \textit{Nótaí Aistriúcháin:}
\begin{itemize}
	\item Téarma díreach le fáil leis an mbrí cheannann chéanna i gcomhthéacs matamaiticiúil i Stórchiste. Is i gcomhthéacs comhchosúil, ach níos leithne, a luaitear an téarma seo sna foclóirí eile.
\end{itemize}


\subsubsection*{inverse (aidiacht): inbhéartach}
 \noindent \textit{Sainmhíniú (ga):} I gcomhthéacs faisnéise i ngraf eolais, mar leagan contrártha d'fhaisnéis eile. Mar shampla, má tá f' mar inbhéarta ar f, agus más fíor (a,f,c), is fíor (c,f',a).
\\
 \noindent \textit{Sainmhíniú (en):} In the context of a predicate in a knowledge graph, being a reverse of another predicate. For example, if p' is the inverse of p, and (s,p,o) is true, then (o,p',s) is true.
\\
 \noindent \textit{Tagairtí:}
\begin{itemize}
	\item inbhéartach: De Bhaldraithe (1978) \cite{de-bhaldraithe}, Ó Dónaill (1977) \cite{odonaill}, Williams et al. (2023) \cite{storchiste}
\end{itemize}

 \noindent \textit{Nótaí Aistriúcháin:}
\begin{itemize}
	\item Téarma díreach ar fáil leis an mbrí chéanna i gcomhthéacs matamaitice.
\end{itemize}


\subsubsection*{inverse (relation) (ainmfhocal): inbhéarta}
 \noindent \textit{Sainmhíniú (ga):} I gcomhthéacs faisnéise i ngraf eolais, leagan contrártha d'fhaisnéis eile. Mar shampla, má tá f' mar inbhéarta ar f, agus más fíor (a,f,c), is fíor (c,f',a).
\\
 \noindent \textit{Sainmhíniú (en):} In the context of a predicate in a knowledge graph, the reverse of another predicate. For example, if p' is the inverse of p, and (s,p,o) is true, then (o,p',s) is true.
\\
 \noindent \textit{Tagairtí:}
\begin{itemize}
	\item inbhéarta: De Bhaldraithe (1978) \cite{de-bhaldraithe}, Ó Dónaill (1977) \cite{odonaill}, Williams et al. (2023) \cite{storchiste}
\end{itemize}

 \noindent \textit{Nótaí Aistriúcháin:}
\begin{itemize}
	\item Téarma díreach ar fáil leis an mbrí chéanna i gcomhthéacs matamaitice.
	\item Féach chomh maith ar an téarma `inverse / inbhéartach'.
\end{itemize}


\phantomsection \subsection*{K}
\addcontentsline{toc}{subsection}{K}
\markboth{K}{K}

\subsubsection*{knowledge graph (KG) (ainmfhocal): graf eolais (GE)}
 \noindent \textit{Sainmhíniú (ga):} Bunachar sonraí a shamhlaíonn eolas mar nóid agus na ceangail eatarthu. Bíonn lipéad ar chuile nód / ceangal, agus bíonn chuile cheangal dírithe.
\\
 \noindent \textit{Sainmhíniú (en):} A database consisting of labelled nodes representing concepts and directed, labelled edges describing the relationships between them.
\\
 \noindent \textit{Tagairtí:}
\begin{itemize}
	\item graf: féach ar an téarma `graph / graf'
	\item eolas: De Bhaldraithe (1978) \cite{de-bhaldraithe}, Dineen (1934) \cite{dineen}, Ó Dónaill et al. (1991) \cite{focloir-beag}, Ó Dónaill (1977) \cite{odonaill}
\end{itemize}

 \noindent \textit{Nótaí Aistriúcháin:}
\begin{itemize}
	\item * Cé go bhfuil `graf' istigh sa bhFoclóir Beag, is i gcomhthéacs graif líne amháin a luaitear é.
	\item Níl aistriúchán déanta ar an téarma seo cheana go bhfios don údar (fiú ar Téarma.ie). Cumtar téarma nua mar sin, as `graf' agus as `eolas' mar a cumadh i mBéarla é.
	\item Féach chomh maith ar an téarma `graph / graf'.
\end{itemize}


\subsubsection*{knowledge graph embedding (KGE) (ainmfhocal): leabú graif eolais (LGE)}
 \noindent \textit{Sainmhíniú (ga):} Próiseas ríomhfhoghlama a bhfuil leabú graif eolais amháin mar thoradh air; nó, leabú amháin a fhaightear mar thoradh air sin.
\\
 \noindent \textit{Sainmhíniú (en):} The machine learning process of embedding a single knowledge graph into vector space; or, a single embedding obtained from said process.
\\
 \noindent \textit{Tagairtí:}
\begin{itemize}
	\item leabú: féach ar an téarma `embedding / leabú'
	\item graf eolais: féach ar an téarma `knowledge graph (KG) / graf eolais (GE)'
\end{itemize}

 \noindent \textit{Nótaí Aistriúcháin:}
\begin{itemize}
	\item Fágtar san uimhir uatha an téarma `graf eolais' toisc nach mbíonn ach graf amháin á leabú ag aon uair amháin / ag aon samhail amháin.
	\item Féach chomh maith ar an téarma `embedding / leabú'.
	\item Féach chomh maith ar an téarma `knowledge graph (KG) / graf eolais (GE)'.
\end{itemize}


\subsubsection*{knowledge graph embedding model (KGEM) (ainmfhocal): samhail leabaithe graif eolais (SLGE)}
 \noindent \textit{Sainmhíniú (ga):} Samhail ríomhfhoghlama a bhfuil mar aidhm aige graf eolais a leabú.
\\
 \noindent \textit{Sainmhíniú (en):} A machine learning model whose aim is to embed a knowledge graph.
\\
 \noindent \textit{Tagairtí:}
\begin{itemize}
	\item leabú: féach ar an téarma `embedding / leabú'
	\item graf eolais: féach ar an téarma `knowledge graph (KG) / graf eolais (GE)'
	\item samhail: féach ar an téarma `model / samhail'
\end{itemize}

 \noindent \textit{Nótaí Aistriúcháin:}
\begin{itemize}
	\item Fágtar san uimhir uatha an téarma `leabú' toisc phróiseas an leabaithe a bheith i gceist, seachas líon na leabuithe ar fad.
	\item Fágtar san uimhir uatha an téarma `graf eolais' toisc nach mbíonn ach graf amháin á leabú ag aon uair amháin / ag aon samhail amháin.
	\item Féach chomh maith ar an téarma `knowledge graph embedding (KGE) / leabú graif eolais (LGE)'.
\end{itemize}


\subsubsection*{Kullback-Leibler (KL) divergence (ainmfhocal): dibhéirseacht Kullback-Leibler (KL)}
 \noindent \textit{Sainmhíniú (ga):} I gcomhthéacs ríomheolaíochta agus matamaitice, feidhm a chomhaireann cé chomh difriúil is atá dhá dháileadh staitistiúil óna chéile. Is minic agus é úsáidte mar fheidhm phionóis, agus is ionann é agus eantrópacht choibhneasta.
\\
 \noindent \textit{Sainmhíniú (en):} In the context of computer science and mathematics, a function that calculates how different two distributions are from each other. It is often used as a loss function, and it is also referred to as relative entropy.
\\
 \noindent \textit{Tagairtí:}
\begin{itemize}
	\item dibhéirseacht: De Bhaldraithe (1978) \cite{de-bhaldraithe}, Dineen (1934) \cite{dineen}, Ó Dónaill et al. (1991) \cite{focloir-beag}, Ó Dónaill (1977) \cite{odonaill}, Williams et al. (2023) \cite{storchiste}
\end{itemize}

 \noindent \textit{Nótaí Aistriúcháin:}
\begin{itemize}
	\item Tá `dibhéirseacht' ar fáil go díreach ó na foinsí thuas. I Stórchiste, luaitear mar théarma matamaitice é.
	\item Is ainm dílis (nó go teicniúil, dhá ainm dílis curtha le chéile) é `Kullback-Leibler'; fágtar gan athrú é mar sin.
	\item Is ionann dibhéirseacht Kullback-Leibler agus eantrópacht choibhneasta.
	\item Féach chomh maith ar an téarma `relative entropy / eantrópacht choibhneasta'.
\end{itemize}


\phantomsection \subsection*{L}
\addcontentsline{toc}{subsection}{L}
\markboth{L}{L}

\subsubsection*{label (ainmfhocal): lipéad}
 \noindent \textit{Sainmhíniú (ga):} I gcomhthéacs graif eolais, téacs atá nasctha le nód nó le ceangal agus a chuireann in iúl céard dó a sheasann an nód / ceangal sin.
\\
 \noindent \textit{Sainmhíniú (en):} In the context of a knowledge graph, text that is linked to a node or edge that indicates what that node / edge represents.
\\
 \noindent \textit{Tagairtí:}
\begin{itemize}
	\item lipéad: De Bhaldraithe (1978) \cite{de-bhaldraithe}, Ó Dónaill et al. (1991) \cite{focloir-beag}, Ó Dónaill (1977) \cite{odonaill}
\end{itemize}

 \noindent \textit{Nótaí Aistriúcháin:}
\begin{itemize}
	\item Téarma díreach ar fáil le brí chomhchosúil.
\end{itemize}


\subsubsection*{labelled (aidiacht): le lipéad}
 \noindent \textit{Sainmhíniú (ga):} Ag tagairt ar nód nó ar ceangal, rud a bhfuil lipéad (uimhir, téacs, nó eile) curtha leis mar shuaitheantas.
\\
 \noindent \textit{Sainmhíniú (en):} Referring to a node or edge, having a label (number, text, etc) attached to it as an identifier.
\\
 \noindent \textit{Tagairtí:}
\begin{itemize}
	\item le: De Bhaldraithe (1978) \cite{de-bhaldraithe}, Dineen (1934) \cite{dineen}, Ó Dónaill et al. (1991) \cite{focloir-beag}, Ó Dónaill (1977) \cite{odonaill}
	\item lipéad: féach ar an téarma `label / lipéad'
\end{itemize}

 \noindent \textit{Nótaí Aistriúcháin:}
\begin{itemize}
	\item Úsáidtear frása réamhfhoclach anseo seachas briathar nua a chumadh.
	\item Is é `lipéadaithe' atá ar Téarma.ie -- ach níl an focal sin le fáil i bhfoclóir dúchasach ar bith. Ní ghlactar leis mar sin.
	\item Féach chomh maith ar an téarma `label / lipéad'.
\end{itemize}


\subsubsection*{language model (LM) (ainmfhocal): samhail theanga (ST)}
 \noindent \textit{Sainmhíniú (ga):} I gcomhthéacs ríomhfhoghlama, samhail a bhfuil mar aidhm aici tuiscint a fháil ar theanga nádúrtha na ndaoine chun í a phróiseáil.
\\
 \noindent \textit{Sainmhíniú (en):} In the context of machine learning, a model that aims to understand and process natural human language.
\\
 \noindent \textit{Tagairtí:}
\begin{itemize}
	\item samhail: féach ar an téarma `model / samhail'
	\item teanga: De Bhaldraithe (1978) \cite{de-bhaldraithe}, Dineen (1934) \cite{dineen}, Ó Dónaill et al. (1991) \cite{focloir-beag}, Ó Dónaill (1977) \cite{odonaill}
\end{itemize}

 \noindent \textit{Nótaí Aistriúcháin:}
\begin{itemize}
	\item Téarma cruthaithe go díreach as na fréamhacha thuas.
	\item Féach chomh maith ar an téarma `model / samhail'.
\end{itemize}


\subsubsection*{large language model (LLM) (ainmfhocal): samhail theanga mhór (STM)}
 \noindent \textit{Sainmhíniú (ga):} I gcomhthéacs ríomhfhoghlama, samhail theanga a bhfuil na billiúin, na trilliúin, nó níos mó paraiméadair aici.
\\
 \noindent \textit{Sainmhíniú (en):} In the context of computer science, a language model with billions, trillions, or more parameters.
\\
 \noindent \textit{Tagairtí:}
\begin{itemize}
	\item samhail theanga: féach ar an téarma `language model (LM) / samhail theanga (ST)'
	\item mór: De Bhaldraithe (1978) \cite{de-bhaldraithe}, Dineen (1934) \cite{dineen}, Ó Dónaill et al. (1991) \cite{focloir-beag}, Ó Dónaill (1977) \cite{odonaill}
\end{itemize}

 \noindent \textit{Nótaí Aistriúcháin:}
\begin{itemize}
	\item Téarma cruthaithe go díreach as na fréamhacha thuas.
	\item Roghnaíodh `samhail theanga mhór' seachas `samhail mhór theanga' toisc gur samhlacha teanga iad gach uile shamhail theanga mhór. Ní úsáidtear `samhail mhór' mar théarma teicniúil riamh, ach chuirfeadh `samhail mhór theanga' in iúl gur sórt samhla móire atá dírithe ar theanga atá i gceist, seachas gur sórt samhla teanga atá an-mhór atá i gceist.
	\item Féach chomh maith ar an téarma `language model (LM) / samhail theanga (ST)'.
\end{itemize}


\subsubsection*{latent (aidiacht): folaigh}
 \noindent \textit{Sainmhíniú (ga):} Ag trácht ar veicteoir, ar leabú, nó ar paraiméadair i samhail ríomhfhoghlama, ag trácht ar eolas atá foghlamtha nó impleachtaithe. Is cisil fholaigh iad chuile chiseal i líonra néarach seachas an ciseal ionchuir agus an ciseal aschuir.
\\
 \noindent \textit{Sainmhíniú (en):} Of a vector, embedding, or parameters in a machine learning model, representing information that is learned or implicit. In a neural network, all layers except the input and output layers are latent layers.
\\
 \noindent \textit{Tagairtí:}
\begin{itemize}
	\item folach: De Bhaldraithe (1978) \cite{de-bhaldraithe}, Dineen (1934) \cite{dineen}*, Ó Dónaill et al. (1991) \cite{focloir-beag}, Ó Dónaill (1977) \cite{odonaill}
\end{itemize}

 \noindent \textit{Nótaí Aistriúcháin:}
\begin{itemize}
	\item Feictear `teas folaigh' mar théarma eolaíochta i bhFoclóir De Bhaldraithe, agus glactar leis sa gcomhthéacs seo mar analach leis sin.
	\item Úsáidtear an focal `folach' toisc go mbíonn leabuithe / paraiméadair / srl folaigh `i bhfolach' -- ní bhítear in ann a rá cad dó a sheasann siad toisc matamaitic na samhlacha ríomhfhoghlama a bheith ró-chasta lena bheith tuigthe go díreach ag dhaoine.
\end{itemize}


\subsubsection*{layer (ainmfhocal): ciseal}
 \noindent \textit{Sainmhíniú (ga):} I líonra néarach, bloc néaróg a bhfuil ionchur agus aschur sainmhínithe dó, agus atá mar chuid in-athúsáidte den líonra néarach iomlán.
\\
 \noindent \textit{Sainmhíniú (en):} In a neural network, a block of neurons that have clearly-defined input and output, and that are a building block of the larger neural network.
\\
 \noindent \textit{Tagairtí:}
\begin{itemize}
	\item ciseal: De Bhaldraithe (1978) \cite{de-bhaldraithe}, Dineen (1934) \cite{dineen}, Ó Dónaill (1977) \cite{odonaill}
\end{itemize}

 \noindent \textit{Nótaí Aistriúcháin:}
\begin{itemize}
	\item Luann Foclóir Uí Dhónaill mar théarma eolaíochta (sa mbitheolaíocht) é seo. Cé nach ionann sin agus comhthéacs ríomheolaíochta, úsáidtear le brí chomhchosúil é -- `layer' a scarann dhá chuid de rud (cill, nó codanna de líonra néarach) óna chéile.
	\item Tá `sraith' ar Téarma.ie ina chomhair seo, ach níor cheart an téarma sin a úsáid. Tugann sraith le fios go bhfuil liosta nó struchtúr sonraí líneach ann. Ní shin mar atá i líonraí néaracha. Nuair a bhítrear ag trácht ar `layer' i líonra néarach, in gach uile cás nach mór, bítear ag tráchar ar maitrís seachas ar sraith. Bíonn sé seo fíor i gcomhthéacs `convolutional layers', `dense layers', `attention / transformer layers', agus ar eile. Mar sin, ní bhíonn `sraith' cruinn ar thaobh na ríomheolaíochta de.
	\item Móide sin, is fearr ciseal ná maitrís (nó sraith, srl) toisc nach mbíonn chuile `layer' ina mhaitrís ach an oiread. Meastar gur fearr é mar théarma mar sin.
\end{itemize}


\subsubsection*{learning rate (ainmfhocal): ráta foghlama}
 \noindent \textit{Sainmhíniú (ga):} I gcomhthéacs samlach ríomhfhoghlama, luach scálach a cinntíonn cé chomh mór is atá chuile athrú ar pharaiméadair na samhla le linn traenála.
\\
 \noindent \textit{Sainmhíniú (en):} In the context of a machine learning mode, a scalar value that determines how large each update to the model's parameters is during training.
\\
 \noindent \textit{Tagairtí:}
\begin{itemize}
	\item ráta: De Bhaldraithe (1978) \cite{de-bhaldraithe}, Ó Dónaill et al. (1991) \cite{focloir-beag}, Ó Dónaill (1977) \cite{odonaill}
	\item foghlaim: féach ar an téarma `machine learning / ríomhfhoghlaim'
\end{itemize}

 \noindent \textit{Nótaí Aistriúcháin:}
\begin{itemize}
	\item Téarmaí díreach ar fáil le bríonna chomhchosúla.
\end{itemize}


\subsubsection*{library (ainmfhocal): leabharlann (chóid)}
 \noindent \textit{Sainmhíniú (ga):} Cnuasach cóid caighdeánach a chuirtear le chéile chun cuidiú le tionscadail chóid eile, agus atá ar fáil (den chuid is mó) go poiblí agus saor in aisce.
\\
 \noindent \textit{Sainmhíniú (en):} A collection of standardised code that is put together to help with other coding projects, and that is (typically) available publicly and for free.
\\
 \noindent \textit{Tagairtí:}
\begin{itemize}
	\item leabharlann: De Bhaldraithe (1978) \cite{de-bhaldraithe}, Dineen (1934) \cite{dineen}, Ó Dónaill et al. (1991) \cite{focloir-beag}, Ó Dónaill (1977) \cite{odonaill}
	\item cód: De Bhaldraithe (1978) \cite{de-bhaldraithe}, Dineen (1934) \cite{dineen}*, Ó Dónaill et al. (1991) \cite{focloir-beag}, Ó Dónaill (1977) \cite{odonaill}
\end{itemize}

 \noindent \textit{Nótaí Aistriúcháin:}
\begin{itemize}
	\item * tá an focal `cód' i bhFoclóir Uí Dhuinín, ach is le brí `a code, a codex, a book' atá sé luaite. Ní dócha go rabhthas ag trácht ar chód ríomhaireachta sa bhFoclóir sin, toisc é a bheith níos sine, agus aidhm níos ginearálta (seachas teicniúil) a bheith aige.
	\item Luann na foclóirí eile `cód' le brí chomhchiallach le `code' ríomhaireachta.
	\item Is trí analach a cruthaíodh an téarma Béarla `library', toisc leabharlann chóid a bheith úsáidte mar chnuasach cóid caighdeánach a bhfuil rochtain go forleathan (agus saor in aisce) air chun córais ríomhaireachta a fhorbairt. Ní droch-analach é sin -- agus ní bhraitheann sé ar choir chainte an Bhéarla ach ar choincheap teibí. Glactar leis mar sin, agus úsáidtear `leabharlann' anseo.
	\item Cuirtear an focal `cód' leis seo (.i. `leabharlann chóid', seachas `leabharlann') toisc nach bhfuil an úsáid seo an-choitianta i nGaeilge, agus toisc gur léire é mar théarma leis an sainmhíniú sin. Sin ráite, ní gá an focal `cód' a choinneáil leis más léir ón gcomhthéacs cad atá i gceist (agus go háirithe má tá an téarma `leabharlann chóid' luaite cheana).
	\item Tá an-chathú le téarma a chuma as nua -- `códlann'. Cé go bhfuil `cód' agus `-lann' in úsáid go forleathan i nGaeilge, ní cainteoir dúchais mé, agus ní mhothóinn compordach téarma mar sin a chruthú nuair atá téarma eile (.i. `leabharlann chóid') in ann an bhrí cheannann chéanna a chur in iúl gan focal ar leith a chumadh as nua.
\end{itemize}


\subsubsection*{linear (aidiacht): líneach}
 \noindent \textit{Sainmhíniú (ga):} I gcomhthéacs matamaitice, comhairthe mar shuim ualaithe de roinnt athróga; m.sh. y = aX1 + bX2 ... + kXn.
\\
 \noindent \textit{Sainmhíniú (en):} In the context of mathematics, calculated as a weighted sum of various variables; for example, y = aX1 + bX2 ... + kXn.
\\
 \noindent \textit{Tagairtí:}
\begin{itemize}
	\item líneach: De Bhaldraithe (1978) \cite{de-bhaldraithe}, Ó Dónaill (1977) \cite{odonaill}, Williams et al. (2023) \cite{storchiste}
\end{itemize}

 \noindent \textit{Nótaí Aistriúcháin:}
\begin{itemize}
	\item Téarma díreach ar fáil leis an mbrí cheannann chéanna ó na foclóirí thuas.
\end{itemize}


\subsubsection*{link prediction (LP) (ainmfhocal): réamhinsint nasc (RN)}
 \noindent \textit{Sainmhíniú (ga):} I gcomhthéacs graif eolais, réamhinsint abairt thriarach nua (nach bhfuil sa ngraf) bunaithe air na habairtí triaracha atá sa ngraf.
\\
 \noindent \textit{Sainmhíniú (en):} In the context of a knowledge graph, the act of predicting a new triple (that is not in the graph) based on the triples that are in the graph.
\\
 \noindent \textit{Tagairtí:}
\begin{itemize}
	\item réamhinsint: féach ar an téarma `prediction / réamhinsint'
	\item nasc: De Bhaldraithe (1978) \cite{de-bhaldraithe}, Dineen (1934) \cite{dineen}, Ó Dónaill et al. (1991) \cite{focloir-beag}, Ó Dónaill (1977) \cite{odonaill}
\end{itemize}

 \noindent \textit{Nótaí Aistriúcháin:}
\begin{itemize}
	\item Luaitear `nasc' den chuid is mó mar snaidhm ag bhíonn ag coinneáil dhá rud le chéile. Sin ráite, is féidir abairt thriarach a shamhlú mar dhá nód agus ceangal nasctha le chéile (agus is dócha gurb in an áit as a dtagann an téarma Béarla `link' chomh maith).
	\item Is é an ginideach iolra (réamhinsint nasc) a úsáidtear anseo toisc gur minice caint a dhéanamh ar réamhinsint roinnt nasc seachas réamhinsint ar nasc amháin. Sin ráite, is é `réamhinsint naisc' an téarma ceart nuair nach bhfuil ach nasc amháin lena bheith réamhinste.
	\item Féach chomh maith ar an téarma `prediction / réamhinsint'.
\end{itemize}


\subsubsection*{link prediction query (ainmfhocal): ceist réamhinsinte nasc}
 \noindent \textit{Sainmhíniú (ga):} Ceist a chuirtear ar réamhinsteoir nasc i bhfoirm abairte triaraí neamhiomlán (s,p,?) nó (?,p,o). Is é an freagra uirthi ná nód $o$ nó $s$ dhéanann abairt iomlán agus cheart as abairt neamhiomlán na ceiste.
\\
 \noindent \textit{Sainmhíniú (en):} A query that is posed to a link predictor in the form of an incomplete triple (s,p,?) or (?,p,o). The answer to this query is a node $o$ or $s$ that makes a complete and true triple out of the query triple.
\\
 \noindent \textit{Tagairtí:}
\begin{itemize}
	\item tasc: De Bhaldraithe (1978) \cite{de-bhaldraithe}, Dineen (1934) \cite{dineen}, Ó Dónaill et al. (1991) \cite{focloir-beag}, Ó Dónaill (1977) \cite{odonaill}
	\item réamhinsint nasc: féach ar an téarma `link prediction (LP) / réamhinsint nasc (RN)'
\end{itemize}

 \noindent \textit{Nótaí Aistriúcháin:}
\begin{itemize}
	\item Féach ar an téarma `link prediction (LP) / réamhinsint nasc (RN)'.
\end{itemize}


\subsubsection*{link prediction task (ainmfhocal): tasc réamhinsinte nasc}
 \noindent \textit{Sainmhíniú (ga):} I gcomhthéacs graif eolais, tasc a bhfuil mar sprioc aige abairt thriarach nua (nach bhfuil a ngraf) a réamhinsint bunaithe ar na habairtí triaracha atá sa ngraf.
\\
 \noindent \textit{Sainmhíniú (en):} In the context of a knowledge graph, the task of predicting a new triple (that is not in the graph) based on the triples that are in the graph.
\\
 \noindent \textit{Tagairtí:}
\begin{itemize}
	\item tasc: De Bhaldraithe (1978) \cite{de-bhaldraithe}, Dineen (1934) \cite{dineen}, Ó Dónaill et al. (1991) \cite{focloir-beag}, Ó Dónaill (1977) \cite{odonaill}
	\item réamhinsint nasc: féach ar an téarma `link prediction (LP) / réamhinsint nasc (RN)'
\end{itemize}

 \noindent \textit{Nótaí Aistriúcháin:}
\begin{itemize}
	\item Féach ar an téarma `link prediction (LP) / réamhinsint nasc (RN)'
\end{itemize}


\subsubsection*{link predictor (ainmfhocal): réamhinsteoir nasc}
 \noindent \textit{Sainmhíniú (ga):} I gcomhthéacs graif eolais, samhail ríomhfhoghlama a dhéanann naisc a réamhinsint.
\\
 \noindent \textit{Sainmhíniú (en):} In the context of a knowledge graph, a machine learning model that performs link prediction.
\\
 \noindent \textit{Tagairtí:}
\begin{itemize}
	\item réamhinsteoir: féach ar an téarma `predictor / réamhinsteoir'
	\item réamhinsint nasc: féach ar an téarma `link prediction (LP) / réamhinsint nasc (RN)'
\end{itemize}

 \noindent \textit{Nótaí Aistriúcháin:}
\begin{itemize}
	\item Féach chomh maith ar an téarma `link prediction (LP) / réamhinsint nasc (RN)'
	\item Féach chomh maith ar an téarma `predictor / réamhinsteoir'.
\end{itemize}


\subsubsection*{linked data (LD) (ainmfhocal): sonraí nasctha (SN)}
 \noindent \textit{Sainmhíniú (ga):} Bailiúchán sonraí i ngraif eolais atá nasctha lena chéile agus atá, go minic, dáilte thar líon mhór suíomhanna / freastalaithe ar an idirlíon.
\\
 \noindent \textit{Sainmhíniú (en):} A collection of data in knowledge graphs that are connected to each other and, often, distributed among many sites / servers on the internet.
\\
 \noindent \textit{Tagairtí:}
\begin{itemize}
	\item sonra: féach ar an téarma `database / bunachar sonraí'
	\item nasc: féach ar an téarma `link prediction (LP) / réamhinsint nasc (RN)'
\end{itemize}

 \noindent \textit{Nótaí Aistriúcháin:}
\begin{itemize}
	\item Téarma cruthaithe go díreach as téarmaí eile sa bhfoclóir seo
	\item Féach chomh maith ar an téarma `database / bunachar sonraí'.
	\item Féach chomh maith ar an téarma `link prediction (LP) / réamhinsint nasc (RN)'.
\end{itemize}


\subsubsection*{linked open data (LOD) (ainmfhocal): sonraí nasctha saor-rochtana (SNSR)}
 \noindent \textit{Sainmhíniú (ga):} Sonraí nasctha a bhfuil rochtain éasca, saor, agus poiblí air.
\\
 \noindent \textit{Sainmhíniú (en):} Linked data that is made available for easy, free, and public access.
\\
 \noindent \textit{Tagairtí:}
\begin{itemize}
	\item sonra: féach ar an téarma `database / bunachar sonraí'
	\item nasc: féach ar an téarma `link prediction (LP) / réamhinsint nasc (RN)'
	\item saor-rochtana: féach ar an téarma `open source / saor-rochtana'
\end{itemize}

 \noindent \textit{Nótaí Aistriúcháin:}
\begin{itemize}
	\item Is é `sonraí nasctha saor-rochtana' atá ar Téarma.ie. Glactar leis sin toisc go gcloíonn sé le sampla na bhfoclóirí dúchasacha.
	\item Féach chomh maith ar an téarma `linked data (LD) / sonraí nasctha (SN)'.
	\item Féach chomh maith ar an téarma `linked data (LD) / sonraí nasctha (SN)'.
	\item Féach chomh maith ar an téarma `open source / saor-rochtana'.
\end{itemize}


\subsubsection*{to load (briathar): lódáil}
 \noindent \textit{Sainmhíniú (ga):} I gcomhthéacs ríomheolaíochta, sonraí nó cód a aistriú ó stóras amháin (m.sh. diosca an ríomhaire) go stóras eile (m.sh. cuimhne an ríomhaire).
\\
 \noindent \textit{Sainmhíniú (en):} In the context of computer science, to transfer data from one storage location (such as the computer's disk) to another (for example, the computer's memory).
\\
 \noindent \textit{Tagairtí:}
\begin{itemize}
	\item lódáil: De Bhaldraithe (1978) \cite{de-bhaldraithe}, Dineen (1934) \cite{dineen}, Ó Dónaill et al. (1991) \cite{focloir-beag}, Ó Dónaill (1977) \cite{odonaill}
\end{itemize}

 \noindent \textit{Nótaí Aistriúcháin:}
\begin{itemize}
	\item Ní luann foclóir ar bith de na foclóirí thuas an focal seo i gcomhthéacs ríomhairí. Cé go bhfuil go leor comhthéacsanna ar leith luaite leis, bíonn trácht i gcónaí air chun lódáil fhisiceach in chur in iúl (m.sh. `lódáil soithigh' i bhFoclóir Uí Dhónaill). Sin ráite -- níl téarma ar bith eile ann a bheadh níos oiriúnaí (.i. nach mbíonn úsáidte go príomha i gcomhthéacs fisicigh). Thairis sin, cé nach rud fisiceach atá i gceist anseo, tá an bun-choincheap céanna i gceist -- rud (lasta nó sonraí) a bhogadh ó stóras amháin go stóras eile.
\end{itemize}


\subsubsection*{local (aidiacht): logánta}
 \noindent \textit{Sainmhíniú (ga):} I gcomhthéacs graif, sonraí, samlach, nó eile, ag trácht ar airíonna a bhaineann le codanna áirithe den ghraf / de na sonraí / den tsamhail, gan trácht airíonna an ruda iomláin. Frithchiallach leis an téarma `uilíoch'.
\\
 \noindent \textit{Sainmhíniú (en):} In the context of a graph, data, a model, etc, relating to features of sub-parts of the graph / data / model without reference to properties of the whole. Antonym to `global'.
\\
 \noindent \textit{Tagairtí:}
\begin{itemize}
	\item logánta: De Bhaldraithe (1978) \cite{de-bhaldraithe}, Dineen (1934) \cite{dineen}, Ó Dónaill et al. (1991) \cite{focloir-beag}, Ó Dónaill (1977) \cite{odonaill}
\end{itemize}

 \noindent \textit{Nótaí Aistriúcháin:}
\begin{itemize}
	\item Téarma díreach ar fáil le brí chomhchosúil ó na foclóirí thuas.
	\item Tá `áitiúil' iomlán ceart go leor chomh maith de réir gach dealraimh. Sin ráite, is é logánta a nglac an Coiste Téarmaíochta leis ar Téarma.ie. Toisc nach bhfuil difríocht ar bith idir `áitiúil' agus `logánta' de réir na bhfoclóirí dúchasacha, glacadh le `logánta' le go mbeadh comhaontú ar théarma caighdeánach amháin ann os a chomhair seo.
\end{itemize}


\subsubsection*{local maximum (ainmfhocal): uasluach logánta}
 \noindent \textit{Sainmhíniú (ga):} I gcomhthéacs feabhsaithe nó ríomhfhoghlama, luach atá ard i gcomparáid le luacha eile in éineacht leis, ach nach gcaithfidh gurb é an luach is airde ar féidir é a fháil (.i. ní chaithfidh sé a bheith mar uaslauch uilíoch).
\\
 \noindent \textit{Sainmhíniú (en):} In the context of optimisation or machine learning, a value that is high compared to other values near it, but that may not be the highest possible value (i.e. that may not be a global maximum).
\\
 \noindent \textit{Tagairtí:}
\begin{itemize}
	\item uaslauch: féach ar an téarma `maximum / uaslauch'
	\item logánta: féach ar an téarma `local / logánta'
\end{itemize}

 \noindent \textit{Nótaí Aistriúcháin:}
\begin{itemize}
	\item Féach ar an téarma `maximum / uaslauch'
	\item Féach chomh maith ar an téarma `local / logánta'
\end{itemize}


\subsubsection*{local minimum (ainmfhocal): íosluach logánta}
 \noindent \textit{Sainmhíniú (ga):} I gcomhthéacs feabhsaithe nó ríomhfhoghlama, luach atá íseal i gcomparáid le luacha eile in éineacht leis, ach nach gcaithfidh gurb é an luach is ísle ar féidir é a fháil (.i. ní chaithfidh sé a bheith mar íoslauch uilíoch).
\\
 \noindent \textit{Sainmhíniú (en):} In the context of optimisation or machine learning, a value that is low compared to other values near it, but that may not be the lowest possible value (i.e. that may not be a global minimum).
\\
 \noindent \textit{Tagairtí:}
\begin{itemize}
	\item íosluach: féach ar an téarma `minimum / íosluach'
	\item logánta: féach ar an téarma `local / logánta'
\end{itemize}

 \noindent \textit{Nótaí Aistriúcháin:}
\begin{itemize}
	\item Féach ar an téarma `minimum / íosluach'
	\item Féach chomh maith ar an téarma `local / logánta'
\end{itemize}


\subsubsection*{logic (ainmfhocal): loighic}
 \noindent \textit{Sainmhíniú (ga):} An réimse staidéar a bhaineann le réasúnaíocht.
\\
 \noindent \textit{Sainmhíniú (en):} The field of study relating to reasoning.
\\
 \noindent \textit{Tagairtí:}
\begin{itemize}
	\item loighic: De Bhaldraithe (1978) \cite{de-bhaldraithe}, Dineen (1934) \cite{dineen}, Ó Dónaill et al. (1991) \cite{focloir-beag}, Ó Dónaill (1977) \cite{odonaill}
\end{itemize}

 \noindent \textit{Nótaí Aistriúcháin:}
\begin{itemize}
	\item Téarma díreach ar fáil ó na foinsí thuas leis an mbrí chéanna.
\end{itemize}


\subsubsection*{loss (ainmfhocal): pionós (foghlama)}
 \noindent \textit{Sainmhíniú (ga):} I gcomhthéacs samhla ríomhfhoghlama, luach uimhriúil a chuireann in iúl cé chomh dona is atá an tsamhail (sin le rá, is ionann pionós níos airde samhail níos measa). Úsáidtear an pionós mar chuid mhatamaiticiúil de ríomhfhoghlaim.
\\
 \noindent \textit{Sainmhíniú (en):} In the context of a machine learning model, a numerical value that represents how bad the model is (that is, a higher penalty means the model is less effective). Loss is used as a part of the mathematical process of machine learning.
\\
 \noindent \textit{Tagairtí:}
\begin{itemize}
	\item pionós: De Bhaldraithe (1978) \cite{de-bhaldraithe}, Dineen (1934) \cite{dineen}, Ó Dónaill et al. (1991) \cite{focloir-beag}, Ó Dónaill (1977) \cite{odonaill}
\end{itemize}

 \noindent \textit{Nótaí Aistriúcháin:}
\begin{itemize}
	\item * Is `pionús' seachas `pionós' atá i bhFoclóir Uí Dhuinín, ach glactar leis gurb in an focal céanna le litriú eile
	\item Tá `caill' / ` caillteanas' ar Téarma.ie mar fhocal ar `loss' an Bhéarla. Meastar go bhfuil siad sin rud beag ró-litriúil; i nGaeilge, is éard atá i gceist le `caill' / ` caillteanas' ná rud atá caillte / gan tuairisc air (de réir Fhoclóir Uí Dhónaill). Ní hionann sin agus an luach `loss', atá úsáidte chun cur síos a dhéanamh ar cé chomh dona is atá samhail ríomhfhoghlama. Thairis sin, níl trácht ar bith ar `caill' mar fhocal matamaiticiúil, eolaíochta, ná sainmhínithe i bhfoclóir dúchasach ar bith. Ní ghlactar le `caill' ná le `caillteanas' mar sin.
	\item Is minic a dhéantar analach idir ríomhfhoghlaim agus foghlaim na ndaoine / ainmhithe. Is é bunús an analacha seo ná go mbíonn `loss' mar phionós ar an ríomhaire (cosúil le pionós a chur ar madra toisc é a bheith dána). Bíonn an tsamhail ag foghlaim toisc an phionóis sin -- chun rudaí a dhéanamh nach bhfuil pionós leo.
	\item Cé nach mbíonn sé seo iomlán soiléir i gcónaí, is `pionós' (nó `penalty') é `loss' go litriúil chomh maith. Nuair a bhíonn ríomhfhoghlaim ar siúl bíonn (i nach mór gach uile cás) paraiméadair infhoghlamtha ar an tsamhail atá le hathrú le linn feabhsaithe. Is ionann na paraiméadair seo a athrú agus treo a thabhairt dóibh. Tabhair mar shampla samhail ríomhfhoghlama an-simplí -- cúlú líneach le dhá pharaiméadar (.i. y = ax + b, ina bhfuil a agus b mar pharaiméadair). Is féidir gach uile shamhail mar sin a shamhlú mar phointe ar bhreacadh Cairtéiseach (.i. (3,2), le a=3 agus b=2). Bíonn luach an phionóis (p) mar ais z (sa tríú toise) -- is é is ísle is atá, is é is fearr. Nuair a bhíonn foghlaim ar siúl, is í is aidhm leis an bhfoghlaim ná pointe (a,b) a fháil a bhfuil an luach phionóis p chomh íseal agus is féidir. Le linn an fheabhsaithe sin, is minic (go háirithe agus feidhm phionóis simplí in úsáid) go mbíonn an tsamhail sáinnithe ar íosmhéid logánta (nach ionann agus an t-íosluach uilíoch). Tarlaíonn sé seo toisc an tsamhail a bheith ag bogadh sa treo mícheart. Chun é seo a sheachaint, déantar feidhmeanna pionóis níos fearr (agus níos casta) a úsáid. Gearrann na feidhmeanna seo pionós níos airde ar an tsamhail má bhogann sé sa treo mícheart trí luach pionóis níos airde a chur leis an pointí sin. Cé nach pionós dlí é seo, is pionós foghlama é go litriúil.
	\item Is minic a bhíonn trácht ar `penalty' i litríocht an réimse i mBéarla díreach toisc an dá phointe thuas
	\item Ní i gcomhthéacs matamaiticiúil a luaitear an focal pionós sna foclóirí thuas, ach glactar leis fós féin toisc na bpointí thuas. Móide sin, moltar `pionós foghlama' a úsáid más gá léiriú cé sóirt pionóis atá i gceist, i dtaca leis na pointí thuas.
\end{itemize}


\subsubsection*{loss function (ainmfhocal): feidhm phionóis}
 \noindent \textit{Sainmhíniú (ga):} I gcomhthéacs samhla ríomhfhoghlama, feidhm a dhéanann luach an phionóis a áireamh don tsamhail sin ar thacar sonraí éigin.
\\
 \noindent \textit{Sainmhíniú (en):} In the context of a machine learning model, a function that calculates the loss value of that model on some set of data points.
\\
 \noindent \textit{Tagairtí:}
\begin{itemize}
	\item feidhm: féach ar an téarma `function / feidhm'
	\item pionós: féach ar an téarma `loss / pionós'
\end{itemize}

 \noindent \textit{Nótaí Aistriúcháin:}
\begin{itemize}
	\item Féach ar an téarma `function / feidhm'.
	\item Féach ar an téarma `loss / pionós'.
\end{itemize}


\phantomsection \subsection*{M}
\addcontentsline{toc}{subsection}{M}
\markboth{M}{M}

\subsubsection*{machine (aidiacht): ríomh-}
 \noindent \textit{Sainmhíniú (ga):} I gcomhthéacs ríomheolaíochta, bunaithe ar ríomhfhoghlaim.
\\
 \noindent \textit{Sainmhíniú (en):} In the context of computer science, based on machine learning.
\\
 \noindent \textit{Tagairtí:}
\begin{itemize}
	\item ríomh-: Ó Dónaill (1977) \cite{odonaill}*
\end{itemize}

 \noindent \textit{Nótaí Aistriúcháin:}
\begin{itemize}
	\item * Níl `ríomh-' ann mar réimír. Cé is moite de sin, feictear é in úsáid mar réimír sa bhfoclóir céanna; .i. `ríomheolaíocht' (murab ionann sin agus `eolaíocht' mar iarmhír)
	\item Tá idir `meaisín-' (mar réimír) agus `ríomh-' (mar réimír) le feiceáil ar Téarma.ie. Ní ghlactar le `meaisín' mar réimír toisc nach mbíonn sé ina réimír ar i bhfoclóir dúchasach ar bith (ach bíonn `ríomh-' úsáidte i bhfocail eile agus i bhfoclóirí dúchasacha).
\end{itemize}


\subsubsection*{machine learning (ainmfhocal): ríomhfhoghlaim}
 \noindent \textit{Sainmhíniú (ga):} Cur chuige feabhsúcháin mhatamaiticiúil a chuireann ar chumas do ríomhairí ceisteanna casta a fhreagairt.
\\
 \noindent \textit{Sainmhíniú (en):} The process of using mathematical optimisation to allow computers to solve complex problems.
\\
 \noindent \textit{Tagairtí:}
\begin{itemize}
	\item foghlaim: De Bhaldraithe (1978) \cite{de-bhaldraithe}, Dineen (1934) \cite{dineen}, Ó Dónaill et al. (1991) \cite{focloir-beag}, Ó Dónaill (1977) \cite{odonaill}
	\item ríomh-: féach ar an téarma `machine / ríomh-'
\end{itemize}

 \noindent \textit{Nótaí Aistriúcháin:}
\begin{itemize}
	\item Bheadh ciall éigin ag dul le `foghlaim na meaisíní' chomh maith, seachas `meaisín' a bheith bainteach lena leithéid de mheaisíní níocháin (féach ar Teanglann) agus gan rian eile de le fheiceáil i gcomhthéacs ríomhaireachta.
	\item Cé nach mbíonn trácht ar `ríomhfhoghalim' i bhfoclóir ar bith, tá `ríomh-' ann mar réimír. Thairis sin, bíonn ann dá leithéid de `ríomhphost' agus `ríomheolaíocht' anois sa gcaint.
	\item Féach chomh maith ar an téarma `machine / ríomh-'.
\end{itemize}


\subsubsection*{to map (briathar): mapáil}
 \noindent \textit{Sainmhíniú (ga):} I gcomhthéacs ríomheolaíochta, mapa a chruthú.
\\
 \noindent \textit{Sainmhíniú (en):} In the context of computer science, to create a map / mapping.
\\
 \noindent \textit{Tagairtí:}
\begin{itemize}
	\item mapáil: De Bhaldraithe (1978) \cite{de-bhaldraithe}, Ó Dónaill et al. (1991) \cite{focloir-beag}, Ó Dónaill (1977) \cite{odonaill}
\end{itemize}

 \noindent \textit{Nótaí Aistriúcháin:}
\begin{itemize}
	\item Luann na foclóirí thuas ar fad `mapáil' i gcomhthéacs mapa den domhan. Cé nach in an comhthéacs céanna is atá i gceist anseo, tá glactar le `mapáil' toisc gur glacadh le `mapa' sa bhfoclóir seo.
	\item Féach chomh maith ar an téarma `mapping / mapa'.
\end{itemize}


\subsubsection*{mapping (ainmfhocal): mapa}
 \noindent \textit{Sainmhíniú (ga):} I gcomhthéacs ríomhaireachta, bunachar a ligeann duit luach amháin (an `luach') a fháil trí cheangal le luach eile (an eochair).
\\
 \noindent \textit{Sainmhíniú (en):} In the context of computer science, a database that allows access to one value (called the `value') using another value (called the `key').
\\
 \noindent \textit{Tagairtí:}
\begin{itemize}
	\item mapa: De Bhaldraithe (1978) \cite{de-bhaldraithe}, Ó Dónaill et al. (1991) \cite{focloir-beag}, Ó Dónaill (1977) \cite{odonaill}
\end{itemize}

 \noindent \textit{Nótaí Aistriúcháin:}
\begin{itemize}
	\item Luann na foclóirí thuas ar fad `mapa' i gcomhthéacs mapa den domhan. Ní shin atá i gceist anseo, ach, toisc gurb é aidhm mapa ríomhaireachta ná ceangal éigin a dhéanamh idir dhá rud (le gur féidir leathrud a fáil ón leathrud eile), agus toisc gurb as an meafar sin a thagann úsáid `mapping' i mBéarla, glactar leis an téarma sin anseo.
	\item Rogha eile a bhí sa bhfocal `léarscáil'. Ní cosúil ó na foclóirí thuas, áfach, cén ceann acu is fearr i gcomhthéacs ríomheolaíochta. Sin ráite, tá 2 bhuntáiste le `mapa' -- tá briathar ceangailte leis ('mapáil'), agus tá idir `mapa' agus `mapáil' ar Téarma.ie cheana. Roghnaíodh `mapa' mar sin.
\end{itemize}


\subsubsection*{margin (ainmfhocal): bearna}
 \noindent \textit{Sainmhíniú (ga):} I gcomhthéacs na feidhme pionóis Pionós Rangaithe le Bearna, hipear-pharaiméadar a chinneann cé chomh fada agus ar ceart scór na n-abairtí triaracha fírinneacha agus scór na bhfrith-shamplaí a choinneáil óna chéile.
\\
 \noindent \textit{Sainmhíniú (en):} In the context of the Margin Ranking Loss loss function, a hyperparameter that determines how far the score of true triples and negative samples should be kept from each other.
\\
 \noindent \textit{Tagairtí:}
\begin{itemize}
	\item bearna: De Bhaldraithe (1978) \cite{de-bhaldraithe}, Dineen (1934) \cite{dineen}, Ó Dónaill et al. (1991) \cite{focloir-beag}, Ó Dónaill (1977) \cite{odonaill}
\end{itemize}

 \noindent \textit{Nótaí Aistriúcháin:}
\begin{itemize}
	\item Téarma ar fáil ó na foclóirí thuas le brí chomhchosúil, ach i gcomhthéacs níos leithne.
\end{itemize}


\subsubsection*{margin ranking loss (MRL) (ainmfhocal): pionós rangaithe le bearna (PRB)}
 \noindent \textit{Sainmhíniú (ga):} Feidhm phionóis cruthaithe ar son an taisc réamhinsinte nasc a iarrann bearna áirithe a choinneáil idir scór na n-abairtí triaracha fírinneacha agus scór na bhfrith-shamplaí.
\\
 \noindent \textit{Sainmhíniú (en):} A loss function created for the link prediction task that attempts to maintain a specific margin between the scores of true triples and the scores of negative samples.
\\
 \noindent \textit{Tagairtí:}
\begin{itemize}
	\item pionós: féach ar an téarma `loss function / feidhm phionóis'
	\item rangaigh: féach ar an téarma `to rank / rangaigh'
	\item bearna: féach ar an téarma `margin / bearna'
\end{itemize}

 \noindent \textit{Nótaí Aistriúcháin:}
\begin{itemize}
	\item Tabhair faoi deara gurb é an leagan ceart de `margin ranking' ná `rangú le bearna'. Sa téarma `pionós rangiathe le bearna' tá `rangaithe' sa tuiseal ginideach; ní aidiacht bhriathartha é. Ní dhearna `rangú le bearna' a chur mar théarma eile sa bhfoclóir seo, áfach, toisc nach mbeadh ciall ar bith leis ina aonar.
	\item Féach chomh maith ar an téarma `loss function / feidhm phionóis'.
	\item Féach chomh maith ar an téarma `to rank / rangaigh'.
	\item Féach chomh maith ar an téarma `margin / bearna'.
\end{itemize}


\subsubsection*{matrix (ainmfhocal): maitrís}
 \noindent \textit{Sainmhíniú (ga):} Grúpa uimhreacha in eangach le sraitheanna agus le colún. Is féidir maitrís a shamhlú mar liosta veicteoirí a bhfuil an toise céanna acu uile.
\\
 \noindent \textit{Sainmhíniú (en):} A group of numbers in a grid of rows and columns. A matrix can be thought of as a list of vectors with the same dimensionality.
\\
 \noindent \textit{Tagairtí:}
\begin{itemize}
	\item maitrís: Williams et al. (2023) \cite{storchiste}
\end{itemize}

 \noindent \textit{Nótaí Aistriúcháin:}
\begin{itemize}
	\item Téarma díreach ar fáil ó Stórchiste leis an mbrí cheannann chéanna.
\end{itemize}


\subsubsection*{maximum (ainmfhocal): uasluach}
 \noindent \textit{Sainmhíniú (ga):} An luach is airde ar féidir a fháil (mar shampla, le linn feabhsaithe samhla ríomhfhoghlama).
\\
 \noindent \textit{Sainmhíniú (en):} The highest value that can be obtained (for example, during the optimisation of a machine learning model).
\\
 \noindent \textit{Tagairtí:}
\begin{itemize}
	\item uasluach: Ó Dónaill et al. (1991) \cite{focloir-beag}, Ó Dónaill (1977) \cite{odonaill}
	\item uas-: De Bhaldraithe (1978) \cite{de-bhaldraithe}, Ó Dónaill et al. (1991) \cite{focloir-beag}, Ó Dónaill (1977) \cite{odonaill}, Williams et al. (2023) \cite{storchiste}
	\item luach: De Bhaldraithe (1978) \cite{de-bhaldraithe}, Dineen (1934) \cite{dineen}, Ó Dónaill et al. (1991) \cite{focloir-beag}, Ó Dónaill (1977) \cite{odonaill}, Williams et al. (2023) \cite{storchiste}
\end{itemize}

 \noindent \textit{Nótaí Aistriúcháin:}
\begin{itemize}
	\item Téarma luaite mar théarma matamaitice i bhFoclóir Uí Dhónaill agus i bhFoclóir Uí Dhónaill agus Uí Mhaoileoin. Tá an dá chuid den téarma seo luaite le bríonna comhchosúla chomh maith sna foclóirí eile thuas.
\end{itemize}


\subsubsection*{mean (aidiacht): meán- nó meánach}
 \noindent \textit{Sainmhíniú (ga):} I gcomhthéacs liosta uimhreacha, bainteach le meán an liosta sin.
\\
 \noindent \textit{Sainmhíniú (en):} In the context of a list of numbers, relating to the mean of that list.
\\
 \noindent \textit{Tagairtí:}
\begin{itemize}
	\item meán-: De Bhaldraithe (1978) \cite{de-bhaldraithe}, Ó Dónaill (1977) \cite{odonaill}, Williams et al. (2023) \cite{storchiste}
	\item meánach: De Bhaldraithe (1978) \cite{de-bhaldraithe}, Ó Dónaill (1977) \cite{odonaill}, Williams et al. (2023) \cite{storchiste}
\end{itemize}

 \noindent \textit{Nótaí Aistriúcháin:}
\begin{itemize}
	\item Téarma díreach ar fáil ó na foclóirí thuas.
	\item Féach chomh maith ar an téarma `mean (statistic) / meán'.
\end{itemize}


\subsubsection*{mean (statistic) (ainmfhocal): meán}
 \noindent \textit{Sainmhíniú (ga):} I gcomhthéacs liosta uimhreacha $L$ ina bhfuil $n$ ball ann, an luach $suim(L) / n$. Úsáidtear an meán mar thomhas ar lár na n-uimhreacha.
\\
 \noindent \textit{Sainmhíniú (en):} In the context of a list of numbers $L$ with $n$ elements, the value $sum(L) / n$. The mean is used as a measure of the center of the list of numbers.
\\
 \noindent \textit{Tagairtí:}
\begin{itemize}
	\item meán: De Bhaldraithe (1978) \cite{de-bhaldraithe}, Ó Dónaill et al. (1991) \cite{focloir-beag}, Ó Dónaill (1977) \cite{odonaill}, Williams et al. (2023) \cite{storchiste}
\end{itemize}

 \noindent \textit{Nótaí Aistriúcháin:}
\begin{itemize}
	\item Téarma díreach ar fáil ó na foclóirí thuas.
\end{itemize}


\subsubsection*{mean rank (MR) (ainmfhocal): meán-rang (MR)}
 \noindent \textit{Sainmhíniú (ga):} I gcomhthéacs liosta ranganna, meán na ranganna ar fad atá istigh ann.
\\
 \noindent \textit{Sainmhíniú (en):} In the context of a ranked list, the mean of all ranks in it.
\\
 \noindent \textit{Tagairtí:}
\begin{itemize}
	\item meán-: féach ar an téarma `mean / meán- nó meánach'
	\item rang: féach ar an téarma `rank / rang'
\end{itemize}

 \noindent \textit{Nótaí Aistriúcháin:}
\begin{itemize}
	\item Tá an focal `meán' i bhFoclóir Uí Dhónaill agus Uí Mhaoileoin, ach ní luaitear mar réimír ann é.
	\item Bheadh ciall le `rang meánach' chomh maith le meán-rang. Sin ráite, is é `meán-' a úsáidtear den chuid is mó i dtéarmaí i bhFoclóir Uí Dhónaill (féach ar `meán-am' / `meán-saorchonair'). Ní fheictear samplaí d'úsáid `meánach' i dtéarmaí eolaíochta / matamaitice ann. Glactar le `meán-rang' mar sin.
	\item Bíonn `meánach' aistrithe mar `mean' agus mar `median' sna foclóirí thuas -- ach ní luann ach `mean' le `meán-' i Stórchiste. Meastar gur léire `meán-rang' mar sin chomh maith, chun débhrí a sheachaint.
	\item Féach chomh maith ar an téarma `mean / meán- nó meánach'.
	\item Féach chomh maith ar an téarma `rank / rang'
	\item Féach chomh maith ar an téarma `ranked list / liosta ranganna'.
\end{itemize}


\subsubsection*{mean reciprocal rank (MRR) (ainmfhocal): meán na ranganna deilíneacha (MRD)}
 \noindent \textit{Sainmhíniú (ga):} I gcomhthéacs liosta ranganna, an tomhas $1 / (me\acute{a}n(1 / rang_i)$ a chomhairtear air (do gach uile rang $rang_1...rang_i...rang_n$).
\\
 \noindent \textit{Sainmhíniú (en):} In the context of a ranked list, the metric $mean(1 / rank_i)$ that is calculated on it (for every rank $rank_1...rank_i...rank_n$).
\\
 \noindent \textit{Tagairtí:}
\begin{itemize}
	\item meán-: féach ar an téarma `mean (statistic) / meán'
	\item rang: féach ar an téarma `rank / rang'
	\item deilíneach: De Bhaldraithe (1978) \cite{de-bhaldraithe}, Ó Dónaill (1977) \cite{odonaill}, Williams et al. (2023) \cite{storchiste}
\end{itemize}

 \noindent \textit{Nótaí Aistriúcháin:}
\begin{itemize}
	\item Cé gurb é `meán-rang deilíneach' an leagan is litriúil ann, bheadh sé rud beag aisteach an leagan sin a úsáid, toisc go bhféadfaí é sin a léamh mar deilín den meán-rang (.i. $1 / (me\acute{a}n-rang)$), seachas mar mheán na ranganna deilíneacha ar fad (.i. $1 / (me\acute{a}n(1 / rang_i)$). Is é an dara rud, cinnte, atá i gceist anseo (meán na ranganna deilíneacha). Ní bhíonn an fhadhb seo i mBéarla -- is ionann `mean reciprocal rank' agus $1 / (me\acute{a}n(1 / rang_i)$ toisc struchtúr theanga an Bhéarla, nach ionann agus struchtúr na Gaeilge. Toisc gur rud amháin é `meán-rang', is éard is `meán-rang deilíneach' ann ná $1 / (me\acute{a}n-rang)$ -- rud nach bhfuil i gceist leis an téarma seo. Ní féidir glacadh le `meán-rang deilíneach' mar sin.
	\item Tá cúpla roghanna ar leith ann chun $1 / (me\acute{a}n(1 / rang_i)$ a chur in iúl -- `rang deilíneach meánach, nó `meán na ranganna deilíneacha'. Meastar go mbeadh meán na ranganna deilíneach níos léire dá bhfuil i gceist, agus glactar leis mar sin.
	\item Tá cosúlacht idir `mean rank' agus `mean recirpocal rank' i mBéarla. Níl sna leaganacha Gaeilge anseo -- déantar é sin d'aon ghnó. Is cuma an bhfuil cosúlacht theangeolaíochta idir an dá théarma, ach amháin go gcuireann siad araon an bhrí cheart in iúl.
	\item Féach chomh maith ar an téarma `mean (statistic) / meán'.
	\item Féach chomh maith ar an téarma `rank / rang'.
\end{itemize}


\subsubsection*{mean squared error (MSE) (ainmfhocal): meán na n-earráidí cearnaithe (MEC)}
 \noindent \textit{Sainmhíniú (ga):} Feidhm phionóis a fhaigheann earráid gach uile réamhinsinte, a chearnaítear an earráid sin, agus a chomhaireann ar deireadh meán na n-earráidí cearnaithe.
\\
 \noindent \textit{Sainmhíniú (en):} A loss function that finds the error of every prediction, squares that error, and then calculates the mean of all of those squared errors.
\\
 \noindent \textit{Tagairtí:}
\begin{itemize}
	\item earráid: féach ar an téarma `error / earráid'
	\item cearnaigh: De Bhaldraithe (1978) \cite{de-bhaldraithe}, Ó Dónaill (1977) \cite{odonaill}, Williams et al. (2023) \cite{storchiste}
	\item meánach: féach ar an téarma `mean / meán- nó meánach'
\end{itemize}

 \noindent \textit{Nótaí Aistriúcháin:}
\begin{itemize}
	\item Cé nach bhfuil an téarma iomlán ar fáil sna foclóirí thuas, táthar in ann é a chumadh go díreach as focail a bhfuil an bhrí chéanna, nó brí chomhchosúil, leo.
	\item Is é `earráid mheán na gcearnóg' atá ar Téarma.ie. Ní ghlactar leis sin toisc gur cosúil gurb ionann sé sin agus earráid(mheán an gcearnóg) -- sin le rá, go bhfuil tomhas earráide comhairthe ar mheán na gcearnóg. Ní shin atá i gceist le `mean square error' ach $me\acute{a}n(earr\acute{a}id^2)$. Meastar go bhfuil `meán na n-earráidí cearnaithe' níos oiriúnaí mar sin sa gcás seo.
	\item Beadh `earráid chearnaithe mheánach' go maith chomh maith mar théarma, ach roghnaíodh `meán na n-earráidí cearnaithe' toisc é a bheith níos léire.
	\item Féach chomh maith ar an téarma `error / earráid'.
	\item Féach chomh maith ar an téarma `mean / meán- nó meánach'.
\end{itemize}


\subsubsection*{measure (ainmfhocal): tomhas}
 \noindent \textit{Sainmhíniú (ga):} I gcomhthéacs matamaitice, luach a dhéanann cur síos cainníochtúil ar shonraí nó ar fheiniméan éigin, go háirithe nuair atá sé úsáidte chun dhá shraith sonraí / dhá fheiniméan chur i gcomparáid lena chéile.
\\
 \noindent \textit{Sainmhíniú (en):} In a mathematical context, a value that gives a quantitative description of data or some phenomenon, especially when used to compare two such data sets or phenomenons.
\\
 \noindent \textit{Tagairtí:}
\begin{itemize}
	\item tomhas: féach ar an téarma `metric / tomhas'
\end{itemize}

 \noindent \textit{Nótaí Aistriúcháin:}
\begin{itemize}
	\item Féach ar an téarma `metric / tomhas'.
\end{itemize}


\subsubsection*{to measure (ainmfhocal): tomhais}
 \noindent \textit{Sainmhíniú (ga):} Tomhas a dhéanamh ar shonraí nó ar phróiseas éigin.
\\
 \noindent \textit{Sainmhíniú (en):} To measure some effect or data.
\\
 \noindent \textit{Tagairtí:}
\begin{itemize}
	\item tomhas: féach ar an téarma `metric / tomhas'
\end{itemize}

 \noindent \textit{Nótaí Aistriúcháin:}
\begin{itemize}
	\item Féach ar an téarma `metric / tomhas'.
\end{itemize}


\subsubsection*{median (aidiacht): airmheánach}
 \noindent \textit{Sainmhíniú (ga):} I gcomhthéacs liosta uimhreacha, bainteach le hairmheán an liosta sin.
\\
 \noindent \textit{Sainmhíniú (en):} In the context of a list of numbers, relating to the median of that list.
\\
 \noindent \textit{Tagairtí:}
\begin{itemize}
	\item meán: féach ar an téarma `mean (statistic) / meán'
	\item meánach: féach ar an téarma `mean / meán- nó meánach'
	\item airmheán: féach ar an téarma `median (statistic) / airmheán'
\end{itemize}

 \noindent \textit{Nótaí Aistriúcháin:}
\begin{itemize}
	\item Níl an téarma `airmheánach' ar fáil i bhfoclóir ar bith. Sin ráite, tá `meán' agus `meánach' ar fáil, chomh maith le `airmheán'. Meastar, mar sin, go bhfuil bunús leis an bhfocal `airmheánach'.
	\item Féach chomh maith ar an téarma `mean (statistic) / meán'.
	\item Féach chomh maith ar an téarma `mean / meán- nó meánach'.
	\item Féach chomh maith ar an téarma `median (statistic) / airmheán'.
\end{itemize}


\subsubsection*{median (statistic) (ainmfhocal): airmheán}
 \noindent \textit{Sainmhíniú (ga):} Ag trácht ar dáileadh, an luach díreach i lár na luachanna ar fad agus iad sórtáilte.
\\
 \noindent \textit{Sainmhíniú (en):} With regards to a distribution, the value directly in the middle of all sorted values.
\\
 \noindent \textit{Tagairtí:}
\begin{itemize}
	\item airmheán: Ó Dónaill (1977) \cite{odonaill}, Williams et al. (2023) \cite{storchiste}
\end{itemize}

 \noindent \textit{Nótaí Aistriúcháin:}
\begin{itemize}
	\item Téarma díreach ar fáil leis an mbrí chéanna sna foclóirí thuas.
\end{itemize}


\subsubsection*{message passing (ainmfhocal): seachadadh teachtaireachtaí}
 \noindent \textit{Sainmhíniú (ga):} I gcomhthéacs ríomhfhoghlama, an próiseas a bhaineann le sonraí a roinnt idir leabuithe i líonra néarach.
\\
 \noindent \textit{Sainmhíniú (en):} In the context of machine learning, the process relating to sharing information between embeddings in a neural network.
\\
 \noindent \textit{Tagairtí:}
\begin{itemize}
	\item seachadadh: De Bhaldraithe (1978) \cite{de-bhaldraithe}, Dineen (1934) \cite{dineen}, Ó Dónaill et al. (1991) \cite{focloir-beag}, Ó Dónaill (1977) \cite{odonaill}
	\item teachtaireacht: De Bhaldraithe (1978) \cite{de-bhaldraithe}, Dineen (1934) \cite{dineen}, Ó Dónaill et al. (1991) \cite{focloir-beag}, Ó Dónaill (1977) \cite{odonaill}
\end{itemize}

 \noindent \textit{Nótaí Aistriúcháin:}
\begin{itemize}
	\item Tá `seachadadh' agus `teachtaireacht' le fáil go díreach ó na foclóirí thuas i gcomhthéacs comhchosúil.
	\item Nuair a dhéantar trácht ar `message passing', is níos mó na teachtaireacht amháin a bhíonn i gceist beagnach i gcónaí. Cuireadh `teachtaireacht' san uimhir iolra mar sin.
	\item Bheadh ciall le `seoladh' seachas `seachadadh' chomh maith. Sin ráite, níl locht ar bith ar `seachadadh', agus is é sin atá ar Téarma.ie. Roghnaítear le go luífeadh an téarma seo lena bhfuil ar Téarma.ie.
\end{itemize}


\subsubsection*{metric (ainmfhocal): tomhas}
 \noindent \textit{Sainmhíniú (ga):} I gcomhthéacs matamaitice, luach a dhéanann cur síos cainníochtúil ar shonraí nó ar fheiniméan éigin, go háirithe nuair atá sé úsáidte chun dhá shraith sonraí / dhá fheiniméan chur i gcomparáid lena chéile.
\\
 \noindent \textit{Sainmhíniú (en):} In a mathematical context, a value that gives a quantitative description of data or some phenomenon, especially when used to compare two such data sets or phenomenons.
\\
 \noindent \textit{Tagairtí:}
\begin{itemize}
	\item tomhas: De Bhaldraithe (1978) \cite{de-bhaldraithe}, Dineen (1934) \cite{dineen}, Ó Dónaill et al. (1991) \cite{focloir-beag}, Ó Dónaill (1977) \cite{odonaill}
\end{itemize}

 \noindent \textit{Nótaí Aistriúcháin:}
\begin{itemize}
	\item Tá an téarma seo comhchiallach leis an téarma `measure / tomhas' sa gcomhthéacs matamaitice / ríomheolaíochta atá i gceist anseo.
	\item Féach chomh maith ar an téarma `measure / tomhas'.
\end{itemize}


\subsubsection*{minimum (ainmfhocal): íosluach}
 \noindent \textit{Sainmhíniú (ga):} An luach is ísle ar féidir a fháil (mar shampla, le linn feabhsaithe samhla ríomhfhoghlama).
\\
 \noindent \textit{Sainmhíniú (en):} The lowest value that can be obtained (for example, during the optimisation of a machine learning model).
\\
 \noindent \textit{Tagairtí:}
\begin{itemize}
	\item íosluach: De Bhaldraithe (1978) \cite{de-bhaldraithe}, Ó Dónaill et al. (1991) \cite{focloir-beag}, Ó Dónaill (1977) \cite{odonaill}
	\item íos-: De Bhaldraithe (1978) \cite{de-bhaldraithe}, Ó Dónaill et al. (1991) \cite{focloir-beag}, Ó Dónaill (1977) \cite{odonaill}, Williams et al. (2023) \cite{storchiste}
	\item luach: De Bhaldraithe (1978) \cite{de-bhaldraithe}, Dineen (1934) \cite{dineen}, Ó Dónaill et al. (1991) \cite{focloir-beag}, Ó Dónaill (1977) \cite{odonaill}, Williams et al. (2023) \cite{storchiste}
\end{itemize}

 \noindent \textit{Nótaí Aistriúcháin:}
\begin{itemize}
	\item Téarma luaite mar théarma matamaitice i bhFoclóir Uí Dhónaill agus i bhFoclóir De Bhaldraithe. Luaitear é (gan chomhthéacs) chomh maith i bhFoclóir Uí Dhónaill agus Uí Mhaoileoin. Tá an dá chuid den téarma seo luaite le bríonna comhchosúla chomh maith sna foclóirí eile thuas.
\end{itemize}


\subsubsection*{to model (ainmfhocal): samhlaigh}
 \noindent \textit{Sainmhíniú (ga):} Samhail ríomhfhoghlama nó staitistiúil a chruthú, nó samhail shonraí a chruthú.
\\
 \noindent \textit{Sainmhíniú (en):} To create a (machine learning or statistical) model, or to create a data model.
\\
 \noindent \textit{Tagairtí:}
\begin{itemize}
	\item samhlaigh: De Bhaldraithe (1978) \cite{de-bhaldraithe}, Dineen (1934) \cite{dineen}, Ó Dónaill et al. (1991) \cite{focloir-beag}, Ó Dónaill (1977) \cite{odonaill}
\end{itemize}

 \noindent \textit{Nótaí Aistriúcháin:}
\begin{itemize}
	\item Is i gcomhthéacs smaointeoireachta a luaitear an focal `samhlaigh', seachas i gcomhthéacs ríomhaireachta / matamaitice, sna foclóirí thuas. Ach toisc go nglactar le `samhail' i gcomhthéacs ríomhfhoghlama sa bhfoclóir seo, glactar leis an mbriathar `samhlaigh' chomh maith.
	\item Féach chomh maith ar an téarma `model / samhail'.
\end{itemize}


\subsubsection*{model (ainmfhocal): samhail}
 \noindent \textit{Sainmhíniú (ga):} I gcomhthéacs ríomhfhoghlama, réad matamaiticiúil a úsáideann cur chuige bunaithe ar calcalas chun tasc ríomhfhoghlama a chur i gcrích. I gcomhthéacs sonraí, an fhormáid agus struchtúr ina bhfuil siad léirithe (m.sh. i ngraf nó i dtábla), agus cé chaoi go sonrach atá na sonraí curtha ann (.i. ointeolaíocht graif nó lipéad na gcolún i dtábla).
\\
 \noindent \textit{Sainmhíniú (en):} In the context of machine learning, a mathematical object that uses a calculus-based approach to solve a machine learning task. In the context of data, the format or structure in which it is contained (ex. in a graph or in a table), as well as precisely how the data is put in it (i.e. the graph's ontology, or the column labels in a table).
\\
 \noindent \textit{Tagairtí:}
\begin{itemize}
	\item samhail: De Bhaldraithe (1978) \cite{de-bhaldraithe}, Dineen (1934) \cite{dineen}, Ó Dónaill et al. (1991) \cite{focloir-beag}, Ó Dónaill (1977) \cite{odonaill}, Williams et al. (2023) \cite{storchiste}
\end{itemize}

 \noindent \textit{Nótaí Aistriúcháin:}
\begin{itemize}
	\item Ní luann foclóir ar bith an téarma seo i gcomhthéacs ríomheolaíochta, ach i gcomhthéacs eile ina bhfuil `samhail' cosúil le `cóip' nó le `cosúlacht'. Sin ráite, tá an bhrí sin oiriúnach don úsáid seo -- cé nach cóip í, bíonn an tsamhail ríomheolaíochta ag iarraidh aschur feidhme matamaiticiúla a réamhinsint go díreach; sin le rá, a chóipeáil.
\end{itemize}


\subsubsection*{modular (aidiacht): modúlach}
 \noindent \textit{Sainmhíniú (ga):} Ag trácht ar samhail nó ar próiseas, a bhfuil modúl mar chuid de. Is féidir a rá go bhfuil cuid de shamhail nó de phróiseas, atá mar mhodúl é féin, modúlach (.i. cuid mhodúlach).
\\
 \noindent \textit{Sainmhíniú (en):} Regarding a model or process, having modules as parts of it. A part of a model or process can be said to be modular if it is itself a module (i.e. a modular component).
\\
 \noindent \textit{Tagairtí:}
\begin{itemize}
	\item modúlach: Ó Dónaill (1977) \cite{odonaill}
\end{itemize}

 \noindent \textit{Nótaí Aistriúcháin:}
\begin{itemize}
	\item Téarma díreach ar fáil leis an mbrí chéanna.
	\item Féach chomh maith ar an téarma `module / modúl'.
\end{itemize}


\subsubsection*{module (ainmfhocal): modúl}
 \noindent \textit{Sainmhíniú (ga):} Cuid de shamhail nó de phróiseas a bhfuil feidhm ar leith aige agus ar féidir é a úsáid i samhail nó i bpróiseas eile, gan é athrú, chun an tasc céanna a dhéanamh.
\\
 \noindent \textit{Sainmhíniú (en):} Part of a model or process that has a specific function and that can be used in other models or processes, without changing it, to do the same task.
\\
 \noindent \textit{Tagairtí:}
\begin{itemize}
	\item modúl: Ó Dónaill et al. (1991) \cite{focloir-beag}, Ó Dónaill (1977) \cite{odonaill}, Williams et al. (2023) \cite{storchiste}
\end{itemize}

 \noindent \textit{Nótaí Aistriúcháin:}
\begin{itemize}
	\item Téarma díreach ar fáil ó na foclóirí thuas.
\end{itemize}


\subsubsection*{motif (ainmfhocal): móitíf}
 \noindent \textit{Sainmhíniú (ga):} I gcomhthéacs graif, patrún struchtúr (go háirithe ceann beag) a fheictear go minic i gcodanna ar leith den ghraf.
\\
 \noindent \textit{Sainmhíniú (en):} In the context of a graph, a structural pattern (especially a small one) that repeats often in different parts of the graph.
\\
 \noindent \textit{Tagairtí:}
\begin{itemize}
	\item móitíf: De Bhaldraithe (1978) \cite{de-bhaldraithe}, Ó Dónaill (1977) \cite{odonaill}
\end{itemize}

 \noindent \textit{Nótaí Aistriúcháin:}
\begin{itemize}
	\item Luann Foclóir De Bhaldraithe go leor téarmaí ar leith a chuireann `motif' in iúl i nGaeilge: `aontréith', `príomhsmaoineamh', `bunábhar (scéil)', agus `móitíf'. (Tá idir `príomhsmaoineamh' agus `móitíf' i bhFoclóir Uí Dhónaill.) Níl `príomhsmaoineamh' ná `bunábhar (scéil)' oiriúnach do chomhthéacs na ngraf. Cé go mbaineann `motif' le tréithe de ghraf, bíonn níos mó ná `motif' amháin i ngach uile ghraf, agus bíonn gach uile `motif' ath-úsáidte go minic sa ngraf céanna. Ní oireann an réimír `aon-' leis an gcomhthéacs seo, mar sin. Thairis sin, ní ghlactar le `tréith' toisc é a bheith ró-leathan mar fhocal -- is iomaí tréithe atá ag graif nach `motifs' iad. Ar deireadh, is cosúil go dtagann an focal `móitíf' as an bhfocal Béarla `motif', agus mar sin is dócha go bhfuil brí chomhchosúil leis (cé nach mbíonn sé sin sainmhínithe sna foclóirí thuas). Glactar le `móitíf' mar sin.
\end{itemize}


\subsubsection*{multi-headed attention (ainmfhocal): aird il-cheannach}
 \noindent \textit{Sainmhíniú (ga):} I gcomhthéacs ríomhfhoghlama, modúl airde a bhfuil níos mó ná ceann airde amháin air.
\\
 \noindent \textit{Sainmhíniú (en):} In the context of computer science, an attention module with more than one head.
\\
 \noindent \textit{Tagairtí:}
\begin{itemize}
	\item aird: féach ar an téarma `attention / aird'
	\item il-: De Bhaldraithe (1978) \cite{de-bhaldraithe}, Dineen (1934) \cite{dineen}, Ó Dónaill et al. (1991) \cite{focloir-beag}, Ó Dónaill (1977) \cite{odonaill}
	\item ceann: féach ar an téarma `attention head / ceann airde'
	\item décheannach: Ó Dónaill (1977) \cite{odonaill}
\end{itemize}

 \noindent \textit{Nótaí Aistriúcháin:}
\begin{itemize}
	\item Is minic agus an iarmhír `-ach' úsáidte chun chur in iúl go bhfuil rud éigin ag rud eile. Mar shampla, is ionann rud `déadach' agus rud a bhfuil déada aige (féach ar Fhoclóir Uí Dhónaill). Thairis sin, tá an focal `décheannach $\rightarrow$ two-headed, two-ended' i bhFoclóir Uí Dhónaill, rud a chuireann in iúl go bhfuil bunús leis an gcaoi a úsáidtear `il-cheannach' anseo.
	\item Féach chomh maith ar an téarma `attention / aird'.
	\item Féach chomh maith ar an téarma `attention head / ceann airde'.
\end{itemize}


\phantomsection \subsection*{N}
\addcontentsline{toc}{subsection}{N}
\markboth{N}{N}

\subsubsection*{n-shot (aidiacht): n-sonra}
 \noindent \textit{Sainmhíniú (ga):} Cur chuige mion-fheabsaithe ina bhfuil samhail réamh-thraenáilte in ann $n$ sonra ó thacar sonraí nua a fheiceáil le linn á mion-fheabhsaithe.
\\
 \noindent \textit{Sainmhíniú (en):} A finetuning protocol in which the pretrained model is able to see $n$ data points from the new data set during finetuning.
\\
 \noindent \textit{Tagairtí:}
\begin{itemize}
	\item sonra: féach ar an téarma `data / sonraí'
\end{itemize}

 \noindent \textit{Nótaí Aistriúcháin:}
\begin{itemize}
	\item Úsáidtear `n' i gcomhthéacs matamaiticiúil chun uimhir éigin (gan sonrú) a chur in iúl.
	\item Úsáidtear n-sonra seachas n-trialach / n-iarrachta nó eile toisc é sin a bheith níos léire. Cuireann `n-sonra' béim ar an méid sonraí atá ar fáil, seachas ar an méid iarrachtaí atá ceadaithe, chun rud a fhoghlaim.
\end{itemize}


\subsubsection*{negative (of number) (aidiacht): diúltach}
 \noindent \textit{Sainmhíniú (ga):} ag caint ar uimhir, faoi 0.
\\
 \noindent \textit{Sainmhíniú (en):} regarding a number, below 0.
\\
 \noindent \textit{Tagairtí:}
\begin{itemize}
	\item diúltach: De Bhaldraithe (1978) \cite{de-bhaldraithe}, Dineen (1934) \cite{dineen}, Ó Dónaill et al. (1991) \cite{focloir-beag}, Ó Dónaill (1977) \cite{odonaill}
\end{itemize}

 \noindent \textit{Nótaí Aistriúcháin:}
\begin{itemize}
	\item Téarma díreach ar fáil sna foclóirí thuas.
	\item Níor cheart an téarma seo a úsáid chun trácht a dhéanamh ar `negative samples' ná ar `negative sampling' (srl).
	\item Féach chomh maith ar an téarma `negative (sample) / frith-shampla'.
	\item Féach chomh maith ar an téarma `negative sampler / frith-shamplóir'.
\end{itemize}


\subsubsection*{negative (sample) (ainmfhocal): frith-shampla}
 \noindent \textit{Sainmhíniú (ga):} I gcomhthéacs graf eolais, abairt thriarach bhréagach a úsáidtear mar fhrith-shampla.
\\
 \noindent \textit{Sainmhíniú (en):} In the context of knowledge graphs, a fake triple that is used as a counterexample.
\\
 \noindent \textit{Tagairtí:}
\begin{itemize}
	\item frith-: féach ar an téarma `counterexample / frith-shampla'
	\item sampla: féach ar an téarma `sample / shampla'
\end{itemize}

 \noindent \textit{Nótaí Aistriúcháin:}
\begin{itemize}
	\item Úsáidtear frith-shampla toisc gurb in, go díreach, a bhfuil i gceist sa gcás seo. Ní uimhir atá ann agus mar sin níl sé `diúltach' in aon chor -- ina ainneoin sin, is sampla bréagach traenála é.
	\item Féach chomh maith ar an téarma `counterexample / frith-shampla'
\end{itemize}


\subsubsection*{negative sampler (ainmfhocal): frith-shamplóir}
 \noindent \textit{Sainmhíniú (ga):} cuid de shamhail leabaithe graif eolais a chruthaíonn frith-shamplaí don tsamhail chéanna.
\\
 \noindent \textit{Sainmhíniú (en):} the part of a knowledge graph embedding model that creates negative samples for the model.
\\
 \noindent \textit{Tagairtí:}
\begin{itemize}
	\item frith-: féach ar an téarma `negative (sample) / frith-shampla'
	\item samplóir: féach ar an téarma `sampler / samplóir'
\end{itemize}

 \noindent \textit{Nótaí Aistriúcháin:}
\begin{itemize}
	\item Ní úsáidtear `*samplóir diúltach' toisc gur samplóir é seo a chruthaíonn frith-shamplaí. Níl tada `diúltach' ag baint leis de réir mar a úsáidtear an focal `diúltach' i nGaeilge.
	\item Féach chomh maith ar an téarma `negative (sample) / frith-shampla',
	\item Féach chomh maith ar an téarma `sampler / samplóir'.
\end{itemize}


\subsubsection*{negative sampling (ainmfhocal): frith-shampláil}
 \noindent \textit{Sainmhíniú (ga):} An próiseas a bhaineann le frith-shamplaí a chruthú bunaithe ar shamplaí fírinneacha.
\\
 \noindent \textit{Sainmhíniú (en):} The process relating to the creation of negative samples based on ground truth samples.
\\
 \noindent \textit{Tagairtí:}
\begin{itemize}
	\item frith-: féach ar an téarma `negative sampler / frith-shamplóir'
	\item sampláil: féach ar an téarma `to sample / sampláil'
\end{itemize}

 \noindent \textit{Nótaí Aistriúcháin:}
\begin{itemize}
	\item Is ainmfhocal amháin é seo. Cé go bhfuil `sampláil' ina briathar, níor cheart `frith-shampláil' a úsáid mar bhriathar. Ní dhéantar i litríocht an réimse i mBéarla é sin, agus is ar éigin a bheadh an úsáid sin de dhíth i nGaeilge ach an oiread.
	\item Féach chomh maith ar an téarma `negative sampler / frith-shamplóir'.
	\item Féach chomh maith ar an téarma `to sample / sampláil'.
\end{itemize}


\subsubsection*{neighbour (ainmfhocal): comharsa}
 \noindent \textit{Sainmhíniú (ga):} I gcomhthéacs nóid i ngraf, ceann ar bith de na nóid eile atá ceangailte leis.
\\
 \noindent \textit{Sainmhíniú (en):} In the context of a node in a graph, any one of the nodes that is connected to it.
\\
 \noindent \textit{Tagairtí:}
\begin{itemize}
	\item comharsa: De Bhaldraithe (1978) \cite{de-bhaldraithe}, Dineen (1934) \cite{dineen}, Ó Dónaill et al. (1991) \cite{focloir-beag}, Ó Dónaill (1977) \cite{odonaill}
\end{itemize}

 \noindent \textit{Nótaí Aistriúcháin:}
\begin{itemize}
	\item Ní luann foclóir ar bith an téarma seo i gcomhthéacs matamaitice / ríomheolaíochta. Úsáidtear an téarma Béarla `neighbour' chun trácht a dhéanamh ar nóid atá ceangailte le nód eile toisc gurb é `neighbour' ná duine a chónaíonn díreach in aici le duine eile. Tá ciall leis an analach sin i nGaeilge chomh maith. Glactar leis an téarma seo mar sin.
\end{itemize}


\subsubsection*{network (ainmfhocal): líonra}
 \noindent \textit{Sainmhíniú (ga):} Tacar ina bhfuil nóid agus na ceangail eatarthu.
\\
 \noindent \textit{Sainmhíniú (en):} A set of nodes and the connections between them.
\\
 \noindent \textit{Tagairtí:}
\begin{itemize}
	\item líonra: De Bhaldraithe (1978) \cite{de-bhaldraithe}, Dineen (1934) \cite{dineen}, Ó Dónaill et al. (1991) \cite{focloir-beag}, Ó Dónaill (1977) \cite{odonaill}
\end{itemize}

 \noindent \textit{Nótaí Aistriúcháin:}
\begin{itemize}
	\item Ní bhíonn `líonra' luaite i gcomhthéacs ríomhaireachta sna foclóirí thuas, cé go mbíonn sé sa gcaint agus i litríocht chomhaimseartha leis an mbrí sin. Thairis sin, is sórt líonra é líonra néarach, cé nach líonra fisiceach é.
\end{itemize}


\subsubsection*{neural (aidiacht): néarach}
 \noindent \textit{Sainmhíniú (ga):} Ag baint le néaróga (bíodh siad fíor nó saorga) nó le líonraí néaracha.
\\
 \noindent \textit{Sainmhíniú (en):} Relating to nerves (be they real or artificial) or to neural networks.
\\
 \noindent \textit{Tagairtí:}
\begin{itemize}
	\item néarach: Ó Dónaill (1977) \cite{odonaill}
\end{itemize}

 \noindent \textit{Nótaí Aistriúcháin:}
\begin{itemize}
	\item Téarma díreach ar fáil i bhFoclóir Uí Dhónaill (i gcomhthéacs fíor-néaróg amháin).
\end{itemize}


\subsubsection*{neural network (NN) (ainmfhocal): líonra néarach (LN)}
 \noindent \textit{Sainmhíniú (ga):} Cur chuige agus struchtúr ríomhfhoghlama bunaithe ar úsáid néaróg saorga.
\\
 \noindent \textit{Sainmhíniú (en):} An approach to, and structure of, machine learning based on artificial neurons.
\\
 \noindent \textit{Tagairtí:}
\begin{itemize}
	\item líonra: féach ar an téarma `network / líonra'
	\item néarach: féach ar an téarma `neural / néarach'
\end{itemize}

 \noindent \textit{Nótaí Aistriúcháin:}
\begin{itemize}
	\item Féach chomh maith ar an téarma `network / líonra'.
	\item Féach chomh maith ar an téarma `neural / néarach'.
\end{itemize}


\subsubsection*{neuro-symbolic (aidiacht): néar-shiombalach}
 \noindent \textit{Sainmhíniú (ga):} I gcomhthéacs ríomhfhoghlama, bunaithe ar úsáid feabhsúchán uimhriúil agus ar rialacha / ar loighic i gcaoi éigin.
\\
 \noindent \textit{Sainmhíniú (en):} In the context of machine learning, based on the use of numeric optimisation as well as logic / rules in some manner.
\\
 \noindent \textit{Tagairtí:}
\begin{itemize}
	\item néar(a)-: Ó Dónaill et al. (1991) \cite{focloir-beag}, Ó Dónaill (1977) \cite{odonaill}, Williams et al. (2023) \cite{storchiste}
	\item siombalach: féach ar an téarma `symbolic / siombalach'
\end{itemize}

 \noindent \textit{Nótaí Aistriúcháin:}
\begin{itemize}
	\item Tá go leor téarmaí teicniúla cruthaithe as an bhfréamh `néar-' le feiceáil i bhFoclóir Uí Dhónaill agus i Stórchiste. Cé go luann Foclóir Uí Dhónaill agus Uí Mhaoileoin `néar-' agus `néara-' mar réimíreanna, is é `néar-' a fheictear i gcónaí i bhFoclóir Uí Dhuinín agus i Stórchiste (m.sh. sna focail `néarshnáithín' agus `néar-ríog'). Is dá bharr seo a roghnaíodh `néar-shiombalach' seachas `* néara-shiombalach'.
	\item Is é `néar-shiombalach' a roghnaíodh seachas `ríomh-shiombalach' (nó mar sin) toisc go mbaineann mórchuid na modhanna ríomhfhoghlama `neuro-symbolic' le líonraí néaracha.
	\item Féach chomh maith ar an téarma `symbolic / siombalach'.
\end{itemize}


\subsubsection*{node (ainmfhocal): nód}
 \noindent \textit{Sainmhíniú (ga):} Cuid de ghraf a chuireann coincheap, bí, nó ainmfhocal in iúl.
\\
 \noindent \textit{Sainmhíniú (en):} An element of a graph that represents a concept, being, or noun.
\\
 \noindent \textit{Tagairtí:}
\begin{itemize}
	\item nód: De Bhaldraithe (1978) \cite{de-bhaldraithe}, Dineen (1934) \cite{dineen}*, Ó Dónaill et al. (1991) \cite{focloir-beag}*, Ó Dónaill (1977) \cite{odonaill}
\end{itemize}

 \noindent \textit{Nótaí Aistriúcháin:}
\begin{itemize}
	\item * Sna foclóirí seo, déantar tagairt don fhocal `nód' mar nód i bplandaí (amháin) gan trácht ar comhthéacs níos leithne.
	\item Focal díreach ar fáil ó na foclóirí thuas.
\end{itemize}


\subsubsection*{noise (ainmfhocal): torann}
 \noindent \textit{Sainmhíniú (ga):} I gcomhthéacs ríomhfhoghlama, an chuid de thacar sonraí atá (i bpáirt nó go hiomlán) randamach, agus nach bhfuil comhartha in-fhoghlamtha inti dá bharr sin.
\\
 \noindent \textit{Sainmhíniú (en):} In the context of machine learning, the part of a data set that is (partially or fully) random, and that does not contain signal for learning as a result.
\\
 \noindent \textit{Tagairtí:}
\begin{itemize}
	\item torann: De Bhaldraithe (1978) \cite{de-bhaldraithe}, Dineen (1934) \cite{dineen}, Ó Dónaill et al. (1991) \cite{focloir-beag}, Ó Dónaill (1977) \cite{odonaill}
\end{itemize}

 \noindent \textit{Nótaí Aistriúcháin:}
\begin{itemize}
	\item Is i gcomhthéacs fuaime amháin a bhíonn an téarma seo luaite sna foclóirí thuas. Sin ráite, úsáidtear i gcomhthéacsanna comhchosúla é -- mar shampla, an frása `bodhar ó thorann an tráchta, deaf from the noise of the traffic' i bhFoclóir Uí Dhónaill. Is rud é `torann' atá in ann cur isteach ar chumas cloiste. Sin díreach cosúil leis an gcaoi a cheileann torann i dtacar sonraí an comhartha in-fhoghlamtha. Gglactar leis an téarma seo mar sin.
	\item Féach chomh maith ar an téarma `signal / comhartha'.
\end{itemize}


\subsubsection*{noisy (aidiacht): torannach}
 \noindent \textit{Sainmhíniú (ga):} I gcomhthéacs tacar sonraí, a bhfuil an-chuid torann randamach ann i gcaoi a chuireann bac ar fhoghlaim ar an tacar sonraí sin.
\\
 \noindent \textit{Sainmhíniú (en):} In the context of a data set, having a lot of random noise that limits learning on that data set.
\\
 \noindent \textit{Tagairtí:}
\begin{itemize}
	\item torannach: Ó Dónaill et al. (1991) \cite{focloir-beag}, Ó Dónaill (1977) \cite{odonaill}
\end{itemize}

 \noindent \textit{Nótaí Aistriúcháin:}
\begin{itemize}
	\item Cé go bhfuil an focal `torannach' i bhFoclóir Uí Dhuinín, is le brí iomlán ar leith atá sé luaite ann.
	\item Féach ar an téarma `noise / torann'.
\end{itemize}


\subsubsection*{non-linear (aidiacht): neamh-líneach}
 \noindent \textit{Sainmhíniú (ga):} I gcomhthéacs matamaitice, gan a bheith líneach.
\\
 \noindent \textit{Sainmhíniú (en):} In the context of mathematics, not linear.
\\
 \noindent \textit{Tagairtí:}
\begin{itemize}
	\item neamh-: De Bhaldraithe (1978) \cite{de-bhaldraithe}, Dineen (1934) \cite{dineen}, Ó Dónaill et al. (1991) \cite{focloir-beag}, Ó Dónaill (1977) \cite{odonaill}, Williams et al. (2023) \cite{storchiste}
	\item líneach: féach ar an téarma `linear / líneach'
\end{itemize}

 \noindent \textit{Nótaí Aistriúcháin:}
\begin{itemize}
	\item Téarma cruthaithe go díreach as na téarmaí thuas.
	\item Féach chomh maith ar an téarma `linear / líneach'.
\end{itemize}


\phantomsection \subsection*{O}
\addcontentsline{toc}{subsection}{O}
\markboth{O}{O}

\subsubsection*{object (ainmfhocal): cuspóir}
 \noindent \textit{Sainmhíniú (ga):} in abairt thriarach $(a,f,c)$, an nód deireanach $c$ atá mar sprioc ag an bhfaisnéis $f$.
\\
 \noindent \textit{Sainmhíniú (en):} in a triple $(s,p,o)$, the final node $o$ that acts as the tail of the predicate $p$.
\\
 \noindent \textit{Tagairtí:}
\begin{itemize}
	\item cuspóir: De Bhaldraithe (1978) \cite{de-bhaldraithe}, Dineen (1934) \cite{dineen}, Ó Dónaill et al. (1991) \cite{focloir-beag}, Ó Dónaill (1977) \cite{odonaill}, Williams et al. (2023) \cite{storchiste}
\end{itemize}

 \noindent \textit{Nótaí Aistriúcháin:}
\begin{itemize}
	\item I mBéarla, samhlaítear abairtí triaracha mar abairtí teangeolaíochta le hainmní, le faisnéis, agus le cuspóir. Glactar leis an analach chéanna i nGaeilge.
\end{itemize}


\subsubsection*{object corruption (ainmfhocal): malartú an chuspóra}
 \noindent \textit{Sainmhíniú (ga):} I gcomhthéacs frith-shamplála, an próiseas a bhaineann le frith-shampla a chruthú tríd an gcuspóir $c$ in abairt thriarach $(a,f,c)$ a ionadú le nód eile.
\\
 \noindent \textit{Sainmhíniú (en):} In the context of negative sampling, the process of creating a negative sample by replacing the object $o$ in a triple $(s,p,o)$ with another node.
\\
 \noindent \textit{Tagairtí:}
\begin{itemize}
	\item malartaigh: féach ar an téarma `to corrupt / malartaigh'
	\item cuspóir: féach ar an téarma `object / cuspóir'
\end{itemize}

 \noindent \textit{Nótaí Aistriúcháin:}
\begin{itemize}
	\item Féach ar an téarma `to corrupt / malartaigh'.
	\item Féach chomh maith ar an téarma `object / cuspóir'.
\end{itemize}


\subsubsection*{object prediction (ainmfhocal): réamhinsint an chuspóra}
 \noindent \textit{Sainmhíniú (ga):} I gcomhthéacs an taisc réamhinsinte nasc, an tasc a bhaineann le cuspóir a réamhinsint chun ceist réamhinsinte nasc sa bhfoirm $(a,f,?)$ a fhreagairt.
\\
 \noindent \textit{Sainmhíniú (en):} In the context of the link prediction task, the task of predicting an object to answer a link prediction query in the form $(s,p,?)$.
\\
 \noindent \textit{Tagairtí:}
\begin{itemize}
	\item réamhinsint: féach ar an téarma `prediction / réamhinsint'
	\item cuspóir: féach ar an téarma `object / cuspóir'
\end{itemize}

 \noindent \textit{Nótaí Aistriúcháin:}
\begin{itemize}
	\item Féach chomh maith ar an téarma `prediction / réamhinsint'.
	\item Féach ar an téarma `object / cuspóir'.
\end{itemize}


\subsubsection*{one-hot encoding (ainmfhocal): códú aon-innéacs}
 \noindent \textit{Sainmhíniú (ga):} Códú a dhéantar ar athróga catagóireacha nuair nach bhfuil slí dhíreach ann chun iad a ionadú le huimhreacha aonaracha. Mar shampla, d'fhéadfadh códú aon-innéacs a dhéanamh ar na luachanna `Frodo, Éowyn, Arwen, Sam' mar `0001, 0010, 0100, 1000'. Ní bhíonn ach an t-aon 1 amháin i ngach uile chódú ann.
\\
 \noindent \textit{Sainmhíniú (en):} Encoding that is done on categorical variables when there is no way to directly represent them as single numbers. For example, one-hot coding could be performed on the values `Frodo, Éowyn, Arwen, Sam' to get `0001, 0010, 0100, 1000'. In each case, the encoding has exactly one 1.
\\
 \noindent \textit{Tagairtí:}
\begin{itemize}
	\item códaigh: De Bhaldraithe (1978) \cite{de-bhaldraithe}, Ó Dónaill et al. (1991) \cite{focloir-beag}, Ó Dónaill (1977) \cite{odonaill}
	\item aon-: De Bhaldraithe (1978) \cite{de-bhaldraithe}, Dineen (1934) \cite{dineen}, Ó Dónaill et al. (1991) \cite{focloir-beag}, Ó Dónaill (1977) \cite{odonaill}
	\item innéacs: De Bhaldraithe (1978) \cite{de-bhaldraithe}, Ó Dónaill et al. (1991) \cite{focloir-beag}, Ó Dónaill (1977) \cite{odonaill}
\end{itemize}

 \noindent \textit{Nótaí Aistriúcháin:}
\begin{itemize}
	\item Is éard atá i gceist le `one-hot encoding' ná códú dénártha nach bhfuil ach an t-aon 1 ceadaithe i ngach uile chódú. I bhfocail eile, bíonn gach uile códú aitheanta de réir an innéacs ina bhfuil an 1 ann. Is mar sin a ghlactar leis an téarma `códú aon-innéacs'.
	\item Cé go luaitear an focal `códaigh' mar `to codify' seachas `to encode', is é `cód' an focal a úsáidtear chun trácht a dhéanamh ar chód ríomhaireachta. Meastar go mba cheart cloí leis anseo mar sin, go háirithe agus gan rogha léir eile ann sna foclóirí dúchasacha.
	\item Is é códú' seachas `códaigh' atá i bhFoclóir Uí Dhónaill agus Uí Mhaoileoin.
\end{itemize}


\subsubsection*{ontology (ainmfhocal): ointeolaíocht}
 \noindent \textit{Sainmhíniú (ga):} I gcomhthéacs graif eolais, scéimre a chuireann in iúl struchtúr loighce an ghraif chéanna (m.sh. cé acu na ceangail atá aistreach nó siméadrach).
\\
 \noindent \textit{Sainmhíniú (en):} In the context of a knowledge graph, a schema that describes the logical structure of the graph (such as which relations are transitive or symmetric).
\\
 \noindent \textit{Tagairtí:}
\begin{itemize}
	\item ointeolaíocht: De Bhaldraithe (1978) \cite{de-bhaldraithe}, Ó Dónaill (1977) \cite{odonaill}
\end{itemize}

 \noindent \textit{Nótaí Aistriúcháin:}
\begin{itemize}
	\item Téarma ar fáil leis an mbrí chéanna sna foclóirí thuas.
\end{itemize}


\subsubsection*{open source (aidiacht): saor-rochtana}
 \noindent \textit{Sainmhíniú (ga):} I gcomhthéacs tionscadail nó foilseacháin, ar fáil go poiblí agus saor in aisce agus gan ceadúnas príobháideach.
\\
 \noindent \textit{Sainmhíniú (en):} In the context of a project or publication, publicly and freely available and not under a private license.
\\
 \noindent \textit{Tagairtí:}
\begin{itemize}
	\item saor-: De Bhaldraithe (1978) \cite{de-bhaldraithe}, Dineen (1934) \cite{dineen}, Ó Dónaill et al. (1991) \cite{focloir-beag}, Ó Dónaill (1977) \cite{odonaill}
	\item rochtain: De Bhaldraithe (1978) \cite{de-bhaldraithe}, Dineen (1934) \cite{dineen}, Ó Dónaill (1977) \cite{odonaill}
\end{itemize}

 \noindent \textit{Nótaí Aistriúcháin:}
\begin{itemize}
	\item Is é `saor-rochtana' a roghnaíodh anseo, seachas `oscailte', toisc go bhfuil `oscailte' ró-litriúil. An bhrí atá leis seo ná go bhfuil rochtain saor air -- agus mar sin is léire agus is simplí `saor-rochtana' ná `oscailte'.
	\item Is é `oscailte' seachas `saor-rochtana' atá ar Téarma.ie os a chomhair seo. Sin ráite, is é `sonraí nasctha saor-rochtana' atá ar Téarma.ie chun trácht a dhéanamh ar `linked open data' -- agus níl sé cinnte cén fáth a roghnaigh siad `oscailte' i gcás amháin agus `saor-rochtana' i gcás eile.
\end{itemize}


\subsubsection*{Open World Assumption (ainmfhocal): Foshuíomh an Domhain Oscailte}
 \noindent \textit{Sainmhíniú (ga):} I gcomhthéacs graf eolais, an foshuíomh a deir go bhfuil líon mór fíricí i réimse graif eolais nach bhfuil sa ngraf eolais toisc go bhfuil i bhfad níos mó fíricí sa domhan nach bhfuil tuiscint againn fós orthu.
\\
 \noindent \textit{Sainmhíniú (en):} In the context of knowledge graphs, the assumption that says that there are many facts in the domain of a knowledge graph which are not in the knowledge graph because most facts about the world remain undiscovered.
\\
 \noindent \textit{Tagairtí:}
\begin{itemize}
	\item foshuíomh: féach ar an téarma `assumption / foshuíomh'
	\item domhan: féach ar an téarma `world / domhan'
	\item oscail: De Bhaldraithe (1978) \cite{de-bhaldraithe}, Dineen (1934) \cite{dineen}, Ó Dónaill et al. (1991) \cite{focloir-beag}*, Ó Dónaill (1977) \cite{odonaill}
\end{itemize}

 \noindent \textit{Nótaí Aistriúcháin:}
\begin{itemize}
	\item Is é `oscailt' seachas `oscail' atá i bhFoclóir Uí Dhuinín agus Uí Mhaoileoin.
	\item Is é `foscail' seachas `oscail' atá i bhFoclóir Uí Dhuinín, ach glactar leis gurb in an focal céanna.
	\item Bheadh ciall éigin ag dul le `Foshuíomh an Eolais Oscailte' nó `Foshuíomh na bhFíricí Oscailte', ach tá fadhb leo siúd -- ní hé go mbíonn na fíricí / an t-eolas oscailte, ach go mbíonn líon na bhfíricí ar fad (nó, `domhan' na bhfíricí) oscailte.
	\item Féach chomh maith ar an téarma `assumption / foshuíomh'.
	\item Féach chomh maith ar an téarma `world / domhan'.
\end{itemize}


\subsubsection*{operation (ainmfhocal): oibriú}
 \noindent \textit{Sainmhíniú (ga):} I gcomhthéacs matamaitice nó ríomheolaíochta, próiseas matamaitice a bhfuil ionchur agus aschur aige. Mar shampla, is oibrithe iad suimiú agus iolrú. Is féidir a rá go bhfuil feidhm ina hoibriú chomh maith.
\\
 \noindent \textit{Sainmhíniú (en):} In the context of mathematics or computer science, a mathematical process that has input and output. For example, addition and multiplication are operations. A function can also be said to be an operation.
\\
 \noindent \textit{Tagairtí:}
\begin{itemize}
	\item oibriú: De Bhaldraithe (1978) \cite{de-bhaldraithe}, Dineen (1934) \cite{dineen}, Ó Dónaill et al. (1991) \cite{focloir-beag}, Ó Dónaill (1977) \cite{odonaill}
\end{itemize}

 \noindent \textit{Nótaí Aistriúcháin:}
\begin{itemize}
	\item Focal luaite mar théarma matamaitice i bhFoclóir Uí Dhónaill agus i bhFoclóir De Bhaldraithe. Sna foclóirí eile, is i gcomhthéacs níos leithne atá sé luaite.
	\item Is é `oibriughadh' atá i bhFoclóir Uí Dhuinín, ach glactar leis gurb in an focal céanna.
\end{itemize}


\subsubsection*{operator (ainmfhocal): oibreoir}
 \noindent \textit{Sainmhíniú (ga):} Siombail nó comhartha a chuireann oibriú in iúl (m.sh. + agus -).
\\
 \noindent \textit{Sainmhíniú (en):} A symbol or character that represents an operation (such as + or -).
\\
 \noindent \textit{Tagairtí:}
\begin{itemize}
	\item oibreoir: Ó Dónaill (1977) \cite{odonaill}, Williams et al. (2023) \cite{storchiste}
\end{itemize}

 \noindent \textit{Nótaí Aistriúcháin:}
\begin{itemize}
	\item Téarma díreach ar fáil mar théarma matamaitice i Stórchiste. I bhFoclóir Uí Dhónaill, ní luaitear comhthéacs leis.
	\item Féach chomh maith ar an téarma `operation / oibriú'.
\end{itemize}


\subsubsection*{optimisation (ainmfhocal): feabhsúchán}
 \noindent \textit{Sainmhíniú (ga):} An próiseas a bhaineann le samhail ríomhfhoghlama a fheabhsú.
\\
 \noindent \textit{Sainmhíniú (en):} The process related to optimising a machine learning model.
\\
 \noindent \textit{Tagairtí:}
\begin{itemize}
	\item feabhsúchán: De Bhaldraithe (1978) \cite{de-bhaldraithe}, Ó Dónaill (1977) \cite{odonaill}
\end{itemize}

 \noindent \textit{Nótaí Aistriúcháin:}
\begin{itemize}
	\item Féach ar an téarma `to optimise / feabhsaigh'.
\end{itemize}


\subsubsection*{to optimise (briathar): feabhsaigh}
 \noindent \textit{Sainmhíniú (ga):} Samhail ríomhfhoghlama a chur chun cinn trína cuid paraiméadar a nuashonrú. Is ionann feabhsú agus foghlaim ar leibhéal matamaiticiúil.
\\
 \noindent \textit{Sainmhíniú (en):} To improve a machine learning model by updating its parameters. At a mathematical level, optimisation is learning.
\\
 \noindent \textit{Tagairtí:}
\begin{itemize}
	\item feabhsaigh: De Bhaldraithe (1978) \cite{de-bhaldraithe}, Dineen (1934) \cite{dineen}, Ó Dónaill et al. (1991) \cite{focloir-beag}*, Ó Dónaill (1977) \cite{odonaill}
\end{itemize}

 \noindent \textit{Nótaí Aistriúcháin:}
\begin{itemize}
	\item Tá an téarma seo ar fáil díreach ó na foclóirí thuas le brí chomhchosúil.
	\item Tá `optamaigh' ar Téarma.ie, ach ní léir ón suíomh sin cén fáth nár leor `feabhsaigh'. Thairis sin, ní bhíonn trácht ar `optamaigh' mar fhocal i bhfoclóir dúchasach ar bith, agus tá an bhrí cheannann chéanna ag `feabhsaigh' sa gcomhthéacs seo. Glactar le `feabhsaigh' seachas le `optamaigh' mar sin.
	\item Is é `feabhsú' (seachas `feabhsaigh') atá i bhFoclóir Uí Dhónaill agus Uí Mhaoileoin.
\end{itemize}


\subsubsection*{optimiser (ainmfhocal): córas feabhsúcháin}
 \noindent \textit{Sainmhíniú (ga):} An córas a dhéanann samhail ríomhfhoghlama a fheabhsú.
\\
 \noindent \textit{Sainmhíniú (en):} The system that optimises a machine learning model.
\\
 \noindent \textit{Tagairtí:}
\begin{itemize}
	\item córas: De Bhaldraithe (1978) \cite{de-bhaldraithe}, Ó Dónaill et al. (1991) \cite{focloir-beag}, Ó Dónaill (1977) \cite{odonaill}
	\item feabhsúchán: féach ar an téarma `optimisation / feabhsúchán'.
\end{itemize}

 \noindent \textit{Nótaí Aistriúcháin:}
\begin{itemize}
	\item Tá réimse leathan téarmaí eile ar fáil a mbeadh brí chomhchosúil leo (córas feabhsaithe, mar shampla), ach meastar gur é `córas feabhsúcháin' an ceann is léire acu sin sa gcomhthéacs seo. Thairis sin, níl `feabhsaitheoir' le feiceáil i bhfoclóir dúchasach ar bith, agus ní ghlactar leis mar sin.
	\item Féach chomh maith ar an téarma `to optimise / feabhsaigh'.
	\item Féach chomh maith ar an téarma `optimisation / feabhsúchán'.
\end{itemize}


\subsubsection*{outlier (ainmfhocal): asluiteach}
 \noindent \textit{Sainmhíniú (ga):} I gcomhthéacs staitistice, luach atá i bhfad amach ó luacha eile agus nach leanann a bpatrún.
\\
 \noindent \textit{Sainmhíniú (en):} In the context of statistics, a value that is very far from all other values, and that does not follow their pattern.
\\
 \noindent \textit{Tagairtí:}
\begin{itemize}
	\item asluiteach: Williams et al. (2023) \cite{storchiste}
\end{itemize}

 \noindent \textit{Nótaí Aistriúcháin:}
\begin{itemize}
	\item Téarma díreach ar fáil mar théarma matamaitice i Stórchiste.
\end{itemize}


\subsubsection*{output (ainmfhocal): aschur}
 \noindent \textit{Sainmhíniú (ga):} I gcomhthéacs córais, próisis, nó feidhme, sonraí a thagann as an bpróiseas céanna agus é críochnaithe.
\\
 \noindent \textit{Sainmhíniú (en):} In the context of a system, process, or function, data that is returned from the process at its end.
\\
 \noindent \textit{Tagairtí:}
\begin{itemize}
	\item aschur: De Bhaldraithe (1978) \cite{de-bhaldraithe}, Ó Dónaill (1977) \cite{odonaill}
\end{itemize}

 \noindent \textit{Nótaí Aistriúcháin:}
\begin{itemize}
	\item Luann Foclóir De Bhaldraithe `aschur' mar théarma teileachumarsáide -- sin le rá, i gcomhthéacs an-chosúil leis an gcomhthéacs seo. Sin ráite, is cosúil go bhfuil an téarma `aschur' (i gcomhthéacs teileachumarsáide) ag trácht ar aschur mar choincheap, seachas mar shonraí nó mar réad ríomhaireachta. Mar sin, is dócha gur cirte a rá `tá dhá uimhir mar aschur ag an bhfeidhm' nó `tá dhá uimhir aschuir ag an bhfeidhm' seachas `* tá dhá aschur ag an bhfeidhm'.
	\item Ní léir ó na foclóirí thuas an féidir briathar a dhéanamh as seo (.i. *aschuir). Mar sin, moltar frása le `aschur' a úsáid nuair atá briathar de dhíth; m.sh. `rinne an fheidhm dhá uimhir a chur amach (mar aschur)'.
	\item Féach chomh maith ar an téarma `to output / cuir amach'.
\end{itemize}


\subsubsection*{to output (ainmfhocal): cuir amach}
 \noindent \textit{Sainmhíniú (ga):} I gcomhthéacs córais, próisis, nó feidhme, sonraí a fháil mar aschur uaidh.
\\
 \noindent \textit{Sainmhíniú (en):} In the context of a system, process, or function, to get data as an output from it.
\\
 \noindent \textit{Tagairtí:}
\begin{itemize}
	\item cuir amach: De Bhaldraithe (1978) \cite{de-bhaldraithe}, Ó Dónaill et al. (1991) \cite{focloir-beag}, Ó Dónaill (1977) \cite{odonaill}
\end{itemize}

 \noindent \textit{Nótaí Aistriúcháin:}
\begin{itemize}
	\item Téarma ar fáil le brí chomhchosúil ó na foclóirí thuas.
	\item Tá go leor slite eile chun é seo a rá. Mar shampla, `chuir an fheidhm X amach', `is é X a bhí mar thoradh ar an bhfeidhm' `bhí X curtha as an bhfeidhm', `rinneadh an fheidhm X a chur amach', srl. Ní ann sa téarma thuas ach sampla amháin.
	\item Cé go ndéanann Foclóir Uí Dhuinín trácht ar an bhfrása seo, úsáideann sé i gcomhthéacsanna ar leith é.
	\item Féach chomh maith ar an téarma `output / aschur'; tá samplaí ann chomh maith ar cé chaoi `to input' a chur in iúl leis an téarma sin.
\end{itemize}


\subsubsection*{overfitting (ainmfhocal): ró-fhoghlaim}
 \noindent \textit{Sainmhíniú (ga):} I gcomhthéacs ríomhfhoghlama, foghlaim de ghlanmheabhair a dhéantar ar an tacar traenála i gcaoi a chuireann bac ar fhoghlaim phatrún ginearálta sa tacar traenála.
\\
 \noindent \textit{Sainmhíniú (en):} In the context of machine learning, memorisation of the training set that precludes learning the general patterns of the training set.
\\
 \noindent \textit{Tagairtí:}
\begin{itemize}
	\item ró-: De Bhaldraithe (1978) \cite{de-bhaldraithe}, Dineen (1934) \cite{dineen}, Ó Dónaill et al. (1991) \cite{focloir-beag}, Ó Dónaill (1977) \cite{odonaill}
	\item foghlaim: féach ar an téarma `machine learning / ríomhfhoghlaim'
\end{itemize}

 \noindent \textit{Nótaí Aistriúcháin:}
\begin{itemize}
	\item Téarma cruthaithe mar chomh-fhocal leis an réimír agus leis an bhfocal thuas.
	\item Féach chomh maith ar an téarma `machine learning / ríomhfhoghlaim'
\end{itemize}


\phantomsection \subsection*{P}
\addcontentsline{toc}{subsection}{P}
\markboth{P}{P}

\subsubsection*{pairwise function (ainmfhocal): feidhm de phéire (pointí)}
 \noindent \textit{Sainmhíniú (ga):} Feidhm a bhfuil péire pointí sonraí mar ionchur aici.
\\
 \noindent \textit{Sainmhíniú (en):} A function that takes a pair of data points as input.
\\
 \noindent \textit{Tagairtí:}
\begin{itemize}
	\item feidhm: féach ar an téarma `function / feidhm'
	\item péire: De Bhaldraithe (1978) \cite{de-bhaldraithe}, Dineen (1934) \cite{dineen}, Ó Dónaill et al. (1991) \cite{focloir-beag}, Ó Dónaill (1977) \cite{odonaill}
	\item pointe: féach ar an téarma `pointwise function / feidhm de phointe'
\end{itemize}

 \noindent \textit{Nótaí Aistriúcháin:}
\begin{itemize}
	\item Féach ar an téarma `pointwise function / feidhm de phointe'.
	\item Féach chomh maith ar an téarma `function / feidhm'.
\end{itemize}


\subsubsection*{parameter (ainmfhocal): paraiméadar}
 \noindent \textit{Sainmhíniú (ga):} I gcomhthéacs samhla ríomhfhoghlama, luach uimhriúil in-fhoghlama a bhíonn ag athrú le linn an phróisis thraenála. Is ionann paraiméadair níos fearr a roghnú do shamhail agus éifeachtacht na samhla a chur chun cinn.
\\
 \noindent \textit{Sainmhíniú (en):} In the context of a machine learning mode, a learnable numerical value that changes during the training process. Choosing better parameters for a model improves the model's performance.
\\
 \noindent \textit{Tagairtí:}
\begin{itemize}
	\item pharaiméadar: De Bhaldraithe (1978) \cite{de-bhaldraithe}, Ó Dónaill (1977) \cite{odonaill}
\end{itemize}

 \noindent \textit{Nótaí Aistriúcháin:}
\begin{itemize}
	\item Téarma díreach ar fáil ó na foclóirí thuas i gcomhthéacs comhchosúil.
	\item Is ionann paraiméadar agus ualach infhoghlamtha.
\end{itemize}


\subsubsection*{percentile (ainmfhocal): peircintíl}
 \noindent \textit{Sainmhíniú (ga):} I gcomhthéacs staitistice, is é an pheircintíl Xú ná an luach i ndáileadh staitistiúil a bhfuil X% de na luachanna eile níos ísle ná é.
\\
 \noindent \textit{Sainmhíniú (en):} In the context of statistics, the Xth percentile is the value in a statistical distribution which has X% of the other values below it.
\\
 \noindent \textit{Tagairtí:}
\begin{itemize}
	\item peircintíl: Williams et al. (2023) \cite{storchiste}
\end{itemize}

 \noindent \textit{Nótaí Aistriúcháin:}
\begin{itemize}
	\item Téarma díreach ar fáil leis an mbrí cheannann chéanna i Stórchiste.
\end{itemize}


\subsubsection*{performance (ainmfhocal): éifeachtacht (ama, taisc)}
 \noindent \textit{Sainmhíniú (ga):} I gcomhthéacs ríomheolaíochta, cé chomh maith is a dhéanann samhail nó próiseas tasc éigin de réir tomhais éigin.
\\
 \noindent \textit{Sainmhíniú (en):} In the context of computer science, how well a model or process does some job as measured by some metric.
\\
 \noindent \textit{Tagairtí:}
\begin{itemize}
	\item éifeachtacht: De Bhaldraithe (1978) \cite{de-bhaldraithe}, Ó Dónaill et al. (1991) \cite{focloir-beag}, Ó Dónaill (1977) \cite{odonaill}
\end{itemize}

 \noindent \textit{Nótaí Aistriúcháin:}
\begin{itemize}
	\item Téarma díreach ar fáil leis an mbrí chéanna ó na foclóirí thuas.
	\item De réir na bhfoclóirí thuas, ní dhéanann an Ghaeilge idirdhealú idir éifeachtacht ama (cé chomh sciobtha is a dhéantar tasc éigin) agus éifeachtacht taisc (cé chomh maith is a dhéantar tasc éigin). Tá an fhadhb céanna ag an mBéarla, mar a tharlaíonn. Mar fhreagra air, moltar tuilleadh comhthéacs a thabhairt nuair is gá (.i. trí `éifeachtacht ama' nó `éifeachtacht taisc' a úsáid).
\end{itemize}


\subsubsection*{plausibility (ainmfhocal): inchreidteacht}
 \noindent \textit{Sainmhíniú (ga):} Airí ruda ar (dócha gur) fíor é.
\\
 \noindent \textit{Sainmhíniú (en):} The property of being (likely) true.
\\
 \noindent \textit{Tagairtí:}
\begin{itemize}
	\item inchreidteacht: De Bhaldraithe (1978) \cite{de-bhaldraithe}, Ó Dónaill (1977) \cite{odonaill}
\end{itemize}

 \noindent \textit{Nótaí Aistriúcháin:}
\begin{itemize}
	\item Téarma ar fáil go díreach ó na foclóirí thuas i gcomhthéacs comhchosúil.
\end{itemize}


\subsubsection*{plausibility score (ainmfhocal): scór inchreidteachta}
 \noindent \textit{Sainmhíniú (ga):} Uimhhir a dhéanann cur síos ar an dóchúlacht go bhfuil rud fíor.
\\
 \noindent \textit{Sainmhíniú (en):} A number that represents the chance that something is true.
\\
 \noindent \textit{Tagairtí:}
\begin{itemize}
	\item scór: féach ar an téarma `score / scór'
	\item inchreidteacht: féach ar an téarma `plausibility / inchreidteacht'
\end{itemize}

 \noindent \textit{Nótaí Aistriúcháin:}
\begin{itemize}
	\item Féach ar an téarma `score / scór'.
	\item Féach chomh maith ar an téarma `plausibility / inchreidteacht'.
\end{itemize}


\subsubsection*{pointwise function (ainmfhocal): feidhm de phointe}
 \noindent \textit{Sainmhíniú (ga):} Feidhm a bhfuil pointe sonraí amháin mar ionchur aici.
\\
 \noindent \textit{Sainmhíniú (en):} A function that takes a single data point as input.
\\
 \noindent \textit{Tagairtí:}
\begin{itemize}
	\item feidhm: féach ar an téarma `function / feidhm'
	\item pointe: De Bhaldraithe (1978) \cite{de-bhaldraithe}, Dineen (1934) \cite{dineen}, Ó Dónaill et al. (1991) \cite{focloir-beag}, Ó Dónaill (1977) \cite{odonaill}, Williams et al. (2023) \cite{storchiste}
\end{itemize}

 \noindent \textit{Nótaí Aistriúcháin:}
\begin{itemize}
	\item Féach ar an téarma `function / feidhm'.
\end{itemize}


\subsubsection*{positive (of number) (aidiacht): deimhneach}
 \noindent \textit{Sainmhíniú (ga):} ag caint ar uimhir, níos mó ná 0.
\\
 \noindent \textit{Sainmhíniú (en):} regarding a number, above 0.
\\
 \noindent \textit{Tagairtí:}
\begin{itemize}
	\item deimhneach: De Bhaldraithe (1978) \cite{de-bhaldraithe}, Ó Dónaill (1977) \cite{odonaill}
\end{itemize}

 \noindent \textit{Nótaí Aistriúcháin:}
\begin{itemize}
	\item Téarma ar fáil leis an mbrí chéanna sna foclóirí thuas.
	\item Níor cheart an téarma seo a úsáid chun trácht ar shamplaí atá fíor (.i. nach bhfuil mar fhrith-shamplaí).
	\item Féach chomh maith ar an téarma `positive (triple)' / fíor-abairt (thriarach).
\end{itemize}


\subsubsection*{positive (triple) (ainmfhocal): fíor-abairt (thriarach)}
 \noindent \textit{Sainmhíniú (ga):} I gcomhthéacs graf eolais, abairt thriarach atá mar chuid de GE, agus a ndéantar frith-shamplaí as.
\\
 \noindent \textit{Sainmhíniú (en):} In the context of knowledge graphs, a ground-truth triple from the KG from which various negatives are generated.
\\
 \noindent \textit{Tagairtí:}
\begin{itemize}
	\item abairt: féach ar an téarma `triple / abairt thriarach'
	\item thriarach: féach ar an téarma `triple / abairt thriarach'
	\item fíor: De Bhaldraithe (1978) \cite{de-bhaldraithe}, Dineen (1934) \cite{dineen}, Ó Dónaill et al. (1991) \cite{focloir-beag}, Ó Dónaill (1977) \cite{odonaill}
\end{itemize}

 \noindent \textit{Nótaí Aistriúcháin:}
\begin{itemize}
	\item Ní bhaineann an téarma seo le huimhreacha deimhneacha, agus níor cheart an aidiacht `deimhneach' a úsáid chun trácht ar fhíor-abairtí triaracha.
	\item Féach chomh maith ar an téarma `counterexample / frith-shampla'.
	\item Féach chomh maith ar an téarma `positive (of number) / deimhneach'.
\end{itemize}


\subsubsection*{predicate (ainmfhocal): faisnéis}
 \noindent \textit{Sainmhíniú (ga):} In abairt thriarach $(a,f,c)$, an ceangal $c$ a cheanglaíonn an t-ainmfhocal $a$ leis an gcuspóir $c$.
\\
 \noindent \textit{Sainmhíniú (en):} In a triple $(s,p,o)$, the predicate $p$ that connects the subject $s$ to the object $o$.
\\
 \noindent \textit{Tagairtí:}
\begin{itemize}
	\item cuspóir: De Bhaldraithe (1978) \cite{de-bhaldraithe}, Dineen (1934) \cite{dineen}, Ó Dónaill et al. (1991) \cite{focloir-beag}, Ó Dónaill (1977) \cite{odonaill}, Williams et al. (2023) \cite{storchiste}
\end{itemize}

 \noindent \textit{Nótaí Aistriúcháin:}
\begin{itemize}
	\item I mBéarla, samhlaítear abairtí triaracha mar abairtí teangeolaíochta le hainmní, le faisnéis, agus le cuspóir. Glactar leis an analach chéanna i nGaeilge.
\end{itemize}


\subsubsection*{to predict (briathar): réamhinis}
 \noindent \textit{Sainmhíniú (ga):} I gcomhthéacs ríomhfhoghlama, meastachán a dhéanamh ar luach sonra éigin.
\\
 \noindent \textit{Sainmhíniú (en):} In the context of computer science, to estimate the value of some data point.
\\
 \noindent \textit{Tagairtí:}
\begin{itemize}
	\item réamhinis: De Bhaldraithe (1978) \cite{de-bhaldraithe}, Ó Dónaill (1977) \cite{odonaill}
\end{itemize}

 \noindent \textit{Nótaí Aistriúcháin:}
\begin{itemize}
	\item Téarma díreach ar fáil le brí chomhchosúil.
	\item Nuair a bhíonn rud á réamhinsint, bíonn samhail ríomhfhoghlama ag iarradh teacht ar an sonra ceart gan ar eolas aici ach sonraí eile (.i. na sonraí traenála). Ní mór di an luach ceart a fháil gan é a fheiceáil -- sin le rá, a insint cén luach atá ann sula bhfeiceann sí é. Cé nach ionann é sin agus sonraí ón todhchaí a réamhinsint, is sórt réamhinsinte fós é.
\end{itemize}


\subsubsection*{prediction (ainmfhocal): réamhinsint}
 \noindent \textit{Sainmhíniú (ga):} Sonra atá réamhinste.
\\
 \noindent \textit{Sainmhíniú (en):} A data point that is predicted.
\\
 \noindent \textit{Tagairtí:}
\begin{itemize}
	\item réamhinsint: De Bhaldraithe (1978) \cite{de-bhaldraithe}, Ó Dónaill (1977) \cite{odonaill}
\end{itemize}

 \noindent \textit{Nótaí Aistriúcháin:}
\begin{itemize}
	\item Tá `réamh-' agus `innsint' i bhFoclóir Uí Dhuinín, ach níl an téarma `réamhinsint' luaite ann.
	\item Tá `réamh-' agus `insint' i bhFoclóir Uí Dhónaill agus Uí Mhaoileoin, ach níl an téarma `réamhinsint' luaite ann.
\end{itemize}


\subsubsection*{predictor (ainmfhocal): réamhinsteoir}
 \noindent \textit{Sainmhíniú (ga):} Córas ríomhfhoghlama a réamhinsíonn sonraí.
\\
 \noindent \textit{Sainmhíniú (en):} A machine learning system that predicts data.
\\
 \noindent \textit{Tagairtí:}
\begin{itemize}
	\item réamhinis: De Bhaldraithe (1978) \cite{de-bhaldraithe}, Ó Dónaill (1977) \cite{odonaill}
	\item insteoir: Ó Dónaill et al. (1991) \cite{focloir-beag}, Ó Dónaill (1977) \cite{odonaill}
\end{itemize}

 \noindent \textit{Nótaí Aistriúcháin:}
\begin{itemize}
	\item Úsáidtear an iarmhír choitianta `-eoir' chun an téarma seo a chruthú. Is féidir an téarma `insteoir' a fheiceáil i bhFoclóir Uí Dhónaill agus i bhFoclóir Uí Dhónaill agus Uí Mhaoileoin, ach níl trácht díreach ar `réamhinsteoir' iontu.
\end{itemize}


\subsubsection*{to pretrain (briathair): réamh-thraenáil}
 \noindent \textit{Sainmhíniú (ga):} Samhail ríomhfhoghlama a thraenáil le plean é a mion-fheabbhsú níos déanaí ar shonraí nua.
\\
 \noindent \textit{Sainmhíniú (en):} To train a machine learning model with intent to finetune it later on new data.
\\
 \noindent \textit{Tagairtí:}
\begin{itemize}
	\item réamh-: féach ar an téarma `pretraining / réamh-thraenáil'
	\item traenáil: féach ar an téarma `training / traenáil'
\end{itemize}

 \noindent \textit{Nótaí Aistriúcháin:}
\begin{itemize}
	\item Féach ar an téarma `pretraining / réamh-thraenáil'
	\item Féach chomh maith ar an téarma training / traenáil'.
\end{itemize}


\subsubsection*{pretraining (ainmfhocal): réamh-thraenáil}
 \noindent \textit{Sainmhíniú (ga):} An próiseas a bhaineann le samhail ríomhfhoghlama a thraenáil le plean é a mion-fheabbhsú níos déanaí ar shonraí nua.
\\
 \noindent \textit{Sainmhíniú (en):} The process of training a machine learning model with intent to finetune it later on new data.
\\
 \noindent \textit{Tagairtí:}
\begin{itemize}
	\item réamh-: De Bhaldraithe (1978) \cite{de-bhaldraithe}, Dineen (1934) \cite{dineen}, Ó Dónaill et al. (1991) \cite{focloir-beag}, Ó Dónaill (1977) \cite{odonaill}
	\item traenáil: féach ar an téarma `training / traenáil'
\end{itemize}

 \noindent \textit{Nótaí Aistriúcháin:}
\begin{itemize}
	\item Téarma cruthaithe as an réimír agus as an bhfocal thuas.
	\item Féach chomh maith ar an téarma `training / traenáil'.
\end{itemize}


\subsubsection*{probability (ainmfhocal): dóchúlacht}
 \noindent \textit{Sainmhíniú (ga):} An seans go dtarlóidh teagmhas randamach.
\\
 \noindent \textit{Sainmhíniú (en):} The chance that a random event will occur.
\\
 \noindent \textit{Tagairtí:}
\begin{itemize}
	\item dóchúlacht: De Bhaldraithe (1978) \cite{de-bhaldraithe}, Dineen (1934) \cite{dineen}*, Ó Dónaill et al. (1991) \cite{focloir-beag}, Ó Dónaill (1977) \cite{odonaill}, Williams et al. (2023) \cite{storchiste}
\end{itemize}

 \noindent \textit{Nótaí Aistriúcháin:}
\begin{itemize}
	\item * Sé `dóigheamhlacht' a fheictear i bhFoclóir Uí Dhuinín, ach meastar gurb in litriú eile ar an bhfocal céanna.
	\item Seachas sin, tá an téarma seo ar fáil go díreach ó na foclóirí thuas.
\end{itemize}


\subsubsection*{proportion (ainmfhocal): comhréir}
 \noindent \textit{Sainmhíniú (ga):} I gcomhthéacs matamaitice, luach ar an eatramh [0,1] a léiríonn cé chomh minic is atá feiniméan éigin, nó cén dóchúlacht atá le teagmhas dóchúlachta.
\\
 \noindent \textit{Sainmhíniú (en):} In the context of mathematics, a value on the interval [0,1] that describes how common some phenomenon is, or what probability some probabilistic event has.
\\
 \noindent \textit{Tagairtí:}
\begin{itemize}
	\item comhréir: De Bhaldraithe (1978) \cite{de-bhaldraithe}, Ó Dónaill et al. (1991) \cite{focloir-beag}, Ó Dónaill (1977) \cite{odonaill}
\end{itemize}

 \noindent \textit{Nótaí Aistriúcháin:}
\begin{itemize}
	\item Téarma díreach ar fáil ó na foinsí thuas. I bhFoclóir Uí Dhónaill agus i bhFoclóir De Bhaldraithe, luaitear é leis an mbrí cheannann chéanna i gcomhthéacs matamaitice.
	\item Cé go bhfuil an focal `cóimhréir' i bhFoclóir Uí Dhuinín, is le brí ar leith atá sé luaite.
\end{itemize}


\subsubsection*{pseudotype (ainmfhocal): aicme chumtha}
 \noindent \textit{Sainmhíniú (ga):} I gcomhthéacs nód i ngraf eolais, aicme measta atá curtha leis (nach dtagann as ointeolaíocht an ghraif, má tá ceann ann).
\\
 \noindent \textit{Sainmhíniú (en):} In the context of a node in a knowledge graph, an estimated class type that is attached to it (wichh does not come from the graph's ontology, if it has one).
\\
 \noindent \textit{Tagairtí:}
\begin{itemize}
	\item cum: De Bhaldraithe (1978) \cite{de-bhaldraithe}, Dineen (1934) \cite{dineen}, Ó Dónaill (1977) \cite{odonaill}
	\item aicme: féach ar an téarma `class / aicme'
\end{itemize}

 \noindent \textit{Nótaí Aistriúcháin:}
\begin{itemize}
	\item Ní úsáidtear `bréag' mar `pseudo' sa gcás seo toisc go bhfuil an téarma sin níos diúltaí i nGaeilge. Ní hé gur bréaga iad na haicmí measta, ach amháin nach dtagann siad as ointeolaíocht an ghraif.
	\item Féach chomh maith ar an téarma `class / aicme'.
\end{itemize}


\subsubsection*{pseudotyped (aidiacht): le haicme chumtha}
 \noindent \textit{Sainmhíniú (ga):} I gcomhthéacs nód i ngraf eolais, le haicme chumtha curtha leis.
\\
 \noindent \textit{Sainmhíniú (en):} In the context of a node in a knowledge graph, having a pseudotype.
\\
 \noindent \textit{Tagairtí:}
\begin{itemize}
	\item aicme chumtha: féach ar an téarma `pseudotype / aicme chumtha'
\end{itemize}

 \noindent \textit{Nótaí Aistriúcháin:}
\begin{itemize}
	\item Má tá tú ag trácht ar ghraf eolais iomlán, seachas ar nód amháin, ba cheart `le haicmí cumtha' a úsáid, toisc go mbíonn níos mó ná aicme amháin cumtha agus aicmí cumtha á gcur leis na nóid ar fad i ngraf eolais.
	\item Féach chomh maith ar an téarma `pseudotype / aicme chumtha'.
\end{itemize}


\phantomsection \subsection*{Q}
\addcontentsline{toc}{subsection}{Q}
\markboth{Q}{Q}

\subsubsection*{qualitative (aidiacht): cineálach}
 \noindent \textit{Sainmhíniú (ga):} I gcomhthéacs sonraí, bunaithe ar fhaisnéis nó ar ghnéithe atá do-chainníochtaithe nó nach bhfuil cainníochtaithe.
\\
 \noindent \textit{Sainmhíniú (en):} In the context of data, based on unquantifiable or unquantified traits or information.
\\
 \noindent \textit{Tagairtí:}
\begin{itemize}
	\item cineálach: De Bhaldraithe (1978) \cite{de-bhaldraithe}, Ó Dónaill (1977) \cite{odonaill}
\end{itemize}

 \noindent \textit{Nótaí Aistriúcháin:}
\begin{itemize}
	\item Téarma díreach ar fáil leis an mbrí cheannann chéanna.
\end{itemize}


\subsubsection*{to quantify (briathar): cainníochtaigh}
 \noindent \textit{Sainmhíniú (ga):} Cur síos uimhriúil a dhéanamh ar fheiniméan éigin.
\\
 \noindent \textit{Sainmhíniú (en):} To give a numerical description of a phenomenon.
\\
 \noindent \textit{Tagairtí:}
\begin{itemize}
	\item cainníochtaigh: Ó Dónaill (1977) \cite{odonaill}
\end{itemize}

 \noindent \textit{Nótaí Aistriúcháin:}
\begin{itemize}
	\item Téarma díreach ar fáil i bhFoclóir Uí Dhónaill leis an mbrí cheannann chéanna.
\end{itemize}


\subsubsection*{quantitative (aidiacht): cainníochtúil}
 \noindent \textit{Sainmhíniú (ga):} I gcomhthéacs sonraí, bunaithe ar fhaisnéis chainníochtaithe.
\\
 \noindent \textit{Sainmhíniú (en):} In the context of data, based on quantified information.
\\
 \noindent \textit{Tagairtí:}
\begin{itemize}
	\item cainníochtúil: De Bhaldraithe (1978) \cite{de-bhaldraithe}, Ó Dónaill (1977) \cite{odonaill}
\end{itemize}

 \noindent \textit{Nótaí Aistriúcháin:}
\begin{itemize}
	\item Téarma díreach ar fáil leis an mbrí cheannann chéanna.
\end{itemize}


\subsubsection*{query (ainmfhocal): ceist}
 \noindent \textit{Sainmhíniú (ga):} I gcomhthéacs bunachar sonraí, ordú a chuirtear ar an mbunachar i bhfoirm cód chun freagra a fháil uaidh. Mar shampla, is féidir ceist a chur ar ghraf eolais faoi an bhfuil abairt thriarach éigin sa ngraf, nó nach bhfuil.
\\
 \noindent \textit{Sainmhíniú (en):} In the context of a database, an order that is given to the database in the form of code to get an answer in return. For example, a query could be put to a knowledge graph asking if a specific triple is a part of the graph, or if it is not.
\\
 \noindent \textit{Tagairtí:}
\begin{itemize}
	\item ceist: De Bhaldraithe (1978) \cite{de-bhaldraithe}, Dineen (1934) \cite{dineen}, Ó Dónaill et al. (1991) \cite{focloir-beag}, Ó Dónaill (1977) \cite{odonaill}
\end{itemize}

 \noindent \textit{Nótaí Aistriúcháin:}
\begin{itemize}
	\item Téarma díreach ar fáil le brí chomhchosúil.
\end{itemize}


\phantomsection \subsection*{R}
\addcontentsline{toc}{subsection}{R}
\markboth{R}{R}

\subsubsection*{random (aidiacht): randamach}
 \noindent \textit{Sainmhíniú (ga):} Ag caint ar próiseas, gan bheith in ann é a réamhinsint ach le dóchúlachtaí.
\\
 \noindent \textit{Sainmhíniú (en):} Regarding a process, unable to be predicted except with probabilities.
\\
 \noindent \textit{Tagairtí:}
\begin{itemize}
	\item randamach: Williams et al. (2023) \cite{storchiste}
\end{itemize}

 \noindent \textit{Nótaí Aistriúcháin:}
\begin{itemize}
	\item Is le brí mhatamaiticiúil a luaitear an téarma seo i Stórchiste.
\end{itemize}


\subsubsection*{random number generator (ainmfhocal): gineadóir uimhreacha randamacha}
 \noindent \textit{Sainmhíniú (ga):} I gcomhthéacs ríomheolaíochta, algartam a úsáideann uimhir fréimhe chun sraith uimhreacha randamacha a chumadh agus a chur amach.
\\
 \noindent \textit{Sainmhíniú (en):} In the context of computer science, an algorithm that uses a seed number to generate and output a series of random numbers.
\\
 \noindent \textit{Tagairtí:}
\begin{itemize}
	\item gineadóir: De Bhaldraithe (1978) \cite{de-bhaldraithe}, Ó Dónaill et al. (1991) \cite{focloir-beag}, Ó Dónaill (1977) \cite{odonaill}
	\item uimhir: De Bhaldraithe (1978) \cite{de-bhaldraithe}, Dineen (1934) \cite{dineen}, Ó Dónaill et al. (1991) \cite{focloir-beag}, Ó Dónaill (1977) \cite{odonaill}, Williams et al. (2023) \cite{storchiste}
	\item randamach: féach ar an téarma `random / randamach'
\end{itemize}

 \noindent \textit{Nótaí Aistriúcháin:}
\begin{itemize}
	\item Úsáidtear `gineadóir' toisc é a bheith luaite mar fhocal ní hamháin ar ghiniúint pháistí, ach ar ghiniúint ghinearálta chomh maith (m.sh. féach ar `generator' i bhFoclóir De Bhaldraithe).
\end{itemize}


\subsubsection*{random sample (ainmfhocal): sampla fánach}
 \noindent \textit{Sainmhíniú (ga):} Sampla a thógtar go randamach.
\\
 \noindent \textit{Sainmhíniú (en):} A sample that is taken randomly.
\\
 \noindent \textit{Tagairtí:}
\begin{itemize}
	\item sampla: féach ar an téarma `sample / sampla'
	\item sampla fánach: De Bhaldraithe (1978) \cite{de-bhaldraithe}, Ó Dónaill et al. (1991) \cite{focloir-beag}, Ó Dónaill (1977) \cite{odonaill}
\end{itemize}

 \noindent \textit{Nótaí Aistriúcháin:}
\begin{itemize}
	\item Tá an téarma seo ina iomlán ar fáil sna foclóirí thuas. Toisc é a bheith ar fáil ina iomlán, ní úsáidtear `randamach' sa gcás seo.
\end{itemize}


\subsubsection*{random search (ainmfhocal): cuardach randamach}
 \noindent \textit{Sainmhíniú (ga):} I gcomhthéacs cuardaithe hipear-pharaiméadar, cuardach a dhéantar ar thacar mór teaglamaí hipear-pharaiméadar, agus ina mbíonn teaglamaí sampláilte go randamach go dtí go bhfuil coinníoll críochnaithe bainte amach.
\\
 \noindent \textit{Sainmhíniú (en):} In the context of a hyperparameter search, a search that is done on a large set of possible hyperparameter combinations, and in which hyperparameter combinations are sampled randomly until some end condition is reached.
\\
 \noindent \textit{Tagairtí:}
\begin{itemize}
	\item cuardach: féach ar an téarma `hyperparameter search / cuardach hipear-pharaiméadar'
	\item randamach: féach ar an téarma `random / randamach'
\end{itemize}

 \noindent \textit{Nótaí Aistriúcháin:}
\begin{itemize}
	\item Féach ar an téarma `hyperparameter search / cuardach hipear-pharaiméadar'.
	\item Féach chomh maith ar an téarma `random / randamach'.
	\item Féach chomh maith ar an téarma `end condition / coinníoll críochnaithe'.
\end{itemize}


\subsubsection*{random seed (ainmfhocal): fréamh randamach}
 \noindent \textit{Sainmhíniú (ga):} I gcomhthéacs ríomheolaíochta, uimhir a thugtar do ghineadóir uimhreacha randamacha lena bheith úsáidte ina algartam chun sraith uimhreacha randamacha a chumadh.
\\
 \noindent \textit{Sainmhíniú (en):} In the context of computer science, a number that is given to a random number generator to be used in its algorithm to generate a series of random numbers.
\\
 \noindent \textit{Tagairtí:}
\begin{itemize}
	\item fréamh: De Bhaldraithe (1978) \cite{de-bhaldraithe}, Dineen (1934) \cite{dineen}, Ó Dónaill et al. (1991) \cite{focloir-beag}, Ó Dónaill (1977) \cite{odonaill}
	\item randamach: féach ar an téarma `random / randamach'
\end{itemize}

 \noindent \textit{Nótaí Aistriúcháin:}
\begin{itemize}
	\item Ní bheadh ciall ar bith ag dul le `síol randamach' (nó mar sin) toisc nach síol litriúil é `random seed'. Is uimhir í a úsáidtear chun sraith ollmhór uimhreacha randamacha a chruthú. I bhfocail eile, is é an `random seed' fréamh na sraithe randamaí. Glactar leis an bhfocal `fréamh', mar sin, seachas lena leithéid de `síol'.
	\item Bheadh ciall le téarmaí eile (m.sh `bunús randamach'). Roghnaíodh `fréamh' ní toisc go bhfuil sé níos fearr ná na roghanna eile, ach toisc gur téarma léir (amháin) é sa gcomhthéacs seo.
	\item Féach chomh maith ar an téarma `random / randamach'.
\end{itemize}


\subsubsection*{random walk (aidiacht): siúlóid fhánach}
 \noindent \textit{Sainmhíniú (ga):} I gcomhthéacs siúlóid ar graf, siúlóid air a dhéantar go randamach.
\\
 \noindent \textit{Sainmhíniú (en):} In the context of a walk on a graph, a walk that is done upon it randomly.
\\
 \noindent \textit{Tagairtí:}
\begin{itemize}
	\item siúlóid: féach ar an téarma `walk / siúlóid'
	\item fánach: féach ar an téarma `random sample / sampla fánach'
\end{itemize}

 \noindent \textit{Nótaí Aistriúcháin:}
\begin{itemize}
	\item Tá `fánach' bainteach go díreach lena bheith ag bogadh. Thairis sin, bíonn sé bainteach le rudaí randamacha (m.sh. sampla fánach). Is sórt sampla fánach é siúlóid fhánach -- sampla a thógtar de nóid an ghraif. Mar sin, úsáidtear `siúlóid fhánach' seachas `siúlóid randamach' anseo, cé go bhfuil ciall leo araon.
	\item Féach chomh maith ar an téarma `walk / siúlóid'.
	\item Féach chomh maith ar an téarma `random sample / sampla fánach'.
\end{itemize}


\subsubsection*{range (ainmfhocal): raon}
 \noindent \textit{Sainmhíniú (ga):} I gcomhthéacs matamaitice, tacar luacha atá mar aschur ag feidhm éigin. I gcomhthéacs faisnéise i ngraf eolais, tacar nód ar féidir leo bheith mar chuspóirí in abairtí triaracha leis an bhfaisnéis sin.
\\
 \noindent \textit{Sainmhíniú (en):} In the context of mathematics, the set of values that can be used as input to some function. In the context of a predicate in a knowledge graph, the set of nodes that can be used as subjects in triples with that predicate.
\\
 \noindent \textit{Tagairtí:}
\begin{itemize}
	\item raon: De Bhaldraithe (1978) \cite{de-bhaldraithe}*, Williams et al. (2023) \cite{storchiste}
\end{itemize}

 \noindent \textit{Nótaí Aistriúcháin:}
\begin{itemize}
	\item Téarma díreach ar ó Stórchiste fáil leis an mbrí chéanna i gcomhthéacs matamaitice.
	\item * Cé go bhfuil an téarma seo i bhFoclóir De Bhaldraithe, ní luaitear comhthéacs ar bith leis, agus níl sé cinnte mar sin an raibh bhrí matamaiticiúil i gceist ann nó nach raibh.
	\item Cé go bhfuil an focal seo i bhFoclóir Uí Dhónaill, i bhFoclóir Uí Dhónaill agus Uí Mhaoileoin, i bhFoclóir De Bhaldraithe, agus i bhFoclóir Uí Dhuinín, is mar bhealach nó mar limistéar talún seachas mar thacar luacha matamaitice a bhíonn sé luaite iontu.
\end{itemize}


\subsubsection*{rank (ainmfhocal): rang}
 \noindent \textit{Sainmhíniú (ga):} Ag trácht ar luach i liosta sórtáilte, an t-innéacs sa liosta sin ag bhfuil an luach.
\\
 \noindent \textit{Sainmhíniú (en):} Regarding a value in a sorted list, the index at which that value is located.
\\
 \noindent \textit{Tagairtí:}
\begin{itemize}
	\item rang: Dineen (1934) \cite{dineen}, Ó Dónaill (1977) \cite{odonaill}, Williams et al. (2023) \cite{storchiste}
\end{itemize}

 \noindent \textit{Nótaí Aistriúcháin:}
\begin{itemize}
	\item Téarma díreach ar fáil le brí chomhchosúil. Luaitear `rang' mar `rank' leis an mbrí cheannann chéanna i gcomhthéacs matamaitice i Stórchiste.
	\item Tá `rang' aistrithe mar `class' ar Téarma.ie, ach ní ghlactar leis sin toisc go bhfuil an focal `aicme' ann cheana, agus toisc go bhfuil fianaise léir ann ó Stórchiste go mba cheart `rang' a úsáid' chun `rank' a chur in iúl.
	\item Cé go bhfuil an focal `rang' i bhFoclóir De Bhaldraithe agus i bhFoclóir Uí Dhónaill agus Uí Mhaoileoin, is i gcomhthéacs iomlán ar leith atá sé iontu.
	\item Féach chomh maith ar an téarma `class / aicme'.
\end{itemize}


\subsubsection*{to rank (briathar): rangaigh}
 \noindent \textit{Sainmhíniú (ga):} I gcomhthéacs tacar uimhreacha, rang a thabhairt do gach uile uimhir ann ar ioann í agus a hinnéacs is liosta sortáilte na n-uimhreacha sin.
\\
 \noindent \textit{Sainmhíniú (en):} In the context of a set of numbers, to give a rank to each number equal to its index in a sorted list of those numbers.
\\
 \noindent \textit{Tagairtí:}
\begin{itemize}
	\item rangaigh: De Bhaldraithe (1978) \cite{de-bhaldraithe}, Ó Dónaill et al. (1991) \cite{focloir-beag}, Ó Dónaill (1977) \cite{odonaill}
	\item rang: féach ar an téarma `rank / rang'
\end{itemize}

 \noindent \textit{Nótaí Aistriúcháin:}
\begin{itemize}
	\item Sna foclóirí thuas ar fad, bíonn `rangaigh' ann mar focal comhchiallach le `aicmigh' seachas mar leagan Gaeilge den fhocal `ranking'. Sin ráite, tagann sé ag an bhfréamh `rang', a bhfuil an bhrí sin aige. Glactar leis an téarma seo mar sin.
	\item Is é `rangú' seachas `rangaigh' atá i bhFoclóir Uí Dhónaill agus Uí Mhaoileoin.
	\item Úsáidtear an téarma seo toisc go bhfuil `rang' úsáidte chun `rank' a chur in iúl.
	\item Féach chomh maith ar an téarma `rank / rang'.
	\item Féach chomh maith ar an téarma `rank / rang'.
\end{itemize}


\subsubsection*{ranked list (ainmfhocal): liosta ranganna}
 \noindent \textit{Sainmhíniú (ga):} I gcomhthéacs measúnú a dhéantar ar réamhinsteoirí nasc, liosta a dhéantar as na ranganna ar fad ar tugadh do gach uile cheist réamhinsinte nasc le linn measúnaithe.
\\
 \noindent \textit{Sainmhíniú (en):} In the context the of evaluation of link predictors, a list composed of all ranks assigned to each link prediction query during the evaluation process.
\\
 \noindent \textit{Tagairtí:}
\begin{itemize}
	\item liosta: De Bhaldraithe (1978) \cite{de-bhaldraithe}, Dineen (1934) \cite{dineen}, Ó Dónaill et al. (1991) \cite{focloir-beag}, Ó Dónaill (1977) \cite{odonaill}
	\item rang: féach ar an téarma `rank / rang'
\end{itemize}

 \noindent \textit{Nótaí Aistriúcháin:}
\begin{itemize}
	\item Ní ghlactar le `liosta rangaithe', ach le `liosta ranganna' toisc gurb é `liosta ranganna' atá i gceist i ndáiríre. Bheadh `liosta rangaithe' níos litriúil cinnte -- ach tugann sé sin le fios go bhfuil rangú déanta ar an liosta, seachas gur liosta é ina bhfuil na ranganna iad féin. I mBéarla  bheadh `list of ranks' níos doiléire ná `ranked list' -- ach ní fheictear cúis ar bith le creidiúint go mbeadh `liosta ranganna' níos doiléire ná `liosta rangaithe' i nGaeilge -- ach a mhalairt.
	\item Féach chomh maith ar an téarma `rank / rang'.
\end{itemize}


\subsubsection*{ranking (ainmfhocal): rangú}
 \noindent \textit{Sainmhíniú (ga):} I gcomhthéacs liosta, an próiseas a bhaineann le rang a thabhairt do gach uile bhall ann.
\\
 \noindent \textit{Sainmhíniú (en):} In the context of a list, the process of assigning a rank to each element in it.
\\
 \noindent \textit{Tagairtí:}
\begin{itemize}
	\item rangú: féach ar an téarma `to rank / rangaigh'
\end{itemize}

 \noindent \textit{Nótaí Aistriúcháin:}
\begin{itemize}
	\item Téarma cruthaithe go díreach as an bhfocal `rangaigh'.
	\item Féach ar an téarma `to rank / rangaigh'.
	\item Féach chomh maith ar an téarma `rank / rang'.
\end{itemize}


\subsubsection*{ratio (ainmfhocal): cóimheas}
 \noindent \textit{Sainmhíniú (ga):} I gcomhthéacs matamaitice, comparáid a dhéantar idir luach dhá athróga nó idir dhá uimhir A agus B, comhairthe mar $A / B$.
\\
 \noindent \textit{Sainmhíniú (en):} In the context of mathematics, a comparison of the values of two numbers or variables A and B, calculated as $A / B$.
\\
 \noindent \textit{Tagairtí:}
\begin{itemize}
	\item cóimheas: De Bhaldraithe (1978) \cite{de-bhaldraithe}, Ó Dónaill et al. (1991) \cite{focloir-beag}, Ó Dónaill (1977) \cite{odonaill}, Williams et al. (2023) \cite{storchiste}
\end{itemize}

 \noindent \textit{Nótaí Aistriúcháin:}
\begin{itemize}
	\item Téarma díreach ar fáil leis an mbrí cheannann chéanna.
	\item Cé go bhfuil an focal seo i bhFoclóir Uí Dhuinín, ní luaitear an bhrí `ratio' leis.
\end{itemize}


\subsubsection*{to reason (on) (briathar): réasúnaíocht a dhéanamh (ar)}
 \noindent \textit{Sainmhíniú (ga):} I gcomhthéacs ríomhfhoghlama, rialacha loighce a úsáid chun fíricí nua a réamhinsint.
\\
 \noindent \textit{Sainmhíniú (en):} In the context of machine learning, to use logical rules to predict new facts.
\\
 \noindent \textit{Tagairtí:}
\begin{itemize}
	\item réasúnaíocht (a dhéanamh): De Bhaldraithe (1978) \cite{de-bhaldraithe}, Ó Dónaill et al. (1991) \cite{focloir-beag}, Ó Dónaill (1977) \cite{odonaill}
\end{itemize}

 \noindent \textit{Nótaí Aistriúcháin:}
\begin{itemize}
	\item Glactar leis an téarma seo seachas le `réasúnaigh' toisc fianaise ó Fhoclóir De Bhaldraithe: bíonn `to reason from premises' aistrithe mar `réasúnaíocht a dhéanamh ó réamhleagan' ann. Níl sampla d'úsáid `réasúnaigh' ann, áfach. Glactar leis an téarma a bhfuil tuilleadh fianaise leis mar sin.
	\item Féach chomh maith ar an téarma `reasoning / réasúnaíocht'.
\end{itemize}


\subsubsection*{reasoner (ainmfhocal): córas réasúnaíochta}
 \noindent \textit{Sainmhíniú (ga):} I gcomhthéacs ríomhfhoghlama, córas a úsáideann rialacha loighce chun fíricí nua a réamhinsint.
\\
 \noindent \textit{Sainmhíniú (en):} In the context of machine learning, a system that uses logical rules to predict new facts.
\\
 \noindent \textit{Tagairtí:}
\begin{itemize}
	\item córas: De Bhaldraithe (1978) \cite{de-bhaldraithe}, Ó Dónaill et al. (1991) \cite{focloir-beag}, Ó Dónaill (1977) \cite{odonaill}
	\item réasúnaíocht: De Bhaldraithe (1978) \cite{de-bhaldraithe}, Ó Dónaill et al. (1991) \cite{focloir-beag}, Ó Dónaill (1977) \cite{odonaill}
\end{itemize}

 \noindent \textit{Nótaí Aistriúcháin:}
\begin{itemize}
	\item Cé go bhfuil `réasúnaí' i bhFoclóir De Bhaldraithe mar `reasoner' is i gcomhthéacs daoine amháin atá an téarma sin luaite. Thairis sin, tá an téarma sin luaite mar leagan den téarma Béarla `rationalist' i bhFoclóir De Bhaldraithe agus i bhFoclóir Uí Dhónaill -- sainmhíniú nach luíonn leis an úsáid atá de dhíth anseo. Ní ghlactar le `réasúnaí' mar sin.
	\item Toisc gur córas ríomhaireachta a bhíonn i gceist i gcónaí le `reasoner', meastar go bhfuil `córas réasúnaíochta' soiléir mar théarma. Thairis sin, cloíonn sé le téarmaí eile sa bhFoclóir seo (atá bunaithe chomh maith ar an bhfocal réasúnaíocht).
	\item Féach chomh maith ar an téarma `reasoning / réasúnaíocht'.
\end{itemize}


\subsubsection*{reasoning (ainmfhocal): réasúnaíocht}
 \noindent \textit{Sainmhíniú (ga):} I gcomhthéacs ríomhfhoghlama, úsáid rialacha loighce chun fíricí nua a réamhinsint.
\\
 \noindent \textit{Sainmhíniú (en):} In the context of machine learning, the use of logical rules to predict new facts.
\\
 \noindent \textit{Tagairtí:}
\begin{itemize}
	\item réasúnaíocht: De Bhaldraithe (1978) \cite{de-bhaldraithe}, Ó Dónaill et al. (1991) \cite{focloir-beag}, Ó Dónaill (1977) \cite{odonaill}
\end{itemize}

 \noindent \textit{Nótaí Aistriúcháin:}
\begin{itemize}
	\item Téarma (agus samplaí de i gcomhthéacs comhchosúil) díreach le fáil ó na foclóirí thuas.
\end{itemize}


\subsubsection*{reference implementation (ainmfhocal): leagan infheidhmithe caighdeánach}
 \noindent \textit{Sainmhíniú (ga):} Leagan infheidhmithe a chuirtear ar fáil mar shampla d'fhorbróirí cóid eile, go háirithe mar leagan caighdeánach de.
\\
 \noindent \textit{Sainmhíniú (en):} An implementation that is made available for other developers, especially as a standard version.
\\
 \noindent \textit{Tagairtí:}
\begin{itemize}
	\item leagan: féach ar an téarma `implementation / leagan infheidhmithe'
	\item feidhmigh: féach ar an téarma `implementation / leagan infheidhmithe'
\end{itemize}

 \noindent \textit{Nótaí Aistriúcháin:}
\begin{itemize}
	\item Bhíothas idir dhá chomhairle maidir leis an téarma seo -- `leagan infheidhmithe caighdeánach' nó `leagan infheidhmithe samplach'. Tá an-chiall leo araon. Glacadh le `leagan infheidhmithe caighdeánach' toisc gurb in atá i gceist le `reference implementation', nach mór i gcónaí ná leagan infheidhmithe a chuirtear i leabharlann chóid.
	\item Féach chomh maith ar an téarma `implementation / leagan infheidhmithe'.
\end{itemize}


\subsubsection*{region (ainmfhocal): réigiún}
 \noindent \textit{Sainmhíniú (ga):} I gcomhthéacs graif, cuid logánta áirithe den ghraf.
\\
 \noindent \textit{Sainmhíniú (en):} In the context of a graph, a specific local part of the graph.
\\
 \noindent \textit{Tagairtí:}
\begin{itemize}
	\item réigiún: De Bhaldraithe (1978) \cite{de-bhaldraithe}, Ó Dónaill et al. (1991) \cite{focloir-beag}, Ó Dónaill (1977) \cite{odonaill}
\end{itemize}

 \noindent \textit{Nótaí Aistriúcháin:}
\begin{itemize}
	\item Cé go mbíonn `réigiún' luaite i gcomhthéacs an domhain fhisicigh amháin, meastar go bhfuil an bhrí sin cosúil go leor le go mbeadh ciall leis an téarma seo i gcomhthéacs graf.
\end{itemize}


\subsubsection*{to regress (briathar): cúlaigh}
 \noindent \textit{Sainmhíniú (ga):} I gcomhthéacs ríomhfhoghlama, cúlú a chur i gcrích.
\\
 \noindent \textit{Sainmhíniú (en):} In the context of machine learning, to perform regression.
\\
 \noindent \textit{Tagairtí:}
\begin{itemize}
	\item cúlaigh: De Bhaldraithe (1978) \cite{de-bhaldraithe}, Ó Dónaill et al. (1991) \cite{focloir-beag}, Ó Dónaill (1977) \cite{odonaill}
	\item cúlú: féach ar an téarma `regression / cúlú'
\end{itemize}

 \noindent \textit{Nótaí Aistriúcháin:}
\begin{itemize}
	\item Is é `cúlú' seachas `cúlaigh' atá i bhFoclóir Uí Dhónaill agus Uí Mhaoileoin.
	\item Ní luann foclóir ar bith thuas an focal `cúlaigh' i gcomhthéacs matamaitice, ach is é `cúlú' atá luaite i gcomhthéacs matamaitice chun `regression' a chur in iúl. Glactar le `cúlaigh' mar sin.
	\item Féach chomh maith ar an téarma `regression / cúlú'.
\end{itemize}


\subsubsection*{regression (ainmfhocal): cúlú}
 \noindent \textit{Sainmhíniú (ga):} I gcomhthéacs matamaitice, an próiseas a bhaineann le cothromóid líne a fháil a dhéanann cur síos cruinn ar phatrúin uimhriúla i sraith sonraí.
\\
 \noindent \textit{Sainmhíniú (en):} The the context of mathematics, the process relating to finding a linear equation that accurately describes the numerical patterns in a data set.
\\
 \noindent \textit{Tagairtí:}
\begin{itemize}
	\item cúlú: De Bhaldraithe (1978) \cite{de-bhaldraithe}, Ó Dónaill (1977) \cite{odonaill}, Williams et al. (2023) \cite{storchiste}
\end{itemize}

 \noindent \textit{Nótaí Aistriúcháin:}
\begin{itemize}
	\item Luann an trí fhoinse thuas an téarma seo mar théarma matamaitice.
\end{itemize}


\subsubsection*{regressor (ainmfhocal): cúlaitheoir}
 \noindent \textit{Sainmhíniú (ga):} I gcomhthéacs ríomhfhoghlama, samhail a chuireann cúlú i gcrích.
\\
 \noindent \textit{Sainmhíniú (en):} In the context of machine learning, a model that performs regression.
\\
 \noindent \textit{Tagairtí:}
\begin{itemize}
	\item cúlaigh: féach ar an téarma `to regress / cúlaigh'
	\item -eoir: De Bhaldraithe (1978) \cite{de-bhaldraithe}, Dineen (1934) \cite{dineen}, Ó Dónaill et al. (1991) \cite{focloir-beag}, Ó Dónaill (1977) \cite{odonaill}, Williams et al. (2023) \cite{storchiste}
\end{itemize}

 \noindent \textit{Nótaí Aistriúcháin:}
\begin{itemize}
	\item Níl iontráil ar leith ag an iarmhír `-eoir' sna foclóirí thuas, ach luann siad uilig go leor focal a úsáideann í díreach mar a úsáidtear anseo.
	\item Féach chomh maith ar an téarma `to regress / cúlaigh'.
\end{itemize}


\subsubsection*{regularisation (ainmfhocal): tabhairt chun rialtachta}
 \noindent \textit{Sainmhíniú (ga):} I gcomhthéacs samhla ríomhfhoghlama, próiseas a bhfuil mar aidhm aige ró-fhoghlaim a laghdú trí mhéid luach na bparaiméadar a shrianadh i gcaoi éigin.
\\
 \noindent \textit{Sainmhíniú (en):} In the context of a machine learning mode, a process that aims to reduce overfitting by restricting the size of parameter values in some way.
\\
 \noindent \textit{Tagairtí:}
\begin{itemize}
	\item tabhair chun rialtachta: féach ar an téarma `to regularise / tabhair chun rialtachta'
\end{itemize}

 \noindent \textit{Nótaí Aistriúcháin:}
\begin{itemize}
	\item Féach ar an téarma `to regularise / tabhair chun rialtachta'.
\end{itemize}


\subsubsection*{to regularise (briathar): tabhair chun rialtachta}
 \noindent \textit{Sainmhíniú (ga):} I gcomhthéacs samhla ríomhfhoghlama, ró-fhoghlaim a laghdú trí mhéid luach na bparaiméadar a shrianadh i caoi éigin.
\\
 \noindent \textit{Sainmhíniú (en):} In the context of a machine learning mode, to reduce overfitting by restricting the size of parameter values in some way.
\\
 \noindent \textit{Tagairtí:}
\begin{itemize}
	\item tabhair chun rialtachta: De Bhaldraithe (1978) \cite{de-bhaldraithe}, Ó Dónaill (1977) \cite{odonaill}
\end{itemize}

 \noindent \textit{Nótaí Aistriúcháin:}
\begin{itemize}
	\item Frása iomlán ar fáil ó na foclóirí thuas i gcomhthéacs ginearálta.
	\item Is é `tabhair chun rialtachta' a úsáidtear toisc gurb é an `rialtacht' atá ann ná laghdú ar cé chomh mór agus is féidir le luach na bparaiméadar a bheith.
\end{itemize}


\subsubsection*{regulariser (ainmfhocal): córas rialtachta}
 \noindent \textit{Sainmhíniú (ga):} Córas a dhéanann samhail ríomhfhoghlama a thabhairt chun rialtachta le linn á traenála.
\\
 \noindent \textit{Sainmhíniú (en):} A system that regularises a machine learning model as it is being trained.
\\
 \noindent \textit{Tagairtí:}
\begin{itemize}
	\item córas: De Bhaldraithe (1978) \cite{de-bhaldraithe}, Ó Dónaill et al. (1991) \cite{focloir-beag}, Ó Dónaill (1977) \cite{odonaill}
	\item rialtacht: féach ar an téarma `to regularise / tabhair chun rialtachta'
\end{itemize}

 \noindent \textit{Nótaí Aistriúcháin:}
\begin{itemize}
	\item Frása iomlán ar fáil ó na foclóirí thuas i gcomhthéacs ginearálta.
	\item Féach chomh maith ar an téarma `to regularise / tabhair chun rialtachta'.
\end{itemize}


\subsubsection*{reification (ainmfhocal): tearcú}
 \noindent \textit{Sainmhíniú (ga):} I gcomhthéacs graf, an próiseas a bhaineann le hipear-graf (nó graf eile a bhfuil lipéid ann nach féidir iad a scríobh go nádúrtha mar abairtí triaracha) aistriú go graf bunaithe ar abairtí triaracha amháin. Chun é seo a dhéanamh, déantar $n$ abairt thriarach as gach aon chuid adamhach den hipear-ghraif, rud a fhágann den chuid is mó go mbíonn an graf tearcaithe níos éadlúithe ná an hipear-ghraif as a gcruthaíodh é.
\\
 \noindent \textit{Sainmhíniú (en):} In the context of graphs, the process relating to transforming a hyper-graph (or other graph in which labels are present that cannot be written as simple triples) into a simpler format composed of only triples. To do this, $n$ triples are created out of every atomic unit of the hyper-graph, which generally results in the reified graph being more sparse than the hyper-graph from which it was made.
\\
 \noindent \textit{Tagairtí:}
\begin{itemize}
	\item tearcaigh: Ó Dónaill (1977) \cite{odonaill}
\end{itemize}

 \noindent \textit{Nótaí Aistriúcháin:}
\begin{itemize}
	\item Níl an téarma `reification' le fáil go díreach i bhfoclóir dúchasach ar bith. Rinneadh an rogha `tearcú' a úsáid toisc gurb é an focal is giorra don bhrí atá de dhíth, agus toisc go n-éiríonn graf níos teirce tar éis do a bheith `reified'.
\end{itemize}


\subsubsection*{to reify (briathar): tearcaigh}
 \noindent \textit{Sainmhíniú (ga):} I gcomhthéacs graif, an próiseas tearcaithe a chur i gcrích.
\\
 \noindent \textit{Sainmhíniú (en):} In the context of a graph, to perform the reification process.
\\
 \noindent \textit{Tagairtí:}
\begin{itemize}
	\item tearcaigh: féach ar an téarma `reification / tearcú'
\end{itemize}

 \noindent \textit{Nótaí Aistriúcháin:}
\begin{itemize}
	\item Féach ar an téarma `reification / tearcú'.
\end{itemize}


\subsubsection*{relation(ship) (ainmfhocal): ceangal}
 \noindent \textit{Sainmhíniú (ga):} Cuid de ghraf a nascann (nó a cheanglaíonn) dhá nód le chéile.
\\
 \noindent \textit{Sainmhíniú (en):} An element of a graph that serves to connect two nodes.
\\
 \noindent \textit{Tagairtí:}
\begin{itemize}
	\item ceangal: De Bhaldraithe (1978) \cite{de-bhaldraithe}, Dineen (1934) \cite{dineen}, Ó Dónaill et al. (1991) \cite{focloir-beag}, Ó Dónaill (1977) \cite{odonaill}
\end{itemize}

 \noindent \textit{Nótaí Aistriúcháin:}
\begin{itemize}
	\item Is mar thagairt d'fheistiú (le rópa) a úsáidtear an téarma seo den chuid is mó sna foclóirí thuas. Sin ráite, is féidir a rá chomh maith go bhfuil dhá nód a bhfuil ceangal eatarthu `feistithe' lena chéile. Ní mheasann an t-údar gur bac ar bith é sin ar úsáid an fhocail `ceangal' leis an mbrí nua seo.
	\item Seo an téarma céanna is a úsáidtear chun `edge' a chur in iúl, toisc go bhfuil an bhrí chéanna le `edge' agus `relationship' i gcomhthéacs graf eolais i mBéarla.
\end{itemize}


\subsubsection*{relative entropy (ainmfhocal): eantrópacht choibhneasta}
 \noindent \textit{Sainmhíniú (ga):} I gcomhthéacs ríomheolaíochta agus matamaitice, feidhm a chomhaireann cé chomh difriúil is atá dhá dháileadh staitistiúil óna chéile. Is minic agus é úsáidte mar fheidhm phionóis, agus is ionann é agus dibhéirseacht Kullback-Leibler.
\\
 \noindent \textit{Sainmhíniú (en):} In the context of computer science and mathematics, a function that calculates how different two distributions are from each other. It is often used as a loss function, and it is also referred to as Kullback-Leibler divergence.
\\
 \noindent \textit{Tagairtí:}
\begin{itemize}
	\item eantrópacht: féach ar an téarma `entropy / eantrópacht'
	\item coibhneasta: De Bhaldraithe (1978) \cite{de-bhaldraithe}, Ó Dónaill et al. (1991) \cite{focloir-beag}, Ó Dónaill (1977) \cite{odonaill}
\end{itemize}

 \noindent \textit{Nótaí Aistriúcháin:}
\begin{itemize}
	\item Téarma cruthaithe go díreach as na focail thuas. Tá `eantrópacht' luaite i gcomhthéacs matamaitice, agus `coibhneasta' luaite i gcomhthéacs níos leithne ach comhchosúil, sna foinsí thuas.
	\item Is ionann eantrópacht choibhneasta agus dibhéirseacht Kullback-Leibler.
	\item Féach chomh maith ar an téarma `Kullback-Leibler (KL) divergence / dibhéirseacht Kullback-Leibler (KL)'.
	\item Féach chomh maith ar an téarma `entropy / eantrópacht'.
\end{itemize}


\subsubsection*{ReLU (ainmfhocal): ReLU}
 \noindent \textit{Sainmhíniú (ga):} An fheidhm ghníomhachtaithe  \noindent \textit{ReLU}.
\\
 \noindent \textit{Sainmhíniú (en):} The ReLU activation function.
\\
 \noindent \textit{Tagairtí:}
\begin{itemize}
	\item ReLU: N / A
\end{itemize}

 \noindent \textit{Nótaí Aistriúcháin:}
\begin{itemize}
	\item Níl an téarma seo le fáil ó fhoinse ar bith (fiú Téarma.ie). Is ainm ar fheidhm ghníomhachtaithe é, agus tá sé ina ainm dílis nach mór ag an bpointe seo. Dá bharr sin, agus toisc go mbeadh aistriúchán / Gaelú air níos doiléire do ríomheolaithe ná an t-ainm sean-bhunaithe a úsáid, fágtar gan aistriú é. Moltar é a chur sa gcló iodálach agus é á scríobh i dtéacs Gaeilge.
\end{itemize}


\subsubsection*{repository (ainmfhocal): stór (cóid)}
 \noindent \textit{Sainmhíniú (ga):} I gcomhthéacs ríomheolaíochta, suíomh ar líne (.i. GitHub) nó fillteán logánta ar ríomhaire (.i. fillteán .git/) ina bhfuil cód, agus stair na sean-leaganacha de, stóráilte.
\\
 \noindent \textit{Sainmhíniú (en):} In the context of computer science, an online site (i.e. GitHub) or local folder on a computer (i.e. a .git/ folder) in which code, and the history of its previous versions, are stored.
\\
 \noindent \textit{Tagairtí:}
\begin{itemize}
	\item stór: De Bhaldraithe (1978) \cite{de-bhaldraithe}, Dineen (1934) \cite{dineen}, Ó Dónaill et al. (1991) \cite{focloir-beag}, Ó Dónaill (1977) \cite{odonaill}
	\item cód: féach ar an téarma `source code / (bun-)chód'
\end{itemize}

 \noindent \textit{Nótaí Aistriúcháin:}
\begin{itemize}
	\item Téarma díreach ar fáil le brí chomhchosúil (ach i gcomhthéacs níos leithe). Luann Foclóir De Bhaldraithe téarma `stór eolais', rud a chuireann in iúl go bhfuil bunús leis an úsáid seo le sórt stóir (nach bhfuil fisiceach) a chur in iúl.
	\item Is focal an-choitianta sa nGaeilge é `stór', nach ionann agus `repository' i mBéarla (ní bhíonn `repository' úsáidte go minic i mBéarla ach chun trácht a dhéanamh ar stór cóid). Mar sin, muna bhfuil an comhthéacs soiléir, moltar `stór cóid' a úsáid chun an bhrí atá i gceist anseo a léiriú.
	\item Féach chomh maith ar an téarma `source code / (bun-)chód'.
\end{itemize}


\subsubsection*{representation (ainmfhocal): leagan}
 \noindent \textit{Sainmhíniú (ga):} I gcomhthéacs leabuithe graif eolais, leabú nó veicteoir a dhéanann ionad (ar leibhéal matamaiticiúil) réada nó coincheapa sa ngraf.
\\
 \noindent \textit{Sainmhíniú (en):} In the context of knowledge graph embeddings, an embedding or vector that (at a mathematical level) stands for an object or concept in the graph.
\\
 \noindent \textit{Tagairtí:}
\begin{itemize}
	\item leagan: De Bhaldraithe (1978) \cite{de-bhaldraithe}, Dineen (1934) \cite{dineen}, Ó Dónaill et al. (1991) \cite{focloir-beag}, Ó Dónaill (1977) \cite{odonaill}
\end{itemize}

 \noindent \textit{Nótaí Aistriúcháin:}
\begin{itemize}
	\item Ní i gcomhthéacs eolaíochta a luaitear an téarma seo, ach is le brí chomhchosúil atá sé luaite.
	\item Is minic a úsáidtear an téarma `representation' i gcomhthéacs nach bhfuil teicniúil / matamaiticiúil; m.sh. `Each node in a knowledge graph is a representation of a real-world object or concept'. Cé go bhfuil an frása sin an-teicniúil ann féin, ní gá téarma sainmhínithe a úsáid chun `representation' a chur in iúl ann. Ina leithéid sin de chás, is leor `cur síos' (nó frása eile cosúil leis sin) chun an bhrí sin a chur in iúl. Mar shampla, `Déanann gach uile nód i ngraf eolas cur síos ar réad nó ar coincheap a bhaineann leis an domhan'.
	\item Ina theannta sin, is minic gur féidir an focal `léiriú' a úsáid. Cé go mbeifeá in ann `léiriú' a úsáid seachas `leagan' anseo, glactar le leagan toisc go bhfuil brí níos cúinge leis, rud a ligeann dó trácht ar rud díreach ar leith gan mearbhall a chur ar an léitheoir.
	\item Más leagan briathair (.i. `to represent') atá uait, moltar: `déan ionad (ruda)', `seas do', frása le `cur síos / léiriú', nó mar sin (féach ar Fhoclóir De Bhaldraithe chun tuilleadh samplaí a fháil).
\end{itemize}


\subsubsection*{representative (aidiacht): ionadaíochta}
 \noindent \textit{Sainmhíniú (ga):} I gcomhthéacs fo-thacar de thacar sonraí, in-úsáidte mar ionadaí ar an tacar sonraí iomlán toisc na patrúin chéanna a bheith ann. I gcomhthéacs samhlacha ríomhfhoghlama, in ann achoimre leathan a dhéanamh ar phatrúin i dtacar sonraí ar a raibh sé traenáilte.
\\
 \noindent \textit{Sainmhíniú (en):} In the context of a subset of a dataset, able to used in place of the whole dataset because it contains the same patterns. In the context of machine learning models, able to provide a general summary of the patterns in the dataset on which it was trained.
\\
 \noindent \textit{Tagairtí:}
\begin{itemize}
	\item ionadaíocht: De Bhaldraithe (1978) \cite{de-bhaldraithe}, Ó Dónaill et al. (1991) \cite{focloir-beag}, Ó Dónaill (1977) \cite{odonaill}
\end{itemize}

 \noindent \textit{Nótaí Aistriúcháin:}
\begin{itemize}
	\item Téarma ar fáil le brí chomhchosúil. I bhFoclóir Uí Dhónaill agus Uí Mhaoileoin, déantar trácht ar an téarma seo i gcomhthéacs daoine / polaitíocht (.i. ionadaí poiblí) amháin.
	\item Déanann `ionadaíocht' trácht ar rud atá úsáidte in ionad ruda eile. Sin go díreach a bhfuil i gceist leis an téarma seo: má tá fo-thacar sonraí ionadaíochta ann, is féidir an fo-thacar sin a úsáid in ionad an tacair mhór as a tháinig sé.
\end{itemize}


\subsubsection*{reproducible (aidiacht): in-athdhéanta}
 \noindent \textit{Sainmhíniú (ga):} I gcomhthéacs turgnaimh, in ann a bheith déanta arís ag taighdeoirí eile, leis na torthaí céanna (nó comhchosúla) is a bhí ag an gcéad turgnamh.
\\
 \noindent \textit{Sainmhíniú (en):} In the context of an experiment, able to be done again by other researchers, with the same or similar results as those originally obtained.
\\
 \noindent \textit{Tagairtí:}
\begin{itemize}
	\item athdhéan: De Bhaldraithe (1978) \cite{de-bhaldraithe}, Ó Dónaill et al. (1991) \cite{focloir-beag}, Ó Dónaill (1977) \cite{odonaill}
	\item ath-: De Bhaldraithe (1978) \cite{de-bhaldraithe}, Dineen (1934) \cite{dineen}, Ó Dónaill et al. (1991) \cite{focloir-beag}, Ó Dónaill (1977) \cite{odonaill}
	\item déan: De Bhaldraithe (1978) \cite{de-bhaldraithe}, Dineen (1934) \cite{dineen}, Ó Dónaill et al. (1991) \cite{focloir-beag}, Ó Dónaill (1977) \cite{odonaill}
\end{itemize}

 \noindent \textit{Nótaí Aistriúcháin:}
\begin{itemize}
	\item Cé go mbíonn an téarma seo úsáidte go minic i gcomhthéacs eolaíochta, níl brí eolaíochta ar leith aige. Mar sin, cloíonn an úsáid seo go díreach leis an úsáid atá luaite leis an bhfocal `athdhéan' sna foclóirí thuas.
\end{itemize}


\subsubsection*{rule (in logic) (ainmfhocal): riail (loighce)}
 \noindent \textit{Sainmhíniú (ga):} Ráiteas loighce ar féidir é a úsáid chun fíricí nua a fáil amach trí réasúnaíocht.
\\
 \noindent \textit{Sainmhíniú (en):} A logical statement that can be used to find new facts based via reasoning.
\\
 \noindent \textit{Tagairtí:}
\begin{itemize}
	\item riail: De Bhaldraithe (1978) \cite{de-bhaldraithe}, Dineen (1934) \cite{dineen}*, Ó Dónaill et al. (1991) \cite{focloir-beag}, Ó Dónaill (1977) \cite{odonaill}
	\item loighic: féach ar an téarma `logic / loighic'
\end{itemize}

 \noindent \textit{Nótaí Aistriúcháin:}
\begin{itemize}
	\item * Is é `riaghail' atá i bhFoclóir Uí Dhuinín.
	\item Téarmaí díreach ar fáil i gcomhthéacs comhchosúil.
	\item Féach chomh maith ar an téarma `logic / loighic'.
\end{itemize}


\subsubsection*{rule-based (aidiacht): bunaithe ar rialacha}
 \noindent \textit{Sainmhíniú (ga):} I gcomhthéacs ríomhfhoghlama, ag baint úsáid as rialacha loighce chun réasúnaíocht a dhéanamh.
\\
 \noindent \textit{Sainmhíniú (en):} In the context of machine learning, using logical rules to perform reasoning.
\\
 \noindent \textit{Tagairtí:}
\begin{itemize}
	\item riail: féach ar an téarma `rule (in logic) / riail (loighce)'
	\item bunaigh: De Bhaldraithe (1978) \cite{de-bhaldraithe}, Ó Dónaill et al. (1991) \cite{focloir-beag}*, Ó Dónaill (1977) \cite{odonaill}
\end{itemize}

 \noindent \textit{Nótaí Aistriúcháin:}
\begin{itemize}
	\item * Is é `bunú' seachas `bunaigh' atá i bhFoclóir Uí Dhónaill agus Uí Mhaoileoin.
	\item Téarmaí díreach ar fáil i gcomhthéacs comhchosúil.
	\item Féach ar an téarma `rule (in logic) / riail (loighce)'.
\end{itemize}


\subsubsection*{to run (briathar): cuir ar siúl}
 \noindent \textit{Sainmhíniú (ga):} I gcomhthéacs ríomheolaíochta, próiseas a thosú.
\\
 \noindent \textit{Sainmhíniú (en):} In the context of computer science, to begin a process.
\\
 \noindent \textit{Tagairtí:}
\begin{itemize}
	\item cuir ar siúl: De Bhaldraithe (1978) \cite{de-bhaldraithe}, Ó Dónaill (1977) \cite{odonaill}'
	\item ar siúl: féach ar an téarma `running / ar siúl'
\end{itemize}

 \noindent \textit{Nótaí Aistriúcháin:}
\begin{itemize}
	\item Tá go leor téarmaí eile ar féidir (agus ar ceart) iad a úsáid chomh maith: tosaigh, próiseáil, cuir ar siúl, cuir ar bun, srl. Níl cúis ar bith gan iad sin a úsáid más fearr leat iad. Tugtar sampla amháin thuas ní toisc gurb é is fearr, ach toisc gur rogha mhaith amháin atá ann.
	\item Tá `rith' ar Téarma.ie agus, de réir Fhoclóir Uí Dhónaill, is cosúil gur féidir é a úsáid sa gcomhthéacs seo chomh maith. Sin ráite, tá `cuir ar siúl' luaite thuas toisc go cloíonn sé leis an téarma `running / ar siúl' (agus toisc nach bhfuil fianaise i bhfoclóir dúchasach ar bith faoi an féidir `ag rith', nó mar sin, a úsáid).
	\item Féach chomh maith ar an téarma `running / ar siúl'
\end{itemize}


\subsubsection*{running (aidiacht): ar siúl}
 \noindent \textit{Sainmhíniú (ga):} Ag trácht ar próiseas, tar éis a beith tosaithe agus fós ag obair (gan a bheith críochnaithe go fóill).
\\
 \noindent \textit{Sainmhíniú (en):} Regarding a process, having been started and still working (not being finished yet).
\\
 \noindent \textit{Tagairtí:}
\begin{itemize}
	\item ar siúl: De Bhaldraithe (1978) \cite{de-bhaldraithe}, Dineen (1934) \cite{dineen}, Ó Dónaill et al. (1991) \cite{focloir-beag}, Ó Dónaill (1977) \cite{odonaill}
\end{itemize}

 \noindent \textit{Nótaí Aistriúcháin:}
\begin{itemize}
	\item Tá go leor téarmaí eile ar féidir (agus ar ceart) iad a úsáid: tosaithe, ag próiseáil, i mbun próiseála, ar siúl, curtha ar bun, srl. Níl cúis ar bith gan iad sin a úsáid más fearr leat iad. Tugtar sampla amháin thuas ní toisc gurb é is fearr, ach toisc gur rogha mhaith amháin atá ann.
\end{itemize}


\phantomsection \subsection*{S}
\addcontentsline{toc}{subsection}{S}
\markboth{S}{S}

\subsubsection*{to sample (briathar): sampláil}
 \noindent \textit{Sainmhíniú (ga):} Sampla a thógáil.
\\
 \noindent \textit{Sainmhíniú (en):} To take a sample.
\\
 \noindent \textit{Tagairtí:}
\begin{itemize}
	\item sampla: De Bhaldraithe (1978) \cite{de-bhaldraithe}, Dineen (1934) \cite{dineen}, Ó Dónaill et al. (1991) \cite{focloir-beag}, Ó Dónaill (1977) \cite{odonaill}, Williams et al. (2023) \cite{storchiste}
\end{itemize}

 \noindent \textit{Nótaí Aistriúcháin:}
\begin{itemize}
	\item Tá an téarma seo (i gcomhthéacs comhchosúil ach níos leithne) díreach ar fáil ó na foclóirí thuas.
	\item Úsáideann Stórchiste `sampláil' mar théarma matamaitice.
\end{itemize}


\subsubsection*{sample (ainmfhocal): sampla}
 \noindent \textit{Sainmhíniú (ga):} Sonra a thógtar as dáileadh staitistiúil nó as próiseas randamach.
\\
 \noindent \textit{Sainmhíniú (en):} A data point that is taken from a statistical distribution or random process.
\\
 \noindent \textit{Tagairtí:}
\begin{itemize}
	\item sampla: De Bhaldraithe (1978) \cite{de-bhaldraithe}, Dineen (1934) \cite{dineen}, Ó Dónaill et al. (1991) \cite{focloir-beag}, Ó Dónaill (1977) \cite{odonaill}, Williams et al. (2023) \cite{storchiste}
\end{itemize}

 \noindent \textit{Nótaí Aistriúcháin:}
\begin{itemize}
	\item Tá an téarma seo (i gcomhthéacs comhchosúil ach níos leithne) díreach ar fáil ó na foclóirí thuas.
	\item Luann Stórchiste `sampla' mar théarma matamaitice.
\end{itemize}


\subsubsection*{sampler (ainmfhocal): samplóir}
 \noindent \textit{Sainmhíniú (ga):} rud (m.sh. algartam ríomhaireachta) a dhéanann sampláil.
\\
 \noindent \textit{Sainmhíniú (en):} a thing (such as a computer algorithm) that samples.
\\
 \noindent \textit{Tagairtí:}
\begin{itemize}
	\item samplóir: De Bhaldraithe (1978) \cite{de-bhaldraithe}, Ó Dónaill et al. (1991) \cite{focloir-beag}, Ó Dónaill (1977) \cite{odonaill}
\end{itemize}

 \noindent \textit{Nótaí Aistriúcháin:}
\begin{itemize}
	\item Tá an téarma seo, leis an mbrí chéanna, díreach ar fáil sna foclóirí thuas.
\end{itemize}


\subsubsection*{scalar (aidiacht): scálach}
 \noindent \textit{Sainmhíniú (ga):} Uimhir nach athróg é a bhíonn á húsáid le huimhir eile a mhéadú fúithi.
\\
 \noindent \textit{Sainmhíniú (en):} A numerical value other than a variable, typically used in multiplication.
\\
 \noindent \textit{Tagairtí:}
\begin{itemize}
	\item comhéifeacht: De Bhaldraithe (1978) \cite{de-bhaldraithe}, Ó Dónaill (1977) \cite{odonaill}, Williams et al. (2023) \cite{storchiste}
\end{itemize}

 \noindent \textit{Nótaí Aistriúcháin:}
\begin{itemize}
	\item Téarma luaite mar théarma matamaitice i bhFoclóir Uí Dhónaill, i bhFoclóir De Bhaldraithe, agus i Stórchiste.
	\item Más ainmfhocal atá uait, úsáid `uimhir scálach' (nó `scálach' mar atá ag Stórchiste).
\end{itemize}


\subsubsection*{scatter plot (or diagram) (ainmfhocal): scaipléaráid}
 \noindent \textit{Sainmhíniú (ga):} Léaráid a léiríonn pointí i ndá thoise, le hais X agus Y.
\\
 \noindent \textit{Sainmhíniú (en):} A diagram that shows points in two dimensions, with an X and a Y axis.
\\
 \noindent \textit{Tagairtí:}
\begin{itemize}
	\item scaipléaráid: Williams et al. (2023) \cite{storchiste}
	\item scaip: De Bhaldraithe (1978) \cite{de-bhaldraithe}, Dineen (1934) \cite{dineen}, Ó Dónaill et al. (1991) \cite{focloir-beag}, Ó Dónaill (1977) \cite{odonaill}, Williams et al. (2023) \cite{storchiste}
	\item léaráid: De Bhaldraithe (1978) \cite{de-bhaldraithe}, Ó Dónaill et al. (1991) \cite{focloir-beag}, Ó Dónaill (1977) \cite{odonaill}
\end{itemize}

 \noindent \textit{Nótaí Aistriúcháin:}
\begin{itemize}
	\item Tá an téarma `scaipléaráid' go díreach le fáil i Stórchiste mar `scatter diagram' i gcomhthéacs matamaitice. Glactar leis sin mar sin.
	\item Luann roinnt foclóirí thuas an téarma `léaráid' mar fhocal ar `diagram' i gcomhthéacs comhchosúil, ach ní luann siad `scaipléaráid'.
	\item Luann roinnt foclóirí thuas an téarma `scaip' mar fhocal ar `to scatter', ach ní luann siad `scaipléaráid', ní ná luann siad `scaip' mar réimír.
\end{itemize}


\subsubsection*{to score (briathar): scóráil}
 \noindent \textit{Sainmhíniú (ga):} Scór a thabhairt do rud (m.sh. samhail ríomhfhoghlama).
\\
 \noindent \textit{Sainmhíniú (en):} To give a score to something (such as a machine learning model).
\\
 \noindent \textit{Tagairtí:}
\begin{itemize}
	\item scóráil: Ó Dónaill (1977) \cite{odonaill}
\end{itemize}

 \noindent \textit{Nótaí Aistriúcháin:}
\begin{itemize}
	\item I gcomhthéacs cluichí a fheictear `scóráil' úsáidte i bhFoclóir Uí Dhónaill, seachas i gcomhthéacs ríomhaireachta. Sin ráite tá an bhrí chéanna leis an bhfocal `scór' sa gcomhthéacs sin.
	\item Féach chomh maith ar an téarma `score / scór'.
\end{itemize}


\subsubsection*{score (ainmfhocal): scór}
 \noindent \textit{Sainmhíniú (ga):} Uimhir a dhéanann cur síos ar cé chomh maith is atá rud (m.sh. cruinneas samhla foghlama).
\\
 \noindent \textit{Sainmhíniú (en):} A number describing how good something is (such as the accuracy of a machine learning model).
\\
 \noindent \textit{Tagairtí:}
\begin{itemize}
	\item scór: De Bhaldraithe (1978) \cite{de-bhaldraithe}, Ó Dónaill (1977) \cite{odonaill}
\end{itemize}

 \noindent \textit{Nótaí Aistriúcháin:}
\begin{itemize}
	\item I gcomhthéacs cluichí a fheictear `scóráil' úsáidte i bhFoclóir Uí Dhónaill, seachas i gcomhthéacs ríomhaireachta. Sin ráite tá an bhrí chéanna leis an bhfocal `scór' sa gcomhthéacs sin.
\end{itemize}


\subsubsection*{scoring function (ainmfhocal): feidhm scórála}
 \noindent \textit{Sainmhíniú (ga):} I gcomhthéacs samhlacha leabaithe graif eolais, feidhm a dhéanann scór inchreidteachta a thabhairt d'abairt thriarach.
\\
 \noindent \textit{Sainmhíniú (en):} In the context of knowledge graph embedding models, a function that assigned a plausibility score to a triple.
\\
 \noindent \textit{Tagairtí:}
\begin{itemize}
	\item feidhm: féach ar an téarma `function / feidhm'
	\item scóráil: féach ar an téarma `to score / scóráil'
\end{itemize}

 \noindent \textit{Nótaí Aistriúcháin:}
\begin{itemize}
	\item Téarma cruthaithe go díreach as dá théarma eile sa bhFoclóir Tráchtais.
	\item Féach chomh maith ar an téarma `function / feidhm'.
	\item Féach chomh maith ar an téarma `to score / scóráil'.
\end{itemize}


\subsubsection*{self-attention (ainmfhocal): féin-aird}
 \noindent \textit{Sainmhíniú (ga):} Oibriú airde ina dtagann an t-ionchur ar fad ón bhfoinse chéanna.
\\
 \noindent \textit{Sainmhíniú (en):} An attention operation in which all inputs come from the same source.
\\
 \noindent \textit{Tagairtí:}
\begin{itemize}
	\item féin-: De Bhaldraithe (1978) \cite{de-bhaldraithe}, Dineen (1934) \cite{dineen}*, Ó Dónaill et al. (1991) \cite{focloir-beag}, Ó Dónaill (1977) \cite{odonaill}
	\item aird: féach ar an téarma `attention / aird'
\end{itemize}

 \noindent \textit{Nótaí Aistriúcháin:}
\begin{itemize}
	\item * Ní bhíonn `féin-' luaite mar réimír i bhFoclóir Uí Dhuinín, ach mar fhocal amháin.
	\item Téarma cruthaithe go díreach as an dá fhréamh thuas.
	\item Féach chomh maith ar an téarma `attention / aird'.
\end{itemize}


\subsubsection*{self-supervised (aidiacht): faoi fhéin-mhaoirseacht}
 \noindent \textit{Sainmhíniú (ga):} I gcomhthéacs ríomhfhoghlama, traenáilte i gcaoi a chruthaíonn lipéid / luacha aschuir chearta ó na sonraí ionchuir (mar shampla, trí focal a bhaint as abairt, agus i ndiaidh sin iarracht a dhéanamh as an bearna sin san abairt a líonadh le réamhinsint ón tsamhail fhoghlama).
\\
 \noindent \textit{Sainmhíniú (en):} In the context of machine learning, trained in such a way that expected output labels / values are created from the training data (for example, by removing a word from a sentence and then attempting to fill the blank in the sentence using a machine learning model).
\\
 \noindent \textit{Tagairtí:}
\begin{itemize}
	\item féin-: De Bhaldraithe (1978) \cite{de-bhaldraithe}, Dineen (1934) \cite{dineen}*, Ó Dónaill et al. (1991) \cite{focloir-beag}, Ó Dónaill (1977) \cite{odonaill}
	\item maoirseacht: féach ar an téarma `supervised / faoi mhaoirseacht'
\end{itemize}

 \noindent \textit{Nótaí Aistriúcháin:}
\begin{itemize}
	\item * Ní bhíonn `féin-' luaite mar réimír i bhFoclóir Uí Dhuinín.
	\item Féach ar an téarma `supervised / faoi mhaoirseacht'.
	\item Féach ar an téarma `supervision / maoirseacht'.
\end{itemize}


\subsubsection*{semantics (ainmfhocal): séimeantaic}
 \noindent \textit{Sainmhíniú (ga):} I gcomhthéacs graf eolais, staidéar ar bhrí na sonraí atá istigh ann, nó an bhrí sin í féin.
\\
 \noindent \textit{Sainmhíniú (en):} In the context of knowledge graphs, study of the meaning of the data they contain, or that meaning itself.
\\
 \noindent \textit{Tagairtí:}
\begin{itemize}
	\item séimeantaic: De Bhaldraithe (1978) \cite{de-bhaldraithe}, Ó Dónaill (1977) \cite{odonaill}
\end{itemize}

 \noindent \textit{Nótaí Aistriúcháin:}
\begin{itemize}
	\item Téarma díreach le fáil leis an mbrí chéanna i gcomhthéacs comhchosúil (.i. teangeolaíocht). Toisc mórchuid na dtéarmaí a bhaineann le graif eolais a bheith úsáidte mar analach le gramadach (m.sh. ainmní, faisnéis, agus cuspóir), meastar go gcloíonn an comhthéacs seo go díreach le comhthéacs na ngraf eolais.
\end{itemize}


\subsubsection*{semi-supervised (aidiacht): faoi leath-mhaoirseacht}
 \noindent \textit{Sainmhíniú (ga):} I gcomhthéacs ríomhfhoghlama, traenáilte ar shonraí ionchur a bhfuil na lipéid / luacha aschuir chearta nasctha leo uaireanta, ach ní i gcónaí.
\\
 \noindent \textit{Sainmhíniú (en):} In the context of machine learning, trained on input data that is sometimes, but not always, linked to expected output labels / values.
\\
 \noindent \textit{Tagairtí:}
\begin{itemize}
	\item leath-: De Bhaldraithe (1978) \cite{de-bhaldraithe}, Dineen (1934) \cite{dineen}, Ó Dónaill et al. (1991) \cite{focloir-beag}, Ó Dónaill (1977) \cite{odonaill}
	\item maoirseacht: féach ar an téarma `supervised / faoi mhaoirseacht'
\end{itemize}

 \noindent \textit{Nótaí Aistriúcháin:}
\begin{itemize}
	\item Luann Foclóir De Bhaldraithe trí réimír ar leith i gcomhair `semi-': leath-, breac-, agus scoth-. Sin ráite, ní fheictear ach `leath-' i gcomhthéacs matamaitice. Thairis sin, is féidir `leath-' a úsáid ní hamháin chun `half' a chur in iúl, ach chun `partial' a chur in iúl chomh maith. Sin an-tábhachtach, toisc gurb in atá i gceist le `semi-supervised' ná go bhfuil cuid de na sonraí ionchuir gan lipéad / luach aschur (nach ionann ach leathchuid acu go díreach i gcónaí). Toisc go bhfuil ciall le `leath-' fós sa gcomhthéacs sin, agus toisc é a bheith úsáidte go minic cheana i gcúrsaí matamaitice, glactar leis.
	\item Féach ar an téarma `supervised / faoi mhaoirseacht'.
	\item Féach ar an téarma `supervision / maoirseacht'.
\end{itemize}


\subsubsection*{set (ainmfhocal): tacar}
 \noindent \textit{Sainmhíniú (ga):} Grúpa rudaí (m.sh. uimhreacha) nach bhfuil áirí an oird ann, agus nach mbíonn an rud céanna faoi dhó ann.
\\
 \noindent \textit{Sainmhíniú (en):} A group of things (such as numbers) that does not have the property of having order, and that does not have repeats.
\\
 \noindent \textit{Tagairtí:}
\begin{itemize}
	\item tacar: De Bhaldraithe (1978) \cite{de-bhaldraithe}, Dineen (1934) \cite{dineen}*, Ó Dónaill et al. (1991) \cite{focloir-beag}*, Ó Dónaill (1977) \cite{odonaill}, Williams et al. (2023) \cite{storchiste}
\end{itemize}

 \noindent \textit{Nótaí Aistriúcháin:}
\begin{itemize}
	\item * Ní i gcomhthéacs matamaiticiúil a luaitear an téarma seo sna foclóirí seo.
	\item Is i gcomhthéacs matamaiticiúil a luaitear an téarma seo sna foclóirí eile.
\end{itemize}


\subsubsection*{setwise function (ainmfhocal): feidhm de thacar (pointí)}
 \noindent \textit{Sainmhíniú (ga):} Feidhm a bhfuil tacar pointí sonraí mar ionchur aici.
\\
 \noindent \textit{Sainmhíniú (en):} A function that takes a set of data points as input.
\\
 \noindent \textit{Tagairtí:}
\begin{itemize}
	\item feidhm: féach ar an téarma `function / feidhm'
	\item tacar: féach ar an téarma `set / tacar'
	\item pointe: féach ar an téarma `pointwise function / feidhm de phointe'
\end{itemize}

 \noindent \textit{Nótaí Aistriúcháin:}
\begin{itemize}
	\item Féach ar an téarma `pointwise function / feidhm de phointe'.
	\item Féach ar an téarma `function / feidhm'.
\end{itemize}


\subsubsection*{sigmoid function (aidiacht): feidhm shiogmóideach}
 \noindent \textit{Sainmhíniú (ga):} An fheidhm $siogma(x) = 1 / (1 + e^(-x))$ (a bhfuil cruth cosúil leis an litir `s' air agus é breactha).
\\
 \noindent \textit{Sainmhíniú (en):} The function $sigma(x) = 1 / (1 + e^(-x))$ (which has a shape similar to the letter s when plotted).
\\
 \noindent \textit{Tagairtí:}
\begin{itemize}
	\item feidhm: féach ar an téarma `function / feidhm'
	\item siogma: Ó Dónaill (1977) \cite{odonaill}
	\item -óideach: Ó Dónaill (1977) \cite{odonaill}
\end{itemize}

 \noindent \textit{Nótaí Aistriúcháin:}
\begin{itemize}
	\item Níl an focal `siogmóideach' ann i bhfoclóir ar bith atá á úsáid agam, ach is féidir an téarma a chruthú i nGaeilge mar a rinneadh i mBéarla as an litir Gréigise (siogma) agus an iarmhír `-óideach'.
\end{itemize}


\subsubsection*{signal (ainmfhocal): comhartha}
 \noindent \textit{Sainmhíniú (ga):} I gcomhthéacs ríomhfhoghlama, foinse faisnéise i dtacar sonraí ar féidir í a úsáid go díreach mar chuid fhiúntach den phróiseas ríomhfhoghlama.
\\
 \noindent \textit{Sainmhíniú (en):} In the context of machine learning, the source of information in a data set that can be directly used as a meaningful part of the machine learning process.
\\
 \noindent \textit{Tagairtí:}
\begin{itemize}
	\item comhartha: De Bhaldraithe (1978) \cite{de-bhaldraithe}, Dineen (1934) \cite{dineen}, Ó Dónaill et al. (1991) \cite{focloir-beag}, Ó Dónaill (1977) \cite{odonaill}
\end{itemize}

 \noindent \textit{Nótaí Aistriúcháin:}
\begin{itemize}
	\item Ní luann na foclóirí thuas an focal seo i gcomhthéacs ríomhfhoghlama. An focal is giorra don bhrí atá de dhíth ná comhartha, toisc (de réir Fhoclóir Uí Dhónaill) go bhfuil na bríonna `indication' agus `notice, heed' leis. Is é an `signal' i dtacar sonraí ná an chuid de na sonraí sin atá in-fhoghlamtha (.i. nach bhfuil mar thorann randamach) -- tugann an comhartha treo do cén chaoi ar cheart don tsamhail a bheith ag foghlaim. Glactar leis an téarma `comhartha' mar sin.
	\item Féach chomh maith ar an téarma `noise / torann'.
\end{itemize}


\subsubsection*{to simulate (briathar): insamhail}
 \noindent \textit{Sainmhíniú (ga):} Samhlail ríomhfhoghlama (nó uirlisí ríomhaireachta eile) a úsáid chun próiseas casta a shamhlú i bhfoirm níos simplí.
\\
 \noindent \textit{Sainmhíniú (en):} To use a machine learning model (or other computational tools) to model a complex process in a simpler form.
\\
 \noindent \textit{Tagairtí:}
\begin{itemize}
	\item insamhail: féach ar an téarma `simulation / insamhladh'
\end{itemize}

 \noindent \textit{Nótaí Aistriúcháin:}
\begin{itemize}
	\item Féach ar an téarma `simulation / insamhladh'.
\end{itemize}


\subsubsection*{simulation (ainmfhocal): insamhladh}
 \noindent \textit{Sainmhíniú (ga):} Úsáid shamhla ríomhfhoghlama (nó uirlisí ríomhaireachta eile) chun próiseas casta a shamhlú i bhfoirm níos simplí.
\\
 \noindent \textit{Sainmhíniú (en):} The use of a machine learning mode (or other computational tools) to model a complex process in a simpler form.
\\
 \noindent \textit{Tagairtí:}
\begin{itemize}
	\item insamhail: Ó Dónaill (1977) \cite{odonaill}
\end{itemize}

 \noindent \textit{Nótaí Aistriúcháin:}
\begin{itemize}
	\item Níl focal ar bith iomlán foirfe don téarma seo sna foclóirí atá á n-úsáid. Cé is moite de sin, tá brí chomhchosúil (nach mór) ag `insamhail', agus tá an-bhuntáiste aige sin go bhfuil sé cosúil leis an bhfocal `samhail', atá in úsáid chomh maith sa tráchtas seo. Roghnaíodh mar sin é.
	\item Is é ionsamhail atá ar Focloir.ie, ach is é `insamhail' atá i bhFoclóir Uí Dhónaill. Mar is iondúil sa saothar seo, tugtar aird d'Fhoclóir Uí Dhónaill amháin sa gcás seo.
\end{itemize}


\subsubsection*{social network (ainmfhocal): líonra cairdis}
 \noindent \textit{Sainmhíniú (ga):} Graf nó graf eolais ina gcuireann nóid daoine in iúl, agus ina mbíonn ceangail ann a léiríonn cén gaol / baint atá ag daoine lena chéile.
\\
 \noindent \textit{Sainmhíniú (en):} A graph or knowledge graph in which nodes represent people, and edges represent the relations / connections the people have with each other.
\\
 \noindent \textit{Tagairtí:}
\begin{itemize}
	\item líonra: féach ar an téarma `network / líonra'
	\item cairdeas: De Bhaldraithe (1978) \cite{de-bhaldraithe}, Dineen (1934) \cite{dineen}, Ó Dónaill et al. (1991) \cite{focloir-beag}, Ó Dónaill (1977) \cite{odonaill}
\end{itemize}

 \noindent \textit{Nótaí Aistriúcháin:}
\begin{itemize}
	\item Tá an dá fhocal (líonra agus cairdeas) díreach ar fáil ó na foclóirí thuas le brí chomhchosúil.
	\item Tá roinnt roghanna eile ann (m.sh. líonra gaolta). Cé go bhfuil brí níos leithne aige sin (ní hionann gaol agus gaol clainne amháin), is mar líonra cairdis i ndáiríre a bhíonn i mórchuid na `social networks'. Thairis sin, is as líonra cairdis ar líne (m.sh. ar Facebook) a tháinig an téarma seo i dtosach.
	\item Níor cheart `líonra sóisialta' a úsáid -- tagann sé sin díreach as an mBéarla gan tuiscint gur ionann `sóisialta' agus rud a bhaineann le sochaí nó le cumas / tuiscint sóisialta na ndaoine.
	\item Féach chomh maith ar an téarma `network / líonra.'
\end{itemize}


\subsubsection*{softmax (ainmfhocal): softmax}
 \noindent \textit{Sainmhíniú (ga):} An fheidhm ghníomhachtaithe  \noindent \textit{softmax}.
\\
 \noindent \textit{Sainmhíniú (en):} The softmax activation function.
\\
 \noindent \textit{Tagairtí:}
\begin{itemize}
	\item softmax: N / A
\end{itemize}

 \noindent \textit{Nótaí Aistriúcháin:}
\begin{itemize}
	\item Níl an téarma seo le fáil ó fhoinse ar bith (fiú Téarma.ie). Is ainm ar fheidhm ghníomhachtaithe é, agus tá sé ina ainm dílis nach mór ag an bpointe seo. Dá bharr sin, agus toisc go mbeadh aistriúchán / Gaelú air níos doiléire do ríomheolaithe ná an t-ainm sean-bhunaithe a úsáid, fágtar gan aistriú é. Moltar é a chur sa gcló iodálach agus é á scríobh i dtéacs Gaeilge.
\end{itemize}


\subsubsection*{source code (ainmfhocal): (bun-)chód}
 \noindent \textit{Sainmhíniú (ga):} An cód taobh thiar de thionscadal cóid éigin (m.sh. an cód taobh thiar de shamhail ríomhfhoghlama).
\\
 \noindent \textit{Sainmhíniú (en):} The code used for some coding project (for example, the code used for a machine learning model).
\\
 \noindent \textit{Tagairtí:}
\begin{itemize}
	\item bun-: De Bhaldraithe (1978) \cite{de-bhaldraithe}, Dineen (1934) \cite{dineen}, Ó Dónaill et al. (1991) \cite{focloir-beag}, Ó Dónaill (1977) \cite{odonaill}
	\item cód: De Bhaldraithe (1978) \cite{de-bhaldraithe}, Ó Dónaill et al. (1991) \cite{focloir-beag}, Ó Dónaill (1977) \cite{odonaill}
\end{itemize}

 \noindent \textit{Nótaí Aistriúcháin:}
\begin{itemize}
	\item Tá an téarma `cód foinseach' ar Téarma.ie, ach níl an focal `foinseach' le feiceáil sna foclóirí dúchasacha. Ní ghlactar leis an téarma sin mar sin.
	\item Cé go bhfuil an focal `foinse' ann i nGaeilge (agus úsáidte go minic ar Téarma.ie chun brí chomhchosúil leis seo a chur in iúl), meastar go bhfuil sé ró-litriúil mar aistriúchán. In ainneoin an focal sin a úsáid, aistrítear bhun-bhrí an téarma seo. Is ionann `source code' agus `bun-chóid' tionscadail, toisc gurb in an cód ar a bhfuil an tionscadal bunaithe. Glactar le `bun-chód' toisc a léire agus a simplí is atá sé.
\end{itemize}


\subsubsection*{sparse (aidiacht): éadlúth}
 \noindent \textit{Sainmhíniú (ga):} I gcomhthéacs graif eolais (nó fo-ghraif), gan móran ceangail le codanna eile den ghraf / den fho-ghraf.
\\
 \noindent \textit{Sainmhíniú (en):} In the context of a knowledge graph (or subgraph), lowly connected with other parts of the same graph / subgraph.
\\
 \noindent \textit{Tagairtí:}
\begin{itemize}
	\item éadlúth: De Bhaldraithe (1978) \cite{de-bhaldraithe}, Ó Dónaill et al. (1991) \cite{focloir-beag}, Ó Dónaill (1977) \cite{odonaill}
\end{itemize}

 \noindent \textit{Nótaí Aistriúcháin:}
\begin{itemize}
	\item Luann Foclóir De Bhaldraithe agus Foclóir Uí Dhónaill an téarma seo mar théarma eolaíochta i gcomhthéacs aeir / an atmaisféir, ach leis an mbrí chéanna.
	\item Tá go leor téarmaí comhchiallacha eile (.i. tearc, gann, srl), ach úsáidtear `éadlúth' toisc gur `dlúth' an focal atá ar a mhalairt de rud.
\end{itemize}


\subsubsection*{sparsity (ainmfhocal): éadlús}
 \noindent \textit{Sainmhíniú (ga):} I gcomhthéacs graif eolais (nó fo-ghraif), cé chomh éadlúth is atá sé.
\\
 \noindent \textit{Sainmhíniú (en):} In the context of a knowledge graph (or subgraph), how sparse it is.
\\
 \noindent \textit{Tagairtí:}
\begin{itemize}
	\item éadlús: De Bhaldraithe (1978) \cite{de-bhaldraithe}, Ó Dónaill (1977) \cite{odonaill}
\end{itemize}

 \noindent \textit{Nótaí Aistriúcháin:}
\begin{itemize}
	\item Luann Foclóir De Bhaldraithe agus Foclóir Uí Dhónaill an téarma seo  mar théarma eolaíochta i gcomhthéacs aeir / an atmaisféir, ach leis an mbrí chéanna.
	\item Tá go leor téarmaí eile (.i. tearc, gann, srl), ach úsáidtear `éadlús' toisc gur `dlús' an focal atá ar a mhalairt de rud.
	\item Féach chomh maith ar an téarma `sparse / éadlúth'.
\end{itemize}


\subsubsection*{standard deviation (ainmfhocal): diall caighdeánach}
 \noindent \textit{Sainmhíniú (ga):} I gcomhthéacs matamaitice, tomhas ar cé chomh héagsúil is atá luachanna i sraith sonraí. Is ionann diall caighdeánach fréamh chearnach athraithis.
\\
 \noindent \textit{Sainmhíniú (en):} In the context of mathematics, a measure of how much the values in a dataset vary. Standard deviation is the square root of variance.
\\
 \noindent \textit{Tagairtí:}
\begin{itemize}
	\item diall: De Bhaldraithe (1978) \cite{de-bhaldraithe}, Ó Dónaill (1977) \cite{odonaill}, Williams et al. (2023) \cite{storchiste}
	\item caighdeánach: De Bhaldraithe (1978) \cite{de-bhaldraithe}, Ó Dónaill et al. (1991) \cite{focloir-beag}, Ó Dónaill (1977) \cite{odonaill}, Williams et al. (2023) \cite{storchiste}
	\item diall caighdeánach: Williams et al. (2023) \cite{storchiste}
\end{itemize}

 \noindent \textit{Nótaí Aistriúcháin:}
\begin{itemize}
	\item Téarma díreach ar fáil leis an mbrí cheannann chéanna i gcomhthéacs matamaitice.
	\item Tá na fo-théarmaí `diall' agus `caighdeánach' le fáil i bhfoclóirí eile ina n-aonar le bríonna comhchosúla, ach níl luann na foclóirí sin i gcomhthéacs matamaitice iad.
\end{itemize}


\subsubsection*{state of the art (best) (ainmfhocal): scoth an réimse}
 \noindent \textit{Sainmhíniú (ga):} An tuiscint, samhail, eolas, nó eile is fearr i réimse eolaíochta éigin.
\\
 \noindent \textit{Sainmhíniú (en):} The best understanding, model, information, etc in a given scientific field.
\\
 \noindent \textit{Tagairtí:}
\begin{itemize}
	\item scoth: De Bhaldraithe (1978) \cite{de-bhaldraithe}, Dineen (1934) \cite{dineen}, Ó Dónaill et al. (1991) \cite{focloir-beag}, Ó Dónaill (1977) \cite{odonaill}
	\item réimse: De Bhaldraithe (1978) \cite{de-bhaldraithe}, Ó Dónaill et al. (1991) \cite{focloir-beag}, Ó Dónaill (1977) \cite{odonaill}
\end{itemize}

 \noindent \textit{Nótaí Aistriúcháin:}
\begin{itemize}
	\item Ní luann Foclóir Uí Dhuinín `scoth' mar `an rud is fearr', ach tá an bhrí sin le feiceáil ann fós féin sa bhfocal `scothamhail' ann.
	\item Mar aon leis sin, is cosúil gur féidir `scothúil' a úsáid mar aidiacht ceangailte leis an t-ainmfhocal seo.
	\item Is é `staid an réimse' an staid ina bhfuil an réimse ann faoi láthair (bíodh sé go maith nó go dona), agus is é `scoth an réimse' an chuid is fearr de.
	\item Féach chomh maith ar an téarma `state of the art (current) / staid an réimse'.
	\item Féach chomh maith ar an téarma `state of the art (the literature) / litríocht an réimse'.
\end{itemize}


\subsubsection*{state of the art (current) (ainmfhocal): staid an réimse}
 \noindent \textit{Sainmhíniú (ga):} An tuiscint / leibhéal taighde atá ann faoi láthair i réimse taighde (bíodh sé go maith nó go dona).
\\
 \noindent \textit{Sainmhíniú (en):} The understanding or level of current research in a given field of research (whether that is good or bad).
\\
 \noindent \textit{Tagairtí:}
\begin{itemize}
	\item staid: De Bhaldraithe (1978) \cite{de-bhaldraithe}, Dineen (1934) \cite{dineen}, Ó Dónaill et al. (1991) \cite{focloir-beag}, Ó Dónaill (1977) \cite{odonaill}
	\item réimse: De Bhaldraithe (1978) \cite{de-bhaldraithe}, Ó Dónaill et al. (1991) \cite{focloir-beag}, Ó Dónaill (1977) \cite{odonaill}
\end{itemize}

 \noindent \textit{Nótaí Aistriúcháin:}
\begin{itemize}
	\item Is é `staid an réimse' an staid ina bhfuil an réimse ann faoi láthair (bíodh sé go maith nó go dona), agus is é `scoth an réimse' an chuid is fearr de.
	\item Féach chomh maith ar an téarma `state of the art (best) / scoth an réimse'.
	\item Féach chomh maith ar an téarma `state of the art (the literature) / litríocht an réimse'.
\end{itemize}


\subsubsection*{state of the art (the literature) (ainmfhocal): litríocht an réimse}
 \noindent \textit{Sainmhíniú (ga):} An litríocht ar fad atá ar fáil i staid an réimse.
\\
 \noindent \textit{Sainmhíniú (en):} All of the literature available in the state of the art.
\\
 \noindent \textit{Tagairtí:}
\begin{itemize}
	\item litríocht: De Bhaldraithe (1978) \cite{de-bhaldraithe}, Ó Dónaill et al. (1991) \cite{focloir-beag}, Ó Dónaill (1977) \cite{odonaill}
	\item réimse: De Bhaldraithe (1978) \cite{de-bhaldraithe}, Ó Dónaill et al. (1991) \cite{focloir-beag}, Ó Dónaill (1977) \cite{odonaill}
\end{itemize}

 \noindent \textit{Nótaí Aistriúcháin:}
\begin{itemize}
	\item Cé gur minic a úsáidtear an focal `litríocht' chun cur síos a dhéanamh ar staidéar ar an litríocht, is léir ó Fhoclóir Uí Dhónaill agus Uí Mhaoileoin gur féidir `litríocht' a úsáid chun trácht ar na saothair litríochta iad féin chomh maith.
	\item Is féidir an focal `litríocht' a úsáid ina aonar chomh maith nuair atá a bhfuil i gceist léir ón gcomhthéacs.
	\item Féach chomh maith ar an téarma `state of the art (best) / scoth an réimse'.
	\item Féach chomh maith ar an téarma `state of the art (current) / scoth an réimse'.
\end{itemize}


\subsubsection*{structural alignment (ainmfhocal): ailíniú struchtúir}
 \noindent \textit{Sainmhíniú (ga):} Cé chomh maith is a luíonn struchtúr ruda (.i. graf eolais) le cáilíocht mhatamaiticiúil eile.
\\
 \noindent \textit{Sainmhíniú (en):} How well the structure of something (i.e. a graph) relates to some other mathematical quantity.
\\
 \noindent \textit{Tagairtí:}
\begin{itemize}
	\item ailíniú: féach ar an téarma `alignment / ailíniú'
	\item struchtúr: féach ar an téarma `structure / struchtúr'
\end{itemize}

 \noindent \textit{Nótaí Aistriúcháin:}
\begin{itemize}
	\item Féach ar an téarma `alignment / ailíniú'.
	\item Féach ar an téarma `structure / struchtúr'.
\end{itemize}


\subsubsection*{structural alignment framework (ainmfhocal): creatlach ailínithe struchtúir}
 \noindent \textit{Sainmhíniú (ga):} An chreatlach atá mar phríomh-thoradh ar an tráchtas seo, a dhéantar trí réamhinsintí na hipitéise ar ailiniú struchtúir a chur i bhfeidhm.
\\
 \noindent \textit{Sainmhíniú (en):} The framework that is the core result of this thesis, which is a result of the predictions of the structural alignment hypothesis.
\\
 \noindent \textit{Tagairtí:}
\begin{itemize}
	\item creatlach: féach ar an téarma `framework / creatlach'
	\item ailíniú: féach ar an téarma `alignment / ailíniú'
	\item struchtúr: féach ar an téarma `structure / struchtúr'
\end{itemize}

 \noindent \textit{Nótaí Aistriúcháin:}
\begin{itemize}
	\item Téarma cruthaithe as téarmaí eile anseo.
	\item Féach chomh maith ar an téarma `framework / creatlach'.
	\item Féach chomh maith ar an téarma `alignment / ailíniú'.
	\item Féach chomh maith ar an téarma `structure / struchtúr'.
\end{itemize}


\subsubsection*{structural alignment hypothesis (ainmfhocal): hipitéis ar ailíniú struchtúir}
 \noindent \textit{Sainmhíniú (ga):} An hipitéis taobh thiar den tráchtas seo, a deir go bhfuil ailíniú idir struchtúr graif eolais agus cé chomh maith agus is féidir réamhinsint nasc a dhéanamh air.
\\
 \noindent \textit{Sainmhíniú (en):} The central hypothesis of this thesis, which states that there is alignment between the structure of a knowledge graph and how well link prediction can be done on it.
\\
 \noindent \textit{Tagairtí:}
\begin{itemize}
	\item hipitéis: De Bhaldraithe (1978) \cite{de-bhaldraithe}, Ó Dónaill et al. (1991) \cite{focloir-beag}, Ó Dónaill (1977) \cite{odonaill}
	\item ailíniú: féach ar an téarma `alignment / ailíniú'
	\item struchtúr: féach ar an téarma `structure / struchtúr'
\end{itemize}

 \noindent \textit{Nótaí Aistriúcháin:}
\begin{itemize}
	\item Téarma cruthaithe as téarmaí eile anseo (agus as `hipitéis', atá luaite sa gcomhthéacs céanna sna foclóirí thuas).
	\item Féach chomh maith ar an téarma `alignment / ailíniú'.
	\item Féach chomh maith ar an téarma `structure / struchtúr'.
\end{itemize}


\subsubsection*{structure (ainmfhocal): struchtúr}
 \noindent \textit{Sainmhíniú (ga):} I gcomhthéacs graif, patrúin, achoimrí uimhriúla, agus staitisticí ar féidir iad a áireamh ar an ngraf (gan trácht ar shéimeantaic an ghraif).
\\
 \noindent \textit{Sainmhíniú (en):} In the context of a graph, patterns, numerical summaries, and statistics that can be calculated on the graph (without reference to the semantics of the graph).
\\
 \noindent \textit{Tagairtí:}
\begin{itemize}
	\item struchtúr: De Bhaldraithe (1978) \cite{de-bhaldraithe}, Ó Dónaill et al. (1991) \cite{focloir-beag}, Ó Dónaill (1977) \cite{odonaill}
\end{itemize}

 \noindent \textit{Nótaí Aistriúcháin:}
\begin{itemize}
	\item Téarma luaite le brí chomhchosúil (ach fisiceach, seachas ríomheolaíochta) sna foclóirí thuas.
\end{itemize}


\subsubsection*{subgraph (ainmfhocal): fo-ghraf}
 \noindent \textit{Sainmhíniú (ga):} Cuid de ghraf eolais, atá mar ghraf eolas (níos lú) é féin.
\\
 \noindent \textit{Sainmhíniú (en):} A part of a knowledge graph that itself is a (smaller) knowledge graph.
\\
 \noindent \textit{Tagairtí:}
\begin{itemize}
	\item fo-: De Bhaldraithe (1978) \cite{de-bhaldraithe}, Ó Dónaill et al. (1991) \cite{focloir-beag}, Ó Dónaill (1977) \cite{odonaill}
	\item graf: féach ar an téarma `graph / graf'
\end{itemize}

 \noindent \textit{Nótaí Aistriúcháin:}
\begin{itemize}
	\item Téarma cruthaithe go díreach as na focail thuas.
	\item Tá `fo-' i bhFoclóir Uí Dhuinín, ach leis an mbrí `faoi' seachas `mar chuid de'.
\end{itemize}


\subsubsection*{subject (ainmfhocal): ainmní}
 \noindent \textit{Sainmhíniú (ga):} in abairt thriarach $(a,f,c)$, an chéad nód $a$ atá mar thús ag an bhfaisnéis $f$.
\\
 \noindent \textit{Sainmhíniú (en):} in a triple $(s,p,o)$, the first node $s$ that acts as the head of the predicate $p$.
\\
 \noindent \textit{Tagairtí:}
\begin{itemize}
	\item ainmfhocal: De Bhaldraithe (1978) \cite{de-bhaldraithe}, Dineen (1934) \cite{dineen}, Ó Dónaill et al. (1991) \cite{focloir-beag}, Ó Dónaill (1977) \cite{odonaill}, Williams et al. (2023) \cite{storchiste}
\end{itemize}

 \noindent \textit{Nótaí Aistriúcháin:}
\begin{itemize}
	\item I mBéarla, samhlaítear abairtí triaracha mar abairtí teangeolaíochta le hainmní, le faisnéis, agus le cuspóir. Glactar leis an analach chéanna i nGaeilge.
\end{itemize}


\subsubsection*{subject corruption (ainmfhocal): malartú an ainmní}
 \noindent \textit{Sainmhíniú (ga):} I gcomhthéacs frith-shamplála, an próiseas a bhaineann le frith-shampla a chruthú tríd an ainmní $a$ in abairt thriarach $(a,f,c)$ a ionadú le nód eile.
\\
 \noindent \textit{Sainmhíniú (en):} In the context of negative sampling, the process of creating a negative sample by replacing the subject $s$ in a triple $(s,p,o)$ with another node.
\\
 \noindent \textit{Tagairtí:}
\begin{itemize}
	\item malartaigh: féach ar an téarma `to corrupt / malartaigh'
	\item ainmní: féach ar an téarma `subject / ainmní'
\end{itemize}

 \noindent \textit{Nótaí Aistriúcháin:}
\begin{itemize}
	\item Féach ar an téarma `to corrupt / malartaigh'.
	\item Féach chomh maith ar an téarma `subject / ainmní'.
\end{itemize}


\subsubsection*{subject prediction (ainmfhocal): réamhinsint an ainmní}
 \noindent \textit{Sainmhíniú (ga):} I gcomhthéacs an taisc réamhinsinte nasc, an tasc a bhaineann le hainmní a réamhinsint chun ceist réamhinsinte nasc sa bhfoirm $(?,f,c)$ a fhreagairt.
\\
 \noindent \textit{Sainmhíniú (en):} In the context of the link prediction task, the task of predicting a subject to answer a link prediction query in the form $(?,p,o)$.
\\
 \noindent \textit{Tagairtí:}
\begin{itemize}
	\item réamhinsint: féach ar an téarma `prediction / réamhinsint'
	\item cuspóir: féach ar an téarma `subject / ainmní'
\end{itemize}

 \noindent \textit{Nótaí Aistriúcháin:}
\begin{itemize}
	\item Féach chomh maith ar an téarma `prediction / réamhinsint'.
	\item Féach ar an téarma `subject / ainmní'.
\end{itemize}


\subsubsection*{subset (ainmfhocal): fo-thacar}
 \noindent \textit{Sainmhíniú (ga):} Tacar a dhéantar as baill ó tacar (nó tacair) eile, agus nach bhfuil ann ach baill ó na tacair sin.
\\
 \noindent \textit{Sainmhíniú (en):} A set whose elements come from another set (or sets), and which only contains those elements.
\\
 \noindent \textit{Tagairtí:}
\begin{itemize}
	\item fo-: De Bhaldraithe (1978) \cite{de-bhaldraithe}, Ó Dónaill et al. (1991) \cite{focloir-beag}, Ó Dónaill (1977) \cite{odonaill}
	\item tacar: féach ar an téarma `set / tacar'
	\item fo-thacar: Ó Dónaill (1977) \cite{odonaill}
\end{itemize}

 \noindent \textit{Nótaí Aistriúcháin:}
\begin{itemize}
	\item Téarmaí díreach ar fáil le bríonna chomhchosúla.
	\item Tá an téarma `fo-thacar' le fáil go díreach i gcomhthéacs matamaiticiúil i bhFoclóir Uí Dhónaill.
	\item Féach chomh maith ar an téarma `set / tacar'.
\end{itemize}


\subsubsection*{supervised (aidiacht): faoi mhaoirseacht}
 \noindent \textit{Sainmhíniú (ga):} I gcomhthéacs ríomhfhoghlama, traenáilte le sonraí ionchuir agus lena bhfuiltear as súil leis mar aschur (m.sh. íomhánna de chait agus de mhadraí, agus lipéad `cat' nó `madra' ar gach uile íomhá ann chun ligean don chóras ríomhfhoghlama a fháil amach an luíonn a aschur féin leis an bhfírinne).
\\
 \noindent \textit{Sainmhíniú (en):} In the context of machine learning, trained with input data and with the expected output values (for example, images of cats and dogs, with the label `cat' or `dog'' on each image so that the machine learning model can determine if its prediction matches the ground truth).
\\
 \noindent \textit{Tagairtí:}
\begin{itemize}
	\item maoirseacht: De Bhaldraithe (1978) \cite{de-bhaldraithe}, Ó Dónaill et al. (1991) \cite{focloir-beag}, Ó Dónaill (1977) \cite{odonaill}
\end{itemize}

 \noindent \textit{Nótaí Aistriúcháin:}
\begin{itemize}
	\item Luann Foclóir Uí Dhónaill `maoirseacht' i gcomhthéacs an-chosúil lena bhfuil de dhíth anseo: `maoirseacht a dhéanamh ar obair, to supervise work.' Is léir nach ag trácht ar chórais ríomhfhoghlama atá sé, áfach. Cé is moite de sin, más cuí `maoirseacht' i gcomhthéacs oibre duine, is cosúil gur cuí i gcomhthéacs oibre ríomhaire é chomh maith.
	\item Tá cúpla roghanna eile ann don téarma seo -- stiúrtha agus riartha. Is léir ó Fhoclóir Uí Dhónaill go bhfuil an briathar `riar' níos ceangailte le riar dlí / gnóthaí / srl, seachas le maoirseacht de shórt éigin a dhéantar ar obair chóras ríomhfhoghlama. Ina theannta sin, baineann `stiúir' mar bhriathar le rudaí a dhíriú nó a threorú, seachas le maoirseacht ar obair.
	\item Níl an téarma seo le fáil ar Téarma.ie i gcomhthéacs ríomhfhoghlama ag uair á chumtha anseo (cé go bhfuil sé ann i gcomhthéacsanna eile ann). Tá idir `faoi mhaoirseacht' agus `maoirsithe' ar Téarma.ie. Ní ghlactar le `maoirsithe' toisc nach bhfuil fianaise ann don fhocal sin i bhfoclóir dúchasach ar bith.
	\item Féach chomh maith ar an téarma `supervision / maoirseacht'.
\end{itemize}


\subsubsection*{supervision (aidiacht): maoirseacht}
 \noindent \textit{Sainmhíniú (ga):} I gcomhthéacs ríomhfhoghlama, cén chaoi a dhéantar samhail a thraenáil de réir na sonraí ionchuir atá aici.
\\
 \noindent \textit{Sainmhíniú (en):} In the context of machine learning, how a model is trained according to its input data.
\\
 \noindent \textit{Tagairtí:}
\begin{itemize}
	\item maoirseacht: féach ar an téarma `supervised / faoi mhaoirseacht'
\end{itemize}

 \noindent \textit{Nótaí Aistriúcháin:}
\begin{itemize}
	\item Más é `semi-supervision' atá i gceist, úsáid `leath-mhaoirseacht'. Más é `self-supervision' atá i gceist, úsáid `féin-mhaoirseacht'.
	\item Féach ar an téarma `supervised / faoi mhaoirseacht'. Muna bhfuil maoirseacht ann le linn thraenála, úsáid `gan mhaoirseacht'.
	\item Féach ar an téarma `unsupervised / gan mhaoirseacht'.
	\item Féach ar an téarma `semi-supervised / faoi leath-mhaoirseacht'.
	\item Féach ar an téarma `self-supervised / faoi fhéin-mhaoirseacht'.
\end{itemize}


\subsubsection*{support vector (ainmfhocal): veicteoir tacaíochta}
 \noindent \textit{Sainmhíniú (ga):} I gcomhthéacs ríomhfhoghlama, veicteoir atá an-ghar don teorainn chinnidh.
\\
 \noindent \textit{Sainmhíniú (en):} In the context of machine learning, a vector that is very close to the decision boundary.
\\
 \noindent \textit{Tagairtí:}
\begin{itemize}
	\item veicteoir: féach ar an téarma `vector / veicteoir'
	\item tacaíocht: De Bhaldraithe (1978) \cite{de-bhaldraithe}, Ó Dónaill et al. (1991) \cite{focloir-beag}, Ó Dónaill (1977) \cite{odonaill}
\end{itemize}

 \noindent \textit{Nótaí Aistriúcháin:}
\begin{itemize}
	\item Níl an focal `support' ina théarma teicniúil anseo. An t-aon bhrí atá i gceist ná gur féidir `support vectors' a úsáid chun tacú leis an bpróiseas foghlama.
	\item Féach chomh maith ar an téarma `vector / veicteoir'.
\end{itemize}


\subsubsection*{support vector classifier (ainmfhocal): aicmitheoir bunaithe ar veicteoirí tacaíochta}
 \noindent \textit{Sainmhíniú (ga):} Samhail bunaithe ar veicteoirí tacaíochta a úsáidtear chun aicmiú a chur i gcrích.
\\
 \noindent \textit{Sainmhíniú (en):} A support vector machine used to perform classification.
\\
 \noindent \textit{Tagairtí:}
\begin{itemize}
	\item aicmitheoir: féach ar an téarma `classifier / aicmitheoir'
	\item veicteoir tacaíochta: féach ar an téarma `support vector / veicteoir tacaíochta'.
\end{itemize}

 \noindent \textit{Nótaí Aistriúcháin:}
\begin{itemize}
	\item Níor cheart `aicmitheoir veicteoirí tacaíochta' a úsáid toisc go gcuirfeadh sé sin in iúl gurb in an aidhm atá ag an aicmitheoir ná veicteoirí tacaíochta (amháin) a aicmiú, seachas aicmiú a chur chun cinn trí díriú go géar ar veicteoirí tacaíochta.
	\item Féach chomh maith ar an téarma `classifier / aicmitheoir'.
	\item Féach chomh maith ar an téarma `support vector / veicteoir tacaíochta'.
	\item Féach chomh maith ar an téarma `support vector machine (SVM) / samhail (bunaithe ar) veicteoirí tacaíochta (SVT)'.
\end{itemize}


\subsubsection*{support vector machine (SVM) (ainmfhocal): samhail (bunaithe ar) veicteoirí tacaíochta (SVT)}
 \noindent \textit{Sainmhíniú (ga):} Samhail ar bith den ghrúpa samhlacha ríomhfhoghlama atá bunaithe ar úsáid veicteoirí tacaíochta chun éifeachtach ríomhfhoghlama a chur chun cinn.
\\
 \noindent \textit{Sainmhíniú (en):} Any one model of the family of machine learning models that are based on the use of support vectors to improve machine learning performance.
\\
 \noindent \textit{Tagairtí:}
\begin{itemize}
	\item samhail: féach ar an téarma `model / samhail'
	\item veicteoir tacaíochta: féach ar an téarma `support vector / veicteoir tacaíochta'
\end{itemize}

 \noindent \textit{Nótaí Aistriúcháin:}
\begin{itemize}
	\item Cuirtear `veicteoir tacaíochta' sa tuiseal ginideach iolra toisc go n-úsáidtear níos mó ná veicteoir tacaíochta amháin i gcónaí agus samhail veicteoirí tacaíochta á úsáid.
	\item Úsáidtear `samhail' anseo seachas a leithéid de `meaisín' toisc gurb in, go bunúsach, a bhfuil i gceist. Is samhlacha ríomhfhoghlama iad `support vector machines' agus ní mheastar go bhfuil cúis ar bith focal mar `meaisín' a úsáid nuair atá an focal Gaeilge `samhail' iomlán cuí chun an bhrí ríomheolaíochta atá de dhíth a chur in iúl.
	\item Is minic agus an téarma seo úsáidte san uimhir iolra (.i. `support vector machines'). Is é an leagan iolra ceart ná `samhlacha (bunaithe ar) veicteoirí tacaíochta'.
	\item Tá idir `samhail veicteoirí tacaíochta' agus `samhail bunaithe ar veicteoirí tacaíochta' iomlán ceart. Sin ráite, is dócha gur léire an dara ceann acu den chuid is mó, toisc go léiríonn sé cén baint atá ann idir an tsamhail agus na veicteoirí tacaíochta.
	\item Féach chomh maith ar an téarma `support vector / veicteoir tacaíochta'.
\end{itemize}


\subsubsection*{support vector regressor (ainmfhocal): cúlaitheoir bunaithe ar veicteoirí tacaíochta}
 \noindent \textit{Sainmhíniú (ga):} Samhail bunaithe ar veicteoirí tacaíochta a úsáidtear chun cúlú a chur i gcrích.
\\
 \noindent \textit{Sainmhíniú (en):} A support vector machine used to perform regression.
\\
 \noindent \textit{Tagairtí:}
\begin{itemize}
	\item cúlaitheoir: féach ar an téarma `regressor / cúlaitheoir'
	\item veicteoir tacaíochta: féach ar an téarma `support vector / veicteoir tacaíochta'.
\end{itemize}

 \noindent \textit{Nótaí Aistriúcháin:}
\begin{itemize}
	\item Níor cheart `cúlaitheoir veicteoirí tacaíochta' a úsáid toisc go gcuirfeadh sé sin in iúl gurb in an aidhm atá ag an gcúlaitheoir ná cúlú a dhéanamh ar veicteoirí tacaíochta (amháin), seachas cúlú a chur chun cinn trí díriú go géar ar veicteoirí tacaíochta.
	\item Féach chomh maith ar an téarma `regressor / cúlaitheoir'.
	\item Féach chomh maith ar an téarma `support vector / veicteoir tacaíochta'.
	\item Féach chomh maith ar an téarma `support vector machine (SVM) / samhail (bunaithe ar) veicteoirí tacaíochta (SVT)'.
\end{itemize}


\subsubsection*{symbol (ainmfhocal): siombail}
 \noindent \textit{Sainmhíniú (ga):} I gcomhthéacs loighce nó rialacha loighce, comhartha a sheasann do shlonn loighce, d'athróg loighce, d'oibríocht loighce, nó do chuid eile de shlonn loighce.
\\
 \noindent \textit{Sainmhíniú (en):} In the context of logic or logical rules, a character that represents a logical expression, a logical variable, a logical operation, or some other part of a logical expression.
\\
 \noindent \textit{Tagairtí:}
\begin{itemize}
	\item siombail: De Bhaldraithe (1978) \cite{de-bhaldraithe}, Ó Dónaill et al. (1991) \cite{focloir-beag}, Ó Dónaill (1977) \cite{odonaill}
\end{itemize}

 \noindent \textit{Nótaí Aistriúcháin:}
\begin{itemize}
	\item Cé go mbíonn `comhartha' luaite mar leagan Gaeilge den fhocal `symbol', is cosúil go mbaineann `comhartha' leis an gcomhartha / marc scríofa níos mó ná leis an gcoincheap teibí atá taobh thiar de (féach ar Fhoclóir Uí Dhónaill, mar shampla). Roghnaíodh `siombail' mar sin.
	\item Is é `siombail' atá ar Téarma.ie ina chomhair seo -- agus meastar gur fearr cloí leis sin toisc go luíonn an téarma ar Téarma.ie le fianaise atá ar fáil sna foclóirí dúchasacha.
\end{itemize}


\subsubsection*{symbolic (aidiacht): siombalach}
 \noindent \textit{Sainmhíniú (ga):} I gcomhthéacs loighce, bainteach le siombailí, nó bunaithe ar úsáid shiombailí.
\\
 \noindent \textit{Sainmhíniú (en):} In the context of logic,relating to symbols, or based on the use of symbols.
\\
 \noindent \textit{Tagairtí:}
\begin{itemize}
	\item siombalach: De Bhaldraithe (1978) \cite{de-bhaldraithe}, Ó Dónaill et al. (1991) \cite{focloir-beag}, Ó Dónaill (1977) \cite{odonaill}
\end{itemize}

 \noindent \textit{Nótaí Aistriúcháin:}
\begin{itemize}
	\item Téarma díreach ar fáil le brí chomhchosúil.
	\item Féach chomh maith ar an téarma `symbol / siombail'.
\end{itemize}


\subsubsection*{symbolic logic (ainmfhocal): loighic shiombalach}
 \noindent \textit{Sainmhíniú (ga):} An brainse loighce a bhaineann le húsáid siombailí chun rialacha agus tairiscintí loighce a shamhlú.
\\
 \noindent \textit{Sainmhíniú (en):} The branch of logic that relates to using symbols to model logical rules and propositions.
\\
 \noindent \textit{Tagairtí:}
\begin{itemize}
	\item loighic: féach ar an téarma `logic / loighic'
	\item siombalach: féach ar an téarma `symbolic / siombalach'
\end{itemize}

 \noindent \textit{Nótaí Aistriúcháin:}
\begin{itemize}
	\item Féach ar an téarma `logic / loighic'.
	\item Féach chomh maith ar an téarma `symbolic / siombalach'.
\end{itemize}


\subsubsection*{symmetric (aidiacht): siméadrach}
 \noindent \textit{Sainmhíniú (ga):} I gcomhthéacs faisnéise (f) i ngraf eolais, leis an impleacht gur fíor (c,f,a) mar abhairt thiarach más fíor (a,f,c).
\\
 \noindent \textit{Sainmhíniú (en):} In the context of a predicate (p) in a knowledge graph, implying that the triple (o,p,s) is true if (s,p,o) is true.
\\
 \noindent \textit{Tagairtí:}
\begin{itemize}
	\item siméadrach: De Bhaldraithe (1978) \cite{de-bhaldraithe}, Ó Dónaill et al. (1991) \cite{focloir-beag}, Ó Dónaill (1977) \cite{odonaill}, Williams et al. (2023) \cite{storchiste}
\end{itemize}

 \noindent \textit{Nótaí Aistriúcháin:}
\begin{itemize}
	\item Téarma díreach ar fáil le brí chomhchosúil.
	\item Cé go mbíonn `comhchruthach' luaite mar leagan den fhocal `symmetrical', ní bhíonn ach `siméadrach' luaite mar théarma matamaitice i Stórchiste. Glactar le `siméadrach' mar sin.
\end{itemize}


\subsubsection*{symmetry (ainmfhocal): siméadracht}
 \noindent \textit{Sainmhíniú (ga):} An t-airí a bhaineann lena bheith siméadrach.
\\
 \noindent \textit{Sainmhíniú (en):} The property of being symmetric.
\\
 \noindent \textit{Tagairtí:}
\begin{itemize}
	\item siméadrach: De Bhaldraithe (1978) \cite{de-bhaldraithe}, Ó Dónaill (1977) \cite{odonaill}
\end{itemize}

 \noindent \textit{Nótaí Aistriúcháin:}
\begin{itemize}
	\item Téarma díreach ar fáil le brí chomhchosúil.
	\item Féach chomh maith ar an téarma `symmetric / siméadrach'.
\end{itemize}


\phantomsection \subsection*{T}
\addcontentsline{toc}{subsection}{T}
\markboth{T}{T}

\subsubsection*{to test (briathar): teisteáil}
 \noindent \textit{Sainmhíniú (ga):} Próiseas teisteála a dhéanamh ar shamhail ríomhfhoghlama.
\\
 \noindent \textit{Sainmhíniú (en):} To perform testing on a machine learning model.
\\
 \noindent \textit{Tagairtí:}
\begin{itemize}
	\item teisteáil: féach ar an téarma `testing / teisteáil'
\end{itemize}

 \noindent \textit{Nótaí Aistriúcháin:}
\begin{itemize}
	\item Féach ar an téarma `testing / teisteáil'
\end{itemize}


\subsubsection*{testing (ainmfhocal): teisteáil}
 \noindent \textit{Sainmhíniú (ga):} An próiseas a úsáidtear chun fáil amach cé chomh maith (nó cé chomh dona) is a fheidhmíonn samhail ríomhfhoghlama tar éis di a bheith traenáilte.
\\
 \noindent \textit{Sainmhíniú (en):} The process that is used to determine how well (or how poorly) a machine learning model works after it has been trained.
\\
 \noindent \textit{Tagairtí:}
\begin{itemize}
	\item teisteáil: Ó Dónaill et al. (1991) \cite{focloir-beag}, Ó Dónaill (1977) \cite{odonaill}
\end{itemize}

 \noindent \textit{Nótaí Aistriúcháin:}
\begin{itemize}
	\item Ní bhíonn an téarma seo luaite i gcomhthéacs ríomhaireachta sna foclóirí thuas, ach is le brí chomhchosúil atá sé luaite.
\end{itemize}


\subsubsection*{testing set (ainmfhocal): tacar teisteála}
 \noindent \textit{Sainmhíniú (ga):} Tacar sonraí a úsáidtear chun samhail ríomhfhoghlama a theisteáil.
\\
 \noindent \textit{Sainmhíniú (en):} The dataset used to test a machine learning model.
\\
 \noindent \textit{Tagairtí:}
\begin{itemize}
	\item tacar: féach ar an téarma `set / tacar'
	\item teisteáil: féach ar an téarma `testing / teisteáil'
\end{itemize}

 \noindent \textit{Nótaí Aistriúcháin:}
\begin{itemize}
	\item Féach ar an téarma `set / tacar'.
	\item Féach ar an téarma `testing / teisteáil'.
\end{itemize}


\subsubsection*{topology (ainmfhocal): toipeolaíocht}
 \noindent \textit{Sainmhíniú (ga):} I gcomhthéacs graif, cur síos matamaiticiúil ar cé chaoi an bhíonn a chuid nód ceangailte lena chéile; nó, réimse staidéir matamaitice bunaithe ar anailísíocht ar struchtúr graf.
\\
 \noindent \textit{Sainmhíniú (en):} In the context of a graph, a mathematical description of how its nodes are connected; or, the field of mathematical study of the analysis of graph structure.
\\
 \noindent \textit{Tagairtí:}
\begin{itemize}
	\item toipeolaíocht: Ó Dónaill (1977) \cite{odonaill}
\end{itemize}

 \noindent \textit{Nótaí Aistriúcháin:}
\begin{itemize}
	\item Téarma díreach ar fáil ó Fhoclóir Uí Dhónaill.
\end{itemize}


\subsubsection*{to train (briathar): traenáil}
 \noindent \textit{Sainmhíniú (ga):} Próiseas traenála a dhéanamh ar shamhail ríomhfhoghlama.
\\
 \noindent \textit{Sainmhíniú (en):} To perform training on a machine learning model.
\\
 \noindent \textit{Tagairtí:}
\begin{itemize}
	\item traenáil: féach ar an téarma `training / traenáil'
\end{itemize}

 \noindent \textit{Nótaí Aistriúcháin:}
\begin{itemize}
	\item Féach ar an téarma `training / traenáil'
\end{itemize}


\subsubsection*{training (ainmfhocal): traenáil}
 \noindent \textit{Sainmhíniú (ga):} An próiseas a bhaineann le feabhsú samhla ríomhfhoghlama trí shonraí a thabhairt di.
\\
 \noindent \textit{Sainmhíniú (en):} The process of optimising a machine learning model by giving it data to learn from.
\\
 \noindent \textit{Tagairtí:}
\begin{itemize}
	\item traenáil: De Bhaldraithe (1978) \cite{de-bhaldraithe}, Ó Dónaill et al. (1991) \cite{focloir-beag}, Ó Dónaill (1977) \cite{odonaill}
\end{itemize}

 \noindent \textit{Nótaí Aistriúcháin:}
\begin{itemize}
	\item Ní bhíonn an téarma seo luaite i gcomhthéacs ríomhaireachta sna foclóirí thuas. Cé is moite de sin, luaitear é i gcomhthéacs cosúil go leor (.i. ainmhí nó duine a thraenáil).
	\item Tá `oiliúint' ar Téarma.ie, ach ní ghlactar leis an téarma sin. De réir Fhoclóir Uí Dhónaill, bíonn tréithe níos daonna ag baint le hoiliúint seachas le traenáil. Thairis sin, is minic agus “oilte” úsáidte chun “skilled” a chur in iúl -- ach ní hionann samhail ríomhfhoghlama a bheith traenáilte agus scil ar bith a bheith aici -- teipeann ar an bpróiseas traenála torthaí maithe a fháil go minic. Ní bhíonn an chlaontacht sin ag baint le `traenáil', agus glactar leis mar sin.
\end{itemize}


\subsubsection*{training loop (ainmfhocal): timthriall traenála}
 \noindent \textit{Sainmhíniú (ga):} An chuid athfhillteach den phróiseas a úsáidtear chun samhail ríomhfhoghlama a thraenáil.
\\
 \noindent \textit{Sainmhíniú (en):} The repeating unit of the process used to train a machine learning model.
\\
 \noindent \textit{Tagairtí:}
\begin{itemize}
	\item timthriall: De Bhaldraithe (1978) \cite{de-bhaldraithe}, Ó Dónaill (1977) \cite{odonaill}
	\item traenáil: féach ar an téarma `training / traenáil'
\end{itemize}

 \noindent \textit{Nótaí Aistriúcháin:}
\begin{itemize}
	\item Féach ar an téarma `training / traenáil'.
	\item Bíonn traenáil samhlacha ríomhfhoghlama (den chuid is mó) ina timthriall -- an próiseas céanna á úsáid arís is arís chun an tsamhail a chur chun cinn céim ar chéim. Is mar sin a dhéantar trácht anseo ar thimthriall traenála seachas ar phróiseas nó ar chur chuige traenála.
\end{itemize}


\subsubsection*{training set (ainmfhocal): tacar traenála}
 \noindent \textit{Sainmhíniú (ga):} Tacar sonraí a úsáidtear chun samhail ríomhfhoghlama a thraenáil.
\\
 \noindent \textit{Sainmhíniú (en):} The dataset used to train a machine learning model.
\\
 \noindent \textit{Tagairtí:}
\begin{itemize}
	\item tacar: féach ar an téarma `set / tacar'
	\item traenáil: féach ar an téarma `training / traenáil'
\end{itemize}

 \noindent \textit{Nótaí Aistriúcháin:}
\begin{itemize}
	\item Féach ar an téarma `set / tacar'.
	\item Féach ar an téarma `training / traenáil'.
\end{itemize}


\subsubsection*{transfer learning (ainmfhocal): tras-fhoghlaim}
 \noindent \textit{Sainmhíniú (ga):} Úsáid cur chuige mion-feabhsaithe chun cur ar chumas do shamhail ríomhfhoghlama atá ann cheana (.i. atá réamh-thraenáilte) tasc eile a dhéanamh.
\\
 \noindent \textit{Sainmhíniú (en):} The practice of fine-training a pre-existing (pre-trained) machine learning model to produce a new model that solves a different task.
\\
 \noindent \textit{Tagairtí:}
\begin{itemize}
	\item tras-: De Bhaldraithe (1978) \cite{de-bhaldraithe}, Ó Dónaill et al. (1991) \cite{focloir-beag}, Ó Dónaill (1977) \cite{odonaill}
	\item foghlaim: féach ar an téarma `machine learning / ríomhfhoghlaim'
\end{itemize}

 \noindent \textit{Nótaí Aistriúcháin:}
\begin{itemize}
	\item Níl aistriúchán déanta ar an téarma seo cheana go bhfios don údar (fiú ar Téarma.ie).
	\item Bhí rogha eile ann chun an téarma seo a aistriú -- `foghlaim aistrithe'. Tá an-chiall leis sin -- déanann `transfer learning' córas ríomhfhoghlama ar traenáladh é i gcomhthéacs amháin a aistriú go comhthéacs eile. Roghnaíodh `trasfhoghlaim' seachas `foghlaim aistrithe' toisc gur rud sainmhínithe atá i gceist le `transfer learning'. Is féidir go leor bríonna a shamhlú le `foghlaim aistrithe' (m.sh. aistriúchán uath-oibríoch, foghlaim bunaithe ar aistriú leabuithe, srl) Meastar gur léire `trasfhoghlaim' mar sin.
	\item Luann Téarma.ie an téarma `traschur' mar `transfer'. Sin ráite, níl an téarma sin le feiceáil i bhFoclóir Uí Dhónaill, i bhFoclóir Uí Dhuinín, i bhFoclóir Uí Dhónaill agus Uí Mhaoileoin, ná i bhFoclóir De Bhaldraithe. Ní ghlactar leis sin mar sin.
\end{itemize}


\subsubsection*{transform (ainmfhocal): trasfhoirm}
 \noindent \textit{Sainmhíniú (ga):} I gcomhthéacs matamaitice, feidhm a úsáidtear chun trasfhoirmiú a chur i gcrích.
\\
 \noindent \textit{Sainmhíniú (en):} In the context of mathematics, a function that is used to perform a transformation.
\\
 \noindent \textit{Tagairtí:}
\begin{itemize}
	\item trasfhoirm: Williams et al. (2023) \cite{storchiste}
\end{itemize}

 \noindent \textit{Nótaí Aistriúcháin:}
\begin{itemize}
	\item Téarma díreach ar fáil ó Stórchiste i gcomhthéacs matamaitice.
	\item Féach chomh maith ar an téarma `to transform / trasfhoirmigh'.
\end{itemize}


\subsubsection*{to transform (briathar): trasfhoirmigh}
 \noindent \textit{Sainmhíniú (ga):} I gcomhthéacs matamaitice, sonraí a athrú le feidhm éigin (mar shampla, chun iad a chuimsiú ar eatramh éigin).
\\
 \noindent \textit{Sainmhíniú (en):} In the context of mathematics, to change data according to some function (for example, to bound them on a given interval).
\\
 \noindent \textit{Tagairtí:}
\begin{itemize}
	\item trasfhoirmigh: Williams et al. (2023) \cite{storchiste}
\end{itemize}

 \noindent \textit{Nótaí Aistriúcháin:}
\begin{itemize}
	\item Téarma díreach ar fáil ó Stórchiste i gcomhthéacs matamaitice.
\end{itemize}


\subsubsection*{transformation (ainmfhocal): trasfhoirmiú}
 \noindent \textit{Sainmhíniú (ga):} I gcomhthéacs matamaitice, an próiseas a bhaineann le trasfhoirm a chur i bhfeidhm ar shonraí.
\\
 \noindent \textit{Sainmhíniú (en):} In the context of mathematics, the process of transforming data.
\\
 \noindent \textit{Tagairtí:}
\begin{itemize}
	\item trasfhoirmiú: Williams et al. (2023) \cite{storchiste}
\end{itemize}

 \noindent \textit{Nótaí Aistriúcháin:}
\begin{itemize}
	\item Téarma díreach ar fáil ó Stórchiste i gcomhthéacs matamaitice.
	\item Féach chomh maith ar an téarma `to transform / trasfhoirmigh'.
\end{itemize}


\subsubsection*{transformer (ainmfhocal): trasfhoirmeoir}
 \noindent \textit{Sainmhíniú (ga):} I gcomhthéacs ríomhfhoghlama, dearadh néarach áirithe a úsáideann sraith chisil airde le chéile.
\\
 \noindent \textit{Sainmhíniú (en):} In the context of machine learning, a specific neural architecture that uses layered attention.
\\
 \noindent \textit{Tagairtí:}
\begin{itemize}
	\item trasfhoirmigh: féach ar an téarma `to transform / trasfhoirmigh'
	\item -eoir: De Bhaldraithe (1978) \cite{de-bhaldraithe}, Dineen (1934) \cite{dineen}, Ó Dónaill et al. (1991) \cite{focloir-beag}, Ó Dónaill (1977) \cite{odonaill}, Williams et al. (2023) \cite{storchiste}
\end{itemize}

 \noindent \textit{Nótaí Aistriúcháin:}
\begin{itemize}
	\item Níl an téarma `trasfhoirmeoir' luaite i bhFoclóir ar bith, ach is féidir é a chruthú go díreach as an bhfréamh `trasfhoirmigh' agus as an iarmhír `-eoir'.
	\item Níl iontráil ar leith den iarmhír `-eoir' sna foclóirí thuas, ach luann siad uilig go leor focal a úsáideann í díreach mar a úsáidtear anseo.
	\item Féach chomh maith ar an téarma `to transform / trasfhoirmigh'.
\end{itemize}


\subsubsection*{transitive (aidiacht): aistreach}
 \noindent \textit{Sainmhíniú (ga):} I gcomhthéacs faisnéise (f) i ngraf eolais, leis an impleacht gur fíor (a,f,c) mar abairt thiarach más fíor (a,f,b) agus (b,f,c).
\\
 \noindent \textit{Sainmhíniú (en):} In the context of a predicate (p) in a knowledge graph, implying that the triple (a,p,c) is true if (a,p,b) ang (b,p,c) are true.
\\
 \noindent \textit{Tagairtí:}
\begin{itemize}
	\item aistreach: De Bhaldraithe (1978) \cite{de-bhaldraithe}, Ó Dónaill (1977) \cite{odonaill}, Williams et al. (2023) \cite{storchiste}
\end{itemize}

 \noindent \textit{Nótaí Aistriúcháin:}
\begin{itemize}
	\item Téarma díreach ar fáil le brí chomhchosúil. I bhFoclóir De Bhaldraithe luaitear é i gcomhthéacs gramadaí. Toisc mórchuid na dtéarmaí a bhaineann le graif eolais a bheith úsáidte mar analach le gramadach (m.sh. ainmní, faisnéis, agus cuspóir), meastar go gcloíonn an comhthéacs seo go díreach le comhthéacs na ngraf eolais.
	\item I Stórchiste, tá an téarma seo luaite mar théarma matamaitice.
	\item Cé go luann Foclóir Uí Dhónaill agus Uí Mhaoileoin an téarma seo, is i gcomhthéacs iomlán ar leith atá sé luaite ann.
\end{itemize}


\subsubsection*{tree (ainmfhocal): crann}
 \noindent \textit{Sainmhíniú (ga):} I gcomhthéacs ríomheolaíochta, struchtúr sonraí nó ríomhfhoghlama a bhfuil nód fréimhe amháin aige agus craobhacha faoi á cheangal le nóid eile sa gcrann. Is féidir crann a úsáid chun próiseas / algartam a chur in iúl, nó chun sonraí a stóráil agus a chuardach go héifeachtach.
\\
 \noindent \textit{Sainmhíniú (en):} In the context of computer science, a data or machine learning structure that has a single root node and branches beneath it that connect to other nodes in the tree. A tree can be used to represent a process / algorithm, or to store and search data efficiently.
\\
 \noindent \textit{Tagairtí:}
\begin{itemize}
	\item crann: De Bhaldraithe (1978) \cite{de-bhaldraithe}, Dineen (1934) \cite{dineen}, Ó Dónaill et al. (1991) \cite{focloir-beag}, Ó Dónaill (1977) \cite{odonaill}
\end{itemize}

 \noindent \textit{Nótaí Aistriúcháin:}
\begin{itemize}
	\item Ní luann foclóir ar bith de na foclóir thuas an focal `crann' i gcomhthéacs matamaitice / ríomheolaíochta. Cruthaíodh an focal seo i mBéarla toisc an chrutha atá ar chrainn ríomheolaíochta agus iad scríofa ar pháipéar. Toisc nach bhfuil focal eile ann sa nGaeilge cheana chun an bhrí atá i gceist anseo a chur in iúl, ní mheastar go bhfuil cúis ar bith gan glacadh leis an analach chéanna i nGaeilge agus `crann' a úsáid.
\end{itemize}


\subsubsection*{triple (ainmfhocal): abairt thriarach}
 \noindent \textit{Sainmhíniú (ga):} abairt trí choda $(a,f,c)$ atá mar aonad eolais bunúsach i ngraif eolais.
\\
 \noindent \textit{Sainmhíniú (en):} a three-part statement $(s,p,o)$ that acts as the basic unit of knowledge in knowledge graphs.
\\
 \noindent \textit{Tagairtí:}
\begin{itemize}
	\item abairt: De Bhaldraithe (1978) \cite{de-bhaldraithe}, Dineen (1934) \cite{dineen}, Ó Dónaill et al. (1991) \cite{focloir-beag}, Ó Dónaill (1977) \cite{odonaill}
	\item triarach: De Bhaldraithe (1978) \cite{de-bhaldraithe}, Dineen (1934) \cite{dineen}, Ó Dónaill (1977) \cite{odonaill}
\end{itemize}

 \noindent \textit{Nótaí Aistriúcháin:}
\begin{itemize}
	\item Cé go bhfuil an focal `triarach' ar Téarma.ie leis an mbrí chéanna (nach mór), ní ghlacfar leis sin toisc nach bhfuil fianaise ar bith ann sna foclóirí dúchasacha gur féidir `triarach' a úsáid mar ainmfhocal. Sna foclóirí dúchasacha, ní luaitear é ach mar aidiacht. Mar sin, úsáidtear mar aidiacht amháin anseo é.
	\item Úsáidtear `abairt' i gcomhthéacs gramadaí, seachas `ráiteas'. Bíonn nach mór chuile rud a bhaineann le graif eolais samhlaithe trí analach le teangeolaíocht / le gramadach. Mar sin, meastar gur léire cloí le `abairt thriarach' seachas `ráiteas triarach' (nó téarma eile mar sin).
\end{itemize}


\phantomsection \subsection*{U}
\addcontentsline{toc}{subsection}{U}
\markboth{U}{U}

\subsubsection*{underfitting (ainmfhocal): foghlaim easnamhach}
 \noindent \textit{Sainmhíniú (ga):} I gcomhthéacs ríomhfhoghlama, foghlaim neamh-iomlán a fhágann nach bhfuil an tsamhail in ann patrúin ghinearálta a fhoghlaim ón tacar traenála.
\\
 \noindent \textit{Sainmhíniú (en):} In the context of machine learning, incomplete learning that results in the model not being able to learn general patterns from the training set.
\\
 \noindent \textit{Tagairtí:}
\begin{itemize}
	\item foghlaim: féach ar an téarma `machine learning / ríomhfhoghlaim'
	\item easnamhach: De Bhaldraithe (1978) \cite{de-bhaldraithe}, Dineen (1934) \cite{dineen}*, Ó Dónaill et al. (1991) \cite{focloir-beag}, Ó Dónaill (1977) \cite{odonaill}
\end{itemize}

 \noindent \textit{Nótaí Aistriúcháin:}
\begin{itemize}
	\item * Is é `easnamh' seachas `easnamhach' atá i bhFoclóir Uí Dhuinín.
	\item Féach chomh maith ar an téarma `machine learning / ríomhfhoghlaim'
\end{itemize}


\subsubsection*{union (ainmfhocal): aontas}
 \noindent \textit{Sainmhíniú (ga):} I gcomhthéacs dhá thacar (nó níos mó) tacar na mball in ar a laghad tacar amháin acu.
\\
 \noindent \textit{Sainmhíniú (en):} In the context of two or more sets, the set of elements contained in at least one of those sets.
\\
 \noindent \textit{Tagairtí:}
\begin{itemize}
	\item aontas: Ó Dónaill (1977) \cite{odonaill}
\end{itemize}

 \noindent \textit{Nótaí Aistriúcháin:}
\begin{itemize}
	\item Téarma le fáil i bhFoclóir Uí Dhónaill leis an mbrí cheannann chéanna.
\end{itemize}


\subsubsection*{universe (ainmfhocal): domhan}
 \noindent \textit{Sainmhíniú (ga):} I gcomhthéacs matamaitice, tacar formhór uilíoch ina bhfuil gach uile réad a bhaineann le réimse éigin ann.
\\
 \noindent \textit{Sainmhíniú (en):} In the context of mathematics, a massive, global set in which all objects belonging to a given domain exist.
\\
 \noindent \textit{Tagairtí:}
\begin{itemize}
	\item domhan: féach ar an téarma `world / domhan'
\end{itemize}

 \noindent \textit{Nótaí Aistriúcháin:}
\begin{itemize}
	\item Féach ar an téarma `world / domhan'.
\end{itemize}


\subsubsection*{unseen (aidiacht): gan feiceáil}
 \noindent \textit{Sainmhíniú (ga):} I gcomhthéacs tacar sonraí (m.sh. tacar teisteála / deimhnithe), gan a bheith úsáidte / feicthe le linn an phróisis thraenála ar shamhail ríomhfhoghlama.
\\
 \noindent \textit{Sainmhíniú (en):} In the context of a dataset (such as the testing / validation set), not being used / seen during the training phase of a machine learning model.
\\
 \noindent \textit{Tagairtí:}
\begin{itemize}
	\item gan feiceáil: De Bhaldraithe (1978) \cite{de-bhaldraithe}, Ó Dónaill (1977) \cite{odonaill}
\end{itemize}

 \noindent \textit{Nótaí Aistriúcháin:}
\begin{itemize}
	\item Téarma iomlán ar fáil ó na foclóirí thuas.
\end{itemize}


\subsubsection*{unsupervised (aidiacht): gan mhaoirseacht}
 \noindent \textit{Sainmhíniú (ga):} I gcomhthéacs ríomhfhoghlama, traenáilte ar shonraí ionchuir gan samplaí ar bith dá bhfuiltear ag súil leis mar aschur. Mar shampla, nuair a dhéantar aicmiú ionduchtach, bíonn pointí ionchur ann, ach ní bhíonn lipéad / aicme cheart na bpointe ar fáil roimh ré.
\\
 \noindent \textit{Sainmhíniú (en):} In the context of machine learning, trained on input data without any examples of what is expected as output. For example, when performing clustering, input points are given, but no information on their labels / classes are given.
\\
 \noindent \textit{Tagairtí:}
\begin{itemize}
	\item maoirseacht: féach ar an téarma `supervised / faoi mhaoirseacht'
\end{itemize}

 \noindent \textit{Nótaí Aistriúcháin:}
\begin{itemize}
	\item Féach ar an téarma `supervised / faoi mhaoirseacht'.
	\item Féach ar an téarma `supervision / maoirseacht'.
\end{itemize}


\phantomsection \subsection*{V}
\addcontentsline{toc}{subsection}{V}
\markboth{V}{V}

\subsubsection*{to validate (ainmfhocal): deimhnigh}
 \noindent \textit{Sainmhíniú (ga):} Próiseas deimhnithe a dhéanamh ar shamhail ríomhfhoghlama.
\\
 \noindent \textit{Sainmhíniú (en):} To perform validation on a machine learning model.
\\
 \noindent \textit{Tagairtí:}
\begin{itemize}
	\item deimhnigh: féach ar an téarma `validation / deimhniú'
\end{itemize}

 \noindent \textit{Nótaí Aistriúcháin:}
\begin{itemize}
	\item Féach ar an téarma `validation / deimhniú'
\end{itemize}


\subsubsection*{validation (ainmfhocal): deimhniú}
 \noindent \textit{Sainmhíniú (ga):} An próiseas a úsáidtear chun a mheas cé chomh maith (nó cé chomh dona) is a fheidhmíonn samhail ríomhfhoghlama le linn a traenála.
\\
 \noindent \textit{Sainmhíniú (en):} The process that is used to estimate how well (or how poorly) a machine learning model works while it is being trained.
\\
 \noindent \textit{Tagairtí:}
\begin{itemize}
	\item deimhniú: De Bhaldraithe (1978) \cite{de-bhaldraithe}, Dineen (1934) \cite{dineen}, Ó Dónaill et al. (1991) \cite{focloir-beag}*, Ó Dónaill (1977) \cite{odonaill}
\end{itemize}

 \noindent \textit{Nótaí Aistriúcháin:}
\begin{itemize}
	\item Níl an téarma seo luaite i gcomhthéacs ríomhaireachta sna foclóirí thuas, ach is le brí chomhchosúil atá sé luaite.
\end{itemize}


\subsubsection*{validation set (ainmfhocal): tacar deimhnithe}
 \noindent \textit{Sainmhíniú (ga):} Tacar sonraí a úsáidtear chun samhail ríomhfhoghlama a dheimhniú.
\\
 \noindent \textit{Sainmhíniú (en):} The dataset used to validate a machine learning model.
\\
 \noindent \textit{Tagairtí:}
\begin{itemize}
	\item tacar: féach ar an téarma `set / tacar'
	\item deimhniú: féach ar an téarma `validation / deimhniú'
\end{itemize}

 \noindent \textit{Nótaí Aistriúcháin:}
\begin{itemize}
	\item Féach ar an téarma `set / tacar'.
	\item Féach ar an téarma `validation / deimhniú'.
\end{itemize}


\subsubsection*{variance (ainmfhocal): athraitheas}
 \noindent \textit{Sainmhíniú (ga):} I gcomhthéacs matamaitice, tomhas ar cé chomh héagsúil is atá luachanna i sraith sonraí. Is ionann athraitheas agus diall caighdeánach cearnaithe.
\\
 \noindent \textit{Sainmhíniú (en):} In the context of mathematics, a measure of how much the values in a dataset vary. Variance is the square of standard deviation.
\\
 \noindent \textit{Tagairtí:}
\begin{itemize}
	\item athraitheas: Williams et al. (2023) \cite{storchiste}
\end{itemize}

 \noindent \textit{Nótaí Aistriúcháin:}
\begin{itemize}
	\item Téarma díreach ar fáil leis an mbrí cheannann chéanna i gcomhthéacs matamaitice.
\end{itemize}


\subsubsection*{vector (ainmfhocal): veicteoir}
 \noindent \textit{Sainmhíniú (ga):} Liosta ordaithe uimhreacha a shamhlaíonn pointe i spás, nó aistriú sa spás céanna.
\\
 \noindent \textit{Sainmhíniú (en):} An ordered list of numbers that represents a displacement in space, or a point in space.
\\
 \noindent \textit{Tagairtí:}
\begin{itemize}
	\item veicteoir: De Bhaldraithe (1978) \cite{de-bhaldraithe}, Ó Dónaill (1977) \cite{odonaill}, Williams et al. (2023) \cite{storchiste}
\end{itemize}

 \noindent \textit{Nótaí Aistriúcháin:}
\begin{itemize}
	\item Luann na foclóirí thuas `veicteoir' mar théarma matamaitice.
\end{itemize}


\phantomsection \subsection*{W}
\addcontentsline{toc}{subsection}{W}
\markboth{W}{W}

\subsubsection*{walk (ainmfhocal): siúlóid}
 \noindent \textit{Sainmhíniú (ga):} I gcomhthéacs graif, imeacht ó nód go nód ar cheangail an ghraif chun an graf ar fad, nó cuid de, a thrasnú.
\\
 \noindent \textit{Sainmhíniú (en):} In the context of a graph, to go from node to node alone the edges in the graph to traverse the whole graph, or a part of it.
\\
 \noindent \textit{Tagairtí:}
\begin{itemize}
	\item siúlóid: De Bhaldraithe (1978) \cite{de-bhaldraithe}, Dineen (1934) \cite{dineen}*, Ó Dónaill et al. (1991) \cite{focloir-beag}, Ó Dónaill (1977) \cite{odonaill}
\end{itemize}

 \noindent \textit{Nótaí Aistriúcháin:}
\begin{itemize}
	\item * Is é `siubhlóid' atá i bhFoclóir Uí Dhuinín, ach glactar leis gurb in an focal céanna.
	\item Ní luann foclóir ar bith an téarma seo i gcomhthéacs ríomheolaíochta. Sin ráite, tá cosúlacht idir siúlóid ar ghraif agus gnáth-shiúlóid -- bíonn siad araon ag dúl ó áit go háit ar bhealaí áirithe. Glactar leis an téarma seo mar sin.
\end{itemize}


\subsubsection*{to weight (briathar): ualaigh}
 \noindent \textit{Sainmhíniú (ga):} I gcomhthéacs ríomhfhoghlama nó matamaiticiúil, ualach a chur le slonn nó le luach.
\\
 \noindent \textit{Sainmhíniú (en):} In the context of machine learning or mathematics, to apply a numerical weight to an expression or value.
\\
 \noindent \textit{Tagairtí:}
\begin{itemize}
	\item ualaigh: De Bhaldraithe (1978) \cite{de-bhaldraithe}, Ó Dónaill (1977) \cite{odonaill}, Williams et al. (2023) \cite{storchiste}
\end{itemize}

 \noindent \textit{Nótaí Aistriúcháin:}
\begin{itemize}
	\item Luann Stórchiste agus Foclóir Uí Dhónaill `ualaigh' mar théarma staitistiúil. (Is sa téarma `meán ualaithe' a fheictear an téarma seo i Stórchiste.)
	\item Féach chomh maith ar an téarma `weight / ualach'.
	\item Athrú ó leagan v1.1 alfa -- bhí téarma mícheart ('uimhir ualaigh a chur le') sa bhFoclóir Tráchtais. Rinneadh é sin a athrú toisc fianaise ó Stórchiste agus ó Fhoclóir Uí Dhónaill.
\end{itemize}


\subsubsection*{weight (ainmfhocal): ualach}
 \noindent \textit{Sainmhíniú (ga):} I gcomhthéacs ríomhfhoghlama nó matamaiticiúil, uimhir scálach a mhéadaítear le slonn (go háirithe chun raon na luachanna aschuir ón slonn sin a athrú nó a shrianadh); nó, uimhir scálach a úsáidtear mar lipéad (mar shampla, ar chodanna de ghraf).
\\
 \noindent \textit{Sainmhíniú (en):} In the context of machine learning or mathematics, a scalar number that is multiplied with an expression (especially to change or restrict the range of values output by that expression); or, a scalar value used as a label (such as on elements of a graph).
\\
 \noindent \textit{Tagairtí:}
\begin{itemize}
	\item ualach: De Bhaldraithe (1978) \cite{de-bhaldraithe}, Dineen (1934) \cite{dineen}, Ó Dónaill et al. (1991) \cite{focloir-beag}*, Ó Dónaill (1977) \cite{odonaill}, Williams et al. (2023) \cite{storchiste}
\end{itemize}

 \noindent \textit{Nótaí Aistriúcháin:}
\begin{itemize}
	\item Luann Foclóir Uí Dhónaill agus Stórchiste `ualaigh' (briathar) mar théarma staitistiúil, ach ní luann siad `ualach' sa gcomhthéacs céanna. Sin ráite, tá an fréamh céanna acu araon, agus ní féidir frása eile le `ualach' (m.sh. uimhir ualaithe) a chumadh toisc go mbeadh ciall ar leith leis sin (.i. uimhir a bhfuil ualach curtha leis, seachas an t-ualach é féin).
	\item Tá ciall leis an téarma `ualach' -- ní Béarlachas é. I gcomhthéacs an meicnice staitistiúla, tá baint díreach idir ualaigh (mar uimhreacha) agus ualaigh mheicniúla. Ní hé gur aistriúchán ar fhocal Béarla é `ualach', ach gur leagan Gaeilge de choincheap eolaíochta atá ann.
	\item Tá `weight $\rightarrow$ ualach' ar fáil ar Téarma.ie -- agus tá ciall leis sin toisc go nglacann Foclóir Uí Dhónaill leis an téarma `ualaigh' mar théarma staitistiúil. 
	\item Cé go bhfuil focail le bríonna comhchosúla leis seo (uimhir, uimhir scálach, paraiméadar, srl), ní oireann ceann ar bith acu don bhrí shainmhínithe atá de dhíth anseo. Is féidir iad a úsáid in ionad an téarma seo go minic, ach ní i gcónaí.
	\item Athrú ó v1.1 alfa -- bhí an téarma seo aistrithe mar `uimhir ualaigh' cheana, as `uimhir' agus `ualach'. Ní shin an téarma ceart, áfach, agus rinneadh é a chur i gceart don leagan seo den Fhoclóir Tráchtais.
\end{itemize}


\subsubsection*{weighted (aidiacht): ualaithe}
 \noindent \textit{Sainmhíniú (ga):} I gcomhthéacs ríomhfhoghlama nó matamaiticiúil, a bhfuil ualach curtha leis.
\\
 \noindent \textit{Sainmhíniú (en):} In the context of machine learning or mathematics, having a numerical weight.
\\
 \noindent \textit{Tagairtí:}
\begin{itemize}
	\item ualaigh: De Bhaldraithe (1978) \cite{de-bhaldraithe}, Ó Dónaill (1977) \cite{odonaill}, Williams et al. (2023) \cite{storchiste}
\end{itemize}

 \noindent \textit{Nótaí Aistriúcháin:}
\begin{itemize}
	\item Luann Stórchiste agus Foclóir Uí Dhónaill `ualaigh' mar théarma staitistiúil. (Is sa téarma `meán ualaithe' a fheictear an téarma seo i Stórchiste.)
	\item Féach chomh maith ar an téarma `weight / ualach'.
\end{itemize}


\subsubsection*{weighted random search (ainmfhocal): cuardach randamach ualaithe}
 \noindent \textit{Sainmhíniú (ga):} I gcomhthéacs cuardaithe hipear-pharaiméadar, cuardach randamach ina bhfuil an sampláil ualaithe de réir cé chomh maith is a bhí samplaí eile a mheasúnaíodh cheana (mar shampla, trí shampláil Bayes a dhéanamh).
\\
 \noindent \textit{Sainmhíniú (en):} In the context of a hyperparameter search, a random search in which the sampling is weighted by how well past samples have performed (for example, through the use of Bayesian sampling).
\\
 \noindent \textit{Tagairtí:}
\begin{itemize}
	\item cuardach randamach: féach ar an téarma `random search / cuardach randamach'
	\item ualaithe: féach ar an téarma `weighted / ualaithe'
\end{itemize}

 \noindent \textit{Nótaí Aistriúcháin:}
\begin{itemize}
	\item Féach ar an téarma `hyperparameter search / cuardach hipear-pharaiméadar'.
	\item Féach chomh maith ar an téarma `random search / cuardach randamach''.
	\item Féach chomh maith ar an téarma `weighted / ualaithe'.
	\item Féach chomh maith ar an téarma `end condition / coinníoll críochnaithe'.
\end{itemize}


\subsubsection*{world (ainmfhocal): domhan}
 \noindent \textit{Sainmhíniú (ga):} I gcomhthéacs matamaitice, tacar formhór uilíoch ina bhfuil gach uile réad a bhaineann le réimse éigin ann.
\\
 \noindent \textit{Sainmhíniú (en):} In the context of mathematics, a massive, global set in which all objects belonging to a given domain exist.
\\
 \noindent \textit{Tagairtí:}
\begin{itemize}
	\item domhan: De Bhaldraithe (1978) \cite{de-bhaldraithe}, Dineen (1934) \cite{dineen}, Ó Dónaill et al. (1991) \cite{focloir-beag}, Ó Dónaill (1977) \cite{odonaill}
\end{itemize}

 \noindent \textit{Nótaí Aistriúcháin:}
\begin{itemize}
	\item I mBéarla, úsáidtear `world' chun trácht ar `the world of all possible facts / things / etc' -- sin le rá, chun trácht ar chuile fhíric / rud atá ann (i réimse éigin). Ní luaitear an bhrí seo go minic le `domhain' i nGaeilge, ach tá brí chosúil go leor leis le feiceáil i bhFoclóir Uí Dhuinín: `Slua, eolas, acmhainn, an domhain, vast crowd, knowledge, resources'. Is féidir nach téarma foirfe é seo -- ach ní léir go mbeadh téarma comhchiallach eile (m.sh. cruinne) níos fearr. Glactar leis mar sin.
\end{itemize}


\phantomsection \subsection*{Z}
\addcontentsline{toc}{subsection}{Z}
\markboth{Z}{Z}

\subsubsection*{z-score (ainmfhocal): scór-z}
 \noindent \textit{Sainmhíniú (ga):} I gcomhthéacs staitistice, scór a léiríonn cá mhéad dialltaí caighdeánacha is atá sonra amháin ó mheán na sonraí.
\\
 \noindent \textit{Sainmhíniú (en):} In the context of statistics, a score that represents how many standard deviations from the mean a data point is.
\\
 \noindent \textit{Tagairtí:}
\begin{itemize}
	\item scór: féach ar an téarma `score / scór'
\end{itemize}

 \noindent \textit{Nótaí Aistriúcháin:}
\begin{itemize}
	\item Is é `z-scór' atá ar Téarma.ie. Ní ghlactar leis sin toisc fianaise ó Stórchiste sa téarma `Student's t-distribtion / dáileadh-t Student' (tabhair faoi deara gur ainm dílis é `Student'). Toisc go ndéanann Stórchiste `t-distribtion' a aistriú mar `dáileadh-t', meastar go mba cheart `z-score' a aistriú mar `scór-z'.
	\item Féach ar an téarma `score / scór'.
\end{itemize}



\newpage
\printbibliography[
title={Tagairtí},
heading=bibintoc
]
\end{document}

